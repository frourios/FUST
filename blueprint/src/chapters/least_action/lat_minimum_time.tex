% FUST Least Action Theorem - Minimum Time Gap
\chapter{Minimum Time Gap (FUST Planck Time)}
\label{chap:mat_minimum_time}

\section{Structural Derivation}
\label{sec:mat_min_time_derivation}

The minimum time is not derived from numerical fitting, but necessarily from the $D_6$ structure.

\textbf{Core facts}:
\begin{itemize}
\item In $\ker(D_6) = \{1, x, x^2\}$, time is not defined ($D_6 = 0$)
\item $n = 3$ is the minimum degree at which ``time begins to exist''
\item $C_3$ is the minimum non-zero dissipation coefficient
\end{itemize}

\begin{theorem}[$C_3$ Value]
\label{thm:mat_C3}
\lean{FUST.C3_eq_12_sqrt5}
\leanok
\[
C_3 = \varphi^9 - 3\varphi^6 + \varphi^3 - \psi^3 + 3\psi^6 - \psi^9 = 12\sqrt{5}
\]
\end{theorem}

\section{Definition of Structural Minimum Time}
\label{sec:mat_structural_min_time}

The minimum scale at which $D_6$ begins to detect temporal change:

\begin{definition}[Structural Minimum Time]
\label{def:mat_min_time}
\lean{FUST.structuralMinTime}
\leanok
\[
t_{\mathrm{FUST}} = \frac{(\sqrt{5})^5}{C_3} = \frac{(\sqrt{5})^5}{12\sqrt{5}} = \frac{(\sqrt{5})^4}{12} = \frac{25}{12}
\]
\end{definition}

\begin{theorem}[Structural Minimum Time Value]
\label{thm:mat_min_time_value}
\lean{FUST.structuralMinTime_eq}
\leanok
$t_{\mathrm{FUST}} = 25/12$
\end{theorem}

\begin{theorem}[Structural Minimum Time Positivity]
\label{thm:mat_min_time_pos}
\lean{FUST.structuralMinTime_positive}
\leanok
$t_{\mathrm{FUST}} > 0$
\end{theorem}

\section{Why It Cannot Be Further Divided}
\label{sec:mat_indivisible}

\begin{theorem}[Structural Reason for Minimum Time]
\label{thm:mat_min_time_reason}
\lean{FUST.minimum_time_structural_reason}
\leanok
\begin{enumerate}
\item $\ker(D_6) = \{1, x, x^2\}$ is not detected by $D_6$
\item Functions with $n \leq 2$ ``do not experience time''
\item $n = 3$ is the minimum degree for temporal change
\item $t_{\mathrm{FUST}} = 25/12$ is derived from this boundary
\end{enumerate}
\end{theorem}

\section{Physical Interpretation}
\label{sec:mat_min_time_physics}

\begin{center}
\begin{tabular}{|c|c|c|}
\hline
Scale & $D_6$ behavior & Physical meaning \\
\hline
$\ker(D_6)$ & $D_6 = 0$ & No time (light-like) \\
$n = 3$ & $C_3 = 12\sqrt{5}$ & Minimum temporal change \\
$n \geq 4$ & $C_n > C_3$ & Stronger dissipation \\
\hline
\end{tabular}
\end{center}

$t_{\mathrm{FUST}} = 25/12$ is:
\begin{itemize}
\item The minimum time interval that $D_6$ can ``resolve''
\item A structural consequence from the $\ker(D_6)$ boundary
\item A purely algebraic derivation, not numerical fitting
\end{itemize}

\section{Dimensional Analysis}
\label{sec:mat_min_time_dim}

The derivation $t_{\mathrm{FUST}} = (\sqrt{5})^5 / C_3$ has dimension:
\[
\dim(t_{\mathrm{FUST}}) = (5, 0, 0) - (0, 1, 0) + (0, 0, 1) = (5, -1, 1)
\]
where $(5,0,0)$ is the normalization denominator $(\sqrt{5})^5$, $(0,1,0)$ is the dissipation coefficient $C_3$, and $(0,0,1)$ marks this as a temporal quantity ($x^3 \notin \ker(D_6)$).

The length dimension $\dim(l) = (5, -1, 0)$ shares $(\sqrt{5}, \delta) = (5, -1)$ with time but has $\tau = 0$ (spatial). The $\tau$ distinction is preserved by the type system: $\dim(c) = \dim(l) - \dim(t) = (0, 0, -1)$.

Lean: \texttt{FUST.Dim.structuralMinTime\_dim}.

