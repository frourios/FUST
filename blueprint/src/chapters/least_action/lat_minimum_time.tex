% FUST Least Action Theorem - Planck Second from Poincaré Temporal Channel
\chapter{Planck Second from Poincar\'{e} Temporal Channel}
\label{chap:mat_minimum_time}

\section{Derivation from \texorpdfstring{$D_\zeta$}{Dζ} 4-Momentum}
\label{sec:mat_planck_derivation}

The Planck second is derived from the Poincar\'{e} $I_4 = \mathrm{Fin}\,3 \oplus \mathrm{Fin}\,1$ decomposition of the $D_\zeta$ operator, not from numerical fitting.

\textbf{Core facts}:
\begin{itemize}
\item $D_\zeta$ decomposes into 3 spatial channels ($\Phi_S$) + 1 temporal channel ($\Phi_A$)
\item The temporal channel coefficient at minimum active mode $s=1$: $\Phi_A(1) = 2\sqrt{5}$
\item $F_\zeta$ temporal eigenvalue $= 5 \cdot \Phi_A(1) \cdot \mathrm{AF} = 10\sqrt{5} \cdot 2i\sqrt{3} = 20i\sqrt{15}$
\end{itemize}

\begin{theorem}[$C_3$ Value]
\label{thm:mat_C3}
\lean{FUST.C3_eq_12_sqrt5}
\leanok
\[
C_3 = \varphi^9 - 3\varphi^6 + \varphi^3 - \psi^3 + 3\psi^6 - \psi^9 = 12\sqrt{5}
\]
\end{theorem}

\section{Planck Second Definition}
\label{sec:mat_planck_second}

\begin{definition}[Planck Second]
\label{def:mat_planck_second}
\lean{FUST.TimeStructure.planckSecond}
\leanok
\[
t_P = \frac{1}{20\sqrt{15}}
\]
Derived as the inverse of the temporal $F_\zeta$ eigenvalue norm at the minimum active mode $s=1$.
\end{definition}

\begin{theorem}[Planck Second Positivity]
\label{thm:mat_planck_second_pos}
\lean{FUST.TimeStructure.planckSecond_pos}
\leanok
$t_P > 0$
\end{theorem}

\begin{theorem}[Planck Second Squared]
\label{thm:mat_planck_second_sq}
\lean{FUST.TimeStructure.planckSecond_sq}
\leanok
$t_P^2 = 1/6000$
\end{theorem}

\section{Temporal Eigenvalue Decomposition}
\label{sec:mat_temporal_decomposition}

\begin{theorem}[Temporal Eigenvalue Norm]
\label{thm:mat_temporal_norm}
\lean{FUST.TimeStructure.temporalEigenNormSq_from_channels}
\leanok
\[
|\text{temporal eigenvalue}|^2 = (20\sqrt{15})^2 = 6000
\]
\end{theorem}

\begin{theorem}[Mass Formula Decomposition]
\label{thm:mat_mass_formula}
\lean{FUST.TimeStructure.temporalEigenNormSq_mass_formula}
\leanok
\[
6000 = 12 \cdot (10\sqrt{5})^2
\]
where $12 = |\mathrm{AF}|^2$ from $\zeta_6$ arithmetic and $10\sqrt{5} = 5 \cdot \Phi_A(1)$.
\end{theorem}

\section{Derivation Chain}
\label{sec:mat_derivation_chain}

\begin{enumerate}
\item $D_\zeta$ components at $\mathrm{Sum.inr}\;0$ = $\Phi_A(s)$ (temporal channel, Gravity.lean)
\item $\Phi_A(1) = 2(\varphi - \psi) = 2\sqrt{5}$ (minimum active mode)
\item $F_\zeta$ temporal eigenvalue $= 5 \cdot \Phi_A(1) \cdot \mathrm{AF} = 10\sqrt{5} \cdot 2i\sqrt{3} = 20i\sqrt{15}$
\item $t_P^2 = 1/|\text{temporal eigenvalue}|^2 = 1/6000$
\item $t_P = 1/(20\sqrt{15})$
\end{enumerate}

\section{Why It Cannot Be Further Divided}
\label{sec:mat_indivisible}

The structural reason for the minimum time is:
\begin{enumerate}
\item $\ker(F_\zeta) = \{1, x, x^2\}$ is not detected by $F_\zeta$
\item Functions with $n \leq 2$ ``do not experience time''
\item $s = 1$ ($n \equiv 1 \pmod{6}$) is the minimum active mode of $F_\zeta$
\item $t_P = 1/(20\sqrt{15})$ is derived from this mode's temporal eigenvalue
\end{enumerate}

\section{Physical Interpretation}
\label{sec:mat_min_time_physics}

\begin{center}
\begin{tabular}{|c|c|c|}
\hline
Scale & $D_\zeta$ behavior & Physical meaning \\
\hline
$\ker(F_\zeta)$ & $F_\zeta = 0$ & No time (vacuum) \\
$s = 1$ & temporal eigenvalue $= 20i\sqrt{15}$ & Minimum temporal change \\
$s \geq 5$ & larger eigenvalues & Stronger dissipation \\
\hline
\end{tabular}
\end{center}

$t_P = 1/(20\sqrt{15})$ is:
\begin{itemize}
\item The minimum time interval derived from Poincar\'{e} temporal channel
\item A structural consequence from the $D_\zeta \to I_4$ decomposition
\item A purely algebraic derivation, not numerical fitting
\end{itemize}
