% FUST Least Action Theorem - Minimum Time Gap
\chapter{Minimum Time Gap (FUST Planck Time)}
\label{chap:mat_minimum_time}

\section{Structural Derivation}
\label{sec:mat_min_time_derivation}

The minimum time is not derived from numerical fitting, but necessarily from the $D_6$ structure.

\textbf{Core facts}:
\begin{itemize}
\item In $\ker(D_6) = \{1, x, x^2\}$, time is not defined ($D_6 = 0$)
\item $n = 3$ is the minimum degree at which ``time begins to exist''
\item $C_3$ is the minimum non-zero dissipation coefficient
\end{itemize}

\begin{theorem}[$C_3$ Value]
\label{thm:mat_C3}
\lean{FUST.C3_eq_12_sqrt5}
\leanok
\[
C_3 = \varphi^9 - 3\varphi^6 + \varphi^3 - \psi^3 + 3\psi^6 - \psi^9 = 12\sqrt{5}
\]
\end{theorem}

\section{Definition of Structural Minimum Time}
\label{sec:mat_structural_min_time}

The minimum scale at which $D_6$ begins to detect temporal change:

\begin{definition}[Structural Minimum Time for $D_6$]
\label{def:mat_min_time}
\lean{FUST.LeastAction.structuralMinTimeD6}
\leanok
\[
t_{\mathrm{FUST}} = \frac{(\sqrt{5})^4}{12} = \frac{25}{12}
\]
This is derived from the spectral formula $t_{\min}^{D_m} = (\sqrt{5})^{m-1} / |C_{d_{\min}}|$ at $m=6$, $d_{\min}=3$, $C_3 = 12\sqrt{5}$.
\end{definition}

\begin{theorem}[Structural Minimum Time Value]
\label{thm:mat_min_time_value}
\lean{FUST.LeastAction.structuralMinTimeD6_eq}
\leanok
$t_{\mathrm{FUST}} = 25/12$
\end{theorem}

\begin{theorem}[Structural Minimum Time Positivity]
\label{thm:mat_min_time_pos}
\lean{FUST.LeastAction.structuralMinTimeD6_positive}
\leanok
$t_{\mathrm{FUST}} > 0$
\end{theorem}

\subsection{Per-Operator Structural Minimum Times}

Each operator $D_m$ has its own structural minimum time $t_{\min}^{D_m} = (\sqrt{5})^{m-1} / |C_{d_{\min}}|$:

\begin{center}
\begin{tabular}{|c|c|c|c|c|c|}
\hline
Operator & $\ker$ & $d_{\min}$ & $|C_{d_{\min}}|$ & Definition & $t_{\min}$ \\
\hline
$D_2$ & $\{1\}$ & 1 & $\sqrt{5}$ & $\sqrt{5}/\sqrt{5}$ & 1 \\
$D_3$ & $\{1\}$ & 1 & 1 & $(\sqrt{5})^2/1$ & 5 \\
$D_4$ & $\{x^2\}$ & 0 & $\sqrt{5}$ & $(\sqrt{5})^3/\sqrt{5}$ & 5 \\
$D_5$ & $\{1,x\}$ & 2 & 6 & $(\sqrt{5})^4/6$ & 25/6 \\
$D_6$ & $\{1,x,x^2\}$ & 3 & $12\sqrt{5}$ & $(\sqrt{5})^4/12$ & 25/12 \\
\hline
\end{tabular}
\end{center}

\begin{theorem}[Structural Minimum Time Hierarchy]
\label{thm:mat_min_time_hierarchy}
$t_{\min}^{D_2} < t_{\min}^{D_6} < t_{\min}^{D_5} < t_{\min}^{D_3} = t_{\min}^{D_4}$, i.e.,
\[
1 < \frac{25}{12} < \frac{25}{6} < 5
\]
(Individual $D_m$ structural minimum times are historical context; the D$\zeta$-unified theory uses only $t_{\mathrm{FUST}} = 25/12$ from D6.)
\end{theorem}

\section{Why It Cannot Be Further Divided}
\label{sec:mat_indivisible}

The structural reason for the minimum time is:
\begin{enumerate}
\item $\ker(D_6) = \{1, x, x^2\}$ is not detected by $D_6$
\item Functions with $n \leq 2$ ``do not experience time''
\item $n = 3$ is the minimum degree for temporal change
\item $t_{\mathrm{FUST}} = 25/12$ is derived from this boundary via the spectral formula
\end{enumerate}

\section{Physical Interpretation}
\label{sec:mat_min_time_physics}

\begin{center}
\begin{tabular}{|c|c|c|}
\hline
Scale & $D_6$ behavior & Physical meaning \\
\hline
$\ker(D_6)$ & $D_6 = 0$ & No time (light-like) \\
$n = 3$ & $C_3 = 12\sqrt{5}$ & Minimum temporal change \\
$n \geq 4$ & $C_n > C_3$ & Stronger dissipation \\
\hline
\end{tabular}
\end{center}

$t_{\mathrm{FUST}} = 25/12$ is:
\begin{itemize}
\item The minimum time interval that $D_6$ can ``resolve''
\item A structural consequence from the $\ker(D_6)$ boundary
\item A purely algebraic derivation, not numerical fitting
\end{itemize}

\section{Dimensional Analysis}
\label{sec:mat_min_time_dim}

The derivation $t_{\mathrm{FUST}} = (\sqrt{5})^4 / 12$ has dimension:
\[
\dim(t_{\mathrm{FUST}}) = (5, -1)
\]
where $p = 5$ is the $\sqrt{5}$-power and $\delta = -1$ is the dissipation component (inverse of $D_6$ output dimension).

Lean: \texttt{FUST.Dim.structuralMinTime\_dim}.

