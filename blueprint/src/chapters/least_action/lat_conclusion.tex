% FUST Least Action Theorem - Conclusion
\chapter{Gauge Invariance and Conclusion}
\label{chap:mat_conclusion}

\section{Gauge-Dependent and Gauge-Invariant Quantities}
\label{sec:mat_gauge_quantities}

\subsection{Gauge-Invariant (Established as Theorems)}

\begin{center}
\begin{tabular}{|c|c|c|}
\hline
Quantity & Lean4 theorem name & Description \\
\hline
$f \in \ker(D_6)$ & IsInKerD6\_implies\_D6\_zero & Non-existence of time (light-like state) \\
$f \notin \ker(D_6)$ & D6\_nonzero\_implies\_time & Existence of time (massive state) \\
Time delay ratio within same degree & same\_degree\_ratio\_gauge\_invariant & Causally comparable quantity \\
D6TimeDifferential = 0 & time\_differential\_zero\_gauge\_invariant & Membership in $\ker(D_6)$ \\
\hline
\end{tabular}
\end{center}

\subsection{Gauge-Dependent (Not Fixed as Theorems)}

\begin{center}
\begin{tabular}{|c|c|}
\hline
Quantity & Dependent parameter \\
\hline
Value of $D_6 f(x)$ & Gauge $x$ \\
Time delay ratio between different degrees & Gauge $x$ \\
Concrete value of perpProjection & Choice of interpolation points (but zero set is invariant) \\
\hline
\end{tabular}
\end{center}

\section{Established Results}
\label{sec:mat_results}

\begin{center}
\begin{tabular}{|c|c|c|}
\hline
Item & Content & Status \\
\hline
Action functional & $\mathcal{A}[f] = \int_0^\infty \|D_6 f\|^2 \frac{dx}{x}$ & Derived \\
$\varphi$-invariant measure & $dx/x$ (Haar measure) & Uniqueness proven \\
Euler-Lagrange & $\ker(D_6^\dagger D_6) = \ker(D_6)$ & Derived \\
Time existence equivalence & $\mathcal{A}[f] > 0 \iff \mathrm{TimeExists}\, f$ & Consistency confirmed \\
Numerical verification & $A \approx 0$ in $\ker(D_6)$, $A > 0$ outside & Verified \\
\hline
\end{tabular}
\end{center}

\section{Established Theorems (All Lean4 Proven)}
\label{sec:mat_theorems}

\begin{enumerate}
\item \textbf{Time Existence Theorem} (time\_existence\_theorem): $f \notin \ker(D_6) \iff$ Time exists
\item \textbf{Higher Order Reduction Theorem} (higher\_order\_reduction): $\ker(D_7) = \ker(D_6)$
\item \textbf{Homogeneous Decomposition Theorem} (same\_degree\_ratio\_gauge\_invariant): Time delay ratio is gauge-invariant only within same degree
\item \textbf{Causal Boundary Theorem} (causal\_boundary\_theorem): $\ker(D_6) =$ Light cone
\item \textbf{Gauge Transformation Rule} (D6\_gauge\_scaling): $D_6[f(c\cdot)](x) = c \cdot D_6[f](cx)$
\item \textbf{Interpolation Uniqueness Theorem} (kernel\_interpolation\_unique): Projection to $\ker(D_6)$ is unique at 3 points
\item \textbf{Time Differential Gauge Invariance} (time\_differential\_zero\_gauge\_invariant): D6TimeDifferential = 0 is equivalent to $\ker(D_6)$
\item \textbf{Mass Gap Theorem} (massGap\_complete): $\Delta = C_3/(\sqrt{5})^5 = 12/25 > 0$ derived from $D_6$ gauge-invariant output
\end{enumerate}

\section{Derived Concepts (Theorems, Not Definitions)}
\label{sec:mat_derived}

\begin{center}
\begin{tabular}{|c|c|}
\hline
Concept & Derivation \\
\hline
Photon state & $\ker(D_6) \cap$ Non-constant functions (has energy, no mass) \\
Mass state & $\ker(D_6)^\perp$ (has components of degree 3 or higher) \\
Existence of time & Deviation from $\ker(D_6)$ \\
Mass gap & $D_6$ gauge-invariant output $\Delta = C_3/(\sqrt{5})^5 = 12/25 = 1/t_{\mathrm{FUST}}$ \\
\hline
\end{tabular}
\end{center}

\section{Physical Correspondence}
\label{sec:mat_correspondence}

\begin{center}
\begin{tabular}{|c|c|}
\hline
Relativistic concept & FUST correspondence \\
\hline
Light cone & $\ker(D_6)$ \\
Photon & IsPhotonState ($\ker(D_6) \cap$ non-constant) \\
Massive particle & IsMassiveState ($\ker(D_6)^\perp$) \\
timelike/spacelike & Same degree/different degrees \\
Speed of light $c$ & Degree mixing prohibition condition \\
Mass gap & Minimum energy having proper time \\
\hline
\end{tabular}
\end{center}

\section{Core Claims}
\label{sec:mat_core_claims}

\begin{quote}
\textbf{The FUST action functional is uniquely determined from $D_6$ structure and contains no arbitrariness.}
\end{quote}

\begin{quote}
\textbf{Positive action is equivalent to proper time existing.}
\end{quote}

\begin{quote}
\textbf{Minimal action solutions belong to $\ker(D_6)$, corresponding to the light cone (photon worldlines).}
\end{quote}

\begin{quote}
\textbf{The ``existence'' of time is a theorem. The ``value'' of time is not a theorem.}
\end{quote}

\begin{quote}
\textbf{The speed of light is not a numerical value, but a structural condition prohibiting comparison between different degrees.}
\end{quote}

\begin{quote}
\textbf{The concepts of photon and mass are not definitions, but theorems derived from $D_6$ structure.}
\end{quote}

\begin{quote}
\textbf{The mass gap is the minimum energy of states that experience proper time.}
\end{quote}

Whether time flows is uniquely determined from $D_6$ structure. How fast it flows is gauge-invariantly defined only within the same degree, and is causally incomparable between different degrees.

\section{Significance}
\label{sec:mat_final_significance}

This result provides the first non-circular answer to the fundamental question of why nature appears to follow variational principles. The variational principle is not a principle, but a theorem \textbf{forced to be derived} from $D_6$ structure and $\varphi$-scale invariance.

