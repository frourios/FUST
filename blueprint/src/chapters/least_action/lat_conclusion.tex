% FUST Least Action Theorem - Conclusion
\chapter{Conclusion: Unified Derivation from \texorpdfstring{$D_\zeta$}{Dζ}}
\label{chap:mat_conclusion}

\section{Complete Derivation Chain}
\label{sec:mat_derivation_chain}

The preceding chapters establish a single derivation chain with no external assumptions:

\[
\boxed{D_\zeta} \;\xrightarrow{\text{Ch.\ref{chap:mat_poincare}}}\; I_4 = \mathrm{Fin}\,3 \oplus \mathrm{Fin}\,1 \;\to\; \eta \;\to\; \mathfrak{so}(3,1) \;\to\; \mathfrak{iso}(3,1)
\]
\[
\xrightarrow{\text{Ch.\ref{chap:mat_time_space}}}\; \text{Time, Space, Arrow of Time} \;\xrightarrow{\text{Ch.\ref{chap:mat_mass_energy}}}\; \text{Mass, Energy, Lagrangian}
\]
\[
\xrightarrow{\text{Ch.\ref{chap:mat_action_functional}}}\; \text{Action Principle} \;\xrightarrow{\text{Ch.\ref{chap:mat_physical_laws}}}\; \text{Thermodynamics, Gravity, Gauge Groups}
\]

Every link in this chain is a Lean4 theorem, not a physical assumption.

\section{Gauge-Invariant Quantities}
\label{sec:mat_gauge_invariant}

\begin{center}
\begin{tabular}{|c|c|c|}
\hline
Quantity & Lean4 theorem & Description \\
\hline
$f \in \ker(F_\zeta)$ & kernel classification & Massless (light-like) state \\
$f \notin \ker(F_\zeta)$ & Fζ\_nonzero\_implies\_time & Massive state with proper time \\
$L[f] = 0$ & Fζ\_lagrangian\_zero\_iff & Membership in $\ker(F_\zeta)$ \\
$m^2 = 14$ & casimirMassSq\_one & Poincar\'{e} Casimir invariant \\
$\Delta = 12/25$ & massScale\_eq & $F_\zeta$ mass scale \\
$\mathrm{SU}(3) \times \mathrm{SU}(2) \times \mathrm{U}(1)$ & standard\_gauge\_group\_unique & Gauge group from $\mathbb{Z}/6\mathbb{Z}$ channels \\
Time delay ratio (same degree) & same\_degree\_ratio\_gauge\_invariant & Causally comparable quantity \\
\hline
\end{tabular}
\end{center}

\section{Gauge-Dependent Quantities}
\label{sec:mat_gauge_dependent}

\begin{center}
\begin{tabular}{|c|c|}
\hline
Quantity & Dependent parameter \\
\hline
Value of $\mathrm{Diff}_6 f(x)$ & Gauge $x$ \\
Time delay ratio between different degrees & Gauge $x$ \\
Concrete value of perpProjection & Choice of interpolation points \\
\hline
\end{tabular}
\end{center}

\section{Derived Concepts (Theorems, Not Definitions)}
\label{sec:mat_derived}

\begin{center}
\begin{tabular}{|c|c|c|}
\hline
Concept & Derivation & Source \\
\hline
Poincar\'{e} group $\mathrm{ISO}(3,1)$ & $D_\zeta$ 3+1 decomposition & Ch.\ref{chap:mat_poincare} \\
Time & $\Phi_A$ temporal channel, $\varphi$-scaling & Ch.\ref{chap:mat_time_space} \\
Space & $\Phi_S$ spatial channels, $\ker(F_\zeta)$ & Ch.\ref{chap:mat_time_space} \\
Arrow of time & $\varphi > 1 > |\psi|$ asymmetry & Ch.\ref{chap:mat_time_space} \\
Mass & Casimir invariant $m^2 = -\eta_{\mu\nu} p^\mu p^\nu$ & Ch.\ref{chap:mat_mass_energy} \\
Mass gap & $m^2(1) = 14 > 0$, $\Delta = 12/25 > 0$ & Ch.\ref{chap:mat_mass_energy} \\
Action principle & $F_\zeta$ linearity + $\varphi$-invariant measure & Ch.\ref{chap:mat_action_functional} \\
Thermodynamics & $S = |F_\zeta f|^2$, $\varphi$-equivariance & Ch.\ref{chap:mat_physical_laws} \\
General relativity & Localized $\mathfrak{so}(3,1)$ on $I_4$ & Ch.\ref{chap:mat_physical_laws} \\
Gauge group & $\mathbb{Z}/6\mathbb{Z}$ channel decomposition & Ch.\ref{chap:mat_physical_laws} \\
Yang-Mills mass gap & $\ker(F_\zeta)$ not a subalgebra & Ch.\ref{chap:mat_physical_laws} \\
\hline
\end{tabular}
\end{center}

\section{Core Principles}
\label{sec:mat_core_principles}

\begin{quote}
\textbf{The Poincar\'{e} group $\mathrm{ISO}(3,1)$ is not assumed but derived from $D_\zeta$ structure.}
\end{quote}

\begin{quote}
\textbf{Time, space, and mass are theorems, not parameters.}
\end{quote}

\begin{quote}
\textbf{Whether time flows is uniquely determined. How fast it flows is gauge-invariant only within the same degree.}
\end{quote}

\begin{quote}
\textbf{The variational principle is a theorem forced by $F_\zeta$ linearity and $\varphi$-scale invariance.}
\end{quote}

\begin{quote}
\textbf{All physical laws---thermodynamics, gravity, gauge interactions---follow from localizing a single algebraic structure.}
\end{quote}

\section{Significance}
\label{sec:mat_final_significance}

This unified derivation provides the first non-circular answer to three fundamental questions:

\begin{enumerate}
\item \textbf{Why $(3,1)$ signature?} Because $D_\zeta$ has exactly 3 spatial sub-operators ($\Phi_S$) and 1 temporal channel ($\Phi_A$), with weight ratio $3:1$.
\item \textbf{Why the Standard Model gauge group?} Because $F_\zeta$'s $\mathbb{Z}/6\mathbb{Z}$ channel structure uniquely determines $\mathrm{SU}(3) \times \mathrm{SU}(2) \times \mathrm{U}(1)$.
\item \textbf{Why a mass gap?} Because the Poincar\'{e} Casimir invariant of $D_\zeta$ 4-momentum gives $m^2 = 14 > 0$ at the minimum active mode.
\end{enumerate}

