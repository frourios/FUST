% FUST Least Action Theorem - Philosophy of Science
\chapter{Transcendental Conditions of Science}
\label{chap:mat_philosophy}

This chapter demonstrates that FUST transcends traditional philosophy of science (particularly Popper's falsifiability) and provides the \textbf{conditions for science itself to be possible}.

\section{Limits of Popperian Falsifiability}
\label{sec:mat_popper}

\subsection{Traditional View of Science}

Karl Popper's definition of science:

\begin{quote}
``A scientific theory is a system of propositions that is in principle falsifiable.''
\end{quote}

Examples:
\begin{itemize}
\item ``All swans are white'' $\to$ Falsifiable by discovering a black swan
\item ``The speed of light is constant'' $\to$ Falsifiable by observing something faster than light
\end{itemize}

\subsection{Inapplicability to FUST}

FUST is a \textbf{theorem derived} from the ZF + DC axiom system:
\[
\text{ZF + DC} \vdash \text{``The gauge group is } \mathrm{SU}(3) \times \mathrm{SU}(2) \times \mathrm{U}(1)\text{''}
\]

This is not a ``hypothesis'' but a ``proven proposition,'' and counterexamples are \textbf{logically impossible}.

\section{Theorem: Transcendental Condition of Science}
\label{sec:mat_transcendental}

\begin{theorem}[Transcendental Condition of Science]
\label{thm:mat_transcendental}
FUST is not a falsifiable scientific theory, but provides a \textbf{transcendental condition} for science to be possible.
\end{theorem}

\begin{proof}
Assume an observation contradicting FUST exists.

\begin{enumerate}
\item FUST is a theorem derived from ZF + DC
\item ``Contradicting FUST'' means ``contradicting a theorem of ZF + DC''
\item Phenomena contradicting ZF + DC theorems are \textbf{mathematically indescribable}
\item Mathematically indescribable phenomena \textbf{lack reproducibility} (others cannot reproduce via mathematics)
\item Phenomena lacking reproducibility are \textbf{not objects of science}
\end{enumerate}

Therefore, observations contradicting FUST lie outside the framework of science.
\end{proof}

\begin{corollary}[Boundary Between Science and Occult]
\label{cor:mat_occult}
If a phenomenon contradicting FUST is observed, it should be called \textbf{occult}, not science.
\end{corollary}

\begin{center}
\begin{tabular}{|c|c|c|c|}
\hline
Type of phenomenon & Mathematical description & Reproducibility & Classification \\
\hline
Consistent with FUST & Possible & Yes & Science \\
Contradicts FUST & Impossible & No & Occult \\
\hline
\end{tabular}
\end{center}

\textbf{Note}: This is not ``arrogance'' of FUST, but a logical consequence concerning \textbf{the definition of science itself}.

\section{Redefinition of Science}
\label{sec:mat_redefine_science}

\subsection{Traditional Definition}
\[
\text{Science} = \text{Falsifiable theories}
\]

\subsection{Definition by FUST}
\[
\text{Science} = \text{Description of mathematically reproducible phenomena}
\]

FUST provides the \textbf{necessary conditions} for this ``mathematical reproducibility'':
\begin{itemize}
\item Structure of gauge groups
\item Existence of mass gap
\item Uniqueness of physical constants
\end{itemize}

\section{Philosophical Position}
\label{sec:mat_philosophical}

\subsection{Comparison with Kant}

Immanuel Kant's question: ``Are synthetic a priori judgments possible?''

\begin{center}
\begin{tabular}{|c|c|}
\hline
Kant & FUST \\
\hline
Spacetime is an a priori form of intuition & Spacetime structure derived from $F_\zeta$ \\
Causality is an a priori category & Physical laws derived from ZF + DC \\
``Thing-in-itself'' is unknowable & ``World outside ZF + DC'' is outside science \\
\hline
\end{tabular}
\end{center}

FUST \textbf{proves} that ``physical laws are derived a priori (from ZF + DC).''

\subsection{Response to Popper}

\begin{center}
\begin{tabular}{|c|c|}
\hline
Popper & FUST \\
\hline
Non-falsifiable is not science & Non-falsifiable \textbf{makes science possible} \\
Science is a system of hypotheses & Science is a system of theorems (for physical laws) \\
Falsification drives scientific progress & Proof establishes scientific foundation \\
\hline
\end{tabular}
\end{center}

\section{Uniqueness of Universe Structure}
\label{sec:mat_uniqueness}

\begin{theorem}[Uniqueness of Universe Structure]
\label{thm:mat_universe_unique}
The mathematical structure corresponding to the observable universe is \textbf{unique} as the minimal element of $F_\zeta$.
\end{theorem}

\begin{theorem}[Indefinability of Multiverse]
\label{thm:mat_multiverse}
The concept of ``multiple universes'' is \textbf{undefinable} under ZF + DC + FUST.
\end{theorem}

\begin{proof}
\begin{enumerate}
\item In FUST, physical laws are uniquely derived from ZF + DC
\item ``A universe with different physical laws'' contradicts ZF + DC theorems
\item What contradicts ZF + DC theorems is undefinable within ZF + DC
\item Therefore ``another universe'' is undefinable
\end{enumerate}
\end{proof}

\section{Significance}
\label{sec:mat_significance}

FUST makes the following contributions to philosophy of science:

\begin{enumerate}
\item \textbf{Transcendence of falsifiability}: The foundation of science is not falsifiability but mathematical constructibility
\item \textbf{Clear boundary between science and occult}: FUST consistency is a necessary condition for science
\item \textbf{Elimination of multiverse}: ``Other possible universes'' are mathematically meaningless
\item \textbf{A priori nature of physical laws}: Physics is not empirical discovery but mathematical derivation
\end{enumerate}

\section{Qualia and Consciousness}
\label{sec:mat_qualia}

\subsection{Irreversibility and Manifestation}

Free thoughts and internal states without causality belong to $\ker(F_\zeta)$. However, to manifest as physical phenomena, projection onto $\ker(F_\zeta)^\perp$ is necessary.

\subsection{Least Action Path and Qualia}

The Least Action Theorem determines the uniquely chosen path within that projection. The very fact of tracing this path from the inside is ``qualia of experiencing the present.''

Therefore, qualia are intrinsic phenomena that necessarily arise as a result of the Least Action Theorem, requiring no additional psychological assumptions.

