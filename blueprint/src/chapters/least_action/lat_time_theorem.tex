% FUST Least Action Theorem - Time Existence Theorem
\chapter{Time Existence Theorem}
\label{chap:mat_time_theorem}

\section{Time Existence Theorem}
\label{sec:mat_time_existence}

\begin{theorem}[Time Existence Theorem]
\label{thm:mat_time_existence}
\lean{FUST.time_existence_theorem}
\leanok
For a state function $f : \mathbb{R} \to \mathbb{R}$, the following are equivalent:
\begin{enumerate}
\item $f \in \ker(D_6)$ (i.e., $f$ is a polynomial of degree 2 or less)
\item $\forall x \neq 0, D_6 f(x) = 0$
\item Time does not exist
\end{enumerate}
\end{theorem}

\textbf{Contrapositive}: $f \notin \ker(D_6) \iff \exists x \neq 0, D_6 f(x) \neq 0 \iff$ Time exists

\begin{proof}
$(1) \Rightarrow (2)$: Proven by \texttt{D6\_polynomial\_deg2}. When $f = a_0 + a_1 x + a_2 x^2$, by linearity of $D_6$ and annihilation of each basis element, $D_6 f = 0$.

$(2) \Rightarrow (1)$: Proven in the second part of \texttt{causal\_boundary\_theorem}. If perpProjection is identically zero, then $f \in \ker(D_6)$.
\end{proof}

\begin{corollary}[Relation Between Action Theorem and Time]
\label{cor:mat_action_time}
\begin{center}
\begin{tabular}{|c|c|c|}
\hline
Condition & Time & Action \\
\hline
$f \in \ker(D_6)$ & Does not exist & $\mathcal{A}[f] = 0$ \\
$f \notin \ker(D_6)$ & Exists & $\mathcal{A}[f] > 0$ \\
\hline
\end{tabular}
\end{center}
\end{corollary}

\textbf{Interpretation}: The action being positive is equivalent to proper time existing.

\begin{corollary}[Gauge Invariance of Time Existence]
\label{cor:mat_time_gauge}
\lean{FUST.time_differential_zero_gauge_invariant}
\leanok
The time existence condition ``$f \in \ker(D_6)$'' does not depend on:
\begin{itemize}
\item Choice of gauge parameter $x$
\item Choice of interpolation points
\end{itemize}
\end{corollary}

This is because $\ker(D_6) = \mathrm{span}\{1, x, x^2\}$ is a fixed algebraic structure.

\section{Interpolation Uniqueness Theorem}
\label{sec:mat_interpolation}

\subsection{Problem Setting}

To define the projection onto $\ker(D_6)$, function values at 3 points must be evaluated. This ``choice of 3 points'' appears arbitrary.

\subsection{Uniqueness Theorem}

\begin{theorem}[Interpolation Uniqueness]
\label{thm:mat_interpolation_unique}
\lean{FUST.kernel_interpolation_unique}
\leanok
If two polynomials $p, q$ in $\ker(D_6)$ agree at any 3 distinct points $t_0, t_1, t_2$, then $p = q$.
\end{theorem}

\begin{proof}[Proof outline]
\begin{enumerate}
\item The difference $p - q$ also belongs to $\ker(D_6)$ (polynomial of degree 2 or less)
\item Agreement at 3 points yields a linear system for the coefficient differences
\item Since the points are distinct, all coefficient differences are zero
\end{enumerate}
\end{proof}

\subsection{Significance}

This theorem guarantees that the choice of interpolation points $\{0, 1, -1\}$ is not arbitrary, and \textbf{the same result is obtained for any 3 points}. The choice $\{0, 1, -1\}$ is a natural selection for symmetry and computational convenience.

