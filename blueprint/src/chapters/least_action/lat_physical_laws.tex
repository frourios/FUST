% FUST Least Action Theorem - Physical Laws from Poincaré
\chapter{Physical Laws from Poincar\'{e} Structure}
\label{chap:mat_physical_laws}

All physical laws are derived from the Poincar\'{e} structure established in Chapter~\ref{chap:mat_poincare}: the Minkowski metric $\eta$, the Lorentz algebra $\mathfrak{so}(3,1)$, and the temporal channel $\Phi_A$.

\section{Thermodynamics from \texorpdfstring{$F_\zeta$}{Fζ}}
\label{sec:mat_thermodynamics}

Entropy is defined from the Poincar\'{e} temporal direction:

\begin{definition}[Entropy]
\label{def:mat_entropy}
\lean{FUST.TimeStructure.entropyAtFζ}
\leanok
\[
S(f)(z) := |F_\zeta f(z)|^2
\]
Entropy measures departure from $\ker(F_\zeta)$ along the Poincar\'{e} temporal channel.
\end{definition}

\subsection{Zeroth Law}

\begin{theorem}[Kernel Equilibrium]
\label{thm:mat_zeroth_law}
\lean{FUST.Thermodynamics.zeroth_law_kernel_equilibrium}
\leanok
If $f, g \in \ker(F_\zeta)$, then $S(f) = S(g) = 0$.
\end{theorem}

Systems in $\ker(F_\zeta)$ share identical entropy (zero), providing the equivalence relation for thermal equilibrium.

\subsection{First Law}

\begin{theorem}[Energy Conservation ($\varphi$-Equivariance)]
\label{thm:mat_first_law}
\lean{FUST.Thermodynamics.first_law_equivariance}
\leanok
\[
S(\mathrm{timeEvolution}(f))(z) = S(f)(\varphi z)
\]
Entropy is $\varphi$-equivariant: time evolution redistributes but conserves the total energy (Noether's theorem for $\varphi$-scaling symmetry).
\end{theorem}

\subsection{Second Law}

\begin{theorem}[Entropy Shift]
\label{thm:mat_second_law}
\lean{FUST.Thermodynamics.second_law_entropy_shift}
\leanok
\[
S(\mathrm{timeEvolution}(f))(z) = S(f)(\varphi z)
\]
Since $\varphi > 1$, time evolution probes the Lagrangian at a larger scale, amplifying $\ker(F_\zeta)^\perp$ components.
\end{theorem}

\subsection{Third Law}

\begin{theorem}[Absolute Zero Unreachable]
\label{thm:mat_third_law}
\lean{FUST.TimeStructure.third_law_Fζ}
\leanok
If $f \notin \ker(F_\zeta)$, then $\exists z: S(f)(z) > 0$.
\end{theorem}

\begin{theorem}[$S \equiv 0 \iff \ker(F_\zeta)$]
\label{thm:mat_entropy_zero}
\lean{FUST.TimeStructure.entropy_zero_iff_kerFζ}
\leanok
$(\forall z, S(f)(z) = 0) \iff f \in \ker(F_\zeta)$.
\end{theorem}

\section{Einstein Equations from Localized \texorpdfstring{$\mathfrak{so}(3,1)$}{so(3,1)}}
\label{sec:mat_einstein}

Localizing the global Lorentz symmetry $\mathfrak{so}(3,1)$ introduces a connection field, yielding general relativity.

\subsection{Lorentz Connection}

\begin{theorem}[Connection Space Dimension]
\label{thm:mat_connection_dim}
\lean{FUST.Physics.Gravity.finrank_connection}
\leanok
\[
\dim(\omega : I_4 \to \mathfrak{so}(3,1)) = |I_4| \times \dim \mathfrak{so}(3,1) = 4 \times 6 = 24
\]
\end{theorem}

\subsection{Curvature}

\begin{definition}[Curvature]
\label{def:mat_curvature}
\lean{FUST.Physics.Gravity.curvature}
\leanok
\[
F_{\mu\nu} = [\omega_\mu, \omega_\nu] \in \mathfrak{so}(3,1)
\]
\end{definition}

\begin{theorem}[Curvature Antisymmetry]
\label{thm:mat_curvature_antisymm}
\lean{FUST.Physics.Gravity.curvature_antisymm}
\leanok
$F_{\mu\nu} = -F_{\nu\mu}$.
\end{theorem}

\begin{theorem}[Curvature Preserves Minkowski Form]
\label{thm:mat_curvature_minkowski}
\lean{FUST.Physics.Gravity.curvature_preserves_minkowski}
\leanok
$B(F_{\mu\nu} v, w) + B(v, F_{\mu\nu} w) = 0$.
\end{theorem}

\subsection{Bianchi Identity}

\begin{theorem}[Algebraic Bianchi Identity]
\label{thm:mat_bianchi}
\lean{FUST.Physics.Gravity.bianchi_identity}
\leanok
\[
[\omega_\mu, [\omega_\nu, \omega_\rho]] + [\omega_\nu, [\omega_\rho, \omega_\mu]] + [\omega_\rho, [\omega_\mu, \omega_\nu]] = 0
\]
This is the Jacobi identity of $\mathfrak{so}(3,1)$.
\end{theorem}

\subsection{Einstein Tensor}

\begin{definition}[Einstein Tensor]
\label{def:mat_einstein}
\lean{FUST.Physics.Gravity.einstein}
\leanok
\[
G_{\mu\nu} = R_{\mu\nu} - \frac{1}{2} \eta_{\mu\nu} R
\]
where $R_{\mu\nu}$ is the Ricci tensor and $R$ is the scalar curvature.
\end{definition}

\begin{theorem}[Vacuum Einstein Equations]
\label{thm:mat_vacuum_einstein}
\lean{FUST.Physics.Gravity.vacuum_einstein}
\leanok
For flat connection ($\omega = 0$): $G_{\mu\nu} = 0$.
\end{theorem}

\begin{theorem}[Einstein Trace]
\label{thm:mat_einstein_trace}
\lean{FUST.Physics.Gravity.einstein_trace}
\leanok
$\eta^{\mu\nu} G_{\mu\nu} = -R$.
\end{theorem}

\section{Gauge Groups from \texorpdfstring{$F_\zeta$}{Fζ} Factorization}
\label{sec:mat_gauge_groups}

The gauge group arises from the factorization freedom of the input state function. The derivation defect $\delta(f,g) = F_\zeta(fg) - f F_\zeta(g) - F_\zeta(f)g$ is invariant under $(f,g) \mapsto (cf, c^{-1}g)$, giving the fundamental gauge invariance.

\begin{theorem}[Gauge Channel Dimensions]
\label{thm:mat_gauge_dimensions}
\lean{FUST.gauge_channel_dimensions}
\leanok
The gauge group representation dimensions $(3, 2, 1)$ are determined by:
\begin{enumerate}
\item Mode vectors $v(s) \in \mathbb{R}^3$ with $\Phi_S$ rank 3 $\Rightarrow$ gauge saturates at $\mathrm{SU}(3)$
\item $\tau(\mathrm{AF\_coeff}) = -\mathrm{AF\_coeff}$ (quaternionic type) $\Rightarrow$ $\mathrm{SU}(2)$
\item Trivial channel ($\dim = 1$) $\Rightarrow$ $\mathrm{U}(1)$
\end{enumerate}
\end{theorem}

\begin{theorem}[$\zeta_6$ Separation]
\label{thm:mat_zeta6_separation}
\lean{FUST.zeta6_ne_phi_pow}
\leanok
$\zeta_6 \neq \varphi^n$ for all $n$, ensuring channel independence.
\end{theorem}

\section{Yang-Mills Mass Gap}
\label{sec:mat_yang_mills}

\begin{theorem}[$\ker(F_\zeta)$ is Not a Subalgebra]
\label{thm:mat_ker_not_subalgebra}
\lean{FUST.YangMills.ker_Fζ_not_subalgebra}
\leanok
$z^3 \in \ker(F_\zeta)$ and $z^4 \in \ker(F_\zeta)$, but $z^7 = z^3 \cdot z^4 \notin \ker(F_\zeta)$ since $7 \equiv 1 \pmod{6}$. This is algebraic confinement.
\end{theorem}

\begin{theorem}[Yang-Mills Mass Gap]
\label{thm:mat_yang_mills_gap}
\lean{FUST.YangMills.yangMills_massGap}
\leanok
The mass gap $\Delta = 12/25 > 0$ with the first active eigenvalue at degree $n = 3$: $F_\zeta[z^3] \neq 0$.
\end{theorem}

\section{Causal Structure}
\label{sec:mat_causality}

\begin{theorem}[Causal Boundary]
\label{thm:mat_causal_boundary}
\lean{FUST.causal_boundary_theorem}
\leanok
$\ker(F_\zeta)$ is the light cone: the boundary between states with proper time (massive, $f \notin \ker(F_\zeta)$) and states without (massless, $f \in \ker(F_\zeta)$).
\end{theorem}

\section{Summary: All Physics from One Structure}
\label{sec:mat_laws_summary}

\begin{center}
\begin{tabular}{|c|c|c|}
\hline
Physical law & $D_\zeta / F_\zeta$ origin & Key Lean theorem \\
\hline
Thermodynamics & $S = |F_\zeta f|^2$, $\varphi$-equivariance & \texttt{third\_law\_Fζ} \\
General relativity & Localized $\mathfrak{so}(3,1)$ on $I_4$ & \texttt{vacuum\_einstein} \\
Gauge group & $\mathbb{Z}/6\mathbb{Z}$ channel decomposition & \texttt{standard\_gauge\_group\_unique} \\
Yang-Mills gap & $\ker(F_\zeta)$ not subalgebra & \texttt{yangMills\_massGap} \\
Causal structure & $\ker(F_\zeta) =$ light cone & \texttt{causal\_boundary\_theorem} \\
\hline
\end{tabular}
\end{center}
