% FUST Least Action Theorem - Action Functional
\chapter{Golden Ratio Scale Invariant Measure and Action Functional}
\label{chap:mat_action_functional}

\section{Derivation of the Scale Invariant Measure}
\label{sec:mat_measure}

\subsection{Invariance Condition}

The measure $\mu$ required for the action functional must be invariant under $\varphi$-scale transformations:
\[
\mu(\varphi A) = \mu(A) \quad \text{for all measurable } A
\]

\subsection{Solution via Logarithmic Coordinates}

In logarithmic coordinates $y = \log(x)$, the $\varphi$-scale transformation $x \to \varphi x$ becomes a translation $y \to y + \log(\varphi)$.

Since Lebesgue measure $dy$ is invariant under translation, in the original coordinates:
\[
d\mu = \frac{dx}{x}
\]

This is the \textbf{Haar measure} on the multiplicative group $(0, \infty)$, and is the unique (up to constant multiple) $\varphi$-scale invariant measure.

\section{Concrete Form of the FUST Action Functional}
\label{sec:mat_action}

\subsection{Definition of the Action Functional}

Define the state space and action as follows:

\begin{definition}[State Space]
\[
\mathcal{S} = \{f : \mathbb{C} \to \mathbb{C} \mid \mathcal{A}[f] < \infty\}
\]
\end{definition}

\begin{definition}[FUST Action Functional]
\label{def:mat_action}
\lean{FUST.action_functional}
\leanok
\[
\boxed{\mathcal{A}[f] = \int_0^\infty |N_6 f(x)|^2 \frac{dx}{x}}
\]
\end{definition}

\subsection{Why L2 Norm}

\begin{enumerate}
\item \textbf{Inner product structure}: $L^2$ provides natural inner product and norm, enabling variational calculus
\item \textbf{Euler-Lagrange equation}: The variation $\delta\mathcal{A}/\delta f = 0$ is well-defined
\item \textbf{$\varphi$-covariance}: $\mathcal{A}[f \circ S_\varphi] = \varphi^2 \cdot \mathcal{A}[f] \circ S_\varphi$ where $S_\varphi(x) = \varphi x$
\item \textbf{Consistency with kernel}: $\mathcal{A}[f] = 0 \iff f \in \ker(F_\zeta)$
\end{enumerate}

\subsection{Properties of the Action Functional (Numerically Verified)}

\begin{center}
\begin{tabular}{|c|c|c|}
\hline
State & Action value & Note \\
\hline
$f = 1 + x + x^2$ & $\approx 0$ & In $\ker(F_\zeta)$ \\
$f = x^3$ & $\approx 576$ & Outside $\ker(F_\zeta)$ (minimal mass state) \\
$f = 1 + x^3$ & $\approx 576$ & Mixed state (kernel component does not contribute) \\
\hline
\end{tabular}
\end{center}

\section{Variational Structure (Lean4 Proven)}
\label{sec:mat_variational}

The Lagrangian $L[f](z) = |F_\zeta f(z)|^2$ admits a rigorous variational calculus because $F_\zeta$ is linear.

\subsection{Linearity of \texorpdfstring{$F_\zeta$}{Fζ}}

\begin{theorem}[$F_\zeta$ Linearity]
\label{thm:mat_Fzeta_linear}
\lean{FUST.LeastAction.Fζ_linear_combination}
\leanok
\[
F_\zeta(af + bg)(z) = a \cdot F_\zeta f(z) + b \cdot F_\zeta g(z)
\]
\end{theorem}

\subsection{First Variation}

\begin{theorem}[Lagrangian Perturbation]
\label{thm:mat_lagrangian_perturbation}
\lean{FUST.LeastAction.lagrangian_perturbation}
\leanok
For perturbation $f \mapsto f + \varepsilon\eta$:
\[
L[f + \varepsilon\eta](z) = |F_\zeta f(z)|^2 + 2\,\mathrm{Re}(\varepsilon \cdot F_\zeta\eta(z) \cdot \overline{F_\zeta f(z)}) + |\varepsilon|^2 |F_\zeta\eta(z)|^2
\]
The first-order term is $2\,\mathrm{Re}(\varepsilon \cdot F_\zeta\eta \cdot \overline{F_\zeta f})$.
\end{theorem}

\subsection{Critical Point Condition}

\begin{theorem}[Critical Point of Witness]
\label{thm:mat_critical_point}
\lean{FUST.LeastAction.critical_point_of_witness}
\leanok
If a nontrivial test variation exists ($\exists \eta: F_\zeta\eta(z) \neq 0$), then:
\[
(\forall \eta,\; F_\zeta f(z) \cdot F_\zeta\eta(z) = 0) \;\Longrightarrow\; F_\zeta f(z) = 0
\]
\end{theorem}

The concrete witness $\eta(t) = t$ satisfies $F_\zeta(\mathrm{id})(z) \neq 0$ for $z \neq 0$
(Lean: \texttt{FUST.WaveEquation.Fζ\_detects\_active}), completing the chain:
\[
\delta L = 0 \;\Longrightarrow\; F_\zeta f = 0 \;\Longleftrightarrow\; f \in \ker(F_\zeta)
\]

\section{Euler-Lagrange Equation}
\label{sec:mat_euler_lagrange}

\subsection{Integral Form}

For the action functional $\mathcal{A}[f] = \int |N_6 f|^2 d\mu$, the integral variation gives:
\[
\delta\mathcal{A} = 2 \int N_6 f \cdot N_6(\delta f) \, d\mu = 2 \int \delta f \cdot N_6^\dagger(N_6 f) \, d\mu
\]

\begin{theorem}[Euler-Lagrange Equation]
\label{thm:mat_euler_lagrange}
\[
N_6^\dagger(N_6 f) = 0
\]
\end{theorem}

Since $N_6^\dagger N_6$ is self-adjoint:
\[
\ker(N_6^\dagger N_6) = \ker(F_\zeta)
\]

\begin{theorem}[Least Action Solutions]
\label{thm:mat_minimal_action}
\lean{FUST.minimal_action_ker}
\leanok
Minimal action solutions belong to $\ker(F_\zeta)$.
\end{theorem}

\subsection{Physical Interpretation}

\begin{center}
\begin{tabular}{|c|c|c|}
\hline
State & Condition & Physics \\
\hline
$\ker(F_\zeta)$ & $\mathcal{A}[f] = 0$ & Light cone (worldline of photon) \\
$\ker(F_\zeta)^\perp$ & $\mathcal{A}[f] > 0$ & Massive state (has proper time) \\
\hline
\end{tabular}
\end{center}

