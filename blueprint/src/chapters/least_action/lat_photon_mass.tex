% FUST Least Action Theorem - Photon and Mass States
\chapter{Derivation of Photon and Mass States}
\label{chap:mat_photon_mass}

\section{Separation of Constant and Non-constant Functions}
\label{sec:mat_constant}

\begin{definition}[IsConstant]
$f$ is constant $\iff \exists c, \forall t, f(t) = c$
\end{definition}

\begin{definition}[HasEnergy]
$f$ is non-constant $\iff \neg\mathrm{IsConstant}\, f$
\end{definition}

\begin{theorem}[Constant Functions in Kernel]
\label{thm:mat_constant_kernel}
\lean{FUST.constant_in_kernel}
\leanok
Constant functions belong to $\ker(F_\zeta)$.
\end{theorem}

\begin{theorem}[Kernel Decomposition]
\label{thm:mat_kernel_decomp}
\lean{FUST.kernel_decomposition}
\leanok
$\ker(F_\zeta) = \text{Constant functions} \lor \text{Non-constant functions}$
\end{theorem}

\section{Derivation of Photon States}
\label{sec:mat_photon}

\textbf{The definition of photon states is not arbitrary, but derived from the following physical correspondence:}

\begin{center}
\begin{tabular}{|c|c|}
\hline
Physical property & Correspondence in $F_\zeta$ structure \\
\hline
Mass $m = 0$ & No components of degree 3 or higher (belongs to $\ker(F_\zeta)$) \\
Energy $E \neq 0$ & Non-constant ($a_1 \neq 0$ or $a_2 \neq 0$) \\
Proper time $d\tau = 0$ & $\mathrm{Diff}_6 f = 0$ \\
\hline
\end{tabular}
\end{center}

\begin{definition}[Photon State]
\label{def:mat_photon}
\lean{FUST.IsPhotonState}
\leanok
$\mathrm{IsPhotonState}\, f \iff \mathrm{IsInKerF\zeta}\, f \land \mathrm{HasEnergy}\, f$
\end{definition}

\begin{theorem}[Photon State Characterization]
\label{thm:mat_photon_char}
\lean{FUST.photon_state_iff_points}
\leanok
\[
\mathrm{IsPhotonState}\, f \iff f \in \ker(F_\zeta) \land (f(1) \neq f(0) \lor f(-1) \neq f(0))
\]
\end{theorem}

\begin{theorem}[Non-constant Kernel Has Energy]
\label{thm:mat_nonconstant_energy}
\lean{FUST.nonconstant_ker_has_energy}
\leanok
Photon states have non-trivial linear or quadratic components.
\end{theorem}

\section{Derivation of Mass States}
\label{sec:mat_mass}

\textbf{The definition of mass states is not arbitrary, but derived as deviation from $\ker(F_\zeta)$:}

\begin{definition}[HasHigherDegree]
$\mathrm{HasHigherDegree}\, f \iff f$ cannot be written as a polynomial of degree 2 or less
\end{definition}

\begin{definition}[Massive State]
\label{def:mat_massive}
\lean{FUST.IsMassiveState}
\leanok
$\mathrm{IsMassiveState}\, f \iff \mathrm{HasHigherDegree}\, f$
\end{definition}

\begin{theorem}[Massive State Equivalence]
\label{thm:mat_massive_time}
\lean{FUST.massive_iff_time_exists}
\leanok
\[
\mathrm{IsMassiveState}\, f \iff \mathrm{TimeExists}\, f
\]
\end{theorem}

\begin{theorem}[Massive State via Perpendicular Projection]
\label{thm:mat_massive_perp}
\lean{FUST.massive_iff_nonzero_perp}
\leanok
\[
\mathrm{IsMassiveState}\, f \iff \exists t, \mathrm{perpProjection}\, f(t) \neq 0
\]
\end{theorem}

\section{Physical Interpretation}
\label{sec:mat_physical_interp}

\begin{center}
\begin{tabular}{|c|c|c|}
\hline
State & $F_\zeta$ structure & Physical meaning \\
\hline
Vacuum & Constant function $\in \ker(F_\zeta)$ & No energy, no mass \\
Photon & Non-constant $\in \ker(F_\zeta)$ & Has energy, no mass, no proper time \\
Matter & $f \notin \ker(F_\zeta)$ & Has energy, has mass, has proper time \\
\hline
\end{tabular}
\end{center}

\section{Homogeneous Decomposition and Time Delay Ratio}
\label{sec:mat_homogeneous}

\subsection{Homogeneous Decomposition Theorem}

Any analytic state function $f$ decomposes into homogeneous components:
\[
f(t) = \sum_{n=0}^{\infty} a_n t^n
\]

By linearity of $\mathrm{Diff}_6$ and the scaling law:
\[
\mathrm{Diff}_6[f](x) = \sum_{n \geq 3} a_n \mathrm{Diff}_6[t^n](x)
\]

\subsection{Gauge Invariance of Time Delay Ratio}

\begin{theorem}[Same Degree Ratio Gauge Invariance]
\label{thm:mat_same_degree}
\lean{FUST.same_degree_ratio_gauge_invariant}
\leanok
For $f_A = a \cdot f$, $f_B = b \cdot f$ (scalar multiples of the same function),
\[
\frac{|\mathrm{Diff}_6 f_A(x)|}{|\mathrm{Diff}_6 f_B(x)|} = \frac{|a|}{|b|}
\]
This is independent of $x$ $\Rightarrow$ \textbf{gauge-invariant}
\end{theorem}

\begin{theorem}[Different Degrees]
\label{thm:mat_diff_degree}
For $f_A = t^n$, $f_B = t^m$ ($n \neq m$),
\[
\frac{|\mathrm{Diff}_6 f_A(x)|}{|\mathrm{Diff}_6 f_B(x)|} \propto x^{n-m}
\]
This depends on $x$ $\Rightarrow$ \textbf{gauge-dependent}
\end{theorem}

\subsection{Physical Interpretation}

``The time delay ratio cannot be defined between different degrees'' is not a defect.

\begin{center}
\begin{tabular}{|c|c|}
\hline
Relativity & FUST \\
\hline
spacelike (causally incomparable) & Different degrees (incomparable) \\
timelike (causally comparable) & Same degree (comparable) \\
\hline
\end{tabular}
\end{center}

\textbf{Conclusion}: The incomparability between different degrees reflects causal structure.

