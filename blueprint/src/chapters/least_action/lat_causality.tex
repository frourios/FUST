% FUST Least Action Theorem - Causality and Light Cone
\chapter{Causal Boundary and Speed of Light}
\label{chap:mat_causality}

\section{D6 as Causal Boundary (Light Cone) Identification}
\label{sec:mat_light_cone}

\subsection{Light Cone in Relativity}

In relativity, the speed of light $c$ is not merely a numerical value but \textbf{the boundary of causal structure}:
\[
ds^2 = 0
\]

\begin{itemize}
\item $ds^2 > 0$ (timelike): Causally connectable, proper time $d\tau > 0$
\item $ds^2 = 0$ (null): On the light cone, proper time $d\tau = 0$
\item $ds^2 < 0$ (spacelike): Causally disconnectable
\end{itemize}

\subsection{Correspondence with D6 Structure}

\begin{center}
\begin{tabular}{|c|c|}
\hline
Relativity & FUST \\
\hline
null ($ds^2 = 0$) & $\ker(D_6)$ \\
timelike ($ds^2 > 0$) & $\ker(D_6)^\perp$ \\
proper time $d\tau = 0$ & $D_6 f = 0$ \\
proper time $d\tau > 0$ & $D_6 f \neq 0$ \\
\hline
\end{tabular}
\end{center}

\subsection{Causal Boundary Theorem}

\begin{theorem}[Causal Boundary Theorem]
\label{thm:mat_causal_boundary}
\lean{FUST.causal_boundary_theorem}
\leanok
\begin{align*}
&(\forall f, f \in \ker(D_6) \Rightarrow \forall x \neq 0, D_6 f(x) = 0) \\
&\land (\forall f, (\forall t, \mathrm{perpProjection}\, f(t) = 0) \Rightarrow f \in \ker(D_6))
\end{align*}
\end{theorem}

\section{Structural Definition of the Speed of Light}
\label{sec:mat_speed_of_light}

\subsection{Essence of the Speed of Light}

In relativity, the speed of light $c$ is:
\begin{itemize}
\item Not merely a numerical constant
\item \textbf{A boundary condition that separates causal structure}
\end{itemize}

\subsection{Speed of Light in FUST}

\begin{quote}
\textbf{The speed of light in FUST is: ``Different homogeneous degrees must not be physically compared'' --- a normalization condition imposed by $D_6$.}
\end{quote}

This implies:
\begin{itemize}
\item The speed of light is automatically normalized to 1 within the theory
\item It does not depend on human unit systems like seconds or meters
\end{itemize}

\subsection{Derivation of Natural Units}

FUST does not assume a unit system but defines the causal boundary from $D_6$ structure. Therefore, $c = 1$ is not a ``choice'' but a ``structural consequence.''

