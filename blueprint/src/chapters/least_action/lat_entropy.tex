% FUST Least Action Theorem - Arrow of Time and Entropy
\chapter{Arrow of Time and Entropy}
\label{chap:mat_entropy}

\section{Golden Ratio Asymmetry}
\label{sec:mat_asymmetry}

The core asymmetry of the $D_6$ structure lies in the absolute values of the golden ratio $\varphi$ and its conjugate $\psi$:

\begin{center}
\begin{tabular}{|c|c|c|}
\hline
Quantity & Value & Property \\
\hline
$\varphi$ & $(1+\sqrt{5})/2 \approx 1.618$ & $\varphi > 1$ (divergent direction) \\
$\psi$ & $(1-\sqrt{5})/2 \approx -0.618$ & $|\psi| < 1$ (convergent direction) \\
$\varphi \cdot |\psi|$ & 1 & Product of conjugates \\
\hline
\end{tabular}
\end{center}

\begin{theorem}[$\varphi > 1$]
\label{thm:mat_phi_gt_one}
\lean{FUST.phi_gt_one}
\leanok
$\varphi > 1$
\end{theorem}

\begin{theorem}[$|\psi| < 1$]
\label{thm:mat_psi_lt_one}
\lean{FUST.abs_psi_lt_one}
\leanok
$|\psi| < 1$
\end{theorem}

\begin{theorem}[$\varphi \cdot |\psi| = 1$]
\label{thm:mat_phi_psi_prod}
\lean{FUST.TimeStructure.phi_mul_abs_psi}
\leanok
$\varphi \cdot |\psi| = 1$
\end{theorem}

\section{\texorpdfstring{$\varphi$}{φ}-Scaling as Time Evolution}
\label{sec:mat_time_evolution}

The $\varphi$-scaling $f \mapsto f(\varphi \cdot)$ is identified as time evolution via the Poincar\'{e} I4 decomposition in Gravity.lean:

\begin{enumerate}
\item The unified $D_\zeta$ operator decomposes into 4 components indexed by $I_4 = \mathrm{Fin}\,3 \oplus \mathrm{Fin}\,1$:
\[
D_\zeta\text{-components}(s) : I_4 \to \mathbb{C}, \quad \mathrm{Sum.inr}\,0 \mapsto \Phi_A\text{-coeff}(s)
\]
\item $\Phi_A\text{-coeff}(s)$ is built from $\varphi, \psi$ powers --- the \textbf{temporal channel depends on $\varphi$}.
\item The Minkowski metric $\eta(\mathrm{Sum.inr}\,0, \mathrm{Sum.inr}\,0) = -1$ identifies $\mathrm{Sum.inr}\,0$ as the \textbf{timelike} direction (Lean: \texttt{FUST.Physics.Gravity.$\eta$\_temporal}).
\item $\Phi_A\text{-coeff}(1) = 2(\varphi - \psi) = 2\sqrt{5} \neq 0$: the temporal channel is non-degenerate (Lean: \texttt{FUST.Physics.Gravity.temporal\_nonzero}).
\end{enumerate}

This justifies calling $\varphi$-scaling ``time evolution'':

\begin{definition}[$\varphi$-Scaling (Time Evolution)]
\label{def:mat_time_evolution}
\lean{FUST.TimeStructure.timeEvolution}
\leanok
$f \mapsto f(\varphi \cdot)$
\[
\mathrm{timeEvolution}(f)(t) := f(\varphi \cdot t)
\]
\end{definition}

$F_\zeta$-equivariance:

\begin{theorem}[$F_\zeta$ Time Evolution]
\label{thm:mat_Fzeta_time_evolution}
\lean{FUST.TimeStructure.Fζ_gauge_scaling}
\leanok
\[
F_\zeta[\mathrm{timeEvolution}(f)](z) = F_\zeta[f](\varphi z)
\]
\end{theorem}

\section{Arrow of Time}
\label{sec:mat_arrow_time}

Given the Poincar\'{e} identification of $\varphi$-scaling with the temporal direction (Section~\ref{sec:mat_time_evolution}), the direction of time is determined from the $\varphi/\psi$ asymmetry:

\begin{itemize}
\item \textbf{$\varphi$ direction ($|\varphi| > 1$)}: Future (information spreads, entropy increases)
\item \textbf{$\psi$ direction ($|\psi| < 1$)}: Past (information converges)
\end{itemize}

Behavior of $D_6$ evaluation points:
\begin{itemize}
\item $\varphi^3 x, \varphi^2 x, \varphi x \to$ Divergent direction when $x > 0$
\item $\psi^3 x, \psi^2 x, \psi x \to$ Convergent direction when $|x| > 0$
\end{itemize}

\section{Invariance of Kernel}
\label{sec:mat_ker_invariance}

Photons ($\ker(D_6)$) are invariant under time evolution:

\begin{theorem}[Kernel Invariance under Time Evolution]
\label{thm:mat_ker_invariant}
\lean{FUST.TimeStructure.ker_Fζ_invariant}
\leanok
\[
f \in \ker(F_\zeta) \Rightarrow \mathrm{timeEvolution}(f) \in \ker(F_\zeta)
\]
\end{theorem}

\textbf{Physical interpretation}: \textbf{Photons do not experience time} (proper time = 0).

\section{Definition of Entropy}
\label{sec:mat_entropy_def}

Quantify the deviation from $\ker(D_6)$ as entropy:

\begin{definition}[Entropy]
\label{def:mat_entropy}
\lean{FUST.TimeStructure.entropyAtFζ}
\leanok
\[
S(f)(t) := (\mathrm{perpProjection}(f)(t))^2
\]
\end{definition}

\textbf{Properties} (Lean4 proven):
\begin{enumerate}
\item $S(f)(t) \geq 0$ (non-negativity)
\item $S(f)(t) = 0 \iff \mathrm{perpProjection}(f)(t) = 0$
\item $(\forall t, S(f)(t) = 0) \iff f \in \ker(F_\zeta)$
\end{enumerate}

\section{Entropy Increase}
\label{sec:mat_entropy_increase}

Time evolution for monomials $f(t) = t^n$:

\begin{theorem}[Monomial Amplification]
\label{thm:mat_monomial_amp}
\lean{FUST.TimeStructure.monomial_amplification}
\leanok
\[
\mathrm{timeEvolution}(t^n)(t) = \varphi^n \cdot t^n
\]
\end{theorem}

\begin{theorem}[$\varphi^n > 1$ for $n \geq 1$]
\label{thm:mat_phi_pow}
\lean{FUST.TimeStructure.phi_pow_gt_one}
\leanok
For $n \geq 1$, $\varphi^n > 1$
\end{theorem}

Thus, time evolution amplifies $\ker(D_6)^\perp$ components ($n \geq 3$) by a factor of $\varphi^n$.

\section{Summary of Arrow of Time}
\label{sec:mat_arrow_summary}

The following are derived from $D_6$ and Poincar\'{e} $I_4$ structure:
\begin{enumerate}
\item $\varphi$-scaling = temporal direction: $D_\zeta$-components at $\mathrm{Sum.inr}\,0 = \Phi_A$-coeff (built from $\varphi,\psi$) with $\eta = -1$ --- \texttt{FUST.Physics.Gravity.$\Phi$\_A\_coeff\_one}, \texttt{FUST.Physics.Gravity.$\eta$\_temporal}
\item $\varphi > 1$ (future direction) --- \texttt{FUST.phi\_gt\_one}
\item $|\psi| < 1$ (past direction) --- \texttt{FUST.abs\_psi\_lt\_one}
\item $\ker(F_\zeta)$ is invariant under $\varphi$-scaling --- \texttt{FUST.TimeStructure.ker\_Fζ\_invariant}
\item $F_\zeta[\mathrm{timeEvolution}(f)](z) = F_\zeta[f](\varphi z)$ --- \texttt{FUST.TimeStructure.Fζ\_gauge\_scaling}
\end{enumerate}

\section{Physical Significance}
\label{sec:mat_entropy_physics}

\begin{center}
\begin{tabular}{|c|c|}
\hline
Thermodynamics & FUST \\
\hline
Entropy $S$ & $\mathrm{perpProjection}^2$ \\
Second law & Amplification by $\varphi^n > 1$ \\
Arrow of time & $\varphi/\psi$ asymmetry \\
Reversible process & Remains in $\ker(D_6)$ \\
Irreversible process & Transition to $\ker(D_6)^\perp$ \\
\hline
\end{tabular}
\end{center}

\textbf{Important}: These are derived from $D_6$ and $D_\zeta$ structure (including the Poincar\'{e} $I_4$ temporal channel identification), without external assumptions (initial conditions, statistical assumptions, external time parameter).

\section{Dimensional Type of Entropy}
\label{sec:mat_entropy_dim}

Entropy $S(f)(t) = (\mathrm{perpProjection}(f)(t))^2$ has the same dimension as the Lagrangian:
\[
\dim(S) = \dim((D_6 f)^2) = (-10, 2) \quad (\text{ScaleQuantity})
\]
Time evolution amplifies $D_6 f$ by $\varphi$, so $S' = \varphi^2 \cdot S$ in coordinate time, consistent with the proper-time energy conservation $E_{\mathrm{conserved}} = H / \varphi^{2 \cdot \mathrm{step}}$.

Lean: \texttt{FUST.TimeStructure.entropyAtFζ} with type \texttt{FζLagrangian}.

