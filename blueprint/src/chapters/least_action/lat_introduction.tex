% FUST Least Action Theorem - Introduction
\chapter{FUST Least Action Theorem: Introduction}
\label{chap:mat_introduction}

\section{Problem Statement}
\label{sec:mat_problem}

The fact that natural laws are described by variational principles has been consistently observed through classical mechanics, relativity, and quantum theory. However, traditional variational principles implicitly assume concepts such as time, distance, energy, and the speed of light, without explaining why these exist in the first place.

Time is the most fundamental concept in physics, yet existing theories introduce it as an ``external parameter'':
\begin{itemize}
\item \textbf{Newtonian mechanics}: Absolute time (assumed without explanation)
\item \textbf{Relativity}: Proper time defined from spacetime metric (metric is assumed)
\end{itemize}

Neither explains \textbf{why time exists}.

\section{Purpose of This Work}
\label{sec:mat_purpose}

\begin{quote}
\textbf{To derive time, the action theorem, and causal structure solely from FUST's $D_6$ structure.}
\end{quote}

We clearly distinguish:
\begin{itemize}
\item \textbf{Established as theorems}: Existence/non-existence of time, conditions for action minimization (gauge-invariant)
\item \textbf{Not established as theorems}: Quantitative values of time (gauge-dependent)
\end{itemize}

\textbf{Important}: FUST has no ``principles.'' Minimal action is not a principle assumed externally, but \textbf{derived as a theorem} from $D_6$ structure.

\section{Overview of FUST and D6 Structure}
\label{sec:mat_d6_overview}

\subsection{Definition of the D6 Operator}

The $D_6$ operator is defined using the golden ratio $\varphi = (1+\sqrt{5})/2$ and its conjugate $\psi = (1-\sqrt{5})/2$:

\begin{definition}[$D_6$ Operator]
\label{def:mat_D6}
\lean{FUST.D6}
\leanok
\[
D_6[f](x) = \frac{f(\varphi^3 x) - 3f(\varphi^2 x) + f(\varphi x) - f(\psi x) + 3f(\psi^2 x) - f(\psi^3 x)}{(\varphi - \psi)^5 x}
\]
\end{definition}

The coefficients $[1, -3, 1, -1, 3, -1]$ are uniquely determined from golden ratio relations (proven in Lean4).

\subsection{Algebraic Properties of the Golden Ratio}

\begin{center}
\begin{tabular}{|c|c|c|}
\hline
Quantity & Value & Property \\
\hline
$\varphi$ & $(1+\sqrt{5})/2 \approx 1.618034$ & $\varphi^2 = \varphi + 1$ \\
$\psi$ & $(1-\sqrt{5})/2 \approx -0.618034$ & $\psi^2 = \psi + 1$ \\
$\varphi\psi$ & $-1$ & Product of conjugates \\
$\varphi + \psi$ & $1$ & Sum of conjugates \\
$\varphi - \psi$ & $\sqrt{5}$ & Difference (normalization factor) \\
\hline
\end{tabular}
\end{center}

$D_7$ and higher do not introduce new structure. $D_6$ determines the ``minimal and complete'' causal structure.

\textbf{Numerical verification}: Coefficient $C_n$ for monomials $t^n$ under $D_6$:

\begin{center}
\begin{tabular}{|c|c|c|}
\hline
$n$ & $C_n$ & $\ker(D_6)$ \\
\hline
0 & 0 & $\checkmark$ \\
1 & 0 & $\checkmark$ \\
2 & 0 & $\checkmark$ \\
3 & 26.83 & $\times$ \\
4 & 187.83 & $\times$ \\
5 & 1006.23 & $\times$ \\
\hline
\end{tabular}
\end{center}

This structure automatically decomposes the state space into ``invisible/reversible'' and ``visible/irreversible.''

\subsection{Gauge Parameter x}

The parameter $x$ in the definition of $D_6$ is a gauge choice, not a physical quantity.

\begin{itemize}
\item $x \neq 0$: Technical condition for the gauge to be well-defined
\item $x = 0$: Coordinate singularity (not a physical singularity)
\end{itemize}

\begin{theorem}[Gauge Scaling]
\label{thm:mat_gauge_scaling}
\lean{FUST.D6_gauge_scaling}
\leanok
\[
D_6[f(c \cdot)](x) = c \cdot D_6[f](cx)
\]
\end{theorem}

\subsection{Scaling Law}

For homogeneous functions $f(t) = t^n$:
\[
D_6[t^n](x) \propto x^{n-1}
\]

Thus, \textbf{degree $n$ serves as an eigenvalue label under scaling}.

