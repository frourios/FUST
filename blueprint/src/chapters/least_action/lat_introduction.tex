% FUST Least Action Theorem - Introduction
\chapter{Introduction: Poincar\'{e} Group as Organizing Principle}
\label{chap:mat_introduction}

\section{Problem Statement}
\label{sec:mat_problem}

Time, space, and mass are the most fundamental concepts in physics, yet existing theories introduce them as external parameters:
\begin{itemize}
\item \textbf{Newtonian mechanics}: Absolute time and Euclidean space (assumed)
\item \textbf{Special relativity}: Minkowski metric $\eta = \mathrm{diag}(+1,+1,+1,-1)$ (assumed)
\item \textbf{General relativity}: Pseudo-Riemannian manifold with Lorentz signature (assumed)
\item \textbf{Quantum field theory}: Poincar\'{e} group $\mathrm{ISO}(3,1)$ as input symmetry (assumed)
\end{itemize}

None of these explains \textbf{why} spacetime has signature $(3,1)$, \textbf{why} the Poincar\'{e} group governs kinematics, or \textbf{why} mass is a Casimir invariant.

\section{Central Thesis}
\label{sec:mat_thesis}

\begin{quote}
\textbf{The Poincar\'{e} group $\mathrm{ISO}(3,1) = \mathrm{SO}(3,1) \ltimes \mathbb{R}^4$ is derived from the algebraic structure of the unified operator $D_\zeta$. Time, space, mass, and all physical laws follow from this single derivation.}
\end{quote}

FUST has no ``principles.'' The least action principle, thermodynamic laws, Einstein equations, and gauge group structure are all \textbf{theorems}, not assumptions.

\section{Overview of \texorpdfstring{$D_\zeta / F_\zeta$}{Dζ/Fζ}}
\label{sec:mat_overview}

The core operator is $\mathrm{Diff}_6$:

\begin{definition}[$\mathrm{Diff}_6$ Operator]
\label{def:mat_D6}
\lean{FUST.Diff6}
\leanok
\[
\mathrm{Diff}_6[f](x) = \frac{f(\varphi^3 x) - 3f(\varphi^2 x) + f(\varphi x) - f(\psi x) + 3f(\psi^2 x) - f(\psi^3 x)}{(\varphi - \psi)^5 x}
\]
\end{definition}

The unified operator $F_\zeta = 5z \cdot D_\zeta$ combines this with $\zeta_6$ arithmetic. Its kernel $\ker(F_\zeta) = \mathrm{span}\{1, z, z^2\}$ is 3-dimensional, and active modes are $n \equiv 1, 5 \pmod{6}$.

\section{Derivation Roadmap}
\label{sec:mat_roadmap}

The following chapters establish the complete derivation chain:

\[
\boxed{D_\zeta} \;\xrightarrow{\text{Ch.\ref{chap:mat_poincare}}}\; I_4 = \mathrm{Fin}\,3 \oplus \mathrm{Fin}\,1 \;\to\; \eta \;\to\; \mathfrak{so}(3,1) \;\to\; \mathfrak{iso}(3,1)
\]
\[
\xrightarrow{\text{Ch.\ref{chap:mat_time_space}}}\; \text{Time, Space, Arrow of Time} \;\xrightarrow{\text{Ch.\ref{chap:mat_mass_energy}}}\; \text{Mass, Energy, Lagrangian}
\]
\[
\xrightarrow{\text{Ch.\ref{chap:mat_action_functional}}}\; \text{Action Principle} \;\xrightarrow{\text{Ch.\ref{chap:mat_physical_laws}}}\; \text{Thermodynamics, Gravity, Gauge Groups}
\]

\begin{center}
\begin{tabular}{|c|c|c|}
\hline
Chapter & Derives & From \\
\hline
\ref{chap:mat_poincare} & $\mathrm{ISO}(3,1)$, $\eta$, Casimir & $D_\zeta$ channel structure \\
\ref{chap:mat_time_space} & Time, space, arrow of time, $t_P$ & Poincar\'{e} temporal/spatial channels \\
\ref{chap:mat_mass_energy} & Mass gap $m^2 = 14$, $\Delta = 12/25$ & Casimir invariant of $D_\zeta$ 4-momentum \\
\ref{chap:mat_action_functional} & Action principle, Euler-Lagrange & $F_\zeta$ linearity and variational structure \\
\ref{chap:mat_physical_laws} & Thermodynamics, Einstein, gauge groups & Localized $\mathfrak{so}(3,1)$, $F_\zeta$ channels \\
\hline
\end{tabular}
\end{center}

Every entry in this table is a \textbf{theorem} proven in Lean4, not an assumption or principle.
