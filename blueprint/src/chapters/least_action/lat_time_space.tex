% FUST Least Action Theorem - Time and Space from Poincaré
\chapter{Time and Space from Poincar\'{e} Structure}
\label{chap:mat_time_space}

Time and space are not external parameters but consequences of the $D_\zeta$ decomposition into $I_4 = \mathrm{Fin}\,1 \oplus \mathrm{Fin}\,3$ (Chapter~\ref{chap:mat_poincare}).

\section{Time = \texorpdfstring{$\varphi$}{φ}-Scaling (Poincar\'{e} Temporal Channel)}
\label{sec:mat_time_definition}

The temporal direction is identified via the $\Phi_A$ channel of $D_\zeta$:

\begin{theorem}[$\Phi_A$-Coefficient at $s=1$]
\label{thm:mat_phiA_one}
\lean{FUST.Physics.Gravity.Φ_A_coeff_one}
\leanok
\[
\Phi_A\text{-coeff}(1) = 2(\varphi - \psi) = 2\sqrt{5}
\]
\end{theorem}

Combined with $\eta(\mathrm{Sum.inl}\,0, \mathrm{Sum.inl}\,0) = +1$ (Theorem~\ref{thm:mat_eta_temporal}), the $\mathrm{Sum.inl}\,0$ component is timelike. The $\Phi_A$ channel is built from $\varphi, \psi$ powers, so $\varphi$-scaling is intrinsically temporal.

\begin{definition}[Time Evolution]
\label{def:mat_time_evolution}
\lean{FUST.TimeStructure.timeEvolution}
\leanok
\[
\mathrm{timeEvolution}(f)(t) := f(\varphi \cdot t)
\]
\end{definition}

\section{Space = \texorpdfstring{$\ker(F_\zeta)$}{ker(Fζ)} Structure}
\label{sec:mat_space_definition}

The 3 spatial directions correspond to the $\Phi_S$ sub-operators:

\begin{itemize}
\item $\sigma_{\mathrm{Diff}_5}(s)$, $\sigma_{\mathrm{Diff}_3}(s)$, $\sigma_{\mathrm{Diff}_2}(s)$ at $\mathrm{Sum.inr}\,0,1,2$
\item $\eta(\mathrm{Sum.inr}\,i, \mathrm{Sum.inr}\,i) = -1$ (spacelike, Theorem~\ref{thm:mat_eta_spatial})
\item $\ker(F_\zeta) = \mathrm{span}\{1, z, z^2\}$ has dimension 3, matching spatial dimension
\end{itemize}

\section{\texorpdfstring{$\varphi/\psi$}{φ/ψ} Duality}
\label{sec:mat_phi_psi_duality}

\begin{theorem}[$\varphi \cdot |\psi| = 1$]
\label{thm:mat_phi_psi}
\lean{FUST.TimeStructure.phi_mul_abs_psi}
\leanok
The golden ratio and its conjugate satisfy $\varphi \cdot |\psi| = 1$.
\end{theorem}

\begin{theorem}[$\varphi > 1$]
\label{thm:mat_phi_gt_one}
\lean{FUST.phi_gt_one}
\leanok
\end{theorem}

\begin{theorem}[$|\psi| < 1$]
\label{thm:mat_psi_lt_one}
\lean{FUST.abs_psi_lt_one}
\leanok
\end{theorem}

This asymmetry $\varphi > 1 > |\psi|$ with $\varphi \cdot |\psi| = 1$ is the algebraic origin of the arrow of time.

\section{Arrow of Time}
\label{sec:mat_arrow_of_time}

\begin{itemize}
\item \textbf{$\varphi$-direction} ($|\varphi| > 1$): future. $\mathrm{Diff}_6$ evaluates at $\varphi^3 z, \varphi^2 z, \varphi z$ which diverge as $z \to \infty$.
\item \textbf{$\psi$-direction} ($|\psi| < 1$): past. $\mathrm{Diff}_6$ evaluates at $\psi z, \psi^2 z, \psi^3 z$ which converge to 0.
\end{itemize}

\begin{theorem}[$\ker(F_\zeta)$ Invariance]
\label{thm:mat_ker_invariance}
\lean{FUST.TimeStructure.ker_Fζ_invariant}
\leanok
If $f \in \ker(F_\zeta)$, then $\mathrm{timeEvolution}(f) \in \ker(F_\zeta)$.
\end{theorem}

Photons (in $\ker(F_\zeta)$) do not experience time: time evolution preserves their kernel membership.

\begin{theorem}[Monomial Amplification]
\label{thm:mat_amplification}
\lean{FUST.TimeStructure.monomial_amplification}
\leanok
$\mathrm{timeEvolution}(t^n)(t) = \varphi^n \cdot t^n$.
\end{theorem}

\begin{theorem}[$\varphi^n > 1$ for $n \geq 1$]
\label{thm:mat_phi_pow_gt}
\lean{FUST.TimeStructure.phi_pow_gt_one}
\leanok
Time evolution amplifies all non-constant monomials.
\end{theorem}

\section{Planck Second from Temporal Eigenvalue}
\label{sec:mat_planck_second}

The minimum time interval is derived from the Poincar\'{e} temporal channel at the minimum active mode $s = 1$:

\begin{definition}[Planck Second]
\label{def:mat_planck_second}
\lean{FUST.TimeStructure.planckSecond}
\leanok
\[
t_P = \frac{1}{20\sqrt{15}}
\]
\end{definition}

\begin{theorem}[$t_P > 0$]
\label{thm:mat_planck_pos}
\lean{FUST.TimeStructure.planckSecond_pos}
\leanok
\end{theorem}

\begin{theorem}[$t_P^2 = 1/6000$]
\label{thm:mat_planck_sq}
\lean{FUST.TimeStructure.planckSecond_sq}
\leanok
\end{theorem}

The derivation chain:
\begin{enumerate}
\item $\Phi_A\text{-coeff}(1) = 2\sqrt{5}$ (temporal channel at $s = 1$)
\item Temporal $F_\zeta$ eigenvalue $= 5 \cdot \Phi_A\text{-coeff}(1) \cdot \mathrm{AF\_coeff} = 10\sqrt{5} \cdot 2i\sqrt{3} = 20i\sqrt{15}$
\item $|\text{temporal eigenvalue}|^2 = (20\sqrt{15})^2 = 6000$
\item $t_P = 1/\sqrt{6000} = 1/(20\sqrt{15})$
\end{enumerate}

\begin{theorem}[Temporal Eigenvalue Decomposition]
\label{thm:mat_temporal_decomposition}
\lean{FUST.TimeStructure.temporalEigenNormSq_mass_formula}
\leanok
\[
6000 = 12 \cdot (10\sqrt{5})^2
\]
where $12 = |\mathrm{AF\_coeff}|^2$ and $10\sqrt{5} = 5 \cdot \Phi_A\text{-coeff}(1)$.
\end{theorem}

\section{Spacetime Dimension Uniqueness}
\label{sec:mat_spacetime_dim}

The 4-dimensional Lorentzian spacetime with signature $(1,3)$ is uniquely determined by $D_\zeta$:

\begin{theorem}[Spacetime Dimension from Channel Weights]
\label{thm:mat_spacetime_dimension}
\lean{FUST.WeinbergAngle.totalWeight_eq}
\leanok
\[
\mathrm{syWeight} + \mathrm{afWeight} = 3 + 1 = 4
\]
\end{theorem}

\begin{theorem}[$D_\zeta$ Determines Spacetime Uniquely]
\label{thm:mat_spacetime_uniqueness}
\lean{FUST.SpacetimeUniqueness.spacetime_from_Dζ}
\leanok
Without any additional assumption beyond ZF+DC:
\begin{enumerate}
\item $\mathrm{AF\_coeff}$ is purely imaginary with positive imaginary part, forcing a 1+3 decomposition
\item The spatial $\Phi_S$ channel has rank 3 (3 independent sub-operators)
\item The temporal $\Phi_A$ channel is nontrivial
\item Signature $(1,3)$: temporal positive, spatial negative
\item $O(3,1)$ preserves the $D_\zeta$-derived Minkowski metric
\end{enumerate}
\end{theorem}

\begin{theorem}[Poincar\'{e} Group Construction]
\label{thm:mat_poincare_group_constructed}
\lean{FUST.SpacetimeUniqueness.instGroupPoincareGroup}
\leanok
$\mathrm{ISO}(3,1) = \mathbb{R}^4 \rtimes O(3,1)$ is a group, constructed as a semidirect product where $O(3,1)$ acts on $\mathbb{R}^4$ by matrix-vector multiplication. The group operation $(a_1, \Lambda_1) \cdot (a_2, \Lambda_2) = (a_1 + \Lambda_1 a_2, \Lambda_1 \Lambda_2)$ is verified by Lean's type checker.
\end{theorem}

\section{Summary}
\label{sec:mat_time_space_summary}

\begin{center}
\begin{tabular}{|c|c|c|}
\hline
Physical concept & $D_\zeta$ origin & Lean reference \\
\hline
Time direction & $\Phi_A$-channel, $\eta(\mathrm{inl}\,0,\mathrm{inl}\,0) = +1$ & \texttt{Φ\_A\_coeff\_one}, \texttt{η\_temporal} \\
Spatial dimensions & $\Phi_S$ sub-operators, $\eta(\mathrm{inr}\,i,\mathrm{inr}\,i) = -1$ & \texttt{η\_spatial} \\
Time evolution & $f \mapsto f(\varphi \cdot)$ & \texttt{timeEvolution} \\
Arrow of time & $\varphi > 1 > |\psi|$, $\varphi \cdot |\psi| = 1$ & \texttt{phi\_gt\_one}, \texttt{phi\_mul\_abs\_psi} \\
Planck second & $t_P = 1/(20\sqrt{15})$ & \texttt{planckSecond} \\
Spacetime dim & $3 + 1 = 4$ & \texttt{totalWeight\_eq} \\
Poincar\'{e} group & $\mathbb{R}^4 \rtimes O(3,1)$ & \texttt{PoincareGroup} \\
\hline
\end{tabular}
\end{center}
