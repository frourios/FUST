% Chapter 8: Conclusion
\chapter{Conclusion}
\label{chap:conclusion}

This paper has demonstrated the following:

\begin{enumerate}
\item The coefficients of $D_5$ and $D_6$ are uniquely determined as $(a,b,A,B)=(-1,-4,3,1)$
\item The half-order hierarchy $D_{5.5}$ necessarily exists in FUST
\item It is the unique intermediate structure connecting $D_5$ and $D_6$
\item The corresponding 5.5-point difference operator can be naturally defined mathematically
\item The half-order hierarchy exists only at $D_5$
\item \textbf{Algebraically non-equivalent difference operators are limited to the 6 types $\{D_2, D_3, D_4, D_5, D_{5.5}, D_6\}$} (Six-Element Completeness Theorem)
\item \textbf{The $\varphi$-dilation eigenvalue structure on $\ker(D_n)$ defines a gauge group parameter space $\mathcal{G}$ (4 sectors: $\varphi$-preserving vs $\varphi$-breaking on each kernel), and $\mathrm{SU}(3) \times \mathrm{SU}(2) \times \mathrm{U}(1)$ is the $\varphi$-breaking point uniquely selected by the describability condition}
\end{enumerate}

This is the \textbf{minimal structure that captures the moment when recursion emerges as mathematics}, providing a unified description from static structure to dynamical evolution equations, cohomological structure, and number-theoretic structure.

