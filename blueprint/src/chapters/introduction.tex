% Chapter 1: Introduction
\chapter{Introduction}
\label{chap:introduction}

\section{Background and Purpose}
\label{sec:background}

The Standard Model of modern physics agrees with experiments to extremely high precision, but the reasons for the choice of its gauge group $\mathrm{SU}(3) \times \mathrm{SU}(2) \times \mathrm{U}(1)$ and its parameters remain unknown, forcing physicists to rely on selection principles external to physics, such as the ``anthropic principle'' or ``multiverse.''

The purpose of this research is to overcome this situation and achieve the following:

\begin{quote}
\textbf{To prove that the structural hierarchy of the observable universe is uniquely determined within the ZF + DC axiom system with the golden ratio as base, without using the full axiom of choice (AC) or physical fine-tuning.}
\end{quote}

\section{Methodology: The Universal Probe Parameter x}
\label{sec:methodology}

The variable $x$ introduced in this theory is not a physical spacetime coordinate, but a \textbf{Universal Probe Parameter} for scanning the algebraic structure of the universe. Through the behavior (kernel structure) of the difference operators $D_n$ with respect to this $x$, the ``type'' of physical laws is determined. Physical spacetime and distance are treated as concepts that derive secondarily from this algebraic structure.

\section{On the Choice of the Golden Ratio}
\label{sec:golden_ratio_choice}

\textbf{Important limitation}: The golden ratio $\varphi$ is \textbf{defined} as the positive root of $x^2 = x + 1$, but ``why this equation is special'' is a \textbf{metamathematical choice}. However, the golden ratio is the ``most difficult irrational number to approximate'' by Hurwitz's theorem, and is the unique basis that maximizes the stability of algebraic structures. The choice of $\varphi$ in FUST is not arbitrary but a mathematical requirement to guarantee the \textbf{robustness} of the structure. FUST's claim is that after this choice is made, the structure is uniquely determined \textbf{without additional physical selection principles}.
