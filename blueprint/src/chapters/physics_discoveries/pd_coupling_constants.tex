% FUST Physics Discoveries - Coupling Constants
\chapter{Coupling Constants from \texorpdfstring{$D_\zeta$}{Dζ} Structure}
\label{chap:pd_coupling}

\section{Strong Coupling Constant}
\label{sec:coupling_strong}

\begin{theorem}[Strong Coupling]
\label{thm:coupling_strong}
\lean{FUST.CouplingConstants.strong_coupling_formula}
\leanok
\[
\alpha_s = \frac{C(3,2)}{C(5,2) + C(6,2)} = \frac{3}{25} = 0.12
\]
This uses $N_3$ (weak sector) vs $N_5$+$N_6$ (unified sector).
\end{theorem}

\section{Cabibbo Angle}
\label{sec:coupling_cabibbo}

\begin{theorem}[Cabibbo Angle]
\label{thm:coupling_cabibbo}
\lean{FUST.CouplingConstants.cabibbo_from_phi}
\leanok
\[
\theta_C = \arctan(1/\varphi^3) \approx 13.28°
\]
The exponent 3 = $C(3,2)$ from $N_3$ pair count.
\end{theorem}

\section{PMNS Mixing Angles}
\label{sec:coupling_pmns}

\begin{theorem}[Solar Angle]
\label{thm:coupling_pmns_solar}
\lean{FUST.CouplingConstants.pmns_solar_from_N3_symmetry}
\leanok
\[
\sin^2\theta_{12} = \frac{1}{3}
\]
from $N_3$ three-fold symmetry.
\end{theorem}

\begin{theorem}[Reactor Angle]
\label{thm:coupling_pmns_reactor}
\lean{FUST.CouplingConstants.pmns_reactor_structural}
\leanok
\[
\sin^2\theta_{13} = \frac{C(2,2)}{C(6,2)} = \frac{1}{15}
\]
\end{theorem}

\section{CP Phase}
\label{sec:coupling_cp_phase}

\begin{theorem}[CP Phase]
\label{thm:coupling_cp_phase}
\lean{FUST.CouplingConstants.cp_phase_from_active_levels}
\leanok
\[
\delta_{CKM} = \frac{2\pi}{5}
\]
where 5 = number of active numerator operator levels ($N_2$ through $N_6$).
\end{theorem}

\section{Fine Structure Constant}
\label{sec:coupling_fine_structure}

\begin{theorem}[Spatial Normalization]
\label{thm:coupling_spatial_norm}
\lean{FUST.CouplingConstants.spatial_normalization}
\leanok
\[
C(D_{\max}, \dim\ker(F_\zeta)) = C(6,3) = 20
\]
This equals the sum of pair counts: $C(2,2)+C(3,2)+C(4,2)+C(5,2) = 1+3+6+10 = 20$.
\end{theorem}

\begin{theorem}[Fine Structure Constant]
\label{thm:coupling_fine_structure}
\lean{FUST.CouplingConstants.fine_structure_from_Fζ_structure}
\leanok
\[
\alpha_0 = \frac{\varphi^{-(D_{\max}-D_{\min})}}{C(D_{\max},\,\dim\ker(F_\zeta))} = \frac{\varphi^{-4}}{C(6,3)}
\]
where:
\begin{itemize}
\item $\varphi^{-4} = \varphi^{-(D_{\max}-D_{\min})}$: D-level propagator across the range $6-2=4$
\item $C(6,3) = 20 = C(D_{\max},\,\dim\ker(F_\zeta))$: spatial configuration count
\end{itemize}
Algebraic form: $\alpha_0 = (7-3\sqrt{5})/40$, giving $1/\alpha_0 = 70+30\sqrt{5} \approx 137.082$ (tree-level, error 0.034\%).
\end{theorem}

\section{CKM Decay Structure}
\label{sec:coupling_ckm}

\begin{theorem}[CKM Geometric Decay]
\label{thm:coupling_ckm_decay}
\lean{FUST.CouplingConstants.ckm_geometric_decay}
\leanok
CKM matrix elements decay as $\varphi^{-3n}$ where $n$ is the numerator operator step distance:
\begin{align}
|V_{us}| &\propto \varphi^{-3} \\
|V_{cb}| &\propto \varphi^{-6} \\
|V_{ub}| &\propto \varphi^{-9}
\end{align}
The exponent 3 = $C(3,2)$ comes from $N_3$ pair count.
\end{theorem}

\section{Connection to Gauge Boson Masses}
\label{sec:coupling_gauge_mass}

The coupling constants connect directly to gauge boson masses:
\begin{itemize}
\item \textbf{W mass exponent}: $C(5,2) + C(6,2) = 25$ (same pair counts as $\alpha_s$ denominator)
\item \textbf{W mass factor}: $C(6,2)/(C(6,2)+C(2,2)) = 15/16$ (operator-level boundary normalization)
\item \textbf{Z mass}: $m_Z = m_W / \cos\theta_W$ using $\sin^2\theta_W = 1/4$
\end{itemize}

The relation $\alpha_s = C(3,2)/(C(5,2)+C(6,2)) = 3/25$ and the W exponent $C(5,2)+C(6,2) = 25$
share the \emph{same denominator}: the strong coupling constant is the ratio of $N_3$ pairs to the
total pairs appearing in the W mass exponent.

\section{Summary}
\label{sec:coupling_summary}

\begin{theorem}[Coupling Constants from Kernel Structure]
\label{thm:coupling_summary}
\lean{FUST.CouplingConstants.coupling_constants_from_kernel_structure}
\leanok
\begin{enumerate}
\item Strong coupling: $\alpha_s = C(3,2)/25 = 3/25$
\item PMNS reactor angle: $\sin^2\theta_{13} = C(2,2)/C(6,2) = 1/15$
\item Active levels = 5
\item $N_3$ gauge invariance: $N_3[1] = 0$
\item $N_5$ extended kernel: $N_5[1] = N_5[x] = 0$
\end{enumerate}
\end{theorem}

\section{Dimensional Types}
\label{sec:coupling_dim}

The dimensional type system classifies coupling constants as follows:

\begin{center}
\begin{tabular}{|c|c|c|}
\hline
Quantity & Type & Lean \\
\hline
$\alpha_s = C(3,2)/(C(5,2)+C(6,2)) = 3/25$ & RatioQuantity & \texttt{strongCoupling} \\
$\sin^2\theta_W = \mathrm{AF\_weight}/\mathrm{total\_weight} = 1/4$ & RatioQuantity & \texttt{weinbergAngle} \\
$\sin^2\theta_{12} = 1/C(3,2) = 1/3$ & RatioQuantity & \texttt{solarMixing} \\
$m_W^2/m_Z^2 = \mathrm{SY\_weight}/\mathrm{total\_weight} = 3/4$ & RatioQuantity & \texttt{wzRatioSq} \\
Active D-levels $= 5$ & CountQuantity & \texttt{activeDLevels} \\
$\delta_{CKM} = 2\pi/5$ & ScaleQ$(0,0,0)$ & \texttt{cpPhase} \\
$\alpha_0 = \varphi^{-4}/C(6,3) = (7-3\sqrt{5})/40$ & ScaleQ$(0,0,0)$ & \texttt{fineStructure} \\
$\theta_C = \arctan(1/\varphi^3)$ & ScaleQ$(0,0,0)$ & \texttt{cabibboAngle} \\
\hline
\end{tabular}
\end{center}

RatioQuantity values are ratios of pair counts $C(m,2)$, ensuring they are dimensionless. ScaleQ$(0,0,0)$ quantities involving $\pi$ ($\delta_{CKM}$) or $\arctan$ ($\theta_C$) are coordinate embeddings (see \S\ref{sec:pd_principles}). All definitions in \texttt{FUST\_dim.CouplingConstants} use structural formulas ($C(m,k)$, $\varphi$), never hardcoded numerals.

