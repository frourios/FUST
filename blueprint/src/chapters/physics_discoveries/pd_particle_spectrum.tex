% FUST Physics Discoveries - Particle Spectrum
\chapter{Particle Spectrum from D-Structure}
\label{chap:pd_particle}

\section{Generation Structure from $\ker(D_6)$}
\label{sec:particle_generations}

The number of fermion generations is not an input parameter but is derived from $\dim\ker(D_6) = 3$.

\begin{theorem}[$\ker(D_6)$ Basis]
\label{thm:particle_D6_kernel_basis}
\lean{FUST.ParticleSpectrum.D6_kernel_basis}
\leanok
$\ker(D_6) = \{1, x, x^2\}$: all three basis elements are annihilated by $D_6$.
\end{theorem}

\begin{theorem}[Maximum Generations $= \dim\ker(D_6)$]
\label{thm:particle_max_generations}
\lean{FUST.ParticleSpectrum.maxGenerations_eq}
\leanok
\[
\text{maxGenerations} = \dim\ker(D_6) = 3
\]
Each basis element of $\ker(D_6)$ corresponds to one generation.
\end{theorem}

\begin{theorem}[Generation-Mass Separation]
\label{thm:particle_generation_mass}
\lean{FUST.ParticleSpectrum.generation_mass_separation}
\leanok
The $\ker(D_6)$ basis $\{1, x, x^2\}$ is separated by the responses of lower $D$-operators:
\begin{itemize}
\item $1 \in \ker(D_3) \cap \ker(D_5) \cap \ker(D_6)$: lightest (electron sector)
\item $x \in \ker(D_6) \setminus \ker(D_3)$: intermediate mass (muon sector)
\item $x^2 \in \ker(D_6) \setminus \ker(D_5)$: heaviest (tau sector)
\end{itemize}
Mass exponents derive from $D$-operator pair counts:
$x^0 \to D_4$: $T(4)+1 = 11$, \quad $x^1 \to D_3$: $T(3) = 6$, \quad $x^2 \to D_3$: $T(3) = 6$.
\end{theorem}

\begin{theorem}[Fourth Generation Forbidden]
\label{thm:particle_4th_gen_forbidden}
\lean{FUST.ParticleSpectrum.fourth_generation_forbidden}
\leanok
$x^3 \notin \ker(D_6)$: $D_6$ detects cubic ($D_6[t^3](x) \neq 0$ for all $x \neq 0$).
Additionally, $D_7$ projects to $D_6$ via Fibonacci recurrence.
\end{theorem}

\section{Charge Constraint}
\label{sec:particle_charges}

\begin{theorem}[Allowed Charges]
\label{thm:particle_allowed_charges}
\lean{FUST.ParticleSpectrum.allowedChargeCount_eq}
\leanok
Allowed charge count from $D_3$ structure:
\[
2 \times C(3,2) + 1 = 7
\]
Allowed charges: $Q = n/3$ for $n \in \{-3, -2, -1, 0, 1, 2, 3\}$.
\end{theorem}

\begin{theorem}[Exotic Charges Forbidden]
\label{thm:particle_exotic_forbidden}
\lean{FUST.ParticleSpectrum.charge_one_fifth_forbidden}
\leanok
Charge $Q = 1/5$ is not representable as $n/3$ and is therefore forbidden.
\end{theorem}

\section{Spin Constraint}
\label{sec:particle_spins}

\begin{theorem}[Maximum Spin]
\label{thm:particle_max_spin}
\lean{FUST.ParticleSpectrum.spin_ceiling}
\leanok
Spin count = spacetime dimension = 4. Allowed spins: $0, 1/2, 1, 2$.

Spin $> 2$ would require $D_7$+, which projects to $D_6$.
\end{theorem}

\section{Colored Leptons Forbidden}
\label{sec:particle_colored_leptons}

\begin{theorem}[Colored Leptons Forbidden]
\label{thm:particle_colored_lepton}
\lean{FUST.ParticleSpectrum.colored_lepton_forbidden}
\leanok
Colored lepton (hypercharge $Y = -1$, color = triplet) is forbidden by $D_3$-$D_4$ embedding.
\end{theorem}

\section{SM Particle Count}
\label{sec:particle_count}

\begin{theorem}[Fermion Count]
\label{thm:particle_fermion_count}
\lean{FUST.ParticleSpectrum.SM_fermion_count_eq}
\leanok
\begin{align}
\text{Leptons} &= 2 \times 3 = 6 \\
\text{Quarks} &= 2 \times 3 \times C(3,2) = 18 \\
\text{Total fermions} &= 24
\end{align}
\end{theorem}

\begin{theorem}[Boson Count]
\label{thm:particle_boson_count}
\lean{FUST.ParticleSpectrum.SM_boson_count_eq}
\leanok
\begin{align}
\text{Gluons} &= C(3,2)^2 - 1 = 8 \\
\text{Weak bosons (W}^\pm\text{, Z)} &= C(3,2) = 3 \\
\text{Photon} &= 1 \\
\text{Higgs} &= 1 \\
\text{Total bosons} &= 13
\end{align}
\end{theorem}

\begin{theorem}[Total SM Particles]
\label{thm:particle_sm_count}
\lean{FUST.ParticleSpectrum.SM_count}
\leanok
\[
\text{SM particles} = 24 + 13 = 37
\]
Including graviton: 38.
\end{theorem}

\section{Predicted Particles}
\label{sec:particle_predictions}

\subsection{Graviton at \texorpdfstring{$D_6$}{D6}}

The graviton is structurally predicted from $D_6$, not postulated.

\begin{theorem}[Graviton Structural Derivation]
\label{thm:graviton_structural}
\lean{FUST.ParticleSpectrum.graviton_spin_derived}
\leanok
The graviton has:
\begin{itemize}
\item $D$-level $= 6$ (highest active level)
\item Spin $= 2$ (from $\mathrm{spatialDim} - \mathrm{timeDim} = 3 - 1$)
\item Massless (from $\Box_\varphi[t^{-1}] = 0$)
\item Gravity sector trace $= L(3) = 4 = \mathrm{spacetimeDim}$
\item Gravity sector discriminant $= 20 = C(6,3)$
\end{itemize}
See Chapter~\ref{chap:pd_gravitational}, \S\ref{sec:grav_graviton} for the full derivation.
\end{theorem}

\subsection{Neutrino Mass Structure at \texorpdfstring{$D_5$}{D5}}

\begin{theorem}[Neutrino Generation Structure]
\label{thm:neutrino_generation}
\lean{FUST.ParticleSpectrum.neutrino_generation_structure}
\leanok
3 neutrino flavors from $\dim\ker(D_6) = 3$ (SU(2) doublet with charged leptons).
Mass states split by $\ker(D_5) \subset \ker(D_6)$:
\begin{itemize}
\item $\ker(D_5) = \{1, x\}$: solar pair $\nu_1, \nu_2$ (2 nearly-degenerate states)
\item $x^2 \in \ker(D_6) \setminus \ker(D_5)$: atmospheric $\nu_3$ (separated)
\end{itemize}
Right-handed neutrinos predicted at $D_5$ level (spin-1/2).
\end{theorem}

\subsection{\texorpdfstring{$D_{5\frac{1}{2}}$}{D5.5} Dark Matter}
Dark matter candidate between $D_5$ and $D_6$ with coupling suppression $\varphi^{-3/2}$.
Mass: $m_{DM}/m_e = \varphi^{25}$ (same exponent as W boson, differing by factor $15/16$).

\section{Summary}
\label{sec:particle_summary}

\begin{theorem}[Particle Spectrum Summary]
\label{thm:particle_summary}
\lean{FUST.ParticleSpectrum.particle_spectrum_summary}
\leanok
\begin{enumerate}
\item SM particle count = 37 (derived from D-structure)
\item Maximum generations $= \dim\ker(D_6) = 3$
\item $x^3 \notin \ker(D_6)$ prevents 4th generation; $D_7$ projects to $D_6$ prevents spin $> 2$
\item Allowed spins = spacetime dimension = 4
\item Allowed charges from $D_3$ structure
\item Graviton: spin-2, massless, $D$-level 6 (structurally derived)
\end{enumerate}
\end{theorem}

\section{Dimensional Types}
\label{sec:particle_dim}

All particle spectrum quantities are CountQuantity ($\mathbb{N}$), derived from kernel dimensions and pair counts:

\begin{center}
\begin{tabular}{|c|c|c|}
\hline
Quantity & Value & Lean \\
\hline
Max generations & $\dim\ker(D_6) = 3$ & \texttt{maxGenerations} \\
SM fermions & $24$ & \texttt{smFermionCount} \\
SM bosons & $13$ & \texttt{smBosonCount} \\
SM total & $37$ & \texttt{smParticleCount} \\
Allowed charges & $2 \times C(3,2) + 1 = 7$ & \texttt{allowedChargeCount} \\
Allowed spins & spacetimeDim $= 4$ & \texttt{allowedSpinCount} \\
$\dim \mathfrak{su}(2) = 3$ & $C(3,2)^2 - 1$ & \texttt{su2Dim} \\
$\dim \mathfrak{su}(3) = 8$ & $C(3,2)^2 - 1$ & \texttt{su3Dim} \\
Spatial dim $= 3$ & $\dim\ker(D_6)$ & \texttt{spatialDim} \\
Spacetime dim $= 4$ & $\dim\ker(D_6) + 1$ & \texttt{spacetimeDim} \\
\hline
\end{tabular}
\end{center}

All definitions in \texttt{FUST\_dim.ParticleSpectrum} reference upstream FUST definitions, not hardcoded numerals.

