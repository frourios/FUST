% FUST Physics Discoveries - Particle Spectrum
\chapter{Particle Spectrum from D-Structure}
\label{chap:pd_particle}

\section{$\ker(D_6)$ Structure}
\label{sec:particle_ker_d6}

\begin{theorem}[$\ker(D_6)$ Basis]
\label{thm:particle_D6_kernel_basis}
\lean{FUST.ParticleSpectrum.D6_kernel_basis}
\leanok
$\ker(D_6) = \{1, x, x^2\}$: all three basis elements are annihilated by $D_6$.
\end{theorem}

\begin{theorem}[$D_6$ Ceiling]
\label{thm:particle_D6_ceiling}
\lean{FUST.ParticleSpectrum.D6_ceiling}
\leanok
$x^3 \notin \ker(D_6)$: $D_6$ detects cubic ($D_6[t^3](x) \neq 0$ for all $x \neq 0$).
Additionally, $D_7$ projects to $D_6$ via Fibonacci recurrence.
\end{theorem}

\section{D-Operator Detection Boundaries}
\label{sec:particle_detection}

The $\ker(D_6)$ basis $\{1, x, x^2\}$ is separated by the responses of lower $D$-operators:
\begin{itemize}
\item $1 \in \ker(D_3) \cap \ker(D_5) \cap \ker(D_6)$
\item $x \in \ker(D_6) \setminus \ker(D_3)$: $D_3$ detects linear
\item $x^2 \in \ker(D_6) \setminus \ker(D_5)$: $D_5$ detects quadratic
\end{itemize}
Mass exponents derive from $D$-operator pair counts:
$x^0 \to D_4$: $T(4)+1 = 11$, \quad $x^1 \to D_3$: $T(3) = 6$, \quad $x^2 \to D_3$: $T(3) = 6$.

\section{Charge Constraint}
\label{sec:particle_charges}

\begin{theorem}[Allowed Charges]
\label{thm:particle_allowed_charges}
\lean{FUST.ParticleSpectrum.allowedChargeCount_eq}
\leanok
Allowed charge count from $D_3$ structure:
\[
2 \times C(3,2) + 1 = 7
\]
Allowed charges: $Q = n/3$ for $n \in \{-3, -2, -1, 0, 1, 2, 3\}$.
\end{theorem}

\begin{theorem}[Exotic Charges Forbidden]
\label{thm:particle_exotic_forbidden}
\lean{FUST.ParticleSpectrum.charge_one_fifth_forbidden}
\leanok
Charge $Q = 1/5$ is not representable as $n/3$ and is therefore forbidden.
\end{theorem}

\section{Spin Constraint}
\label{sec:particle_spins}

\begin{theorem}[Maximum Spin]
\label{thm:particle_max_spin}
\lean{FUST.ParticleSpectrum.spin_ceiling}
\leanok
Spin count = spacetime dimension = 4. Allowed spins: $0, 1/2, 1, 2$.

Spin $> 2$ would require $D_7$+, which projects to $D_6$.
\end{theorem}

\section{Colored Leptons Forbidden}
\label{sec:particle_colored_leptons}

\begin{theorem}[Colored Leptons Forbidden]
\label{thm:particle_colored_lepton}
\lean{FUST.ParticleSpectrum.colored_lepton_forbidden}
\leanok
Colored lepton (hypercharge $Y = -1$, color = triplet) is forbidden by $D_3$-$D_4$ embedding.
\end{theorem}

\section{SM Particle Count}
\label{sec:particle_count}

\begin{theorem}[Fermion Count]
\label{thm:particle_fermion_count}
\lean{FUST.ParticleSpectrum.SM_fermion_count_eq}
\leanok
Fermion flavor count $= \dim\ker(D_6) = 3$.
\begin{align}
\text{Leptons} &= 2 \times 3 = 6 \\
\text{Quarks} &= 2 \times 3 \times C(3,2) = 18 \\
\text{Total fermions} &= 24
\end{align}
\end{theorem}

\begin{theorem}[Boson Count]
\label{thm:particle_boson_count}
\lean{FUST.ParticleSpectrum.SM_boson_count_eq}
\leanok
\begin{align}
\text{Gluons} &= C(3,2)^2 - 1 = 8 \\
\text{Weak bosons (W}^\pm\text{, Z)} &= C(3,2) = 3 \\
\text{Photon} &= 1 \\
\text{Higgs} &= 1 \\
\text{Total bosons} &= 13
\end{align}
\end{theorem}

\begin{theorem}[Total SM Particles]
\label{thm:particle_sm_count}
\lean{FUST.ParticleSpectrum.SM_count}
\leanok
\[
\text{SM particles} = 24 + 13 = 37
\]
Including graviton: 38.
\end{theorem}

\section{Predicted Particles}
\label{sec:particle_predictions}

\subsection{Graviton at \texorpdfstring{$D_6$}{D6}}

The graviton is structurally predicted from $D_6$, not postulated.

\begin{theorem}[Graviton Structural Derivation]
\label{thm:graviton_structural}
\lean{FUST.ParticleSpectrum.graviton_spin_derived}
\leanok
The graviton has:
\begin{itemize}
\item $D$-level $= 6$ (highest active level)
\item Spin $= 2$ (from $\mathrm{spatialDim} - \mathrm{timeDim} = 3 - 1$)
\item Massless (from $\Box_\varphi[t^{-1}] = 0$)
\item Gravity sector trace $= \varphi^3 + \psi^3 = 4 = \mathrm{spacetimeDim}$
\item Gravity sector discriminant $= 20 = C(6,3)$
\end{itemize}
See Chapter~\ref{chap:pd_gravitational}, \S\ref{sec:grav_graviton} for the full derivation.
\end{theorem}

\subsection{Neutrino Mass Structure at \texorpdfstring{$D_5$}{D5}}

\begin{theorem}[Neutrino Kernel Structure]
\label{thm:neutrino_kernel}
\lean{FUST.ParticleSpectrum.neutrino_kernel_structure}
\leanok
3 neutrino flavors from $\dim\ker(D_6) = 3$ (SU(2) doublet with charged leptons).
Mass states split by $\ker(D_5) \subset \ker(D_6)$:
\begin{itemize}
\item $\ker(D_5) = \{1, x\}$: solar pair $\nu_1, \nu_2$ (2 nearly-degenerate states)
\item $x^2 \in \ker(D_6) \setminus \ker(D_5)$: atmospheric $\nu_3$ (separated)
\end{itemize}
Right-handed neutrinos predicted at $D_5$ level (spin-1/2).
\end{theorem}

\subsection{\texorpdfstring{$D_{5\frac{1}{2}}$}{D5.5} Dark Matter}

$D_{5\frac{1}{2}}$ is a half-order operator between $D_5$ and $D_6$:
\[
D_{5\frac{1}{2}} f(x) = D_5 f(x) + \mu \cdot (f(\varphi x) - f(\psi x)), \quad \mu = \frac{2}{\varphi + 2}.
\]

Kernel hierarchy:
$\ker(D_{5\frac{1}{2}}) \subset \ker(D_5) \subset \ker(D_6)$, \quad $\dim: 1 \to 2 \to 3$.
\begin{itemize}
\item $\ker(D_{5\frac{1}{2}}) = \{1\}$: only constants (same as $D_3$)
\item $D_{5\frac{1}{2}}[x] \neq 0$: detects linear functions (unlike $D_5$ and $D_6$)
\item Gauge-invariant output: $D_{5\frac{1}{2}}[x](x)/x = 2/\varphi \approx 1.236$
\end{itemize}

$\mathrm{deriveFDim}(D_{5\frac{1}{2}}) = (-4, 0, 1)$:
same denominator structure as $D_5$ ($p = -4$), antisymmetric ($\delta = 0$),
but $\ker\dim = 1$ gives $\tau = 2 - 1 = 1$ (vs.\ $D_5$: $\tau = 0$, $D_6$: $\tau = -1$).

\begin{theorem}[Dark Matter FDim]
\label{thm:dark_matter_fdim}
\lean{FUST.ParticleSpectrum.dimDarkMatter_ne_dimWBoson}
\leanok
\[
\mathrm{dimDarkMatter} = \mathrm{deriveFDim}(D_{5\frac{1}{2}}) \times \mathrm{dimTimeD2}^{25} = (21, -25, -24)
\]
This is distinct from $\mathrm{dimWBoson} = (20, -24, -26)$: the $D_{5\frac{1}{2}}$ sector base
$(-4,0,1)$ differs from the $D_6$ sector base $(-5,1,-1)$.
Mass: $m_{DM}/m_e = \varphi^{25}$, coupling suppression $\varphi^{-3/2}$.
\end{theorem}

\section{Unique FDim Catalog}
\label{sec:particle_fdim}

Each massive particle has a unique FDim determined by its sector base and $\varphi$-scaling index $n$.
Dimensions determine physical properties: no two distinct particles share the same FDim.

\begin{center}
\begin{tabular}{|c|c|c|c|}
\hline
Particle & FDim $(p, \delta, \tau)$ & $n$ & Sector \\
\hline
$e$ & $(-5,1,-1)$ & 0 & $D_6$ \\
$\mu$ & $(6,-10,-12)$ & 11 & $D_6$ \\
$\tau$ & $(12,-16,-18)$ & 17 & $D_6$ \\
$p$ & $(9,-13,-15)$ & 14 & $D_6$ \\
$n$ & $(8,-12,-14)$ & 14 & $D_6 \times D_2$ \\
$W$ & $(20,-24,-26)$ & 25 & $D_6$ \\
DM & $(21,-25,-24)$ & 25 & $D_{5\frac{1}{2}}$ \\
\hline
$\nu_3$ & $(-42,34,30)$ & $-32$ & $D_6^2$ \\
$\nu_2$ & $(-43,35,31)$ & $-32$ & $D_6^2 \times D_2$ \\
$Z^2$ comp1 & $(40,-48,-52)$ & 50 & $D_6^2$ \\
$Z^2$ comp2 & $(42,-46,-52)$ & --- & Mixed \\
$H$ comp1 & $(21,-25,-27)$ & 26 & $D_6$ \\
$H$ comp2 & $(18,-22,-24)$ & 23 & $D_6$ \\
\hline
\end{tabular}
\end{center}

\subsection{Massless Particles and FDim}

Mass $= 0$ means $f \in \ker(D_6)$, but this does \emph{not} imply FDim $= (0,0,0)$.
Each $\ker(D_6)$ basis element $x^d$ has $\mathrm{FDim} = \mathrm{dimTimeD2}^d$,
and the $D$-operator detection matrix separates them:

\begin{center}
\begin{tabular}{|c|c|c|c|c|}
\hline
Basis & FDim & $D_3$ & $D_{5\frac{1}{2}}$ & $D_5$ \\
\hline
$1$ ($d=0$) & $(0,0,0)$ & $= 0$ & $= 0$ & $= 0$ \\
$x$ ($d=1$) & $(1,-1,-1)$ & $\neq 0$ & $\neq 0$ & $= 0$ \\
$x^2$ ($d=2$) & $(2,-2,-2)$ & $\neq 0$ & $\neq 0$ & $\neq 0$ \\
\hline
\end{tabular}
\end{center}

This 3-layer structure within $\ker(D_6)$ distinguishes massless particle types:
\begin{itemize}
\item Layer 1 ($\ker(D_{5\frac{1}{2}})$): vacuum, truly gauge-invisible
\item Layer 2 ($\ker(D_5) \setminus \ker(D_{5\frac{1}{2}})$): $D_{5\frac{1}{2}}$ detects $x$;
  photon/gluon transitions
\item Layer 3 ($\ker(D_6) \setminus \ker(D_5)$): $D_5$ detects $x^2$;
  Higgs DOF, W/Z mass mechanism
\end{itemize}

\subsection{Sector Classification}

\begin{itemize}
\item \textbf{$D_6$ sector}: base $= \mathrm{deriveFDim}(6) = (-5,1,-1)$.
  Electron, muon, tau, proton, W, Higgs components.
\item \textbf{$D_{5\frac{1}{2}}$ sector}: base $= \mathrm{deriveFDim}(D_{5\frac{1}{2}}) = (-4,0,1)$.
  Dark matter. Same $\varphi$-exponent as W ($n = 25$) but distinct FDim.
\item \textbf{$D_6^2$ sector}: base $= \mathrm{deriveFDim}(6)^2 = (-10,2,-2)$.
  Neutrinos, $Z^2$ comp1.
\item \textbf{Mixed sector}: $Z^2$ comp2.
  Involves $D_2$ and $D_3$ corrections.
\item \textbf{Ratio sector}: pure $\mathrm{dimTimeD2}^n$.
  Dimensionless mass ratios $m_s/m_d$, coupling constants.
\end{itemize}

\subsection{Dimension Uniqueness}

\begin{theorem}[Particle Dimensions All Distinct]
\label{thm:particle_dims_distinct}
\lean{FUST.ParticleSpectrum.particleDims_all_distinct}
\leanok
All particle FDim values are pairwise distinct.
This follows from the injectivity of the map $(a, n) \mapsto \mathrm{deriveFDim}(6)^a \times \mathrm{dimTimeD2}^n$.
\end{theorem}

\section{Generation Structure}
\label{sec:generation_structure}

The number of fermion generations is not a free parameter but is determined
by the minimal polynomial of $\varphi$.

\subsection{Why 3 Generations}

\begin{theorem}[Three Annihilation Conditions]
\label{thm:three_annihilation}
\lean{FUST.Dim.three_annihilation_conditions}
\leanok
$\ker(D_6) = \{1, x, x^2\}$ has dimension 3 because $D_6$'s coefficients
$(1, -3, 1, -1, 3, -1)$ satisfy exactly 3 polynomial identities:
\begin{align}
d=0: &\quad 1 - 3 + 1 - 1 + 3 - 1 = 0 & \text{(coefficient sum)} \\
d=1: &\quad \sum c_i s_i = 0 & \text{(uses } \varphi + \psi = 1\text{)} \\
d=2: &\quad \sum c_i s_i^2 = 0 & \text{(uses } \varphi^2 + \psi^2 = 3, \; \varphi\psi = -1\text{)}
\end{align}
$d=3$ fails: $D_6[t^3](x) \neq 0$ for all $x \neq 0$.

The golden ratio satisfies $x^2 - x - 1 = 0$ (degree 2), providing
2 symmetric-function identities $(\sigma_1 = 1, \sigma_2 = -1)$
plus the coefficient-sum constraint, giving exactly 3 annihilation conditions:
\[
\boxed{3 \text{ generations} = \deg(x^2 - x - 1) + 1 = 2 + 1 = 3}
\]
\end{theorem}

\subsection{The \texorpdfstring{$D_4$}{D4} Anomaly}

\begin{theorem}[$D_4$ Anomaly]
\label{thm:D4_anomaly}
\lean{FUST.Dim.D4_anomaly}
\leanok
$D_4$ has non-contiguous kernel: $\ker(D_4) = \{x^2\}$, not $\{1\}$.
\begin{itemize}
\item $D_4[1](x) \neq 0$: $D_4$ detects constants
\item $D_4[x^2](x) = 0$: $D_4$ annihilates quadratics
\end{itemize}
The cancellation at $d=2$ arises because $D_4[x^2] = ((\varphi^2)^2 - \varphi^2 \cdot \varphi^2 + \psi^2 \cdot \psi^2 - (\psi^2)^2)x^2 = (\varphi^4 - \varphi^4 + \psi^4 - \psi^4)x^2 = 0$.
\end{theorem}

\begin{theorem}[$D_4$ Unique Constant Detector]
\label{thm:D4_unique}
\lean{FUST.Dim.D4_unique_constant_detector}
\leanok
$D_4$ is the only operator that detects constants within $\ker(D_6)$:
\[
D_2[1] = 0, \quad D_3[1] = 0, \quad D_4[1] \neq 0, \quad D_5[1] = 0, \quad D_6[1] = 0.
\]
\end{theorem}

\subsection{Detection Matrix}

The following table shows which $D_m$ detects which $x^d$ within $\ker(D_6)$:

\begin{center}
\begin{tabular}{|c|c|c|c|c|c|}
\hline
$d$ & $D_2$ & $D_3$ & $D_4$ & $D_5$ & $D_6$ \\
\hline
0 & $= 0$ & $= 0$ & $\neq 0$ & $= 0$ & $= 0$ \\
1 & $\neq 0$ & $\neq 0$ & $\neq 0$ & $= 0$ & $= 0$ \\
2 & $\neq 0$ & $\neq 0$ & $= 0$ & $\neq 0$ & $= 0$ \\
\hline
\end{tabular}
\end{center}

\begin{theorem}[Detection Matrix]
\label{thm:detection_matrix}
\lean{FUST.Dim.detection_d0}
\lean{FUST.Dim.detection_d1}
\lean{FUST.Dim.detection_d2}
\leanok
Each row is formally verified:
\begin{itemize}
\item $d=0$: only $D_4$ detects (the $D_4$ anomaly)
\item $d=1$: $D_2, D_3, D_4$ detect; $D_5, D_6$ annihilate
\item $d=2$: $D_2, D_3, D_5$ detect; $D_4, D_6$ annihilate
\end{itemize}
\end{theorem}

\subsection{Kernel Filtration}

\begin{theorem}[Kernel Filtration]
\label{thm:kernel_filtration}
\lean{FUST.Dim.kernel_filtration}
\leanok
The 3-layer kernel filtration creates 3 generations:
\[
\ker(D_{5\frac{1}{2}}) \subset \ker(D_5) \subset \ker(D_6), \quad \dim: 1 \to 2 \to 3.
\]
Each dimension increase corresponds to one fermion generation.
\end{theorem}

\subsection{Generation Exponents from Pair Counts}

All mass exponents are sums of pair counts $C(m,2)$ of the 5 active $D$-operators.

\begin{theorem}[Five Pair Counts]
\label{thm:five_pair_counts}
\lean{FUST.Dim.five_pair_counts}
\leanok
\[
C(2,2) = 1, \quad C(3,2) = 3, \quad C(4,2) = 6, \quad C(5,2) = 10, \quad C(6,2) = 15.
\]
These are the first 5 triangular numbers $T(1), \ldots, T(5)$.
\end{theorem}

\begin{theorem}[Lepton Generation Exponents]
\label{thm:lepton_exponents}
\lean{FUST.Dim.lepton_generation_exponents}
\leanok
\begin{align}
m_\mu / m_e &= \varphi^{11}, & 11 &= C(5,2) + C(2,2) \\
m_\tau / m_\mu &= \varphi^{6}, & 6 &= C(4,2) \\
m_\tau / m_e &= \varphi^{17}, & 17 &= C(5,2) + C(2,2) + C(4,2)
\end{align}
\begin{itemize}
\item $e \to \mu$: $C(5,2) + C(2,2) = 11$ from $D_5$ kernel expansion ($\ker\dim: 1 \to 2$)
  plus $D_2$ boundary.
\item $\mu \to \tau$: $C(4,2) = 6$ from $D_4$'s anomalous kernel (the only operator
  detecting constants but annihilating $x^2$).
\end{itemize}
\end{theorem}

\begin{theorem}[W Boson Exponent]
\label{thm:wboson_exponent}
\lean{FUST.Dim.wBoson_exponent}
\leanok
\[
m_W / m_e \propto \varphi^{25}, \quad 25 = C(5,2) + C(6,2) = 10 + 15.
\]
The W boson exponent is the total pair count of $D_5 + D_6$.
\end{theorem}

\begin{theorem}[Generation Structure Summary]
\label{thm:generation_summary}
\lean{FUST.Dim.generation_structure}
\leanok
The complete generation structure follows from 3 algebraic facts:
\begin{enumerate}
\item $\varphi^2 = \varphi + 1$ (minimal polynomial degree 2)
  $\Rightarrow$ 3 annihilation conditions $\Rightarrow$ $\dim\ker(D_6) = 3$ $\Rightarrow$ 3 generations.
\item $D_4[x^2] = 0$ but $D_4[1] \neq 0$ (anomalous kernel)
  $\Rightarrow$ $\tau/\mu$ and $s/d$ exponent $C(4,2) = 6$.
\item $\ker(D_5) \subsetneq \ker(D_6)$ (kernel inclusion chain)
  $\Rightarrow$ $\mu/e$ exponent $C(5,2) + C(2,2) = 11$;
  W exponent $C(5,2) + C(6,2) = 25$.
\end{enumerate}
No free parameters. Every mass exponent is built from $\{1, 3, 6, 10, 15\}$.
\end{theorem}

\section{Summary}
\label{sec:particle_summary}

\begin{theorem}[Particle Spectrum Summary]
\label{thm:particle_summary}
\lean{FUST.ParticleSpectrum.particle_spectrum_summary}
\leanok
\begin{enumerate}
\item SM particle count = 37 (derived from D-structure)
\item Fermion flavor count $= \dim\ker(D_6) = 3$
\item $x^3 \notin \ker(D_6)$: $D_6$ ceiling; $D_7$ projects to $D_6$ prevents spin $> 2$
\item Allowed spins = spacetime dimension = 4
\item Allowed charges from $D_3$ structure
\item Graviton: spin-2, massless, $D$-level 6 (structurally derived)
\end{enumerate}
\end{theorem}

\section{Dimensional Types}
\label{sec:particle_dim}

Particle spectrum quantities split into two categories:

\textbf{CountQuantity} ($\mathbb{N}$): derived from kernel dimensions and pair counts.

\begin{center}
\begin{tabular}{|c|c|c|}
\hline
Quantity & Value & Lean \\
\hline
Fermion flavors & $\dim\ker(D_6) = 3$ & \texttt{fermionFlavorCount} \\
SM fermions & $24$ & \texttt{smFermionCount} \\
SM bosons & $13$ & \texttt{smBosonCount} \\
SM total & $37$ & \texttt{smParticleCount} \\
Allowed charges & $2 \times C(3,2) + 1 = 7$ & \texttt{allowedChargeCount} \\
Allowed spins & spacetimeDim $= 4$ & \texttt{allowedSpinCount} \\
\hline
\end{tabular}
\end{center}

\textbf{ScaleQuantity / DimSum2}: dimensioned masses with unique FDim per particle.
See \S\ref{sec:particle_fdim} for the complete catalog.
Mass ratios computed via \texttt{.val} (ScaleQ) or \texttt{.eval} (DimSum2) are $\mathbb{R}$-valued
and invariant under FDim relabeling.
