% FUST Physics Discoveries - Particle Spectrum
\chapter{Particle Spectrum from \texorpdfstring{$D_\zeta$}{Dζ} Structure}
\label{chap:pd_particle}

\section{\texorpdfstring{$F_\zeta$}{Fζ} Channel Structure}
\label{sec:particle_channel_structure}

The $F_\zeta$ normSq decomposition $|D_\zeta|^2 = 12(3a^2 + b^2)$ yields two channel weights:
\begin{itemize}
\item SY channel: weight 3 (carries SU(3) color, rank 3)
\item AF channel: weight 1 (carries SU(2) weak, rank 1)
\item Total weight = $3 + 1 = 4$ = spacetimeDim
\end{itemize}

The SY channel weight $= 3$ determines the fermion generation count.

Mass exponents derive from pair counts $C(m,2)$:
$e \to \mu$: $C(5,2)+C(2,2) = 11$, \quad $\mu \to \tau$: $C(4,2) = 6$, \quad $e \to \tau$: $11+6 = 17$.

\section{Charge Constraint}
\label{sec:particle_charges}

\begin{theorem}[Allowed Charges]
\label{thm:particle_allowed_charges}
\lean{FUST.ParticleSpectrum.allowedChargeCount_eq}
\leanok
Allowed charge count from $\mathrm{Diff}_3$ structure:
\[
2 \times C(3,2) + 1 = 7
\]
Allowed charges: $Q = n/3$ for $n \in \{-3, -2, -1, 0, 1, 2, 3\}$.
\end{theorem}

\begin{theorem}[Exotic Charges Forbidden]
\label{thm:particle_exotic_forbidden}
\lean{FUST.ParticleSpectrum.charge_one_fifth_forbidden}
\leanok
Charge $Q = 1/5$ is not representable as $n/3$ and is therefore forbidden.
\end{theorem}

\section{Spin Constraint}
\label{sec:particle_spins}

\begin{theorem}[Maximum Spin]
\label{thm:particle_max_spin}
\lean{FUST.ParticleSpectrum.max_spin}
\leanok
Spin count = totalWeight = 4. Allowed spins: $0, 1/2, 1, 2$.
Maximum spin $= 2$ from the allowed spin list.
\end{theorem}

\section{Colored Leptons Forbidden}
\label{sec:particle_colored_leptons}

\begin{theorem}[Colored Leptons Forbidden]
\label{thm:particle_colored_lepton}
\lean{FUST.ParticleSpectrum.colored_lepton_forbidden}
\leanok
Colored lepton (hypercharge $Y = -1$, color = triplet) is forbidden by $\mathrm{Diff}_3$-$\mathrm{Diff}_4$ embedding.
\end{theorem}

\section{SM Particle Count}
\label{sec:particle_count}

\begin{theorem}[Fermion Count]
\label{thm:particle_fermion_count}
\lean{FUST.ParticleSpectrum.SM_fermion_count_eq}
\leanok
Fermion flavor count $= \mathrm{syWeight} = 3$ (from $|D_\zeta|^2 = 12(3a^2 + b^2)$).
\begin{align}
\text{Leptons} &= 2 \times 3 = 6 \\
\text{Quarks} &= 2 \times 3 \times C(3,2) = 18 \\
\text{Total fermions} &= 24
\end{align}
\end{theorem}

\begin{theorem}[Boson Count]
\label{thm:particle_boson_count}
\lean{FUST.ParticleSpectrum.SM_boson_count_eq}
\leanok
\begin{align}
\text{Gluons} &= C(3,2)^2 - 1 = 8 \\
\text{Weak bosons (W}^\pm\text{, Z)} &= C(3,2) = 3 \\
\text{Photon} &= 1 \\
\text{Higgs} &= 1 \\
\text{Total bosons} &= 13
\end{align}
\end{theorem}

\begin{theorem}[Total SM Particles]
\label{thm:particle_sm_count}
\lean{FUST.ParticleSpectrum.SM_count}
\leanok
\[
\text{SM particles} = 24 + 13 = 37
\]
Including graviton: 38.
\end{theorem}

\section{Predicted Particles}
\label{sec:particle_predictions}

\subsection{Graviton at \texorpdfstring{$F_\zeta$}{Fζ}}

The graviton is structurally predicted from $F_\zeta$, not postulated.

\begin{theorem}[Graviton Structural Derivation]
\label{thm:graviton_structural}
\lean{FUST.ParticleSpectrum.graviton_spin_derived}
\leanok
The graviton has:
\begin{itemize}
\item Level $= 6$ (highest active level)
\item Spin $= 2$ (from $\mathrm{spatialDim} - \mathrm{timeDim} = 3 - 1$)
\item Massless (from $\Box_\varphi[t^{-1}] = 0$)
\item Gravity sector trace $= \varphi^3 + \psi^3 = 4 = \mathrm{spacetimeDim}$
\item Gravity sector discriminant $= 20 = C(6,3)$
\end{itemize}
See Chapter~\ref{chap:pd_gravitational}, \S\ref{sec:grav_graviton} for the full derivation.
\end{theorem}

\subsection{Neutrino Mass Structure at \texorpdfstring{$\mathrm{Diff}_5$}{Diff5}}

\begin{theorem}[Neutrino Flavor Count]
\label{thm:neutrino_kernel}
3 neutrino flavors from $\mathrm{syWeight} = 3$ (SU(2) doublet with charged leptons).
The SY channel weight determines the number of neutrino mass eigenstates.
Right-handed neutrinos predicted at $\mathrm{Diff}_5$ level (spin-1/2).
\end{theorem}

\subsection{\texorpdfstring{$\Phi_S$}{Φ_S} Dark Matter}

$\Phi_S$ is the SY channel operator between $\mathrm{Diff}_5$ and $F_\zeta$:
\[
\Phi_S f(x) = \mathrm{Diff}_5 f(x) + \mu \cdot (f(\varphi x) - f(\psi x)), \quad \mu = \frac{2}{\varphi + 2}.
\]

Kernel hierarchy:
$\ker(\Phi_S) \subset \ker(\mathrm{Diff}_5) \subset \ker(F_\zeta)$, \quad $\dim: 1 \to 2 \to 3$.
\begin{itemize}
\item $\ker(\Phi_S) = \{1\}$: only constants (same as $\mathrm{Diff}_3$)
\item $\Phi_S[x] \neq 0$: detects linear functions (unlike $\mathrm{Diff}_5$ and $F_\zeta$)
\item Gauge-invariant output: $\Phi_S[x](x)/x = 2/\varphi \approx 1.236$
\end{itemize}

$\mathrm{deriveFDim}(\Phi_S) = (-4, 0, 1)$:
same denominator structure as $\mathrm{Diff}_5$ ($p = -4$), antisymmetric ($\delta = 0$),
but $\ker\dim = 1$ gives $\tau = 2 - 1 = 1$ (vs.\ $\mathrm{Diff}_5$: $\tau = 0$, $F_\zeta$: $\tau = -1$).

\begin{theorem}[Dark Matter FDim]
\label{thm:dark_matter_fdim}
\lean{FUST.ParticleSpectrum.dimDarkMatter_ne_dimWBoson}
\leanok
\[
\mathrm{dimDarkMatter} = \mathrm{deriveFDim}(\Phi_S) \times \mathrm{dimTimeD2}^{25} = (21, -25, -24)
\]
This is distinct from $\mathrm{dimWBoson} = (20, -24, -26)$: the $\Phi_S$ sector base
$(-4,0,1)$ differs from the $F_\zeta$ sector base $(-5,1,-1)$.
Mass: $m_{DM}/m_e = \varphi^{25}$, coupling suppression $\varphi^{-3/2}$.
\end{theorem}

\section{Unique FDim Catalog}
\label{sec:particle_fdim}

Each massive particle has a unique FDim determined by its sector base and $\varphi$-scaling index $n$.
Dimensions determine physical properties: no two distinct particles share the same FDim.

\begin{center}
\begin{tabular}{|c|c|c|c|}
\hline
Particle & FDim $(p, \delta, \tau)$ & $n$ & Sector \\
\hline
$e$ & $(-5,1,-1)$ & 0 & $F_\zeta$ \\
$\mu$ & $(6,-10,-12)$ & 11 & $F_\zeta$ \\
$\tau$ & $(12,-16,-18)$ & 17 & $F_\zeta$ \\
$p$ & $(9,-13,-15)$ & 14 & $F_\zeta$ \\
$n$ & $(8,-12,-14)$ & 14 & $F_\zeta \times \mathrm{Diff}_2$ \\
$W$ & $(20,-24,-26)$ & 25 & $F_\zeta$ \\
DM & $(21,-25,-24)$ & 25 & $\Phi_S$ \\
\hline
$\nu_3$ & $(-42,34,30)$ & $-32$ & $F_\zeta^2$ \\
$\nu_2$ & $(-43,35,31)$ & $-32$ & $F_\zeta^2 \times \mathrm{Diff}_2$ \\
$Z^2$ comp1 & $(40,-48,-52)$ & 50 & $F_\zeta^2$ \\
$Z^2$ comp2 & $(42,-46,-52)$ & --- & Mixed \\
$H$ comp1 & $(21,-25,-27)$ & 26 & $F_\zeta$ \\
$H$ comp2 & $(18,-22,-24)$ & 23 & $F_\zeta$ \\
\hline
\end{tabular}
\end{center}

\subsection{Massless Particles and FDim}

Mass $= 0$ means $f \in \ker(F_\zeta)$, but this does \emph{not} imply FDim $= (0,0,0)$.
Each $\ker(F_\zeta)$ basis element $x^d$ has $\mathrm{FDim} = \mathrm{dimTimeD2}^d$,
and the numerator operator detection matrix separates them:

\begin{center}
\begin{tabular}{|c|c|c|c|c|}
\hline
Basis & FDim & $\mathrm{Diff}_3$ & $\Phi_S$ & $\mathrm{Diff}_5$ \\
\hline
$1$ ($d=0$) & $(0,0,0)$ & $= 0$ & $= 0$ & $= 0$ \\
$x$ ($d=1$) & $(1,-1,-1)$ & $\neq 0$ & $\neq 0$ & $= 0$ \\
$x^2$ ($d=2$) & $(2,-2,-2)$ & $\neq 0$ & $\neq 0$ & $\neq 0$ \\
\hline
\end{tabular}
\end{center}

This 3-layer structure within $\ker(F_\zeta)$ distinguishes massless particle types:
\begin{itemize}
\item Layer 1 ($\ker(\Phi_S)$): vacuum, truly gauge-invisible
\item Layer 2 ($\ker(\mathrm{Diff}_5) \setminus \ker(\Phi_S)$): $\Phi_S$ detects $x$;
  photon/gluon transitions
\item Layer 3 ($\ker(F_\zeta) \setminus \ker(\mathrm{Diff}_5)$): $\mathrm{Diff}_5$ detects $x^2$;
  Higgs DOF, W/Z mass mechanism
\end{itemize}

\subsection{Sector Classification}

\begin{itemize}
\item \textbf{$F_\zeta$ sector}: base $= \mathrm{deriveFDim}(6) = (-5,1,-1)$.
  Electron, muon, tau, proton, W, Higgs components.
\item \textbf{$\Phi_S$ sector}: base $= \mathrm{deriveFDim}(\Phi_S) = (-4,0,1)$.
  Dark matter. Same $\varphi$-exponent as W ($n = 25$) but distinct FDim.
\item \textbf{$F_\zeta^2$ sector}: base $= \mathrm{deriveFDim}(6)^2 = (-10,2,-2)$.
  Neutrinos, $Z^2$ comp1.
\item \textbf{Mixed sector}: $Z^2$ comp2.
  Involves $\mathrm{Diff}_2$ and $\mathrm{Diff}_3$ corrections.
\item \textbf{Ratio sector}: pure $\mathrm{dimTimeD2}^n$.
  Dimensionless mass ratios $m_s/m_d$, coupling constants.
\end{itemize}

\subsection{Dimension Uniqueness}

\begin{theorem}[Particle Dimensions All Distinct]
\label{thm:particle_dims_distinct}
\lean{FUST.ParticleSpectrum.particleDims_all_distinct}
\leanok
All particle FDim values are pairwise distinct.
This follows from the injectivity of the map $(a, n) \mapsto \mathrm{deriveFDim}(6)^a \times \mathrm{dimTimeD2}^n$.
\end{theorem}

\section{Generation Structure}
\label{sec:generation_structure}

The number of fermion generations is not a free parameter but equals
the $F_\zeta$ SY channel weight $= 3$, which in turn is determined
by the minimal polynomial of $\varphi$.

\subsection{Why 3 Generations}

\begin{theorem}[Three Generations from $F_\zeta$ SY Channel]
\label{thm:three_annihilation}
The number of fermion generations equals the SY channel weight from
$|D_\zeta|^2 = 12(\underbrace{3}_{\mathrm{syWeight}} a^2 + b^2)$.

The golden ratio satisfies $x^2 - x - 1 = 0$ (degree 2), providing
2 symmetric-function identities $(\sigma_1 = 1, \sigma_2 = -1)$
plus the coefficient-sum constraint, giving exactly 3 annihilation conditions:
\[
\boxed{3 \text{ generations} = \mathrm{syWeight} = \deg(x^2 - x - 1) + 1 = 2 + 1 = 3}
\]
\end{theorem}

\subsection{Pair Count Origin}

The generation exponents arise from pair counts $C(m,2)$ of the active operators.
The $\mathrm{Diff}_4$ anomaly ($\mathrm{Diff}_4[1] \neq 0$ but $\mathrm{Diff}_4[x^2] = 0$) explains why the
$\tau/\mu$ exponent equals $C(4,2) = 6$: the non-contiguous kernel of $\mathrm{Diff}_4$
forces a specific pair-count contribution to the mass hierarchy.

\subsection{Generation Exponents from Pair Counts}

All mass exponents are sums of pair counts $C(m,2)$ of the 5 active numerator operators.

\begin{theorem}[Five Pair Counts]
\label{thm:five_pair_counts}
\lean{FUST.Dim.five_pair_counts}
\leanok
\[
C(2,2) = 1, \quad C(3,2) = 3, \quad C(4,2) = 6, \quad C(5,2) = 10, \quad C(6,2) = 15.
\]
These are the pair counts $C(m,2)$ of the 5 active numerator operators $\mathrm{Diff}_2, \ldots, \mathrm{Diff}_6$.
\end{theorem}

\begin{theorem}[Lepton Generation Exponents]
\label{thm:lepton_exponents}
\lean{FUST.Dim.lepton_generation_exponents}
\leanok
\begin{align}
m_\mu / m_e &= \varphi^{11}, & 11 &= C(5,2) + C(2,2) \\
m_\tau / m_\mu &= \varphi^{6}, & 6 &= C(4,2) \\
m_\tau / m_e &= \varphi^{17}, & 17 &= C(5,2) + C(2,2) + C(4,2)
\end{align}
\begin{itemize}
\item $e \to \mu$: $C(5,2) + C(2,2) = 11$ from $\mathrm{Diff}_5$ kernel expansion ($\ker\dim: 1 \to 2$)
  plus $\mathrm{Diff}_2$ boundary.
\item $\mu \to \tau$: $C(4,2) = 6$ from $\mathrm{Diff}_4$'s anomalous kernel (the only operator
  detecting constants but annihilating $x^2$).
\end{itemize}
\end{theorem}

\begin{theorem}[W Boson Exponent]
\label{thm:wboson_exponent}
\lean{FUST.Dim.wBoson_exponent}
\leanok
\[
m_W / m_e \propto \varphi^{25}, \quad 25 = C(5,2) + C(6,2) = 10 + 15.
\]
The W boson exponent is the total pair count of $\mathrm{Diff}_5 + \mathrm{Diff}_6$.
\end{theorem}

\begin{theorem}[Generation Structure Summary]
\label{thm:generation_summary}
\lean{FUST.Dim.generation_structure}
\leanok
The complete generation structure follows from pair count sums:
\begin{enumerate}
\item $\mathrm{syWeight} = 3$ from $|D_\zeta|^2 = 12(3a^2 + b^2)$ $\Rightarrow$ 3 generations.
\item Lepton exponents: $C(5,2) + C(2,2) = 11$, $C(4,2) = 6$.
\item Proton exponent: $C(5,2) + C(3,2) + C(2,2) = 14$.
\item W exponent: $C(5,2) + C(6,2) = 25$.
\end{enumerate}
No free parameters. Every mass exponent is built from $\{1, 3, 6, 10, 15\}$.
\end{theorem}

\section{Summary}
\label{sec:particle_summary}

\begin{theorem}[Particle Spectrum Summary]
\label{thm:particle_summary}
\lean{FUST.ParticleSpectrum.particle_spectrum_summary}
\leanok
\begin{enumerate}
\item SM particle count = 37 (derived from $F_\zeta$ structure)
\item Fermion flavor count $= \mathrm{syWeight} = 3$
\item Allowed spins = totalWeight = 4
\item Allowed charges from pair count structure
\item Graviton: spin-2, massless, level 6 (structurally derived)
\end{enumerate}
\end{theorem}

\section{Dimensional Types}
\label{sec:particle_dim}

Particle spectrum quantities split into two categories:

\textbf{CountQuantity} ($\mathbb{N}$): derived from $F_\zeta$ channel weights and pair counts.

\begin{center}
\begin{tabular}{|c|c|c|}
\hline
Quantity & Value & Lean \\
\hline
Fermion flavors & $\mathrm{syWeight} = 3$ & \texttt{fermionFlavorCount} \\
SM fermions & $24$ & \texttt{smFermionCount} \\
SM bosons & $13$ & \texttt{smBosonCount} \\
SM total & $37$ & \texttt{smParticleCount} \\
Allowed charges & $2 \times C(3,2) + 1 = 7$ & \texttt{allowedChargeCount} \\
Allowed spins & totalWeight $= 4$ & \texttt{allowedSpinCount} \\
\hline
\end{tabular}
\end{center}

\textbf{ScaleQuantity / DimSum2}: dimensioned masses with unique FDim per particle.
See \S\ref{sec:particle_fdim} for the complete catalog.
Mass ratios computed via \texttt{.val} (ScaleQ) or \texttt{.eval} (DimSum2) are $\mathbb{R}$-valued
and invariant under FDim relabeling.
