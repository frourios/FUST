% FUST Physics Discoveries - Mass Ratios
\chapter{Mass Ratio Predictions from \texorpdfstring{$D_\zeta$}{Dζ}}
\label{chap:pd_mass_ratios}

\section{Quark Mass Ratios}
\label{sec:mass_quark}

\begin{theorem}[Up/Down Ratio]
\label{thm:mass_mu_md}
\lean{FUST.QuarkMassRatios.mu_md_from_D2}
\leanok
\[
\frac{m_u}{m_d} = \frac{1}{2}
\]
from $D_2$ isospin structure. Experimental: 0.47 (error 6.4\%).
\end{theorem}

\begin{theorem}[Strange/Down Ratio]
\label{thm:mass_ms_md}
\lean{FUST.QuarkMassRatios.ms_md_structural_eq}
\leanok
\[
\frac{m_s}{m_d} = \varphi^{T(3)} = \varphi^6 \approx 17.94
\]
where $T(3) = C(4,2) = 6$. Experimental: 19.5 (error $\sim 8\%$).
\end{theorem}

\begin{theorem}[Charm/Strange Ratio]
\label{thm:mass_mc_ms}
\lean{FUST.QuarkMassRatios.mc_ms_value}
\leanok
\[
\frac{m_c}{m_s} = C(5,2) + 2 = 10 + 2 = 12
\]
Experimental: 11.7 (error 2.6\%).
\end{theorem}

\begin{theorem}[Bottom/Charm Ratio]
\label{thm:mass_mb_mc}
\lean{FUST.QuarkMassRatios.mb_mc_value}
\leanok
\[
\frac{m_b}{m_c} = C(3,2) = 3
\]
Experimental: 3.0 (error 0\%, exact match).
\end{theorem}

\begin{theorem}[Top/Bottom Ratio]
\label{thm:mass_mt_mb}
\lean{FUST.QuarkMassRatios.mt_mb_structural_eq}
\leanok
\[
\frac{m_t}{m_b} = \varphi^7 + \varphi^5 \approx 40.12
\]
where $7 = C(4,2) + C(2,2)$ and $5 = C(4,2) - C(2,2)$. Experimental: 40.8 (error 1.7\%).
\end{theorem}

\section{Lepton Mass Ratios from D-Operator Pair Counts}
\label{sec:mass_lepton}

Each $\ker(D_6)$ basis element $t^n$ maps to a $D$-operator index $k$, with mass exponent $T(k) + \delta$:
\begin{center}
\begin{tabular}{|c|c|c|c|c|}
\hline
$\ker(D_6)$ basis & Lepton & $D$-index & Exponent & Reason \\
\hline
$x^0 = 1$ & $e$ (lightest) & $D_4$ & $T(4)+1=11$ & $1 \in \ker(D_3)$, descent $D_4 \to D_3$ \\
$x^1$ & $\mu$ (middle) & $D_3$ & $T(3)+0=6$ & $D_3(x) \neq 0$ \\
$x^2$ & $\tau$ (heaviest) & $D_3$ & $T(3)+0=6$ & $D_5(x^2) \neq 0$ \\
\hline
\end{tabular}
\end{center}

\begin{theorem}[Tau/Mu Ratio]
\label{thm:mass_tau_mu}
\lean{FUST.MassRatioDerivation.tau_mu_ratio}
\leanok
\[
\frac{m_\tau}{m_\mu} = \varphi^{T(3)+0} = \varphi^6
\]
Both $\tau$ and $\mu$ use $D_3$ pair count; correction $\delta = 0$ for same-level.
\end{theorem}

\begin{theorem}[Mu/Electron Ratio]
\label{thm:mass_mu_e}
\lean{FUST.MassRatioDerivation.mu_e_ratio}
\leanok
\[
\frac{m_\mu}{m_e} = \varphi^{T(4)+1} = \varphi^{11}
\]
Electron uses $D_4$ (since $1 \in \ker(D_3)$, requiring higher pair count); $\delta = 1$ for $D_4 \to D_3$ descent.
\end{theorem}

\begin{theorem}[Tau/Electron Ratio]
\label{thm:mass_tau_e}
\lean{FUST.MassRatioDerivation.tau_e_ratio}
\leanok
\[
\frac{m_\tau}{m_e} = \varphi^{6+11} = \varphi^{17}
\]
\end{theorem}

\section{Coefficient Corrections}
\label{sec:mass_corrections}

\begin{theorem}[$D_6$ Correction Factor]
\label{thm:mass_d6_correction}
\lean{FUST.MassRatioDerivation.D6_correction_6pt}
\leanok
\[
\kappa_6 = \frac{B}{6A} = \frac{C(2,2)}{6 \times C(3,2)} = \frac{1}{18}
\]
\end{theorem}

\begin{theorem}[$D_5$ Correction Factor]
\label{thm:mass_d5_correction}
\lean{FUST.MassRatioDerivation.D5_correction_11pt}
\leanok
\[
\eta_{11} = \frac{|a|}{11|b|} = \frac{1}{44}
\]
\end{theorem}

\section{Dimensioned Particle Masses}
\label{sec:mass_dimensioned}

Each particle has a unique FDim $= \mathrm{deriveFDim}(6)^a \times \mathrm{dimTimeD2}^n$.
The electron baseline $(-5, 1, -1)$ corresponds to $a=1, n=0$;
other particles acquire distinct FDim via the $\varphi$-scaling index $n$.
All $\mathbb{R}$-valued mass ratios are invariant under FDim relabeling.

\begin{theorem}[Electron Mass]
\label{thm:mass_electron}
\lean{FUST.Dim.electronMass_eq_massGap}
\leanok
\[
m_e = \lambda_{\min} = \frac{C_3}{(\sqrt{5})^5} = \frac{12}{25} \quad [\mathrm{dimElectron} = (-5,1,-1)]
\]
The electron, as the lightest \emph{charged} fermion, is identified with the $D_6$ minimum eigenvalue. Neutrinos acquire mass via $\ker(D_5) \subset \ker(D_6)$ perturbation (see \S\ref{sec:mass_neutrino}) and are not bounded by $\lambda_{\min}$.
\end{theorem}

\begin{theorem}[Proton Mass]
\label{thm:mass_proton}
\lean{FUST.Dim.protonMass_val}
\leanok
\[
m_p = \Delta \times \varphi^{11} \times \frac{C(6,3) \times C(4,2)}{C(3,2)+C(5,2)} = \frac{12}{25} \cdot \varphi^{11} \cdot \frac{120}{13} \quad [\mathrm{dimProton} = (9,-13,-15)]
\]
\end{theorem}

\begin{theorem}[Proton/Electron Ratio]
\label{thm:mass_pe}
\lean{FUST.Dim.protonElectronRatio_from_masses}
\leanok
\[
\frac{m_p}{m_e} = \varphi^{11} \times \frac{120}{13} \approx 1836.97
\]
Experimental: 1836.15 (error 0.045\%). Derived as quotient of independent dimensioned masses; $\Delta$ cancels.
\end{theorem}

Structural decomposition of $120/13$:
\begin{itemize}
\item $C(6,3) = 20$: spatial normalization ($C(D_{\max}, \dim\ker D_6)$)
\item $C(4,2) = 6$: baryon pair count from $D_4$ hierarchy
\item $C(3,2)+C(5,2) = 13$: kernel and $D_5$ pair normalization
\end{itemize}

Each particle has a unique FDim determined by its $\varphi$-scaling index:
\begin{center}
\begin{tabular}{|c|c|c|c|}
\hline
Particle & Formula & FDim $(\sqrt{5}, \delta, \tau)$ & Lean \\
\hline
$e$ & $\Delta$ & $(-5,1,-1)$ & \texttt{dimElectron} \\
$\mu$ & $\Delta \cdot \varphi^{11}$ & $(6,-10,-12)$ & \texttt{dimMuon} \\
$\tau$ & $\Delta \cdot \varphi^{17}$ & $(12,-16,-18)$ & \texttt{dimTau} \\
$p$ & $\Delta \cdot \varphi^{11} \cdot 120/13$ & $(9,-13,-15)$ & \texttt{dimProton} \\
$n$ & $m_p \times n_p$ & $(8,-12,-14)$ & \texttt{dimNeutron} \\
$W$ & $\Delta \cdot \varphi^{25} \cdot 15/16$ & $(20,-24,-26)$ & \texttt{dimWBoson} \\
DM & $\Delta \cdot \varphi^{25}$ & $(21,-25,-24)$ & \texttt{dimDarkMatter} \\
$\nu_3$ & $\Delta^2 \cdot \varphi^{-32}$ & $(-42,34,30)$ & \texttt{dimNu3} \\
$\nu_2$ & $m_{\nu_3} \cdot \sqrt{1/30}$ & $(-43,35,31)$ & \texttt{dimNu2} \\
\hline
$Z$ & \texttt{DimSum2} & $(40,-48,-52) \oplus (42,-46,-52)$ & \texttt{zBosonMassSq} \\
$H$ & \texttt{DimSum2} & $(21,-25,-27) \oplus (18,-22,-24)$ & \texttt{higgsMass} \\
\hline
\end{tabular}
\end{center}
$Z$ and $H$ are \texttt{DimSum2}: formal sums of two ScaleQ with different FDim.
The $\mathbb{R}$-evaluation (\texttt{.eval}) recovers the physical mass value.

\section{Gauge Boson Masses}
\label{sec:mass_gauge_bosons}

The W boson mass is derived from the kernel hierarchy $\ker(D_5) \subsetneq \ker(D_6)$.
The key degree of freedom is $x^2 \in \ker(D_6) \setminus \ker(D_5)$: this acts as the Higgs mechanism,
giving W/Z bosons their mass while gluons (in $\ker(D_6)$) and photons (from grading symmetry) remain massless.

\subsection{W Boson / Electron Mass Ratio}

\begin{theorem}[W/Electron Mass Ratio]
\label{thm:mass_w_e}
\lean{FUST.MassRatioPredictions.WElectronRatio_from_kernel}
\leanok
\[
\frac{m_W}{m_e} = \varphi^{C(5,2)+C(6,2)} \times \frac{C(6,2)}{C(6,2)+C(2,2)}
= \varphi^{25} \times \frac{15}{16}
\]
\begin{itemize}
\item Exponent $25 = C(5,2)+C(6,2)$: total pair interactions in $D_5$+$D_6$ sectors
\item Factor $15/16 = C(6,2)/(C(6,2)+C(2,2))$: normalization spanning $D_6$ to $D_2$
\item Kernel condition: $D_5(x^2) \neq 0$ and $D_6(x^2) = 0$ (W couples to the Higgs DOF)
\end{itemize}
Experimental: $m_W/m_e \approx 157\,279$, predicted $\approx 157\,276$ (error $0.002\%$).
\end{theorem}

\begin{theorem}[W Boson Dimensioned Mass]
\label{thm:mass_w_dimensioned}
\lean{FUST.Dim.wBosonMass_val}
\leanok
\[
m_W = \Delta \times \varphi^{25} \times \frac{15}{16} \quad [\mathrm{dimWBoson} = (20,-24,-26)]
\]
\end{theorem}

\subsection{Z Boson Mass}

\begin{theorem}[Z Boson Mass Squared]
\label{thm:mass_z}
\lean{FUST.Dim.zBosonMass_val}
\leanok
$m_Z^2$ is a \texttt{DimSum2} with two components of different FDim:
\[
m_Z^2 = \underbrace{m_W^2}_{\mathrm{dimZSqComp1} = (40,-48,-52)}
+ \underbrace{m_W^2 \times \frac{3}{10}}_{\mathrm{dimZSqComp2} = (42,-46,-52)}
\]
where $3/10 = C(3,2)/C(5,2) = \sin^2\theta_W / \cos^2\theta_W$.
The $\mathbb{R}$-evaluation gives $m_Z^2.\mathrm{eval} = m_W^2 \times 13/10$, recovering $m_Z = m_W / \sqrt{10/13}$.
\end{theorem}

\begin{theorem}[W/Z Ratio]
\label{thm:mass_w_z}
\lean{FUST.MassRatioPredictions.WZRatio_from_kernel_structure}
\leanok
\[
\frac{m_W}{m_Z} = \sqrt{\frac{C(5,2)}{C(3,2)+C(5,2)}} = \sqrt{\frac{10}{13}} \approx 0.877
\]
Experimental: 0.881 (error 0.5\%).
\end{theorem}

\subsection{Higgs Mass}

\begin{theorem}[Higgs/W Ratio]
\label{thm:mass_higgs_w}
\lean{FUST.MassRatioPredictions.higgsWRatio_structure}
\leanok
\[
\frac{m_H}{m_W} = \varphi - \frac{1}{C(5,2)} = \varphi - \frac{1}{10} \approx 1.518
\]
Experimental: 1.559 (error 2.6\%).
\end{theorem}

\begin{theorem}[Higgs Dimensioned Mass]
\label{thm:mass_higgs_dimensioned}
\lean{FUST.Dim.higgsMass_val}
\leanok
$m_H$ is a \texttt{DimSum2} with two components:
\[
m_H = \underbrace{m_W \cdot \varphi}_{\mathrm{dimHiggsVacuum} = (21,-25,-27)}
+ \underbrace{(-m_W / 10)}_{\mathrm{dimHiggsCorrection} = (18,-22,-24)}
\]
The $\mathbb{R}$-evaluation gives $m_H.\mathrm{eval} = m_W \times (\varphi - 1/10)$.
\end{theorem}

\subsection{Gauge Boson Mass Mechanism}

\begin{theorem}[Gauge Boson Mass Hierarchy]
\label{thm:gauge_boson_hierarchy}
\lean{FUST.MassRatioPredictions.gauge_boson_mass_hierarchy}
\leanok
The mass/massless distinction follows from kernel structure:
\begin{center}
\begin{tabular}{|c|c|c|c|c|}
\hline
Particle & Source & Mass mechanism & Mass & FDim \\
\hline
gluon & $\ker(D_6)$ & $D_6 = 0$ & 0 & --- \\
$\gamma$ & grading & all $D_m$ preserve & 0 & --- \\
$W^\pm$ & $\ker(D_5)$ & $D_5(x^2) \neq 0$ & ScaleQ & $(20,-24,-26)$ \\
$Z$ & $\ker(D_5)$ & $m_W / \cos\theta_W$ & DimSum2 & $(40,...) \oplus (42,...)$ \\
$H$ & $\ker(D_6) \setminus \ker(D_5)$ & Higgs DOF & DimSum2 & $(21,...) \oplus (18,...)$ \\
\hline
\end{tabular}
\end{center}
\end{theorem}

\begin{theorem}[Complete Gauge Boson Chain]
\label{thm:gauge_boson_chain}
\lean{FUST.Dim.gauge_boson_chain}
\leanok
\begin{enumerate}
\item $m_W / m_e = \varphi^{25} \times 15/16$
\item $m_Z = m_W / \sqrt{10/13}$
\item $m_H = m_W \times (\varphi - 1/10)$
\end{enumerate}
All gauge boson masses are determined by $\Delta$, $\varphi$, and pair counts $C(m,2)$.
\end{theorem}

\subsection{Numerical Comparison}

Numerical predictions use $\mathbb{R}$-evaluation of each mass (\texttt{.val} or \texttt{.eval}):
\begin{center}
\begin{tabular}{|c|c|c|c|c|}
\hline
Particle & Mass formula & Predicted (GeV) & Experimental (GeV) & Error \\
\hline
$W^\pm$ & $m_e \times \varphi^{25} \times 15/16$ & 80.37 & 80.37 & 0.002\% \\
$Z$ & $m_Z^2.\mathrm{eval} = m_W^2 \times 13/10$ & 91.66 & 91.19 & 0.5\% \\
$H$ & $m_H.\mathrm{eval} = m_W(\varphi - 1/10)$ & 121.99 & 125.25 & 2.6\% \\
gluon & 0 (in $\ker D_6$) & 0 & 0 & exact \\
$\gamma$ & 0 (grading symmetry) & 0 & 0 & exact \\
\hline
\end{tabular}
\end{center}

\section{Beyond SM Mass Ratios}
\label{sec:mass_beyond_sm}

\begin{theorem}[Dark Matter/Electron Ratio]
\label{thm:mass_dm_e}
\lean{FUST.MassRatioPredictions.darkMatterElectronRatio_eq}
\leanok
\[
\frac{m_{DM}}{m_e} = \varphi^{C(5,2)+C(6,2)} = \varphi^{25} \approx 1.68 \times 10^5
\]
(WIMP scale $\sim 100$ GeV). The $\varphi$-exponent 25 is the same as for $m_W/m_e$,
but dark matter belongs to the $D_{5\frac{1}{2}}$ sector:
\[
\mathrm{dimDarkMatter} = \mathrm{deriveFDim}(D_{5\frac{1}{2}}) \times \mathrm{dimTimeD2}^{25} = (21, -25, -24)
\neq (20, -24, -26) = \mathrm{dimWBoson}.
\]
The distinction arises from the sector base: $(-4,0,1)$ for $D_{5\frac{1}{2}}$ vs.\ $(-5,1,-1)$ for $D_6$.
\end{theorem}

\section{Neutrino Mass Hierarchy}
\label{sec:mass_neutrino}

Neutrinos have 3 flavors (from $\dim\ker(D_6) = 3$ via SU(2) doublet with charged leptons)
but their mass structure is governed by $\ker(D_5) \subset \ker(D_6)$:

\begin{itemize}
\item $\ker(D_5) = \{1, x\}$ ($\dim = 2$): solar pair $\nu_1, \nu_2$ (nearly degenerate)
\item $\ker(D_6) \setminus \ker(D_5) = \{x^2\}$ ($\dim = 1$): atmospheric state $\nu_3$ (separated)
\item $D_5(x^2) \neq 0$: $x^2$ is not in $\ker(D_5)$, giving $\nu_3$ its distinct mass
\end{itemize}

\begin{theorem}[Neutrino Mass Squared Ratio]
\label{thm:mass_neutrino}
\lean{FUST.NeutrinoMass.neutrinoMassSqRatio_eq}
\leanok
\[
\frac{\Delta m^2_{21}}{\Delta m^2_{31}} = \frac{1}{\dim\ker(D_5) \times C(6,2)} = \frac{1}{2 \times 15} = \frac{1}{30}
\]
where $\dim\ker(D_5) = 2$ counts the solar pair states and $C(6,2) = 15$ is the $D_6$ pair count.
Experimental: $\sim 1/33$ (error $\sim 10\%$).
\end{theorem}

\begin{theorem}[Neutrino Hierarchy from Kernel Filtration]
\label{thm:mass_neutrino_hierarchy}
\lean{FUST.NeutrinoMass.neutrino_hierarchy_from_kernel}
\leanok
The kernel filtration $\ker(D_2) \subset \ker(D_5) \subset \ker(D_6)$ with $\dim = 1, 2, 3$
determines the neutrino mass hierarchy:
\begin{enumerate}
\item $\dim\ker(D_5) = 2$: two nearly-degenerate states (solar pair)
\item $D_5(x^2) \neq 0$: one separated state (atmospheric)
\item $\text{neutrinoMassSqDenom} = \dim\ker(D_5) \times C(6,2) = 30$
\end{enumerate}
\end{theorem}

Neutrino masses are not bounded by $\lambda_{\min}$: neutrinos acquire mass via $D_5$-level perturbation from $\ker(D_6)$. The $D_6$ minimum eigenvalue $\lambda_{\min}$ applies only to charged fermions that couple directly to $D_6$ output.

\section{Baryon Asymmetry}
\label{sec:mass_baryon}

\begin{theorem}[Baryon Asymmetry Parameter]
\label{thm:mass_baryon}
\lean{FUST.MassRatioPredictions.baryonAsymmetry_structure}
\leanok
\[
\eta = \varphi^{-44} \times \sin\left(\frac{2\pi}{5}\right) \approx 6 \times 10^{-10}
\]
where $44 = T(4) \times 4 + 4$ and $5$ = active D-levels.
Experimental: $\sim 6 \times 10^{-10}$ (error $\sim 1\%$).
\end{theorem}

\section{Summary}
\label{sec:mass_summary}

\begin{theorem}[All Exponents from Pair Counts]
\label{thm:mass_all_exponents}
\lean{FUST.QuarkMassRatios.all_exponents_from_pair_counts}
\leanok
All mass ratio exponents are derived from D-structure pair counts:
\begin{center}
\begin{tabular}{|c|c|c|}
\hline
Ratio & Formula & D-structure origin \\
\hline
$m_s/m_d$ & $\varphi^6$ & $T(3) = C(4,2)$ \\
$m_c/m_s$ & 12 & $C(5,2) + 2$ \\
$m_b/m_c$ & 3 & $C(3,2)$ \\
$m_t/m_b$ & $\varphi^7 + \varphi^5$ & $C(4,2) \pm C(2,2)$ \\
\hline
\end{tabular}
\end{center}
\end{theorem}

\section{Dimensional Types}
\label{sec:mass_dim}

Mass ratios split into two structural types:

\begin{center}
\begin{tabular}{|c|c|c|}
\hline
Ratio & Type & Lean \\
\hline
$m_s/m_d = \varphi^6$ & ScaleQ$(0,0,0)$ (dimensionless) & \texttt{msMdRatio} \\
$m_t/m_b = \varphi^7 + \varphi^5$ & ScaleQ$(0,0,0)$ (dimensionless) & \texttt{mtMbRatio} \\
$m_u/m_d = C(2,2)/2 = 1/2$ & RatioQuantity & \texttt{muMdRatio} \\
$m_b/m_c = C(3,2) = 3$ & CountQuantity & \texttt{mbMcValue} \\
$m_c/m_s = C(5,2)+2 = 12$ & CountQuantity & \texttt{mcMsValue} \\
$\Delta m^2_{21}/\Delta m^2_{31} = 1/(\dim\ker(D_5) \times C(6,2)) = 1/30$ & RatioQuantity & \texttt{neutrinoMassSqRatio} \\
\hline
\end{tabular}
\end{center}

The exponents are CountQuantity: $T(3) = C(4,2) = 6$ (\texttt{msMdExponent}), $7 = C(4,2)+C(2,2)$ (\texttt{mtMbExpHigh}), $5 = C(4,2)-C(2,2)$ (\texttt{mtMbExpLow}). All exponents trace to D-structure pair counts, never hardcoded numerals.

$m_s/m_d = \varphi^6$ (ScaleQuantity, $\approx 17.94$) is structurally distinct from $\lambda_{\min} = 12/25$ (ScaleQuantity at dimTime$^{-1}$, $D_6$ minimum eigenvalue from gauge-invariant output).

