% FUST Physics Discoveries - Gravitational Collapse from D₆ Structure
\chapter{Gravitational Collapse from \texorpdfstring{$D_\zeta$}{Dζ} Structure}
\label{chap:pd_black_hole}

All results derived from the unified $D_\zeta$ operator algebra and $\varphi$-$\psi$ duality.
No continuous theory (GR, QFT, Bekenstein-Hawking) is assumed.

\section{\texorpdfstring{$D_6$}{D6} Completeness (No Singularity)}
\label{sec:bh_no_singularity}

\begin{theorem}[$D_6$ Completeness]
\label{thm:bh_d6_completeness}
\lean{FUST.BlackHole.D6_completeness}
\leanok
$6 = \binom{3}{2} + \binom{3}{2}$. $D_6$ is the maximum D-level.
\end{theorem}

\begin{theorem}[No Singularity from $D_6$ Completeness]
\label{thm:bh_no_singularity}
\lean{FUST.BlackHole.no_singularity_from_D6_completeness}
\leanok
For all $n \geq 7$, $D_n$ projects to $D_6$:
\[
\forall n \geq 7: \mathrm{projectToD6}(n) = 6
\]
Physics has finite resolution. Infinite density (singularity) cannot exist.
\end{theorem}

\section{Discrete Scale Lattice}
\label{sec:bh_discrete_scales}

\begin{theorem}[Discrete Scales]
\label{thm:bh_discrete_scales}
\lean{FUST.BlackHole.no_continuous_limit}
\leanok
Physical scales form a discrete lattice $\varphi^n$ ($n \in \mathbb{Z}$).
No continuous limit exists. The ratio between adjacent scales is exactly $\varphi$:
\[
\frac{\varphi^{n+1}}{\varphi^n} = \varphi
\]
\end{theorem}

\begin{theorem}[Discrete Levels Distinct]
\label{thm:bh_discrete_levels_distinct}
\lean{FUST.BlackHole.discrete_levels_distinct}
\leanok
$\varphi > 1$ implies injectivity:
\[
n \neq m \Rightarrow \varphi^n \neq \varphi^m
\]
\end{theorem}

\section{\texorpdfstring{$\varphi$-$\psi$}{φ-ψ} Duality (Algebraic Reversibility)}
\label{sec:bh_duality}

\begin{theorem}[$\varphi$-$\psi$ Duality]
\label{thm:bh_phi_psi_duality}
\lean{FUST.BlackHole.phi_psi_duality}
\leanok
From $x^2 = x + 1$:
\[
\varphi \cdot |\psi| = 1
\]
\end{theorem}

\begin{theorem}[Algebraic Reversibility]
\label{thm:bh_algebraic_reversibility}
\lean{FUST.BlackHole.algebraic_reversibility}
\leanok
For all $n \in \mathbb{N}$:
\[
\varphi^n \cdot |\psi|^n = 1
\]
\end{theorem}

\begin{theorem}[Unitarity from Duality]
\label{thm:bh_unitarity}
\lean{FUST.BlackHole.unitarity_from_duality}
\leanok
Scale inversion:
\[
\forall n \in \mathbb{Z}: \varphi^n \cdot \varphi^{-n} = 1
\]
\end{theorem}

\section{\texorpdfstring{$\ker(D_6)$}{ker(D6)} Structure}
\label{sec:bh_kernel}

\begin{theorem}[Spatial Dimension from Kernel]
\label{thm:bh_spatial_dim}
\lean{FUST.BlackHole.spatial_from_kernel}
\leanok
$\ker(D_6) = \mathrm{span}\{1, x, x^2\}$ (dimension 3).
Spatial dimension is 3, derived from $D_6$ algebra, not imported from GR.
\end{theorem}

\begin{theorem}[Kernel Dimension]
\label{thm:bh_kernel_dimension}
\lean{FUST.BlackHole.kernel_dimension}
\leanok
$\{1, x, x^2\}$ are annihilated by $D_6$, but $x^3$ is detected:
\begin{enumerate}
\item $D_6(1)(x) = 0$ for $x \neq 0$
\item $D_6(\mathrm{id})(x) = 0$ for $x \neq 0$
\item $D_6(t^2)(x) = 0$ for $x \neq 0$
\item $D_6(t^3)(x) \neq 0$ for $x \neq 0$
\end{enumerate}
\end{theorem}

\section{\texorpdfstring{$D_6$}{D6} Dissipation Rate}
\label{sec:bh_dissipation}

$D_6$ output decays geometrically by $|\psi|$ per step.
This replaces continuous Hawking temperature without importing QFT.

\begin{theorem}[Dissipation Geometric Decay]
\label{thm:bh_dissipation_geometric}
\lean{FUST.BlackHole.dissipation_geometric}
\leanok
\[
\mathrm{dissipationRate}(n+1) = \mathrm{dissipationRate}(n) \cdot |\psi|
\]
where $\mathrm{dissipationRate}(n) = |\psi|^n$.
\end{theorem}

\begin{theorem}[Dissipation Ratio]
\label{thm:bh_dissipation_ratio}
\lean{FUST.BlackHole.dissipation_ratio}
\leanok
\[
\frac{\mathrm{dissipationRate}(n+1)}{\mathrm{dissipationRate}(n)} = |\psi|
\]
The ratio $|\psi| = \varphi^{-1}$ is a purely algebraic consequence of $\varphi \cdot |\psi| = 1$.
\end{theorem}

\section{Scale Separation (Discrete Resolution)}
\label{sec:bh_scale_separation}

The minimum scale separation is $\varphi^{-k}$ for $k$ steps.
This is the $D_6$ resolution limit, not a ``horizon width'' from GR.

\begin{theorem}[Scale Resolution Decreasing]
\label{thm:bh_scale_resolution_decreasing}
\lean{FUST.BlackHole.scaleResolution_decreasing}
\leanok
\[
\mathrm{scaleResolution}(k+1) < \mathrm{scaleResolution}(k)
\]
where $\mathrm{scaleResolution}(k) = \varphi^{-k} > 0$.
\end{theorem}

\section{State Space Dimension from \texorpdfstring{$D_6$}{D6}}
\label{sec:bh_dof}

The number of independent degrees of freedom at scale level $k$
is determined by $\ker(D_6)$ dimension and the number of scale steps.
This replaces Bekenstein-Hawking $S \propto A$ without importing continuous GR.

\begin{theorem}[Degrees of Freedom from Kernel Dimension]
\label{thm:bh_dof}
\lean{FUST.BlackHole.dof_from_kernel_dim}
\leanok
\[
\mathrm{degreesOfFreedom}(k) = \dim\ker(D_6) \times k = 3k
\]
\end{theorem}

\begin{theorem}[Degrees of Freedom Monotone]
\label{thm:bh_dof_monotone}
\lean{FUST.BlackHole.dof_monotone}
\leanok
\[
\mathrm{degreesOfFreedom}(k) \leq \mathrm{degreesOfFreedom}(k+1)
\]
\end{theorem}

\section{\texorpdfstring{$\psi$}{ψ}-Contraction (Time-Reversed Evolution)}
\label{sec:bh_contraction}

$|\psi|^n$ is the contraction dual of $\varphi^n$ expansion.
Purely algebraic ($\varphi \cdot |\psi| = 1$), not ``white holes'' from GR.

\begin{theorem}[Expansion-Contraction Identity]
\label{thm:bh_expansion_contraction}
\lean{FUST.BlackHole.expansion_contraction_identity}
\leanok
\[
\varphi^n \cdot |\psi|^n = 1 \quad \forall n \in \mathbb{N}
\]
\end{theorem}

\section{Algebraic Completion (\texorpdfstring{$x \to 0$}{x → 0})}
\label{sec:bh_completion}

In FUST coordinate $x \in (0, 1)$, the sequence $\varphi^{-n} \to 0$ provides
algebraic completion. This is NOT a ``Big Bang singularity'' from GR.

\begin{theorem}[Completion Bounded by $D_6$]
\label{thm:bh_completion_bounded}
\lean{FUST.BlackHole.completion_bounded}
\leanok
$D_6$ boundary prevents trans-Planckian extension:
\[
\forall n \geq 7: \mathrm{projectToD6}(n) = 6
\]
\end{theorem}

\section{Scale Unboundedness}
\label{sec:bh_unbounded}

\begin{theorem}[Scale Unbounded]
\label{thm:bh_scale_unbounded}
\lean{FUST.BlackHole.scale_unbounded}
\leanok
$\varphi^n$ grows without bound ($\varphi > 1$):
\[
\forall M \in \mathbb{R},\; \exists k \in \mathbb{N}: \varphi^k > M
\]
\end{theorem}

\section{\texorpdfstring{$D_6$}{D6} Critical Scale}
\label{sec:bh_critical_scale}

From $\lambda_{\min} = 12/25$ (D6MinEigenvalue) and mass $m = \lambda_{\min} \cdot x_0^2$,
the conserved energy $E_{\mathrm{cons}} = \varphi^{4n} \cdot m^2$ reaches 1
at a critical scale $n_{\mathrm{crit}}$, giving coordinate distance
$r_{\mathrm{crit}} = x_0 \cdot |\psi|$.

\begin{theorem}[$|\psi| = \varphi^{-1}$]
\label{thm:bh_abs_psi_eq_inv_phi}
\lean{FUST.BlackHole.abs_psi_eq_inv_phi}
\leanok
Algebraic consequence of $\varphi \cdot |\psi| = 1$:
\[
|\psi| = \varphi^{-1}
\]
\end{theorem}

\begin{theorem}[Critical Scale Positive]
\label{thm:bh_critical_scale_pos}
\lean{FUST.BlackHole.criticalScale_pos}
\leanok
For $x_0 > 0$:
\[
\mathrm{criticalScale}(x_0) = x_0 \cdot |\psi| > 0
\]
\end{theorem}

\begin{theorem}[Critical Scale Below Initial Position]
\label{thm:bh_critical_scale_lt}
\lean{FUST.BlackHole.criticalScale_lt}
\leanok
For $x_0 > 0$:
\[
\mathrm{criticalScale}(x_0) < x_0
\]
since $|\psi| < 1$.
\end{theorem}

\begin{theorem}[Critical Scale at Lattice Point]
\label{thm:bh_critical_scale_lattice}
\lean{FUST.BlackHole.criticalScale_at_lattice}
\leanok
For $x_0 = \varphi^{-k}$:
\[
\mathrm{criticalScale}(\varphi^{-k}) = \varphi^{-(k+1)}
\]
\end{theorem}

\begin{theorem}[Critical Scale = Next Resolution Level]
\label{thm:bh_critical_scale_resolution}
\lean{FUST.BlackHole.criticalScale_eq_scaleResolution}
\leanok
\[
\mathrm{criticalScale}(\mathrm{scaleResolution}(k)) = \mathrm{scaleResolution}(k+1)
\]
\end{theorem}

\begin{theorem}[Critical Scale Monotone]
\label{thm:bh_critical_scale_monotone}
\lean{FUST.BlackHole.criticalScale_monotone}
\leanok
\[
x_1 \leq x_2 \Rightarrow \mathrm{criticalScale}(x_1) \leq \mathrm{criticalScale}(x_2)
\]
\end{theorem}

\section{Summary: All from \texorpdfstring{$D_6$}{D6} Algebra}
\label{sec:bh_summary}

\begin{theorem}[All from $D_6$ Algebra]
\label{thm:bh_all_from_d6}
\lean{FUST.BlackHole.all_from_D6_algebra}
\leanok
Every result uses only:
\begin{itemize}
\item $\varphi, \psi$: roots of $x^2 = x + 1$
\item $D_6$: difference operator with $\ker = \mathrm{span}\{1, x, x^2\}$
\item $D_6$ completeness: $6 = \binom{3}{2} + \binom{3}{2}$
\end{itemize}
No GR metric, no QFT, no Bekenstein-Hawking, no Hawking temperature assumed.
\end{theorem}
