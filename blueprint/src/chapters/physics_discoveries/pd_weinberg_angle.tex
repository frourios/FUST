% FUST Physics Discoveries - Weinberg Angle
\chapter{Weinberg Angle from D-Structure}
\label{chap:pd_weinberg}

\section{Kernel Dimension Transition}
\label{sec:weinberg_kernel_transition}

The Weinberg mixing angle is derived from the kernel structure of difference operators.

\begin{theorem}[$D_3$ Gauge Invariance]
\label{thm:weinberg_d3_gauge}
\lean{FUST.WeinbergAngle.D3_gauge}
\leanok
$D_3$ annihilates constants (kernel dimension 1):
\[
\forall x \neq 0: D_3[1](x) = 0
\]
\end{theorem}

\begin{theorem}[$D_5$ Extended Kernel]
\label{thm:weinberg_d5_extended}
\lean{FUST.WeinbergAngle.D5_extended_kernel}
\leanok
$D_5$ has extended kernel (dimension $\geq 2$):
\[
\forall x \neq 0: D_5[1](x) = 0 \text{ and } D_5[x](x) = 0
\]
\end{theorem}

\begin{theorem}[Kernel Dimension Transition]
\label{thm:weinberg_kernel_transition}
\lean{FUST.WeinbergAngle.kernel_dim_transition}
\leanok
$D_3 \to D_5$ is the unique transition where kernel dimension increases from 1 to $\geq 2$:
\begin{itemize}
\item $D_3$: kernel dim = 1 (annihilates only constants)
\item $D_3$ does NOT annihilate linear: $\exists x \neq 0: D_3[x](x) \neq 0$
\item $D_5$: kernel dim $\geq 2$ (annihilates constants AND linear)
\end{itemize}
\end{theorem}

\section{Pair Count Formula}
\label{sec:weinberg_pair_count}

\begin{theorem}[$D_3$ Pair Count]
\label{thm:weinberg_d3_pairs}
\lean{FUST.WeinbergAngle.D3_pair_count}
\leanok
$D_3$ uses 3 evaluation points ($\varphi x$, $x$, $\psi x$):
\[
C(3,2) = 3 \text{ pairs}
\]
\end{theorem}

\begin{theorem}[$D_5$ Pair Count]
\label{thm:weinberg_d5_pairs}
\lean{FUST.WeinbergAngle.D5_pair_count}
\leanok
$D_5$ uses 5 evaluation points ($\varphi^2 x$, $\varphi x$, $x$, $\psi x$, $\psi^2 x$):
\[
C(5,2) = 10 \text{ pairs}
\]
\end{theorem}

\section{Weinberg Angle Formula}
\label{sec:weinberg_formula}

\begin{theorem}[Weinberg Angle]
\label{thm:weinberg_formula}
\lean{FUST.WeinbergAngle.weinberg_angle_formula}
\leanok
\[
\sin^2\theta_W = \frac{C(3,2)}{C(3,2) + C(5,2)} = \frac{3}{13} \approx 0.2308
\]
Experimental value: $0.231$ (error $< 1\%$).
\end{theorem}

\section{Physical Interpretation}
\label{sec:weinberg_interpretation}

The electroweak mixing arises from the interplay between:
\begin{itemize}
\item $D_3$: $\mathrm{SU}(2)_L$ weak isospin (minimal gauge structure)
\item $D_5$: First structure with extended kernel (unified structure)
\end{itemize}

The ratio of pair counts determines the mixing strength.

\section{Derivation Chain}
\label{sec:weinberg_derivation}

\begin{theorem}[Complete Derivation Chain]
\label{thm:weinberg_derivation_chain}
\lean{FUST.WeinbergAngle.weinberg_derivation_chain}
\leanok
\begin{enumerate}
\item $D_3$ has kernel dim 1 (gauge invariance only)
\item $D_5$ has kernel dim $\geq 2$ (first extended)
\item Pair counts from evaluation points: $C(3,2) = 3$, $C(5,2) = 10$
\item Mixing ratio: $3/13$
\end{enumerate}
\end{theorem}

\section{\texorpdfstring{$D_5$}{D5} Coefficient Uniqueness}
\label{sec:weinberg_d5_uniqueness}

\begin{theorem}[$D_5$ Coefficient Uniqueness]
\label{thm:weinberg_d5_coefficients}
\lean{FUST.WeinbergAngle.D5_coefficient_uniqueness_needs_two_conditions}
\leanok
$D_5$ coefficients are uniquely determined by requiring both $D_5[1] = 0$ and $D_5[x] = 0$:
\[
a = -1, \quad b = -4
\]
\end{theorem}

\section{Summary}
\label{sec:weinberg_summary}

\begin{theorem}[Weinberg Summary]
\label{thm:weinberg_summary}
\lean{FUST.WeinbergAngle.weinberg_summary}
\leanok
\begin{enumerate}
\item $D_3$ gauge invariant (kernel dim 1)
\item $D_5$ has extended kernel (kernel dim $\geq 2$)
\item $D_3 \to D_5$ is kernel dimension transition
\item $\sin^2\theta_W = C(3,2)/(C(3,2)+C(5,2)) = 3/13$
\end{enumerate}
\end{theorem}

\section{Dimensional Type}
\label{sec:weinberg_dim}

$\sin^2\theta_W = 3/13$ is a \textbf{RatioQuantity}: a ratio of pair counts $C(3,2)/(C(3,2)+C(5,2))$, inherently dimensionless.

Lean: \texttt{FUST\_dim.CouplingConstants.weinbergAngle} with type \texttt{RatioQ}. The value theorem \texttt{weinbergAngle\_val} is proven by \texttt{simp only [weinbergAngle, Nat.choose]; norm\_num}, verifying the structural derivation.

