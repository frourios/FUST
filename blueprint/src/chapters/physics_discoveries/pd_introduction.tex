% FUST Physics Discoveries - Introduction
\chapter{FUST Physics Discoveries from \texorpdfstring{$D_\zeta$}{Dζ}}
\label{chap:pd_introduction}

\section{Overview}
\label{sec:pd_overview}

This part presents physical predictions derived from the unified $D_\zeta$ framework. All results follow from the algebraic structure of $D_\zeta$ and its $\mathbb{Z}/6\mathbb{Z}$ channel decomposition:

\begin{enumerate}
\item \textbf{$D_\zeta$ Unification}: Composition of hyperbolic ($\varphi$), elliptic ($\zeta_6$), and parabolic (normalization) geometries
\item \textbf{Channel Decomposition}: $\Phi_A$ (antisymmetric, AF) and $\Phi_S$ (symmetric, SY) channels
\item \textbf{Kernel Hierarchy}: $\ker(D_2) \subset \ker(D_5) \subset \ker(D_6)$, $\dim: 1 \to 2 \to 3$
\item \textbf{Gauge Group}: $\mathrm{SU}(3) \times \mathrm{SU}(2) \times \mathrm{U}(1)$ uniquely from $D_\zeta$
\item \textbf{Pair Counts}: $C(n,2) = n(n-1)/2$ from the evaluation points of each $D_n$ projection
\end{enumerate}

\section{Structure of This Part}
\label{sec:pd_structure}

The following chapters cover:

\begin{itemize}
\item \textbf{Black Hole Physics}: Information paradox resolution, discrete Hawking spectrum
\item \textbf{Cosmological Structure}: Scale lattice $\{\varphi^n\}$, energy density scaling
\item \textbf{Electroweak Parameters}: Weinberg angle $\sin^2\theta_W = 1/4$
\item \textbf{Coupling Constants}: Strong coupling, Cabibbo angle, CP phase
\item \textbf{Particle Spectrum}: SM particle count, forbidden particles
\item \textbf{Mass Predictions}: Quark ratios, lepton ratios, gravitational coupling
\end{itemize}

\section{Derivation Principles}
\label{sec:pd_principles}

All predictions use only:
\begin{enumerate}
\item The unified $D_\zeta$ and its $\mathbb{Z}/6\mathbb{Z}$ Fourier decomposition
\item Binomial coefficients $C(n,k)$ from evaluation point combinatorics
\item The golden ratio $\varphi$ and $\zeta_6$ (roots of $x^2 = x \pm 1$)
\end{enumerate}

Since $D_\zeta$ is a discrete algebraic operator on the $\langle \varphi, \zeta_6 \rangle$ lattice, transcendental constants such as $\pi$ and $e$ do not participate in structural quantities. They may appear in coordinate embeddings (e.g.\ $\delta_{CKM}=2\pi/5$, $\theta_C=\arctan(1/\varphi^3)$) but never in coupling strengths or ratios intrinsic to $D_\zeta$.

No phenomenological fitting is involved.
