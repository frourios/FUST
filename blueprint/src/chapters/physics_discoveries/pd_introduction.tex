% FUST Physics Discoveries - Introduction
\chapter{FUST Physics Discoveries}
\label{chap:pd_introduction}

\section{Overview}
\label{sec:pd_overview}

This part presents additional physical predictions derived from the FUST D-structure framework. All results follow from the fundamental principles established in previous parts:

\begin{enumerate}
\item \textbf{$D_6$ Completeness}: $D_7$+ reduces to $D_6$ via Fibonacci recurrence
\item \textbf{Kernel Structure}: $\ker(D_6) = \mathrm{span}\{1, x, x^2\}$ (dimension 3)
\item \textbf{Golden Ratio}: $\varphi = (1+\sqrt{5})/2$, $\psi = (1-\sqrt{5})/2$
\item \textbf{Pair Counts}: $C(n,2) = n(n-1)/2$ from $D_n$ evaluation points
\end{enumerate}

\section{Structure of This Part}
\label{sec:pd_structure}

The following chapters cover:

\begin{itemize}
\item \textbf{Black Hole Physics}: Information paradox resolution, discrete Hawking spectrum
\item \textbf{Cosmological Structure}: Scale lattice $\{\varphi^n\}$, energy density scaling
\item \textbf{Electroweak Parameters}: Weinberg angle $\sin^2\theta_W = 3/13$
\item \textbf{Coupling Constants}: Strong coupling, Cabibbo angle, CP phase
\item \textbf{Particle Spectrum}: SM particle count, forbidden particles
\item \textbf{Mass Predictions}: Quark ratios, lepton ratios, gravitational coupling
\end{itemize}

\section{Derivation Principles}
\label{sec:pd_principles}

All predictions use only:
\begin{enumerate}
\item Binomial coefficients $C(n,2)$ from D-structure pair counts
\item Triangular numbers $T(n) = n(n+1)/2 = C(n+1,2)$
\item Kernel dimension transitions ($D_3$, $D_4$ have dim 1; $D_5$, $D_6$ have extended kernels)
\item The golden ratio $\varphi$ and its algebraic properties
\end{enumerate}

No phenomenological fitting is involved.

