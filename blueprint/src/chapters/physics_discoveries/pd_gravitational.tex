% FUST Physics Discoveries - Gravitational Coupling
\chapter{Gravitational Coupling from D-Structure}
\label{chap:pd_gravitational}

\section{Lepton Mass Exponent}
\label{sec:grav_lepton_exponent}

\begin{theorem}[Lepton Mass Exponent]
\label{thm:grav_lepton_exponent}
\lean{FUST.GravitationalCoupling.leptonMassExponent_eq}
\leanok
\[
107 = p_3 + e_3 + d = 92 + 12 + 3
\]
where $p_3 = \sum t_k^3$, $e_3 = \prod t_k$, $d = \dim\ker(D_6) = 3$
(see Theorem~\ref{thm:lepton_sector}).
\end{theorem}

\section{Fractional Correction}
\label{sec:grav_fractional}

\begin{theorem}[Correction Denominator]
\label{thm:grav_correction_denom}
\lean{FUST.GravitationalCoupling.gravityCorrectionDenom_eq}
\leanok
\[
63 = C(3,2) \times T(6) = 3 \times 21
\]
\end{theorem}

\begin{theorem}[Alpha Exponent]
\label{thm:grav_alpha_exponent}
\lean{FUST.GravitationalCoupling.alpha_exponent_eq}
\leanok
\[
5 = 6 - 2 + 1 \text{ (active D-levels)}
\]
\end{theorem}

\section{Electron-Planck Mass Ratio}
\label{sec:grav_electron_planck}

\begin{theorem}[Electron-Planck Ratio]
\label{thm:grav_electron_planck}
\lean{FUST.GravitationalCoupling.electronPlanckRatio_eq}
\leanok
\[
\frac{m_e}{m_{Pl}} = \varphi^{-(107 + 5/63)}
\]
\end{theorem}

\section{Gravitational Coupling}
\label{sec:grav_coupling}

\begin{theorem}[Gravitational Coupling]
\label{thm:grav_coupling}
\lean{FUST.GravitationalCoupling.gravitationalCoupling_exponent}
\leanok
\[
\alpha_G = \left(\frac{m_e}{m_{Pl}}\right)^2 = \varphi^{-(214 + 10/63)}
\]
\end{theorem}

\section{D-Hierarchy Pair Counts}
\label{sec:grav_pair_counts}

\begin{theorem}[Total D-Hierarchy Pairs]
\label{thm:grav_total_pairs}
\lean{FUST.GravitationalCoupling.totalDHierarchyPairs_eq}
\leanok
\[
C(2,2) + C(3,2) + C(4,2) + C(5,2) + C(6,2) = 1 + 3 + 6 + 10 + 15 = 35
\]
\end{theorem}

\section{CMB Temperature}
\label{sec:grav_cmb}

The CMB temperature exponent decomposes as $152 = 107 + 45$,
separating mass scale ($\varphi^{-107} \approx m_e/m_{Pl}$)
from the thermal factor ($\varphi^{-45} = T_{CMB}/m_e$ in natural units).
Both terms are dimensionless exponents.

\begin{theorem}[CMB Decoupling Factor]
\label{thm:grav_cmb_decoupling}
\lean{FUST.GravitationalCoupling.cmbDecouplingFactor_eq}
\leanok
\[
45 = C(3,2) \times T(5) = 3 \times 15 = T(9)
\]
\end{theorem}

\begin{theorem}[CMB Temperature Exponent]
\label{thm:grav_cmb_exponent}
\lean{FUST.GravitationalCoupling.cmbTemperatureExponent_eq}
\leanok
\[
152 = 107 + 45
\]
\end{theorem}

\begin{theorem}[CMB Temperature Ratio]
\label{thm:grav_cmb_ratio}
\[
\frac{T_{CMB}}{T_{Pl}} = \varphi^{-152}
= \underbrace{\varphi^{-107}}_{\approx\, m_e/m_{Pl}} \times \underbrace{\varphi^{-45}}_{T_{CMB}/m_e}
\]
\end{theorem}

\section{Cosmological Constant}
\label{sec:grav_cosmological}

The cosmological exponent uses dimensional separation via Stefan--Boltzmann
$\rho \propto T^4$: the energy density ratio factors as
$\varphi^{-582} = (\varphi^{-152})^4 \times \varphi^{26}$,
giving $582 = 4 \times 152 - 26$ where $4 = d{+}1$ is the spacetime dimension
and $26 = t_1^2 + t_2^2 + t_3^2$ is the sector trace square sum.

\begin{theorem}[Cosmological Exponent]
\label{thm:grav_cosmological_exponent}
\lean{FUST.GravitationalCoupling.cosmologicalExponent_eq}
\leanok
\[
582 = 4 \times 152 - 26
= \underbrace{(d{+}1)}_{\text{Stefan--Boltzmann}}
  \times \underbrace{152}_{\text{CMB exponent}}
  - \underbrace{t_1^2{+}t_2^2{+}t_3^2}_{\text{sector trace squares}}
\]
\end{theorem}

\begin{theorem}[Cosmological Density Ratio]
\label{thm:grav_cosmological_density}
\[
\frac{\rho_\Lambda}{\rho_{Pl}} = \varphi^{-582}
= \left(\frac{T_{CMB}}{T_{Pl}}\right)^{\!4} \times \varphi^{26}
\]
\end{theorem}

\section{Summary}
\label{sec:grav_summary}

\begin{theorem}[Gravitational Coupling Summary]
\label{thm:grav_summary}
\lean{FUST.GravitationalCoupling.gravitational_coupling_summary}
\leanok
\begin{enumerate}
\item Exponent $107 = p_3 + e_3 + d$ (sector cube sum + product + vacuum dim)
\item Fractional correction: $5/63 = 5/(C(3,2) \times T(6))$
\item D-structure pairs: $1 + 3 + 6 + 10 + 15 = 35$
\item $D_3$ gauge invariance: $D_3[1] = 0$
\end{enumerate}
\end{theorem}

\begin{theorem}[Cosmological Summary]
\label{thm:grav_cosmological_summary}
\lean{FUST.GravitationalCoupling.cosmological_summary}
\leanok
\begin{enumerate}
\item CMB temperature exponent: $152 = 107 + 45$ (mass scale + thermal factor)
\item Cosmological constant exponent: $582 = 4 \times 152 - 26$ (Stefan--Boltzmann + sector correction)
\item Decoupling factor: $45 = C(3,2) \times T(5) = T(9)$
\item Sector trace squares: $26 = t_1^2 + t_2^2 + t_3^2$
\end{enumerate}
\end{theorem}

\section{Inverse Square Law from D-Structure}
\label{sec:grav_inverse_square}

Newton's inverse square law $F \propto 1/r^2$ is derived purely from the $D_6$
operator structure, without importing any assumption from continuous field theory.

\subsection{Golden Conjugate Inversion}

The derivation rests on the algebraic identity $\varphi\psi = -1$, which implies:

\begin{lemma}[Golden Conjugate Inversion]
\label{lem:golden_conjugate_inversion}
\lean{FUST.GravitationalCoupling.phi_inv_eq_neg_psi}
\lean{FUST.GravitationalCoupling.psi_inv_eq_neg_phi}
\leanok
\[
\varphi^{-1} = -\psi, \qquad \psi^{-1} = -\varphi.
\]
In particular, $\varphi^{-2k} = \psi^{2k}$ and $\psi^{-2k} = \varphi^{2k}$ for all $k$.
\end{lemma}

\subsection{Extended Kernel of D-Structure}

Define the dissipation coefficient for the monomial $t^n$:
\[
C(n) = \varphi^{3n} - 3\varphi^{2n} + \varphi^n - \psi^n + 3\psi^{2n} - \psi^{3n}.
\]
The kernel $\ker(D_6) = \operatorname{span}\{1, t, t^2\}$ corresponds to $C(0) = C(1) = C(2) = 0$.

By the golden conjugate inversion, the coefficient at $n = -2$ satisfies
\[
C(-2) = \varphi^{-6} - 3\varphi^{-4} + \varphi^{-2} - \psi^{-2} + 3\psi^{-4} - \psi^{-6}
= \psi^{6} - 3\psi^{4} + \psi^{2} - \varphi^{2} + 3\varphi^{4} - \varphi^{6} = -C(2) = 0.
\]

\begin{theorem}[$D_6$ Annihilates $t^{-2}$]
\label{thm:D6_inv_sq_zero}
\lean{FUST.GravitationalCoupling.D6_inv_sq_zero}
\leanok
For all $x \neq 0$,
\[
D_6[t^{-2}](x) = 0.
\]
\end{theorem}

\begin{proof}
Expand $D_6[t^{-2}](x)$ using $(ax)^{-2} = a^{-2}x^{-2}$.
The numerator factors as $x^{-2} \cdot C(-2) = 0$ since $\varphi^{-2k} = \psi^{2k}$
reduces the coefficient to $-C(2) = 0$.
\end{proof}

\subsection{Inverse-Square Force}

The $D_6$ operator applied to the $1/r$ potential $\Phi(t) = t^{-1}$ yields:

\begin{theorem}[Inverse-Square Force from $D_6$]
\label{thm:D6_inv_one}
\lean{FUST.GravitationalCoupling.D6_inv_one}
\leanok
For all $x \neq 0$,
\[
D_6[t^{-1}](x) = \frac{6}{(\varphi - \psi)^4 \cdot x^2} = \frac{6}{25} \cdot \frac{1}{x^2}.
\]
\end{theorem}

\begin{proof}
Expand the numerator using $(ax)^{-1} = a^{-1}x^{-1}$ and the substitutions
$\varphi^{-1} = -\psi$, $\psi^{-1} = -\varphi$:
\begin{align*}
\text{numerator}
&= x^{-1}\bigl(-\psi^3 - 3\psi^2 - \psi + \varphi + 3\varphi^2 + \varphi^3\bigr) \\
&= x^{-1}\bigl(-(2\psi+1) - 3(\psi+1) - \psi + \varphi + 3(\varphi+1) + (2\varphi+1)\bigr) \\
&= x^{-1} \cdot 6(\varphi - \psi) = 6\sqrt{5} \cdot x^{-1}.
\end{align*}
Dividing by the denominator $(\varphi-\psi)^5 x = (\sqrt{5})^5 x$ gives
$6\sqrt{5}/((\sqrt{5})^5 x^2) = 6/25 \cdot x^{-2}$.
\end{proof}

The nonzero coefficient $C(-1) = 6\sqrt{5} \neq 0$ confirms the gravitational force
exists and is exactly inverse-square.

\subsection{Harmonic Potential}

\begin{theorem}[$1/r$ Potential is Harmonic]
\label{thm:dAlembertian_inv_zero}
\lean{FUST.GravitationalCoupling.dAlembertian_inv_zero}
\leanok
For all $x \neq 0$,
\[
\Box_\varphi[t^{-1}](x) = D_6(D_6[t^{-1}])(x) = 0.
\]
\end{theorem}

\begin{proof}
By Theorem~\ref{thm:D6_inv_one}, $D_6[t^{-1}](y) = \tfrac{6}{(\varphi-\psi)^4} \cdot y^{-2}$
for $y \neq 0$.  Since $D_6$ evaluates only at nonzero points $\varphi^k x$ and $\psi^k x$,
the outer $D_6$ sees a constant multiple of $t^{-2}$.
By $D_6$-linearity (homogeneity) and Theorem~\ref{thm:D6_inv_sq_zero}:
\[
D_6\!\Bigl[\frac{6}{(\varphi-\psi)^4} \cdot t^{-2}\Bigr](x)
= \frac{6}{(\varphi-\psi)^4} \cdot D_6[t^{-2}](x) = 0. \qedhere
\]
\end{proof}

\subsection{Summary}

\begin{theorem}[Inverse Square Law Derivation]
\label{thm:inverse_square_law}
\lean{FUST.GravitationalCoupling.inverse_square_law_derivation}
\leanok
From the $D_6$ structure alone:
\begin{enumerate}
\item $\dim\ker(D_6) = 3$ determines spatial dimension $d = 3$.
\item $D_6[t^{-2}] = 0$: the inverse-square monomial is in the extended kernel.
\item $D_6[t^{-1}](x) = \tfrac{6}{25} x^{-2}$: force is inverse-square with nonzero coefficient.
\item $\Box_\varphi[t^{-1}] = 0$: the $1/r$ potential is harmonic under the FUST d'Alembertian.
\end{enumerate}
\end{theorem}

The derivation chain is:
\[
\varphi\psi = -1
\;\;\Longrightarrow\;\;
C(-2) = 0
\;\;\Longrightarrow\;\;
D_6[t^{-2}] = 0
\;\;\Longrightarrow\;\;
\Box_\varphi[t^{-1}] = 0
\;\;\Longrightarrow\;\;
F = D_6[t^{-1}] \propto \frac{1}{r^2}.
\]

\section{D-Structure Algebraic Relations}
\label{sec:grav_d_structure_relations}

Systematic exploration of D$_6$ algebraic properties reveals deep connections
to Fibonacci number theory and the sector factorization.

\subsection{Dissipation--Fibonacci Decomposition}

\begin{theorem}[Dissipation--Fibonacci Decomposition]
\label{thm:dissipation_fibonacci}
\lean{FUST.NavierStokes.dissipation_fibonacci_decomposition}
\leanok
For all $n \in \mathbb{N}$,
\[
C(n) = (\varphi - \psi) \cdot \bigl(F(3n) - 3F(2n) + F(n)\bigr)
\]
where $F(k) = (\varphi^k - \psi^k)/\sqrt{5}$ is the $k$-th Fibonacci number
and $C(n)$ is the dissipation coefficient.
\end{theorem}

\subsection{$D_6$ Coefficient Structure}

\begin{theorem}[$D_6$ Evaluation Point Sum]
\label{thm:D6_eval_sum}
\lean{FUST.GravitationalCoupling.D6_eval_multiplier_sum}
\leanok
\[
\varphi^3 + \varphi^2 + \varphi + \psi + \psi^2 + \psi^3 = 8 = t_1 + t_2 + t_3.
\]
\end{theorem}

\begin{theorem}[$D_6$ Coefficient Properties]
\label{thm:D6_coeff_props}
\lean{FUST.GravitationalCoupling.D6_coeff_sum}
\lean{FUST.GravitationalCoupling.D6_coeff_abs_sum}
\leanok
The $D_6$ coefficients $[1, -3, 1, -1, 3, -1]$ satisfy:
\begin{enumerate}
\item Signed sum $= 0$ (kills constants).
\item Absolute sum $= 10 = C(5,2)$.
\item Positive part sum $=$ negative part $= 5 = $ active levels.
\end{enumerate}
\end{theorem}

\subsection{D$_6$ Spectral Invariants}

The elementary symmetric polynomials $e_k$ of the evaluation multipliers
$\{\varphi^3, \varphi^2, \varphi, \psi, \psi^2, \psi^3\}$ are:

\begin{theorem}[$D_6$ Spectral Invariants]
\label{thm:D6_spectral_invariants}
\lean{FUST.GravitationalCoupling.D6_spectral_invariants}
\leanok
\begin{align*}
e_1 &= 8, & e_2 &= 18, & e_3 &= 6, \\
e_4 &= -12, & e_5 &= -2, & e_6 &= (\varphi\psi)^{1+2+3} = (-1)^6 = 1.
\end{align*}
\end{theorem}

These define the characteristic polynomial
\[
p(x) = x^6 - 8x^5 + 18x^4 - 6x^3 - 12x^2 + 2x + 1
\]
whose roots are precisely the $D_6$ evaluation multipliers.

\begin{theorem}[$D_6$ Characteristic Polynomial Roots]
\label{thm:D6_charPoly_roots}
\lean{FUST.GravitationalCoupling.D6_charPoly_roots}
\leanok
\[
p(\varphi^k) = 0, \quad p(\psi^k) = 0 \quad \text{for } k = 1, 2, 3.
\]
\end{theorem}

\subsection{\texorpdfstring{$D_6$}{D₆} Sector Factorization}

The characteristic polynomial factors into three sectors:

\begin{theorem}[$D_6$ Characteristic Polynomial Factorization]
\label{thm:D6_charPoly_factorization}
\lean{FUST.GravitationalCoupling.D6_charPoly_factorization}
\leanok
\[
p(x) = (x^2 - x - 1)(x^2 - 3x + 1)(x^2 - 4x - 1)
\]
corresponding to matter $(\varphi,\psi)$, gauge $(\varphi^2,\psi^2)$,
and gravity $(\varphi^3,\psi^3)$ sectors.
\end{theorem}

\begin{theorem}[Sector Trace Squares]
\label{thm:sector_trace_sq_sum}
\lean{FUST.GravitationalCoupling.sector_trace_sq_sum}
\leanok
\[
t_1^2 + t_2^2 + t_3^2 = 1 + 9 + 16 = 26.
\]
\end{theorem}

\subsection{Gravity Sector Structural Properties}

The gravity sector polynomial $x^2 - 4x - 1$ encodes key spacetime properties.

\begin{theorem}[Gravity Sector Trace = Spacetime Dimension]
\label{thm:gravity_trace_spacetime}
\lean{FUST.GravitationalCoupling.gravity_trace_eq_spacetimeDim}
\leanok
\[
\varphi^3 + \psi^3 = 4 = \mathrm{spacetimeDim}.
\]
The gravity sector trace equals the spacetime dimension.
This is the structural bridge between the gravity sector and spacetime geometry.
\end{theorem}

\begin{theorem}[Gravity Sector Determinant]
\label{thm:gravity_sector_det}
\lean{FUST.GravitationalCoupling.gravity_sector_det}
\leanok
\[
(\varphi\psi)^3 = (-1)^3 = -1.
\]
The gravity sector has determinant $-1$, inherited from the matter sector via cubing.
\end{theorem}

\begin{theorem}[Gravity Sector Discriminant]
\label{thm:gravity_sector_disc}
\lean{FUST.GravitationalCoupling.gravity_sector_discriminant}
\leanok
\[
\Delta(x^2 - 4x - 1) = 4^2 + 4 \cdot 1 = 20 = C(6,3).
\]
\end{theorem}

\begin{theorem}[Discriminant Decomposition]
\label{thm:gravity_disc_decomposition}
\lean{FUST.GravitationalCoupling.gravity_disc_eq_spacetime_times_active}
\leanok
\[
C(6,3) = 20 = \mathrm{spacetimeDim} \times \mathrm{activeDLevels} = 4 \times 5.
\]
The spatial normalization $C(6,3)$ factors as the product of spacetime dimension and active D-levels.
This is the same $C(6,3) = 20$ appearing in the fine structure constant $\alpha_0 = \varphi^{-4}/C(6,3)$.
\end{theorem}

\begin{theorem}[Sector Discriminant Pattern]
\label{thm:sector_discriminants}
\lean{FUST.GravitationalCoupling.sector_discriminants}
\leanok
\begin{center}
\begin{tabular}{|c|c|c|c|}
\hline
Sector & Polynomial & Discriminant & Meaning \\
\hline
Matter & $x^2 - x - 1$ & $1 + 4 = 5$ & activeDLevels \\
Gauge & $x^2 - 3x + 1$ & $9 - 4 = 5$ & activeDLevels \\
Gravity & $x^2 - 4x - 1$ & $16 + 4 = 20 = C(6,3)$ & $4 \times 5$ \\
\hline
\end{tabular}
\end{center}
The matter and gauge sectors share discriminant $5 =$ activeDLevels.
The gravity sector discriminant is $4 \times 5 = \mathrm{spacetimeDim} \times \mathrm{activeDLevels}$.
\end{theorem}

\subsection{Sector-Invariant Derivation of Physical Exponents}

The $D_6$ characteristic polynomial factors into three sectors with
traces $t_k = \varphi^k + \psi^k$: $t_1 = 1$ (matter), $t_2 = 3$ (gauge), $t_3 = 4$ (gravity).
The physical exponents are expressed using independent spectral invariants:
power sums $p_k = \sum t_i^k$, elementary symmetric functions $e_k$,
and the vacuum dimension $d = \dim\ker(D_6) = 3$.

\begin{theorem}[Sector Trace Cube Sum]
\label{thm:sector_trace_cube_sum}
\lean{FUST.GravitationalCoupling.sector_trace_cube_sum}
\leanok
\[
p_3 = t_1^3 + t_2^3 + t_3^3 = 1 + 27 + 64 = 92.
\]
\end{theorem}

\begin{theorem}[Sector Trace Product]
\label{thm:sector_trace_product}
\lean{FUST.GravitationalCoupling.sector_trace_product}
\leanok
\[
e_3 = t_1 \times t_2 \times t_3 = 1 \times 3 \times 4 = 12.
\]
\end{theorem}

\begin{theorem}[Sector Product + Kernel = Pair Count]
\label{thm:sector_product_plus_ker}
\lean{FUST.GravitationalCoupling.sector_product_plus_kerDim_eq_pairs}
\leanok
\[
e_3 + d = t_1 \cdot t_2 \cdot t_3 + \dim\ker(D_6) = 12 + 3 = 15 = C(6,2).
\]
The sector product plus vacuum dimension equals the $D_6$ pair count.
\end{theorem}

\begin{theorem}[Lepton Exponent from Sector Invariants]
\label{thm:lepton_sector}
\lean{FUST.GravitationalCoupling.leptonExponent_from_sector_invariants}
\leanok
\[
107 = p_3 + e_3 + d = \sum t_k^3 + \prod t_k + \dim\ker(D_6) = 92 + 12 + 3.
\]
Each term is an independent spectral invariant with no degree mixing.
\end{theorem}

\begin{theorem}[CMB Decoupling from Sector Invariants]
\label{thm:cmb_decoupling_sector}
\lean{FUST.GravitationalCoupling.cmbDecoupling_from_sector_invariants}
\leanok
\[
45 = d \times (e_3 + d) = 3 \times 15.
\]
\end{theorem}

\begin{theorem}[CMB Exponent from Sector Invariants]
\label{thm:cmb_sector}
\lean{FUST.GravitationalCoupling.cmbExponent_from_sector_invariants}
\leanok
\[
152 = p_3 + D \times (e_3 + d) = 92 + 4 \times 15.
\]
where $D = t_3 = \mathrm{spacetimeDim} = 4$.
\end{theorem}

\begin{theorem}[Cosmological Exponent from Sector Invariants]
\label{thm:cosmo_sector}
\lean{FUST.GravitationalCoupling.cosmoExponent_from_sector_invariants}
\leanok
\[
582 = D \times 152 - \sigma = 4 \times 152 - 26.
\]
where $\sigma = t_1^2 + t_2^2 + t_3^2 = 26$ is the sector trace square sum.
\end{theorem}

\begin{theorem}[Complete Sector-Invariant Derivation]
\label{thm:sector_invariant_derivation}
\lean{FUST.GravitationalCoupling.sector_invariant_derivation}
\leanok
All physical exponents derive from sector traces $t_k = \varphi^k + \psi^k$ for $k=1,2,3$:
\begin{align*}
p_3 &= 1^3 + 3^3 + 4^3 = 92 \\
e_3 &= 1 \times 3 \times 4 = 12, \quad e_3 + d = 15 = C(6,2) \\
\sigma &= 1^2 + 3^2 + 4^2 = 26 \\
107 &= p_3 + e_3 + d \\
45 &= d \times (e_3 + d) \\
152 &= p_3 + D \times (e_3 + d) \\
582 &= D \times 152 - \sigma
\end{align*}
\end{theorem}

\subsection{Dissipation Recurrence}

\begin{theorem}[6th-Order Recurrence for Dissipation Coefficients]
\label{thm:dissipation_recurrence}
\lean{FUST.NavierStokes.dissipation_recurrence}
\leanok
For all $n \in \mathbb{N}$,
\[
C(n{+}6) = 8C(n{+}5) - 18C(n{+}4) + 6C(n{+}3)
 + 12C(n{+}2) - 2C(n{+}1) - C(n).
\]
The recurrence coefficients are the negated elementary symmetric polynomials:
$\{8, -18, 6, 12, -2, -1\} = \{e_1, -e_2, e_3, -e_4, e_5, -e_6\}$.
\end{theorem}

\begin{theorem}[Power Sum $p_2$]
\label{thm:D6_power_sum_2}
\lean{FUST.NavierStokes.D6_power_sum_2}
\leanok
\[
\varphi^6 + \varphi^4 + \varphi^2 + \psi^2 + \psi^4 + \psi^6 = 28 = (\varphi^6{+}\psi^6) + (\varphi^4{+}\psi^4) + (\varphi^2{+}\psi^2).
\]
\end{theorem}

\subsection{Extended d'Alembertian Kernel}

\begin{theorem}[Extended Kernel of $\Box_\varphi$]
\label{thm:dAlembertian_extended_kernel}
\lean{FUST.GravitationalCoupling.dAlembertian_extended_kernel}
\leanok
The FUST d'Alembertian $\Box_\varphi = D_6 \circ D_6$ annihilates:
\[
\Box_\varphi[t^n] = 0 \quad \text{for } n \in \{-1, 0, 1, 2, 3\}.
\]
For $n = 0, 1, 2$: $D_6[t^n] = 0$ (kernel), so $\Box_\varphi[t^n] = D_6(0) = 0$.
For $n = 3$: $D_6[t^3] \propto t^2$, then $D_6[t^2] = 0$.
For $n = -1$: $D_6[t^{-1}] \propto t^{-2}$, then $D_6[t^{-2}] = 0$.
\end{theorem}

\section{Dimensional Derivation Structure}
\label{sec:grav_derivation_structure}

The $D_6$ operator maps $D_6[t^n](x) = C(n) \cdot x^n / (\varphi-\psi)^5$
where $C(n)$ is the dissipation coefficient.
The monomial eigenvalue $\Lambda(n) = C(n)/(\varphi-\psi)^5$ vanishes
for $n \in \{0, 1, 2\}$:
\begin{itemize}
\item $\Delta = 0$: constants (dimensionless couplings)
\item $\Delta = 1$: mass operators ($m_e$, $m_{Pl}$)
\item $\Delta = 2$: kinetic energy operators ($p^2$, $\nabla^2$)
\end{itemize}
Since $D_6$ annihilates $\Delta = 1$, mass ratios $m_e/m_{Pl}$
are \emph{boundary data} of the D-hierarchy, not determined by $D_6$ eigenvalues.

\begin{theorem}[$D_6$ Kernel Dimensions]
\label{thm:D6_kernel_dimensions}
\lean{FUST.GravitationalCoupling.D6_kernel_dimensions}
\leanok
$D_6[1] = D_6[t] = D_6[t^2] = 0$ (kernel: $\Delta = 0, 1, 2$),
while $D_6[t^{-1}] \neq 0$ (force operator: outside kernel).
\end{theorem}

The physical exponents form a three-layer structure:

\subsection*{Layer 1: $D_6$ Eigenvalue Structure (structural)}

Determined purely by $D_6$ spectral data:
\begin{itemize}
\item $d = \dim(\ker D_6) = 3$ (spatial dimension)
\item $d{+}1 = 4$ (spacetime dimension)
\item $\sigma = t_1^2 + t_2^2 + t_3^2 = 26$ (sector trace squares)
\item $F \propto 1/r^2$ (inverse square law from $\varphi\psi = -1$)
\item $p(x) = (x^2{-}x{-}1)(x^2{-}3x{+}1)(x^2{-}4x{-}1)$ (sector factorization)
\end{itemize}

\subsection*{Layer 2: Sector-Invariant Exponents}

The physical exponents derive from sector traces $t_k = \varphi^k + \psi^k$
of the $D_6$ characteristic polynomial factorization
$p(x) = (x^2{-}x{-}1)(x^2{-}3x{+}1)(x^2{-}4x{-}1)$:
\begin{itemize}
\item $107 = p_3 + e_3 + d = \sum t_k^3 + \prod t_k + \dim\ker(D_6)$
\item $45 = d \times (e_3 + d) = 3 \times C(6,2)$
\item $5/63 = \text{activeLevels}/\text{correctionDenom}$
\end{itemize}
Each term is an independent spectral invariant with no degree mixing
(Theorems~\ref{thm:lepton_sector}--\ref{thm:sector_invariant_derivation}).

\subsection*{Layer 3: Physical Assembly (dimensionally consistent)}

The physical quantities are assembled with dimensional intermediates:
\begin{enumerate}
\item $\varphi^{-152} = \varphi^{-107} \times \varphi^{-45}$:
      $T_{CMB}/T_{Pl} = (m_e/m_{Pl}) \times (T_{CMB}/m_e)$,
      passing through $[M]$ intermediate.
\item $\varphi^{-582} = (\varphi^{-152})^4 \times \varphi^{26}$:
      $\rho_\Lambda/\rho_{Pl} = (T_{CMB}/T_{Pl})^4 \times \varphi^{26}$,
      passing through $[M^4]$ intermediate via Stefan--Boltzmann $\rho \propto T^4$.
\end{enumerate}

\begin{theorem}[Three-Layer Assembly]
\label{thm:derivation_layer3}
\lean{FUST.GravitationalCoupling.derivation_layer3_assembly}
\leanok
\begin{align*}
152 &= 107 + 45 & &\text{(mass scale + thermal factor, via $[M]$)} \\
582 &= 4 \times 152 - 26 & &\text{(Stefan--Boltzmann + sector correction, via $[M^4]$)}
\end{align*}
\end{theorem}

\noindent\textbf{Note on derivation.}
The sector-invariant formulas (Theorems~\ref{thm:lepton_sector}--\ref{thm:sector_invariant_derivation})
express all physical exponents using independent spectral invariants ($p_3$, $e_3$, $\sigma$, $d$, $D$)
from the $D_6$ sector factorization, avoiding the degree-inhomogeneity of earlier spectral formulas.

\section{Graviton Structural Prediction}
\label{sec:grav_graviton}

The graviton is not postulated but derived from the $D_6$ structure.
Seven properties follow from the characteristic polynomial and kernel analysis.

\begin{theorem}[Graviton Masslessness]
\label{thm:graviton_massless}
\lean{FUST.GravitationalCoupling.graviton_massless}
\leanok
For all $x \neq 0$,
\[
\Box_\varphi[t^{-1}](x) = D_6(D_6[t^{-1}])(x) = 0.
\]
The graviton propagator has no mass term because $D_6[t^{-1}] \propto t^{-2}$
and $t^{-2}$ is in the extended kernel of $D_6$ (Theorem~\ref{thm:D6_inv_sq_zero}).
\end{theorem}

\begin{theorem}[Graviton Spin from Spacetime]
\label{thm:graviton_spin}
\lean{FUST.GravitationalCoupling.graviton_spin_from_spacetime}
\leanok
\[
\mathrm{spin}_{\text{graviton}} = \mathrm{spatialDim} - \mathrm{timeDim} = 3 - 1 = 2.
\]
The maximum helicity for a massless particle in $(d{+}1)$-dimensional spacetime is $d - 1$
(Weinberg's theorem). With $\dim\ker(D_6) = 3$, the graviton is uniquely spin-2.
\end{theorem}

\begin{theorem}[Graviton Prediction from $D_6$ Structure]
\label{thm:graviton_prediction}
\lean{FUST.GravitationalCoupling.graviton_prediction}
\leanok
From the $D_6$ characteristic polynomial and kernel alone:
\begin{enumerate}
\item \textbf{Existence}: $p(x) = (x^2{-}x{-}1)(x^2{-}3x{+}1)(x^2{-}4x{-}1)$ has a gravity sector.
\item \textbf{Spacetime bridge}: $t_3 = \varphi^3 + \psi^3 = 4 = \mathrm{spacetimeDim}$.
\item \textbf{Sector determinant}: $(\varphi\psi)^3 = -1$.
\item \textbf{Sector discriminant}: $\Delta = 20 = C(6,3) = 4 \times 5$.
\item \textbf{Massless}: $\Box_\varphi[t^{-1}] = 0$.
\item \textbf{Spin-2}: $\mathrm{spatialDim} - \mathrm{timeDim} = 2$.
\item \textbf{Inverse square}: $D_6[t^{-1}](x) = \tfrac{6}{25} x^{-2}$.
\end{enumerate}
\end{theorem}

The derivation chain is:
\[
D_6 \text{ charPoly}
\;\Longrightarrow\;
\text{gravity sector } (x^2{-}4x{-}1)
\;\Longrightarrow\;
\begin{cases}
t_3 = 4 & \text{(spacetimeDim)} \\
\Delta = 20 & \text{(spatial normalization)} \\
\Box_\varphi[t^{-1}] = 0 & \text{(massless)} \\
F \propto 1/r^2 & \text{(inverse square)}
\end{cases}
\]
