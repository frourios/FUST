% Chapter 7: Gauge Groups and Parameter Space
\chapter{Gauge Groups and Parameter Space}
\label{chap:gauge_groups}

\section{Mathematical Structure of the Gauge Sector}
\label{sec:gauge_sector}

This chapter derives the \textbf{gauge group parameter space} from the $\varphi$-dilation eigenvalue structure of the kernel of difference operators.

\begin{theorem}[Gauge Parameter Space from $\varphi$-Dilation]
\label{thm:gauge_parameter_space}
\lean{FUST.gauge\_parameter\_space}
\leanok
The $\varphi$-dilation operator $U_\varphi: f(z) \mapsto f(\varphi z)$ acts on the monomial basis $\{z^k\}$ of each kernel with eigenvalues $\{\varphi^k\}$. Since these eigenvalues are pairwise distinct, any gauge transformation commuting with $\varphi$-dilation must be diagonal. This gives a binary classification for each kernel:

\begin{center}
\begin{tabular}{|c|c|c|c|}
\hline
Kernel & Dim & $\varphi$-preserving & $\varphi$-breaking \\
\hline
$\ker(D_5)$ & 2 & $\mathrm{U}(1)^2$ (diagonal) & $\mathrm{SU}(2)$ (non-diagonal) \\
$\ker(D_6)$ & 3 & $\mathrm{U}(1)^3$ (diagonal) & $\mathrm{SU}(3)$ (non-diagonal) \\
\hline
\end{tabular}
\end{center}

Scaling $x \to e^{i\theta}x$ gives $G_1 = \mathrm{U}(1)$ (unique).

The gauge parameter space is $\mathcal{G} = G_6 \times G_5$ with $|\mathcal{G}| = 2 \times 2 = 4$.
\end{theorem}

\subsection{\texorpdfstring{$\varphi$}{φ}-Dilation Commutant Theorem}

\begin{theorem}[$\varphi$-Dilation Commutant is Diagonal]
\label{thm:commutant_diagonal}
\lean{FUST.phiDilation\_commutant\_diagonal}
\leanok
Let $S = \mathrm{diag}(1, \varphi, \varphi^2)$ be the $\varphi$-dilation scaling matrix on $\ker(D_6)$. If $M \in \mathrm{GL}(3, \mathbb{C})$ satisfies $MS = SM$, then $M$ is diagonal: $M_{ij} = 0$ for $i \neq j$.
\end{theorem}

\begin{proof}
From $MS = SM$, entry-wise:
\[
M_{ij} \cdot \varphi^j = \varphi^i \cdot M_{ij}
\]
Thus $M_{ij}(\varphi^j - \varphi^i) = 0$. Since the eigenvalues $\{1, \varphi, \varphi^2\} = \{1, \varphi, \varphi + 1\}$ are pairwise distinct ($\varphi > 1$ implies $1 \neq \varphi \neq \varphi + 1 \neq 1$), we have $\varphi^j - \varphi^i \neq 0$ for $i \neq j$, hence $M_{ij} = 0$.
\end{proof}

\textbf{Consequence}: The gauge transformations commuting with $\varphi$-dilation are exactly the diagonal unitary matrices $\mathrm{U}(1)^n$, one independent $\mathrm{U}(1)$ per eigenspace. Non-diagonal transformations such as $\mathrm{SU}(n)$ necessarily break $\varphi$-dilation symmetry.

\subsection{Infinite Dihedral Group and Weyl Reflection}

The fundamental group of Frourio algebra is:
\[
D_\infty = \langle u, r \mid r^2 = 1,\; rur = u^{-1} \rangle
\]
where $U: f(x) \mapsto f(\varphi x)$ (scale transformation) and $R: f(x) \mapsto f(-x)$ (inversion).

The relation $rur = u^{-1}$ represents the \textbf{Weyl reflection} $u \mapsto u^{-1}$, corresponding to the Weyl group $\mathbb{Z}_2$ of $\mathrm{SU}(2)$.

\subsection{Phase Quantization and Half-Integer Spin}

For any positive number $a > 0$, $a^{i\pi/\log a} = e^{i\pi} = -1$ holds. This is not unique to the golden ratio.

\textbf{Why $\varphi$ is special}: The reason lies in its extremal properties (see below). The meaning of phase quantization is as follows:

For the scale generator $u: f(x) \mapsto f(\varphi x)$ of $D_\infty$, consider the complex power $u^s = \varphi^s$. The equality $u^{i\pi/\log\varphi} = -1$ implies phase quantization at angle $\theta = \pi$.

\textbf{Consequence: Selection of half-integer spin}

$-1 = e^{i\pi}$ implies representations that change sign under $2\pi$ rotation (fermions):
\begin{itemize}
\item In the spin-$j$ representation of $\mathrm{SU}(2)$, the eigenvalue of $e^{2\pi i J_z}$ is $e^{2\pi i m}$ ($m = -j, \ldots, j$)
\item For $j = 1/2$, $m = \pm 1/2$ gives $e^{\pi i} = -1$
\end{itemize}

\subsection{Extremal Properties of the Golden Ratio}

In the metallic ratio family $\Lambda_p = \frac{p + \sqrt{p^2+4}}{2}$, $p = 1$ (golden ratio $\varphi$) has the following \textbf{extremal properties}:

\begin{enumerate}
\item \textbf{Worst approximable irrational}: By Hurwitz's theorem, for any irrational $\alpha$, there exist infinitely many integers $p, q$ with $|q\alpha - p| < 1/(\sqrt{5} q)$. The constant $\sqrt{5}$ is optimal for $\varphi$; all other irrationals have better approximations.
\item \textbf{Minimal operator norm}: In the metallic ratio family, $\|D_\Phi\|$ is minimal.
\item \textbf{Maximal zero spacing in Mellin symbol}: The spectral structure is simplest.
\end{enumerate}

\section{Kernel Dimension Structure and Gauge Group Constraints}
\label{sec:kernel_dimension}

Gauge group constraints are obtained from the \textbf{$\varphi$-dilation eigenvalue structure on the kernels of $D_5$ and $D_6$}.

\textbf{Two regimes}: The gauge group on each kernel depends on whether $\varphi$-dilation symmetry is preserved:

\begin{center}
\begin{tabular}{|c|c|c|c|}
\hline
$\varphi$-dilation & $\ker(D_5)$ (dim 2) & $\ker(D_6)$ (dim 3) & Physical identification \\
\hline
Preserving & $\mathrm{U}(1)^2$ (diagonal) & $\mathrm{U}(1)^3$ (diagonal) & Abelian theory \\
Breaking & $\mathrm{SU}(2)$ (non-diagonal) & $\mathrm{SU}(3)$ (non-diagonal) & Standard Model \\
\hline
\end{tabular}
\end{center}

\textbf{Scaling symmetry of polynomial variable} (preserves grading and product structure):

\begin{center}
\begin{tabular}{|c|c|c|c|}
\hline
Domain & Derivation & Gauge group & Physical identification \\
\hline
Polynomial variable $x$ & $x \to e^{i\theta}x$ & U(1) & Hypercharge $\mathrm{U}(1)_Y$ \\
\hline
\end{tabular}
\end{center}

Note: $x$ is not a spacetime coordinate but a variable of the polynomial ring $\mathbb{C}[x]$.

\subsection{Why SO(3) and SO(2) are Excluded}

Under the $\mathbb{R}$-formulation, $\mathrm{SO}(3)$ and $\mathrm{SO}(2)$ appeared as additional possibilities. The complex extension eliminates them:
\begin{itemize}
\item \textbf{SO(3)}: As a subgroup of the unitary group on $\mathbb{C}^3$, the real form $\mathrm{SO}(3)$ does not arise naturally. Since $\ker(D_6)$ is a complex vector space, the natural group is $\mathrm{SU}(3)$ (not $\mathrm{SO}(3)$).
\item \textbf{SO(2)}: $\mathrm{SO}(2) \cong \mathrm{U}(1)$, so it is already contained in $\mathrm{U}(1)^2$. It does not constitute an independent choice.
\item \textbf{SU(2)$\times$U(1)}: On $\ker(D_6)$, $\varphi$-dilation acts with 3 distinct eigenvalues $\{1, \varphi, \varphi+1\}$. An $\mathrm{SU}(2)\times\mathrm{U}(1)$ gauge group would require a 2-dimensional degenerate eigenspace, which does not exist.
\end{itemize}

\begin{theorem}[Kernel Dimensions]
\label{thm:kernel_dim}
\lean{FUST.ker\_D5\_dim, FUST.ker\_D6\_dim}
\leanok
\[
\ker(D_5) = \mathrm{span}\{1, x\} \quad (\dim = 2)
\]
\[
\ker(D_6) = \mathrm{span}\{1, x, x^2\} \quad (\dim = 3)
\]
\end{theorem}

\begin{proof}
Solve $D_5[x^c]/x^c = \varphi^{2c} + \varphi^c - 4 + \psi^c + \psi^{2c}$:
\begin{itemize}
\item $c = 0$: $1 + 1 - 4 + 1 + 1 = 0$ $\checkmark$
\item $c = 1$: $(\varphi^2 + \psi^2) + (\varphi + \psi) - 4 = 3 + 1 - 4 = 0$ $\checkmark$
\item $c = 2$: $(\varphi^4 + \psi^4) + (\varphi^2 + \psi^2) - 4 = 7 + 3 - 4 = 6 \neq 0$
\end{itemize}
Thus $\ker(D_5) = \mathrm{span}\{1, x\}$.

By Binet's formula, $D_6[x^c] \propto F_{3c} - 3F_{2c} + F_c$:
\begin{itemize}
\item $c = 0$: $0 - 0 + 0 = 0$ $\checkmark$
\item $c = 1$: $2 - 3 \cdot 1 + 1 = 0$ $\checkmark$
\item $c = 2$: $8 - 3 \cdot 3 + 1 = 0$ $\checkmark$
\item $c = 3$: $34 - 3 \cdot 8 + 2 = 12 \neq 0$
\end{itemize}
Thus $\ker(D_6) = \mathrm{span}\{1, x, x^2\}$.
\end{proof}

\textbf{$\varphi$-Dilation eigenvalues}:

\begin{theorem}[$\varphi$-Dilation Eigenvalues are Pairwise Distinct]
\label{thm:eigenvalues_distinct}
\lean{FUST.kerD6\_eigenvalues\_distinct}
\leanok
The $\varphi$-dilation eigenvalues on $\ker(D_6)$ are:
\[
\varphi^0 = 1, \quad \varphi^1 = \varphi, \quad \varphi^2 = \varphi + 1
\]
These are pairwise distinct since $\varphi > 1$.
\end{theorem}

\textbf{Kernel hierarchy}:
\[
\ker(D_2) \subset \ker(D_5) \subset \ker(D_6), \quad \dim: 1 \to 2 \to 3
\]

\section{Derivation of U(1)}
\label{sec:U1_derivation}

Scaling of the polynomial variable $x \to e^{i\theta} x$ acts on $\ker(D_6) = \mathrm{span}\{1, x, x^2\}$ as:
\[
1 \mapsto 1, \quad x \mapsto e^{i\theta} x, \quad x^2 \mapsto e^{2i\theta} x^2
\]
defining a $\mathrm{U}(1)$ representation with degrees $\{0, 1, 2\}$. This is \textbf{hypercharge $\mathrm{U}(1)_Y$}.

\textbf{Uniqueness}: The only 1-dimensional compact connected Lie group is $\mathrm{U}(1) \cong S^1$.

\section{Derivation of SU(2)}
\label{sec:SU2_derivation}

\subsection{su(2) Structure on Kernel of D5}

View $V_0 := \ker(D_5) = \mathrm{span}\{1, x\}$ as a 2-dimensional vector space. For the basis $|{\uparrow}\rangle = 1$, $|{\downarrow}\rangle = x$, define ladder operators:
\[
\sigma_+ := \frac{d}{dx}\Big|_{V_0}: x \mapsto 1, \; 1 \mapsto 0
\]
\[
\sigma_- := P_0 \circ M_x\Big|_{V_0}: 1 \mapsto x
\]
where $P_0$ is the projection onto $V_0$ and $M_x$ is multiplication by $x$.

\subsection{Computation of Commutation Relations}

\begin{align*}
[\sigma_+, \sigma_-](1) &= \sigma_+ \sigma_-(1) - \sigma_- \sigma_+(1) = \sigma_+(x) - \sigma_-(0) = 1 = |{\uparrow}\rangle \\
[\sigma_+, \sigma_-](x) &= \sigma_+ \sigma_-(x) - \sigma_- \sigma_+(x) = 0 - x = -|{\downarrow}\rangle
\end{align*}

Setting $\sigma_z := [\sigma_+, \sigma_-]$:
\[
\sigma_z |{\uparrow}\rangle = |{\uparrow}\rangle, \quad \sigma_z |{\downarrow}\rangle = -|{\downarrow}\rangle
\]

\subsection{Verification of su(2) Commutation Relations}

$[\sigma_z, \sigma_+](x) = \sigma_z(1) - \sigma_+(-x) = 1 + 1 = 2 = 2\sigma_+(x)$

Thus $[\sigma_z, \sigma_+] = 2\sigma_+$, $[\sigma_z, \sigma_-] = -2\sigma_-$, $[\sigma_+, \sigma_-] = \sigma_z$.

These are the \textbf{standard su(2) commutation relations}, giving the spin-1/2 representation.

\subsection{SU(2) as \texorpdfstring{$\varphi$}{φ}-Dilation Breaking}

The $\varphi$-dilation eigenvalues on $\ker(D_5) = \mathrm{span}\{1, x\}$ are $\{1, \varphi\}$, which are distinct. By the commutant theorem, the gauge group commuting with $\varphi$-dilation is $\mathrm{U}(1)^2$. The SU(2) gauge group \textbf{breaks} $\varphi$-dilation symmetry: it mixes the eigenspaces with distinct eigenvalues.

\section{Derivation of SU(3)}
\label{sec:SU3_derivation}

\subsection{Unitary Group on Kernel of D6}

View $V := \ker(D_6) = \mathrm{span}\{1, x, x^2\}$ as a 3-dimensional complex vector space. The unitary group on $V$ is $\mathrm{U}(3)$, the special unitary group is $\mathrm{SU}(3)$.

\begin{theorem}[Gauge Group from $\varphi$-Dilation Breaking]
\label{thm:gauge_space}
The $\varphi$-dilation eigenvalues $\{1, \varphi, \varphi+1\}$ on $\ker(D_6)$ are pairwise distinct. Therefore:
\begin{itemize}
\item $\varphi$-preserving gauge: $\mathrm{U}(1)^3$ (diagonal unitary, unique by commutant theorem)
\item $\varphi$-breaking gauge: $\mathrm{SU}(3)$ (non-diagonal unitary mixing 3 eigenspaces)
\end{itemize}
\end{theorem}

\subsection{Justification via Kernel Hierarchy}

From the kernel inclusion $\ker(D_2) \subset \ker(D_5) \subset \ker(D_6)$, three independent subspaces emerge naturally:
\[
V_0 := \ker(D_2) = \mathrm{span}\{1\}
\]
\[
V_1 := \ker(D_5) / \ker(D_2) \cong \mathrm{span}\{x\}
\]
\[
V_2 := \ker(D_6) / \ker(D_5) \cong \mathrm{span}\{x^2\}
\]

We have $\ker(D_6) = V_0 \oplus V_1 \oplus V_2$, and the \textbf{unitary transformations mixing these 3 directions} constitute $\mathrm{SU}(3)$.

\subsection{Symmetry Breaking by D5}

Eigenvalue decomposition of $D_5|_V$:
\[
D_5(1) = 0, \quad D_5(x) = 0, \quad D_5(x^2) = 6 \cdot x^2
\]

$D_5$ decomposes $V$ into eigenspaces $V_0 \oplus V_1$ and $V_2$. This corresponds to symmetry breaking $\mathrm{SU}(3) \to \mathrm{SU}(2) \times \mathrm{U}(1)$.

\section{Consistency of Direct Product Structure}
\label{sec:direct_product}

\begin{theorem}[Gauge Group Consistency Theorem]
\label{thm:gauge_consistency}
\lean{FUST.gauge\_group\_consistency}
\leanok
The kernel structure of the Frourio difference hierarchy is consistent with the gauge group
\[
G = \mathrm{SU}(3) \times \mathrm{SU}(2) \times \mathrm{U}(1)
\]
This is the $\varphi$-dilation breaking point in the gauge parameter space $\mathcal{G}$.
\end{theorem}

\begin{proof}
\begin{enumerate}
\item \textbf{U(1)}: Derived from natural scaling on $\ker(D_6)$. The only 1-dimensional compact connected Lie group is $\mathrm{U}(1)$.
\item \textbf{SU(2)}: su(2) algebra is naturally realized on $\ker(D_5) = \mathrm{span}\{1, x\}$. The only 3-dimensional compact simple Lie group is $\mathrm{SU}(2)$.
\item \textbf{SU(3)}: Unitary group mixing the 3 directions of $\ker(D_6) = V_0 \oplus V_1 \oplus V_2$. The only 8-dimensional compact simple Lie group is $\mathrm{SU}(3)$.
\item \textbf{Direct product structure}: Guaranteed by the following algebraic mechanisms:
   \begin{itemize}
   \item $[D_5, D_6] = 0$ (commutativity of operators)
   \item PBW-type standard form: $\mathcal{F}(A) \cong (A \rtimes k[D_\infty]) \oplus (A \rtimes k[D_\infty])\Delta$
   \item Ore exchange relation $\Delta M_a - M_{\sigma_u(a)}\Delta = \tilde{\delta}(a)$ resolves critical pairs
   \item Kernel hierarchy $\ker(D_2) \subset \ker(D_5) \subset \ker(D_6)$ gives an \textbf{increasing sequence of dimensions $1 \to 2 \to 3$}, guaranteeing independence of each group
   \end{itemize}
\end{enumerate}
\end{proof}

\subsection{Dimension Matching}

\begin{center}
\begin{tabular}{|c|c|c|c|}
\hline
Origin & Derivation & Gauge group & Dimension \\
\hline
$\ker(D_6)$ scaling & dim 1 connected compact group & U(1) & 1 \\
su(2) on $\ker(D_5)$ & dim 3 simple compact group & SU(2) & 3 \\
$\ker(D_6) = V_0 \oplus V_1 \oplus V_2$ & dim 8 simple compact group & SU(3) & 8 \\
\hline
\end{tabular}
\end{center}

Total: $1 + 3 + 8 = 12$ dimensions (matching the Standard Model gauge group).

\section{Gauge Parameter Space and \texorpdfstring{$\varphi$}{φ}-Dilation Classification}
\label{sec:phi_dilation_classification}

The $\varphi$-dilation commutant theorem reduces the gauge parameter space from 12 (under $\mathbb{R}$) to 4 sectors:

\begin{center}
\begin{tabular}{|c|c|c|c|}
\hline
Sector & $G_6$ ($\ker D_6$) & $G_5$ ($\ker D_5$) & Physical theory \\
\hline
1 & $\mathrm{SU}(3)$ ($\varphi$-breaking) & $\mathrm{SU}(2)$ ($\varphi$-breaking) & Standard Model \\
2 & $\mathrm{U}(1)^3$ ($\varphi$-preserving) & $\mathrm{U}(1)^2$ ($\varphi$-preserving) & Fully abelian \\
3 & $\mathrm{SU}(3)$ ($\varphi$-breaking) & $\mathrm{U}(1)^2$ ($\varphi$-preserving) & Mixed I \\
4 & $\mathrm{U}(1)^3$ ($\varphi$-preserving) & $\mathrm{SU}(2)$ ($\varphi$-breaking) & Mixed II \\
\hline
\end{tabular}
\end{center}

$G_1 = \mathrm{U}(1)$ is unique in all sectors (no choice).

\textbf{Key structural result}: Each gauge choice is a binary decision (preserve or break $\varphi$-dilation on each kernel), and the 4 sectors are the only mathematically consistent possibilities.

\subsection{Selection by Describability}

All 4 sectors are mathematically consistent, but \textbf{the sectors where ``an intelligence capable of describing this theory'' can exist are limited}:
\begin{itemize}
\item Without \textbf{confinement} (non-abelian nature of SU(3)), stable hadrons cannot form $\Rightarrow$ no atomic nuclei
\item Without \textbf{electroweak symmetry} (SU(2)$\times$U(1)), stable electron orbits enabling chemical bonds cannot exist
\end{itemize}

\begin{center}
\begin{tabular}{|c|c|c|}
\hline
Sector & Mathematical consistency & Possibility of describer \\
\hline
$(\mathrm{SU}(3), \mathrm{SU}(2), \mathrm{U}(1))$ & $\checkmark$ & $\checkmark$ (confinement + stable atoms) \\
$(\mathrm{U}(1)^3, \mathrm{U}(1)^2, \mathrm{U}(1))$ & $\checkmark$ & $\times$ (no confinement) \\
$(\mathrm{SU}(3), \mathrm{U}(1)^2, \mathrm{U}(1))$ & $\checkmark$ & $\times$ (no electroweak) \\
$(\mathrm{U}(1)^3, \mathrm{SU}(2), \mathrm{U}(1))$ & $\checkmark$ & $\times$ (no confinement) \\
\hline
\end{tabular}
\end{center}

\textbf{FUST's claim}: All 4 sectors are mathematically equivalent, but the fact that ``FUST is being described'' is only valid in a sector where describers (observers) exist. This is not the physical anthropic principle but \textbf{logical self-consistency}.

\section{Gauge Boson Mass Mechanism from Kernel Hierarchy}
\label{sec:gauge_mass_mechanism}

The kernel inclusion $\ker(D_5) \subsetneq \ker(D_6)$ provides a structural mass mechanism
for gauge bosons, playing the role of the Higgs mechanism in the Standard Model.

\subsection{The Higgs Degree of Freedom}

The key observation is:
\[
x^2 \in \ker(D_6) \setminus \ker(D_5): \quad D_6(x^2) = 0 \text{ but } D_5(x^2) \neq 0.
\]
This extra degree of freedom $x^2$ acts as the \textbf{Higgs field}: it belongs to the color sector
($\ker D_6$) but not the weak sector ($\ker D_5$).

\subsection{Mass/Massless Classification}

\begin{center}
\begin{tabular}{|c|c|c|c|}
\hline
Gauge boson & Symmetry origin & Mass mechanism & FDim \\
\hline
Gluon (8) & $\ker(D_6)$, $\mathrm{SU}(3)$ & $D_6 = 0$ $\to$ massless & --- \\
Photon & Grading, $\mathrm{U}(1)$ & all $D_m$ preserve $\to$ massless & --- \\
$W^\pm$ & $\ker(D_5)$, $\mathrm{SU}(2)$ & $D_5(x^2) \neq 0$ $\to$ massive & $(20,-24,-26)$ \\
$Z$ & $\ker(D_5)$, $\mathrm{SU}(2)$ & DimSum2 via $\theta_W$ & $(40,...) \oplus (42,...)$ \\
$H$ & $\ker(D_6) \setminus \ker(D_5)$ & Higgs DOF & $(21,...) \oplus (18,...)$ \\
DM & $D_{5\frac{1}{2}}$ sector & $D_{5\frac{1}{2}}[x] \neq 0$ $\to$ massive & $(21,-25,-24)$ \\
Graviton & $D_6$ gravity sector & $\Box_\varphi[t^{-1}] = 0$ $\to$ massless & --- \\
\hline
\end{tabular}
\end{center}

The W boson mass is quantitatively determined:
\[
\frac{m_W}{m_e} = \varphi^{C(5,2)+C(6,2)} \times \frac{C(6,2)}{C(6,2)+C(2,2)} = \varphi^{25} \times \frac{15}{16}
\]
with 0.002\% agreement against PDG data.
$m_Z^2$ and $m_H$ are \texttt{DimSum2}: formal sums of ScaleQ with different FDim,
whose $\mathbb{R}$-evaluation gives the physical mass.
See Chapter~\ref{chap:pd_mass_ratios} for details.
