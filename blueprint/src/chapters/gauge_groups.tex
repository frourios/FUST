% Chapter 7: Standard Gauge Group from Dζ
\chapter{Standard Gauge Group from \texorpdfstring{$D_\zeta$}{Dζ}}
\label{chap:gauge_groups}

\section{Overview: Unique Derivation from \texorpdfstring{$D_\zeta$}{Dζ}}
\label{sec:gauge_overview}

This chapter proves that the Standard Model gauge group $\mathrm{SU}(3) \times \mathrm{SU}(2) \times \mathrm{U}(1)$ is \textbf{uniquely determined} by the unified operator $D_\zeta$, with no free parameters or selection principles. The derivation uses the $\mathbb{Z}/6\mathbb{Z}$ channel decomposition of $D_\zeta$ and the kernel hierarchy of its component operators.

\begin{theorem}[Standard Gauge Group Uniqueness]
\label{thm:standard_gauge_unique}
\lean{FUST.standard_gauge_group_unique}
\leanok
The following properties of $D_\zeta$ uniquely determine the gauge group $\mathrm{SU}(3) \times \mathrm{SU}(2) \times \mathrm{U}(1)$:
\begin{enumerate}
\item $\Phi_S$ has rank 3 (3 linearly independent sub-operators) $\Rightarrow$ $\mathrm{SU}(3)$
\item $\mathrm{AF\_coeff} = 2i\sqrt{3} \neq 0$ is purely imaginary $\Rightarrow$ $\mathrm{SU}(2)$
\item $\ker(D_2) = \mathrm{span}\{1\}$ is 1-dimensional $\Rightarrow$ $\mathrm{U}(1)$
\item $\zeta_6 \neq \varphi^n$ for all $n$ (ensures non-reducibility)
\item $|6a + \mathrm{AF\_coeff} \cdot b|^2 = 12(3a^2 + b^2)$ (weight ratio 3:1)
\end{enumerate}
\end{theorem}

\section{\texorpdfstring{$\zeta_6$}{ζ6} Separation from \texorpdfstring{$\varphi$}{φ}-Eigenspectrum}
\label{sec:zeta6_separation}

\begin{theorem}[$\zeta_6 \neq \varphi^n$]
\label{thm:zeta6_ne_phi}
\lean{FUST.zeta6_ne_phi_pow}
\leanok
For all $n \in \mathbb{N}$, $\zeta_6 \neq \varphi^n$, since $\mathrm{Im}(\zeta_6) = \sqrt{3}/2 \neq 0$ while $\varphi^n \in \mathbb{R}$.
\end{theorem}

This separation guarantees that $\zeta_6$-rotation mixes $\varphi$-dilation eigenspaces, preventing gauge group reduction. On $\ker(D_6)$ with basis $\{1, z, z^2\}$, the $\varphi$-dilation eigenvalues are $\{1, \varphi, \varphi^2\}$ (all real and distinct), but $\zeta_6$ acts non-diagonally on this basis, providing the off-diagonal mixing necessary for non-abelian gauge structure.

\section{Channel Non-Degeneracy}
\label{sec:channel_nondegeneracy}

\begin{theorem}[AF Channel is Non-Degenerate]
\label{thm:AF_nondegenerate}
\lean{FUST.AF_channel_nondegenerate}
\leanok
$\mathrm{AF\_coeff} \neq 0$, since $|\mathrm{AF\_coeff}|^2 = 12 \neq 0$.
\end{theorem}

\begin{theorem}[AF Channel is Purely Imaginary]
\label{thm:AF_imaginary}
\lean{FUST.AF_coeff_re_zero}
\leanok
$\mathrm{Re}(\mathrm{AF\_coeff}) = 0$, i.e., $\mathrm{AF\_coeff} = 2i\sqrt{3}$.
\end{theorem}

The AF channel being purely imaginary means it acts \textbf{orthogonally} to the real $\varphi$-dilation diagonal. On $\ker(D_5) = \mathrm{span}\{1, z\}$, the AF direction provides the off-diagonal generator needed for $\mathrm{su}(2)$.

\section{Kernel Hierarchy and Dimensions}
\label{sec:kernel_hierarchy}

\begin{theorem}[Kernel Hierarchy]
\label{thm:kernel_hierarchy}
\lean{FUST.kernel_hierarchy}
\leanok
\[
\ker(D_2) \subset \ker(D_5) \subset \ker(D_6), \quad \dim: 1 \to 2 \to 3
\]
\begin{itemize}
\item $D_2[1] = 0$, $D_2[z] \neq 0$ ($\ker(D_2) = \mathrm{span}\{1\}$, $\dim = \mathrm{card}(\mathrm{Fin}\,1) = 1$)
\item $D_5[1] = D_5[z] = 0$, $D_5[z^2] \neq 0$ ($\ker(D_5) = \mathrm{span}\{1, z\}$, $\dim = \mathrm{card}(\mathrm{Fin}\,2) = 2$)
\item $D_6[1] = D_6[z] = D_6[z^2] = 0$, $D_6[z^3] \neq 0$ ($\ker(D_6) = \mathrm{span}\{1, z, z^2\}$, $\dim = \mathrm{card}(\mathrm{Fin}\,3) = 3$)
\end{itemize}
\end{theorem}

\section{\texorpdfstring{$\varphi$}{φ}-Dilation Eigenvalue Structure}
\label{sec:phi_dilation}

\begin{theorem}[$\varphi$-Dilation Eigenvalues are Pairwise Distinct]
\label{thm:eigenvalues_distinct}
\lean{FUST.kerD6_eigenvalues_distinct}
\leanok
On $\ker(D_6)$, the $\varphi$-dilation $U_\varphi: f(z) \mapsto f(\varphi z)$ has eigenvalues:
\[
\varphi^0 = 1, \quad \varphi^1 = \varphi, \quad \varphi^2 = \varphi + 1
\]
These are pairwise distinct since $\varphi > 1$.
\end{theorem}

\begin{theorem}[$\varphi$-Dilation Commutant is Diagonal]
\label{thm:commutant_diagonal}
\lean{FUST.phiDilation_commutant_diagonal}
\leanok
Let $S = \mathrm{diag}(1, \varphi, \varphi^2)$ be the scaling matrix on $\ker(D_6)$. If $M \in \mathrm{GL}(3, \mathbb{C})$ satisfies $MS = SM$, then $M$ is diagonal.
\end{theorem}

\begin{proof}
From $MS = SM$, entry-wise: $M_{ij} \cdot \varphi^j = \varphi^i \cdot M_{ij}$, so $M_{ij}(\varphi^j - \varphi^i) = 0$. Since eigenvalues are pairwise distinct, $M_{ij} = 0$ for $i \neq j$.
\end{proof}

\section{Symmetric Channel \texorpdfstring{$\Phi_S \to \mathrm{SU}(3)$}{Φ_S → SU(3)}}
\label{sec:SU3_derivation}

\begin{theorem}[Symmetric Channel Determines SU(3)]
\label{thm:symmetric_SU3}
\lean{FUST.Zeta6.Phi_S_rank_three}
\leanok
$\Phi_S = 2 \cdot N_5 + N_3 + \mu \cdot N_2$ decomposes into 3 linearly independent sub-operators.
\begin{itemize}
\item The $3 \times 3$ determinant of $(\sigma_{N_5}, \sigma_{N_3}, \sigma_{N_2})$ at $s = 1, 5, 7$ equals $-6952(\varphi - \psi) \neq 0$
\item These 3 independent components act on $\ker(D_6) = \mathrm{span}\{1, z, z^2\}$ (dimension 3)
\item Rank $=$ dim $\Rightarrow$ irreducible action on a 3-dimensional space
\item $\zeta_6 \neq \varphi^n$ ensures $\zeta_6$-rotation mixes $\varphi$-eigenspaces non-diagonally
\item The unique compact simple Lie group acting irreducibly on $\mathbb{C}^3$ is $\mathrm{SU}(3)$
\end{itemize}
\end{theorem}

$\mathrm{SU}(3)$ has dimension $3^2 - 1 = 8$ (8 gluons).

\section{Antisymmetric Channel \texorpdfstring{$\Phi_A \to \mathrm{SU}(2)$}{Φ_A → SU(2)}}
\label{sec:SU2_derivation}

\begin{theorem}[Antisymmetric Channel Determines SU(2)]
\label{thm:antisymmetric_SU2}
\lean{FUST.AF_channel_nondegenerate, FUST.AF_coeff_re_zero}
\leanok
\begin{itemize}
\item $\mathrm{AF\_coeff} = 2i\sqrt{3}$ is nonzero and purely imaginary
\item On $\ker(D_5) = \mathrm{span}\{1, z\}$ (dimension 2), $\varphi$-dilation has eigenvalues $\{1, \varphi\}$ (diagonal, real)
\item The AF direction provides the off-diagonal, purely imaginary generator
\item One real direction ($\varphi$-dilation Cartan) $+$ one imaginary direction (AF root) $=$ $\mathrm{su}(2)$ algebra
\item The unique compact simple Lie group on $\mathbb{C}^2$ is $\mathrm{SU}(2)$
\end{itemize}
\end{theorem}

$\mathrm{SU}(2)$ has dimension $2^2 - 1 = 3$ ($W^\pm, Z$ bosons).

\subsection{su(2) Commutation Relations}

View $V_0 := \ker(D_5) = \mathrm{span}\{1, z\}$ as a 2-dimensional space with basis $|{\uparrow}\rangle = 1$, $|{\downarrow}\rangle = z$.

Ladder operators: $\sigma_+ := d/dz|_{V_0}$, $\sigma_- := P_0 \circ M_z|_{V_0}$, $\sigma_z := [\sigma_+, \sigma_-]$.

These satisfy $[\sigma_z, \sigma_+] = 2\sigma_+$, $[\sigma_z, \sigma_-] = -2\sigma_-$, $[\sigma_+, \sigma_-] = \sigma_z$ --- the standard $\mathrm{su}(2)$ commutation relations (spin-1/2).

\section{Trivial Channel \texorpdfstring{$\to \mathrm{U}(1)$}{→ U(1)}}
\label{sec:U1_derivation}

\begin{theorem}[Trivial Channel Determines U(1)]
\label{thm:trivial_U1}
On $\ker(D_2) = \mathrm{span}\{1\}$ (dimension 1), the only compact connected gauge group is $\mathrm{U}(1)$.
\end{theorem}

\section{Weight Ratio and Spacetime Structure}
\label{sec:weight_ratio}

\begin{theorem}[$D_\zeta$ Norm-Squared Decomposition]
\label{thm:normSq_decomposition}
\lean{FUST.Zeta6.Dzeta_normSq_decomposition}
\leanok
\[
|6a + \mathrm{AF\_coeff} \cdot b|^2 = 12(3a^2 + b^2)
\]
\end{theorem}

The weight ratio $3 : 1$ between symmetric ($a$, SU(3)) and antisymmetric ($b$, SU(2)) channels encodes the spacetime decomposition $I_4 = \mathrm{Fin}\,3 \oplus \mathrm{Fin}\,1$ (3 spatial + 1 temporal dimension).

\section{Dimension Matching}
\label{sec:dimension_matching}

\begin{center}
\begin{tabular}{|c|c|c|c|c|}
\hline
$D_\zeta$ channel & Kernel & $\dim$ & Gauge group & $\dim(G)$ \\
\hline
Symmetric ($\Phi_S$) & $\ker(D_6) = \{1, z, z^2\}$ & 3 & SU(3) & 8 \\
Antisymmetric ($\Phi_A$) & $\ker(D_5) = \{1, z\}$ & 2 & SU(2) & 3 \\
Trivial & $\ker(D_2) = \{1\}$ & 1 & U(1) & 1 \\
\hline
\multicolumn{4}{|r|}{Total gauge dimensions:} & 12 \\
\hline
\end{tabular}
\end{center}

$12 = 8 + 3 + 1$: matching the 12 gauge bosons of the Standard Model (8 gluons $+ W^\pm + Z +$ photon).

\section{Gauge Boson Mass Mechanism}
\label{sec:gauge_mass_mechanism}

The kernel inclusion $\ker(D_5) \subsetneq \ker(D_6)$ provides a structural mass mechanism:

\[
z^2 \in \ker(D_6) \setminus \ker(D_5): \quad D_6(z^2) = 0 \text{ but } D_5(z^2) \neq 0.
\]

This extra degree of freedom $z^2$ acts as the \textbf{Higgs field}: it belongs to $\mathrm{SU}(3)$ ($\ker D_6$) but not to $\mathrm{SU}(2)$ ($\ker D_5$), breaking electroweak symmetry.

\begin{center}
\begin{tabular}{|c|c|c|c|}
\hline
Gauge boson & Origin in $D_\zeta$ & Mass mechanism & Mass \\
\hline
Gluon (8) & $\Phi_S$ on $\ker(D_6)$ & $D_6 = 0$ & massless \\
Photon & U(1) grading & All $D_m$ preserve & massless \\
$W^\pm$ & $\Phi_A$ on $\ker(D_5)$ & $D_5(z^2) \neq 0$ & massive \\
$Z$ & $\Phi_A$ on $\ker(D_5)$ & via $\theta_W$ mixing & massive \\
$H$ & $\ker(D_6) \setminus \ker(D_5)$ & Higgs DOF & massive \\
\hline
\end{tabular}
\end{center}

\section{Extremal Properties of the Golden Ratio}
\label{sec:extremal_properties}

In the metallic ratio family $\Lambda_p = \frac{p + \sqrt{p^2+4}}{2}$, $p = 1$ (golden ratio $\varphi$) has:

\begin{enumerate}
\item \textbf{Worst approximable irrational}: By Hurwitz's theorem, the constant $\sqrt{5}$ is optimal for $\varphi$
\item \textbf{Minimal operator norm}: $\|D_\Phi\|$ is minimal in the metallic ratio family
\item \textbf{Maximal zero spacing in Mellin symbol}: Simplest spectral structure
\end{enumerate}

\section{Infinite Dihedral Group}
\label{sec:dihedral}

The symmetry group of Frourio algebra is $D_\infty = \langle u, r \mid r^2 = 1,\; rur = u^{-1} \rangle$
where $U: f(z) \mapsto f(\varphi z)$ and $R: f(z) \mapsto f(-z)$. The Weyl reflection $u \mapsto u^{-1}$ corresponds to the Weyl group $\mathbb{Z}_2$ of $\mathrm{SU}(2)$.

\section{Summary}
\label{sec:gauge_summary}

\textbf{No free parameters}: The gauge group $\mathrm{SU}(3) \times \mathrm{SU}(2) \times \mathrm{U}(1)$ is the unique output of $D_\zeta$'s algebraic structure. The derivation uses:
\begin{itemize}
\item $\Phi_S$ rank 3 (from $D_\zeta$'s $\mathbb{Z}/6\mathbb{Z}$ symmetric channel)
\item AF$\_$coeff $= 2i\sqrt{3}$ (from $D_\zeta$'s antisymmetric channel)
\item Kernel hierarchy $1 \to 2 \to 3$ (from $D_2 \subset D_5 \subset D_6$)
\item $\zeta_6 \neq \varphi^n$ (elliptic-hyperbolic separation)
\item $|6a + 2i\sqrt{3} \cdot b|^2 = 12(3a^2 + b^2)$ (spacetime weight ratio)
\end{itemize}
