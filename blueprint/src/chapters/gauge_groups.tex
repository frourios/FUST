% Chapter 7: Gauge Groups and Parameter Space
\chapter{Gauge Groups and Parameter Space}
\label{chap:gauge_groups}

\section{Mathematical Structure of the Gauge Sector}
\label{sec:gauge_sector}

This chapter analyzes the \textbf{gauge group parameter space} defined by FUST's kernel structure and its relationship to observational constraints.

\begin{theorem}[Gauge Parameter Space]
\label{thm:gauge_parameter_space}
\lean{FUST.gauge_parameter_space}
\leanok
The kernel dimensions $\dim(\ker D_5) = 2$ and $\dim(\ker D_6) = 3$ define a gauge group parameter space $\mathcal{G}$:

\begin{center}
\begin{tabular}{|c|c|c|}
\hline
Kernel & Dimension & Admissible gauge groups \\
\hline
$\ker(D_5)$ & 2 & $G_5 \in \{\mathrm{SU}(2), \mathrm{U}(1)^2, \mathrm{SO}(2)\}$ \\
$\ker(D_6)$ & 3 & $G_6 \in \{\mathrm{SU}(3), \mathrm{U}(1)^3, \mathrm{SU}(2) \times \mathrm{U}(1), \mathrm{SO}(3)\}$ \\
Scaling & 1 & $G_1 = \mathrm{U}(1)$ \\
\hline
\end{tabular}
\end{center}
\end{theorem}

\textbf{FUST's mathematical claim}: All elements of the parameter space $\mathcal{G} = G_6 \times G_5 \times G_1$ are mathematically admissible.

\textbf{Observational constraints}: Cosmological data (CMB temperature $T = 2.725 \pm 0.001$ K, spectral index $n_s = 0.965 \pm 0.004$, baryon density ratio $\Omega_b/\Omega_m = 0.157 \pm 0.004$) specify a single point in $\mathcal{G}$.

\subsection{Infinite Dihedral Group and Weyl Reflection}

The fundamental group of Frourio algebra is:
\[
D_\infty = \langle u, r \mid r^2 = 1,\; rur = u^{-1} \rangle
\]
where $U: f(x) \mapsto f(\varphi x)$ (scale transformation) and $R: f(x) \mapsto f(-x)$ (inversion).

The relation $rur = u^{-1}$ represents the \textbf{Weyl reflection} $u \mapsto u^{-1}$, corresponding to the Weyl group $\mathbb{Z}_2$ of $\mathrm{SU}(2)$.

\subsection{Phase Quantization and Half-Integer Spin}

For any positive number $a > 0$, $a^{i\pi/\log a} = e^{i\pi} = -1$ holds. This is not unique to the golden ratio.

\textbf{Why $\varphi$ is special}: The reason lies in its extremal properties (see below). The meaning of phase quantization is as follows:

For the scale generator $u: f(x) \mapsto f(\varphi x)$ of $D_\infty$, consider the complex power $u^s = \varphi^s$. The equality $u^{i\pi/\log\varphi} = -1$ implies phase quantization at angle $\theta = \pi$.

\textbf{Consequence: Selection of half-integer spin}

$-1 = e^{i\pi}$ implies representations that change sign under $2\pi$ rotation (fermions):
\begin{itemize}
\item In the spin-$j$ representation of $\mathrm{SU}(2)$, the eigenvalue of $e^{2\pi i J_z}$ is $e^{2\pi i m}$ ($m = -j, \ldots, j$)
\item For $j = 1/2$, $m = \pm 1/2$ gives $e^{\pi i} = -1$
\end{itemize}

\subsection{Extremal Properties of the Golden Ratio}

In the metallic ratio family $\Lambda_p = \frac{p + \sqrt{p^2+4}}{2}$, $p = 1$ (golden ratio $\varphi$) has the following \textbf{extremal properties}:

\begin{enumerate}
\item \textbf{Worst approximable irrational}: By Hurwitz's theorem, for any irrational $\alpha$, there exist infinitely many integers $p, q$ with $|q\alpha - p| < 1/(\sqrt{5} q)$. The constant $\sqrt{5}$ is optimal for $\varphi$; all other irrationals have better approximations.
\item \textbf{Minimal operator norm}: In the metallic ratio family, $\|D_\Phi\|$ is minimal.
\item \textbf{Maximal zero spacing in Mellin symbol}: The spectral structure is simplest.
\end{enumerate}

\section{Kernel Dimension Structure and Gauge Group Constraints}
\label{sec:kernel_dimension}

Gauge group constraints are obtained from the \textbf{kernel dimension structure of $D_5$ and $D_6$}.

\textbf{Unifying principle}: The derivation of gauge groups involves two types of symmetry:

\textbf{Internal symmetry} (linear transformations on kernels as vector spaces):

\begin{center}
\begin{tabular}{|c|c|c|c|c|}
\hline
Domain & $\dim$ & Derivation & Gauge group & Physical identification \\
\hline
$\ker(D_5)$ & 2 & Ladder operators $\to$ su(2) & SU(2) & Weak isospin \\
$\ker(D_6)$ & 3 & Unitary mixing $\to$ su(3) & SU(3) & Color \\
\hline
\end{tabular}
\end{center}

\textbf{Scaling symmetry of polynomial variable} (preserves grading and product structure):

\begin{center}
\begin{tabular}{|c|c|c|c|}
\hline
Domain & Derivation & Gauge group & Physical identification \\
\hline
Polynomial variable $x$ & $x \to e^{i\theta}x$ & U(1) & Hypercharge $\mathrm{U}(1)_Y$ \\
\hline
\end{tabular}
\end{center}

Note: $x$ is not a spacetime coordinate but a variable of the polynomial ring $\mathbb{C}[x]$.

\textbf{Preservation of product structure}:
\begin{itemize}
\item SU(2) acts on $\ker(D_5)$ mixing bases
\item SU(3) acts on $\ker(D_6)$ mixing bases
\item U(1) scales the polynomial variable (different phases for different degrees)
\item Scaling is diagonal, linear transformations mix bases $\Rightarrow$ commutative $\Rightarrow$ direct product $\mathrm{SU}(3) \times \mathrm{SU}(2) \times \mathrm{U}(1)$
\end{itemize}

\begin{theorem}[Kernel Dimensions]
\label{thm:kernel_dim}
\lean{FUST.ker_D5_dim, FUST.ker_D6_dim}
\leanok
\[
\ker(D_5) = \mathrm{span}\{1, x\} \quad (\dim = 2)
\]
\[
\ker(D_6) = \mathrm{span}\{1, x, x^2\} \quad (\dim = 3)
\]
\end{theorem}

\begin{proof}
Solve $D_5[x^c]/x^c = \varphi^{2c} + \varphi^c - 4 + \psi^c + \psi^{2c}$:
\begin{itemize}
\item $c = 0$: $1 + 1 - 4 + 1 + 1 = 0$ \checkmark
\item $c = 1$: $(\varphi^2 + \psi^2) + (\varphi + \psi) - 4 = 3 + 1 - 4 = 0$ \checkmark
\item $c = 2$: $(\varphi^4 + \psi^4) + (\varphi^2 + \psi^2) - 4 = 7 + 3 - 4 = 6 \neq 0$
\end{itemize}
Thus $\ker(D_5) = \mathrm{span}\{1, x\}$.

By Binet's formula, $D_6[x^c] \propto F_{3c} - 3F_{2c} + F_c$:
\begin{itemize}
\item $c = 0$: $0 - 0 + 0 = 0$ \checkmark
\item $c = 1$: $2 - 3 \cdot 1 + 1 = 0$ \checkmark
\item $c = 2$: $8 - 3 \cdot 3 + 1 = 0$ \checkmark
\item $c = 3$: $34 - 3 \cdot 8 + 2 = 12 \neq 0$
\end{itemize}
Thus $\ker(D_6) = \mathrm{span}\{1, x, x^2\}$.
\end{proof}

\textbf{Kernel hierarchy}:
\[
\ker(D_2) \subset \ker(D_5) \subset \ker(D_6), \quad \dim: 1 \to 2 \to 3
\]

\section{Derivation of U(1)}
\label{sec:U1_derivation}

Scaling of the polynomial variable $x \to e^{i\theta} x$ acts on $\ker(D_6) = \mathrm{span}\{1, x, x^2\}$ as:
\[
1 \mapsto 1, \quad x \mapsto e^{i\theta} x, \quad x^2 \mapsto e^{2i\theta} x^2
\]
defining a $\mathrm{U}(1)$ representation with degrees $\{0, 1, 2\}$. This is \textbf{hypercharge $\mathrm{U}(1)_Y$}.

\textbf{Uniqueness}: The only 1-dimensional compact connected Lie group is $\mathrm{U}(1) \cong S^1$.

\section{Derivation of SU(2)}
\label{sec:SU2_derivation}

\subsection{su(2) Structure on Kernel of D5}

View $V_0 := \ker(D_5) = \mathrm{span}\{1, x\}$ as a 2-dimensional vector space. For the basis $|{\uparrow}\rangle = 1$, $|{\downarrow}\rangle = x$, define ladder operators:
\[
\sigma_+ := \frac{d}{dx}\Big|_{V_0}: x \mapsto 1, \; 1 \mapsto 0
\]
\[
\sigma_- := P_0 \circ M_x\Big|_{V_0}: 1 \mapsto x
\]
where $P_0$ is the projection onto $V_0$ and $M_x$ is multiplication by $x$.

\subsection{Computation of Commutation Relations}

\begin{align*}
[\sigma_+, \sigma_-](1) &= \sigma_+ \sigma_-(1) - \sigma_- \sigma_+(1) = \sigma_+(x) - \sigma_-(0) = 1 = |{\uparrow}\rangle \\
[\sigma_+, \sigma_-](x) &= \sigma_+ \sigma_-(x) - \sigma_- \sigma_+(x) = 0 - x = -|{\downarrow}\rangle
\end{align*}

Setting $\sigma_z := [\sigma_+, \sigma_-]$:
\[
\sigma_z |{\uparrow}\rangle = |{\uparrow}\rangle, \quad \sigma_z |{\downarrow}\rangle = -|{\downarrow}\rangle
\]

\subsection{Verification of su(2) Commutation Relations}

$[\sigma_z, \sigma_+](x) = \sigma_z(1) - \sigma_+(-x) = 1 + 1 = 2 = 2\sigma_+(x)$

Thus $[\sigma_z, \sigma_+] = 2\sigma_+$, $[\sigma_z, \sigma_-] = -2\sigma_-$, $[\sigma_+, \sigma_-] = \sigma_z$.

These are the \textbf{standard su(2) commutation relations}, giving the spin-1/2 representation.

\subsection{Why SU(2) and not SO(2)}

From $\dim \ker(D_5) = 2 = 2j + 1$, we have $j = 1/2$ (half-integer spin). Frourio's identity $\varphi^{i\pi/\log\varphi} = -1$ selects half-integer spin, so $\mathrm{Spin}(3) \cong \mathrm{SU}(2)$ is uniquely determined rather than $\mathrm{SO}(2)$.

\section{Derivation of SU(3)}
\label{sec:SU3_derivation}

\subsection{Unitary Group on Kernel of D6}

View $V := \ker(D_6) = \mathrm{span}\{1, x, x^2\}$ as a 3-dimensional complex vector space. The unitary group on $V$ is $\mathrm{U}(3)$, the special unitary group is $\mathrm{SU}(3)$.

\begin{theorem}[Gauge Group Parameter Space Theorem]
\label{thm:gauge_space}
In FUST, the unitary symmetry on $\ker(D_6)$ constitutes $\mathrm{U}(3) = \mathrm{SU}(3) \times \mathrm{U}(1)$, where both SU(3) and U(3) are admissible points in the parameter space.
\end{theorem}

\subsection{Justification via Kernel Hierarchy}

From the kernel inclusion $\ker(D_2) \subset \ker(D_5) \subset \ker(D_6)$, three independent subspaces emerge naturally:
\[
V_0 := \ker(D_2) = \mathrm{span}\{1\}
\]
\[
V_1 := \ker(D_5) / \ker(D_2) \cong \mathrm{span}\{x\}
\]
\[
V_2 := \ker(D_6) / \ker(D_5) \cong \mathrm{span}\{x^2\}
\]

We have $\ker(D_6) = V_0 \oplus V_1 \oplus V_2$, and the \textbf{unitary transformations mixing these 3 directions} constitute $\mathrm{SU}(3)$.

\subsection{Symmetry Breaking by D5}

Eigenvalue decomposition of $D_5|_V$:
\[
D_5(1) = 0, \quad D_5(x) = 0, \quad D_5(x^2) = 6 \cdot x^2
\]

$D_5$ decomposes $V$ into eigenspaces $V_0 \oplus V_1$ and $V_2$. This corresponds to symmetry breaking $\mathrm{SU}(3) \to \mathrm{SU}(2) \times \mathrm{U}(1)$.

\subsection{Why SU(3) and not SO(3)}

Frourio's identity $\varphi^{i\pi/\log\varphi} = -1$ selects complex representations (half-integer spin). The Weyl group of $\ker(D_6)$ is $S_3$ (permutation group of 3 elements), coinciding with the Weyl group of $\mathrm{SU}(3)$. The structure of $D_\infty$ and Frourio's identity prioritize $\mathrm{SU}(3)$ over the real group $\mathrm{SO}(3)$.

\section{Consistency of Direct Product Structure}
\label{sec:direct_product}

\begin{theorem}[Gauge Group Consistency Theorem]
\label{thm:gauge_consistency}
\lean{FUST.gauge_group_consistency}
\leanok
The kernel structure of the Frourio difference hierarchy is consistent with the gauge group
\[
G = \mathrm{SU}(3) \times \mathrm{SU}(2) \times \mathrm{U}(1)
\]
Other points in $\mathcal{G}$ (e.g., $G' = \mathrm{U}(1)^3 \times \mathrm{U}(1)^2 \times \mathrm{U}(1)$) are mathematically admissible, but the describability condition transcendentally selects the Standard Model point uniquely.
\end{theorem}

\begin{proof}
\begin{enumerate}
\item \textbf{U(1)}: Derived from natural scaling on $\ker(D_6)$. The only 1-dimensional compact connected Lie group is $\mathrm{U}(1)$.
\item \textbf{SU(2)}: su(2) algebra is naturally realized on $\ker(D_5) = \mathrm{span}\{1, x\}$. The only 3-dimensional compact simple Lie group is $\mathrm{SU}(2)$.
\item \textbf{SU(3)}: Unitary group mixing the 3 directions of $\ker(D_6) = V_0 \oplus V_1 \oplus V_2$. The only 8-dimensional compact simple Lie group is $\mathrm{SU}(3)$.
\item \textbf{Direct product structure}: Guaranteed by the following algebraic mechanisms:
   \begin{itemize}
   \item $[D_5, D_6] = 0$ (commutativity of operators)
   \item PBW-type standard form: $\mathcal{F}(A) \cong (A \rtimes k[D_\infty]) \oplus (A \rtimes k[D_\infty])\Delta$
   \item Ore exchange relation $\Delta M_a - M_{\sigma_u(a)}\Delta = \tilde{\delta}(a)$ resolves critical pairs
   \item Kernel hierarchy $\ker(D_2) \subset \ker(D_5) \subset \ker(D_6)$ gives an \textbf{increasing sequence of dimensions $1 \to 2 \to 3$}, guaranteeing independence of each group
   \end{itemize}
\end{enumerate}
\end{proof}

\subsection{Dimension Matching}

\begin{center}
\begin{tabular}{|c|c|c|c|}
\hline
Origin & Derivation & Gauge group & Dimension \\
\hline
$\ker(D_6)$ scaling & dim 1 connected compact group & U(1) & 1 \\
su(2) on $\ker(D_5)$ & dim 3 simple compact group & SU(2) & 3 \\
$\ker(D_6) = V_0 \oplus V_1 \oplus V_2$ & dim 8 simple compact group & SU(3) & 8 \\
\hline
\end{tabular}
\end{center}

Total: $1 + 3 + 8 = 12$ dimensions (matching the Standard Model gauge group).

\section{Transcendental Selection by Describability}
\label{sec:describability}

FUST defines a gauge parameter space $\mathcal{G}$ with 12 points, \textbf{all mathematically equivalent}:

\begin{enumerate}
\item $\mathrm{SU}(3) \times \mathrm{SU}(2) \times \mathrm{U}(1)$ (Standard Model)
\item $\mathrm{U}(1)^3 \times \mathrm{U}(1)^2 \times \mathrm{U}(1)$ (completely abelian)
\item 10 other combinations
\end{enumerate}

\textbf{Necessary selection by describability}: All 12 points are mathematically equivalent, but \textbf{the points where ``an intelligence capable of describing this theory'' can exist are limited}:
\begin{itemize}
\item Without \textbf{confinement} (non-abelian nature of SU(3)), stable hadrons cannot form $\Rightarrow$ no atomic nuclei
\item Without \textbf{electroweak symmetry} (SU(2)$\times$U(1)), stable electron orbits enabling chemical bonds cannot exist
\item At the completely abelian point $\mathrm{U}(1)^6$, confinement does not occur, and complex structures cannot form
\end{itemize}

Therefore, the Standard Model point is not an ``arbitrary choice'' but a \textbf{transcendental necessity for the theory to be described}. This is not the physical anthropic principle (``selected because observers exist'') but \textbf{logical self-consistency of theory existence}: the very fact that ``FUST is being described'' uniquely selects the Standard Model.

\begin{center}
\begin{tabular}{|c|c|c|}
\hline
Sector & Mathematical consistency & Possibility of describer \\
\hline
$(\mathrm{SU}(3), \mathrm{SU}(2), \mathrm{U}(1))$ & \checkmark & \checkmark (confinement + stable atoms $\to$ intelligent life) \\
$(\mathrm{U}(1)^3, \mathrm{U}(1)^2, \mathrm{U}(1))$ & \checkmark & $\times$ (no structure formation) \\
Other 10 sectors & \checkmark & $\times$ (unstable nucleons or proton decay) \\
\hline
\end{tabular}
\end{center}

\textbf{FUST's claim}: All 12 sectors are mathematically equivalent, but the fact that ``FUST is being described'' is only valid in a sector where describers (observers) exist. This is not the physical anthropic principle but \textbf{logical self-consistency}.

