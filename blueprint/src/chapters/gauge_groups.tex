% Chapter 7: Standard Gauge Group from Fζ Factorization Gauge Freedom
\chapter{Standard Gauge Group from \texorpdfstring{$F_\zeta$}{Fζ} Factorization}
\label{chap:gauge_groups}

\section{Overview: Gauge Freedom from Factorization}
\label{sec:gauge_overview}

The gauge group arises from the \textbf{input side}: the ambiguity in factoring a state function $f = f_1 \cdot f_2 \cdots f_k$. The derivation defect
\[
\delta(f,g) := F_\zeta(fg) - f \cdot F_\zeta(g) - F_\zeta(f) \cdot g
\]
measures how $F_\zeta$ fails to be a derivation. The key gauge invariance is:

\begin{theorem}[Factorization Gauge Invariance]
\label{thm:factorization_gauge}
\lean{FUST.FζOperator.derivDefect_const_gauge}
\leanok
For $c \in \mathbb{C}^\times$, $\delta(cf, c^{-1}g) = \delta(f,g)$.
\end{theorem}

This means the physics encoded in $\delta$ is invariant under $(f,g) \mapsto (cf, c^{-1}g)$, giving a $\mathrm{U}(1)$ scalar gauge. For multi-mode superpositions with $k$ modes, the mode coefficients can be mixed by $\mathrm{GL}(k)$ while preserving the product, giving $\mathrm{SU}(k)$ after unitarity. The maximum $k$ is bounded by the rank of the mode vector space.

\begin{theorem}[Gauge Channel Dimensions]
\label{thm:gauge_channel_dimensions}
\lean{FUST.gauge_channel_dimensions}
\leanok
The following properties of $F_\zeta$ determine the gauge group representation dimensions $(3, 2, 1)$:
\begin{enumerate}
\item $\Phi_S$ has rank 3 (mode vectors span $\mathbb{R}^3$) $\Rightarrow$ gauge saturates at $\mathrm{SU}(3)$
\item $\tau(\mathrm{AF\_coeff}) = -\mathrm{AF\_coeff}$ (quaternionic type) $\Rightarrow$ $\mathrm{SU}(2) \cong \mathrm{Sp}(1)$
\item Residual 1-dimensional trivial channel $\Rightarrow$ $\mathrm{U}(1)$
\item $\zeta_6 \neq \varphi^n$ for all $n$ (channel independence)
\item $|6a + \mathrm{AF\_coeff} \cdot b|^2 = 12(3a^2 + b^2)$ (weight ratio $3:1$)
\end{enumerate}
\end{theorem}

\section{\texorpdfstring{$\zeta_6$}{ζ₆} Separation: Channel Independence}
\label{sec:zeta6_separation}

\begin{theorem}[$\zeta_6 \neq \varphi^n$]
\label{thm:zeta6_ne_phi}
\lean{FUST.zeta6_ne_phi_pow}
\leanok
For all $n \in \mathbb{N}$, $\zeta_6 \neq \varphi^n$, since $\mathrm{Im}(\zeta_6) = \sqrt{3}/2 \neq 0$ while $\varphi^n \in \mathbb{R}$.
\end{theorem}

This separation guarantees that the SY channel ($\varphi$-eigenspectrum) and the AF channel ($\zeta_6$-rotation) act independently, preventing gauge group reduction.

\section{SY Channel: Mode Vector Rank \texorpdfstring{$\to \mathrm{SU}(3)$}{→ SU(3)}}
\label{sec:SU3_derivation}

For each active mode $s$, the SY sub-operator eigenvalues give a \textbf{mode vector}:
\[
v(s) = (\sigma_{\mathrm{Diff}_5}(s),\; \sigma_{\mathrm{Diff}_3}(s),\; \sigma_{\mathrm{Diff}_2}(s)) \in \mathbb{R}^3
\]
where $\sigma_{\mathrm{Diff}_5}(s) = \varphi^{2s} + \varphi^s - 4 + \psi^s + \psi^{2s}$, $\sigma_{\mathrm{Diff}_3}(s) = \varphi^s - 2 + \psi^s$, $\sigma_{\mathrm{Diff}_2}(s) = \varphi^s - \psi^s$.

\begin{theorem}[Mode Space Rank Three]
\label{thm:mode_space_rank}
\lean{FUST.mode_space_rank_three}
\leanok
The $3 \times 3$ determinant
\[
\det \begin{pmatrix} \sigma_{\mathrm{Diff}_5}(1) & \sigma_{\mathrm{Diff}_3}(1) & \sigma_{\mathrm{Diff}_2}(1) \\
\sigma_{\mathrm{Diff}_5}(5) & \sigma_{\mathrm{Diff}_3}(5) & \sigma_{\mathrm{Diff}_2}(5) \\
\sigma_{\mathrm{Diff}_5}(7) & \sigma_{\mathrm{Diff}_3}(7) & \sigma_{\mathrm{Diff}_2}(7)
\end{pmatrix} \neq 0
\]
so the mode vectors $v(1), v(5), v(7)$ are linearly independent in $\mathbb{R}^3$.
\end{theorem}

Since $\dim \mathbb{R}^3 = 3$, any 4th mode vector is a linear combination of the first 3. Mode-mixing gauge transformations on $k$ independent modes give $\mathrm{GL}(k) \to \mathrm{SU}(k)$. With at most 3 independent mode vectors, the gauge group saturates at $\mathrm{SU}(3)$.

$\mathrm{SU}(3)$ has dimension $3^2 - 1 = 8$ (8 gluons).

\section{AF Channel: \texorpdfstring{$\tau$}{τ}-Anti-invariance \texorpdfstring{$\to \mathrm{SU}(2)$}{→ SU(2)}}
\label{sec:SU2_derivation}

\begin{theorem}[$\tau$-Anti-invariance of AF Coefficient]
\label{thm:AF_tau_neg}
\lean{FUST.AF_coeff_tau_neg}
\leanok
$\tau(\mathrm{AF\_coeff\_gei}) = -\mathrm{AF\_coeff\_gei}$, where $\mathrm{AF\_coeff\_gei} = \langle -2, 0, 4, 0 \rangle \in \mathbb{Z}[\varphi, \zeta_6]$.
\end{theorem}

The Galois automorphism $\tau: \zeta_6 \mapsto 1 - \zeta_6$ fixes $\varphi$ and acts on $\mathbb{Z}[\varphi, \zeta_6]$. The condition $\tau(x) = -x$ is the signature of a \textbf{quaternionic representation type}, which by Frobenius' theorem uniquely determines $\mathrm{SU}(2) \cong \mathrm{Sp}(1)$ as the gauge group on the 2-dimensional AF space.

\begin{theorem}[AF Channel Non-Degeneracy]
\label{thm:AF_nondegenerate}
\lean{FUST.AF_coeff_nonzero}
\leanok
$\mathrm{AF\_coeff} \neq 0$, since $|\mathrm{AF\_coeff}|^2 = 12 \neq 0$.
\end{theorem}

\begin{theorem}[AF Coefficient Purely Imaginary]
\label{thm:AF_imaginary}
\lean{FUST.AF_coeff_purely_imaginary}
\leanok
$\mathrm{Re}(\mathrm{AF\_coeff}) = 0$, i.e., $\mathrm{AF\_coeff} = 2i\sqrt{3}$.
\end{theorem}

$\mathrm{SU}(2)$ has dimension $2^2 - 1 = 3$ ($W^\pm, Z$ bosons).

\section{Trivial Channel \texorpdfstring{$\to \mathrm{U}(1)$}{→ U(1)}}
\label{sec:U1_derivation}

On the trivial channel (dimension 1), the only compact connected gauge group is $\mathrm{U}(1)$. This is also the scalar gauge from the basic factorization invariance $\delta(cf, c^{-1}g) = \delta(f,g)$.

\section{Norm Decomposition: Weight Ratio 3:1}
\label{sec:weight_ratio}

\begin{theorem}[$F_\zeta$ Norm-Squared Decomposition]
\label{thm:normSq_decomposition}
\lean{FUST.norm_weight_ratio}
\leanok
\[
|6a + \mathrm{AF\_coeff} \cdot b|^2 = 12(3a^2 + b^2)
\]
\end{theorem}

The weight ratio $3 : 1$ between the SY channel ($a$, weight 3) and the AF channel ($b$, weight 1) encodes the spacetime decomposition $I_4 = \mathrm{Fin}\,3 \oplus \mathrm{Fin}\,1$ (3 spatial + 1 temporal dimension).

\section{Dimension Matching}
\label{sec:dimension_matching}

\begin{center}
\begin{tabular}{|c|c|c|c|}
\hline
$F_\zeta$ channel & Gauge mechanism & Gauge group & $\dim(G)$ \\
\hline
SY ($\Phi_S$, rank 3) & Mode mixing, $k \leq 3$ & SU(3) & 8 \\
AF ($\Phi_A$, $\tau$-anti-inv.) & Quaternionic type & SU(2) & 3 \\
Trivial (dim 1) & Scalar gauge & U(1) & 1 \\
\hline
\multicolumn{3}{|r|}{Total gauge dimensions:} & 12 \\
\hline
\end{tabular}
\end{center}

$12 = 8 + 3 + 1$: matching the 12 gauge bosons of the Standard Model (8 gluons $+ W^\pm + Z +$ photon).

\section{Summary}
\label{sec:gauge_summary}

The gauge group $\mathrm{SU}(3) \times \mathrm{SU}(2) \times \mathrm{U}(1)$ arises from the \textbf{factorization freedom} of the input state function:
\begin{itemize}
\item \textbf{SU(3)}: Mode-mixing on the SY channel is bounded by $\Phi_S$ rank 3 (mode vectors $v(s) \in \mathbb{R}^3$, $\dim \mathbb{R}^3 = 3$)
\item \textbf{SU(2)}: $\tau$-anti-invariance of AF\_coeff ($\tau(x) = -x$, quaternionic type $\Rightarrow$ $\mathrm{Sp}(1) \cong \mathrm{SU}(2)$)
\item \textbf{U(1)}: Trivial channel (1-dim) and scalar factorization gauge $\delta(cf, c^{-1}g) = \delta(f,g)$
\item Channel independence: $\zeta_6 \neq \varphi^n$ (imaginary part separation)
\item Weight ratio: $|6a + \mathrm{AF\_coeff} \cdot b|^2 = 12(3a^2 + b^2)$ encodes $3{+}1$ spacetime
\end{itemize}
