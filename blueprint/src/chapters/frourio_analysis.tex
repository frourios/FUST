% Chapter 2: Foundations of Frourio Analysis
\chapter{Foundations of Frourio Analysis}
\label{chap:frourio_analysis}

\section{The Three Geometric Types}
\label{sec:three_geometries}

FUST is built on the quadratic family $x^2 = x + c$, which produces three geometric types depending on $c \in \{+1, 0, -1\}$.

\begin{definition}[Hyperbolic pair: $c = +1$]
\label{def:hyperbolic}
\lean{FUST.phi, FUST.psi}
\leanok
The equation $x^2 = x + 1$ has roots:
\[
\varphi = \frac{1+\sqrt{5}}{2} \approx 1.618, \quad \psi = \frac{1-\sqrt{5}}{2} \approx -0.618
\]
with product $\varphi\psi = -1$ and sum $\varphi + \psi = 1$.
\end{definition}

\begin{definition}[Elliptic pair: $c = -1$]
\label{def:elliptic}
\lean{FUST.Zeta6.zeta6}
\leanok
The equation $x^2 = x - 1$ has roots:
\[
\zeta_6 = \frac{1+i\sqrt{3}}{2} = e^{i\pi/3}, \quad \bar{\zeta}_6 = \frac{1-i\sqrt{3}}{2} = e^{-i\pi/3}
\]
with product $\zeta_6 \bar{\zeta}_6 = +1$ and sum $\zeta_6 + \bar{\zeta}_6 = 1$.
These are primitive 6th roots of unity: $\zeta_6^6 = 1$, $\zeta_6^3 = -1$.
\end{definition}

\begin{definition}[Parabolic pair: $c = 0$]
\label{def:parabolic}
The equation $x^2 = x$ has roots $1$ and $0$, with product $1 \cdot 0 = 0$ and sum $1 + 0 = 1$.
\end{definition}

\textbf{Structural comparison}:

\begin{center}
\begin{tabular}{|c|c|c|c|c|c|}
\hline
$c$ & Roots & Product & $|\text{roots}|$ & Geometry & Role in $D_\zeta$ \\
\hline
$+1$ & $\varphi, \psi$ & $-1$ & $\varphi > 1, |\psi| < 1$ & Hyperbolic & $\varphi$-dilation lattice \\
$-1$ & $\zeta_6, \bar{\zeta}_6$ & $+1$ & $|\zeta_6| = 1$ & Elliptic & $\mathbb{Z}/6\mathbb{Z}$ DFT \\
$0$ & $1, 0$ & $0$ & $1, 0$ & Parabolic & $z$-normalization \\
\hline
\end{tabular}
\end{center}

All three types share the trace $\sigma_1 = 1$ (sum of roots), but differ in determinant $\sigma_2 = -c$. The unified operator $D_\zeta$ composes all three into a single algebraic object.

\section{Algebraic Properties}
\label{sec:algebraic_properties}

\begin{theorem}[Golden Ratio Properties]
\label{thm:golden_properties}
\lean{FUST.phi_sq, FUST.phi_mul_psi}
\leanok
\begin{align}
\varphi^2 &= \varphi + 1, \quad \varphi \cdot \psi = -1, \quad \varphi - \psi = \sqrt{5}
\end{align}
\end{theorem}

\begin{theorem}[$\zeta_6$ Properties]
\label{thm:zeta6_properties}
\lean{FUST.Zeta6.zeta6_sq, FUST.Zeta6.zeta6_mul_conj}
\leanok
\begin{align}
\zeta_6^2 &= \zeta_6 - 1, \quad \zeta_6 \cdot \bar{\zeta}_6 = 1, \quad \zeta_6 - \bar{\zeta}_6 = i\sqrt{3}
\end{align}
\end{theorem}

\textbf{Key structural parallel}: Both $\varphi - \psi = \sqrt{5}$ and $\zeta_6 - \bar{\zeta}_6 = i\sqrt{3}$ are algebraic irrationals that serve as normalization denominators in the difference operators. Both $\sqrt{5}$ and $i\sqrt{3}$ arise from the discriminant $\sqrt{1 + 4c}$ of $x^2 - x - c = 0$.

\begin{theorem}[$\zeta_6$ is outside the $\varphi$-eigenspectrum]
\label{thm:zeta6_ne_phi}
\lean{FUST.zeta6_ne_phi_pow}
\leanok
For all $n \in \mathbb{N}$, $\zeta_6 \neq \varphi^n$, because $\varphi^n \in \mathbb{R}$ while $\mathrm{Im}(\zeta_6) = \sqrt{3}/2 \neq 0$.
\end{theorem}

This separation between the hyperbolic and elliptic sectors is essential: it ensures that $\zeta_6$-rotation mixes $\varphi$-dilation eigenspaces, preventing gauge group reduction.

\section{Fibonacci Decomposition}
\label{sec:fib_decomposition}

\begin{theorem}[Fibonacci Decomposition]
\label{thm:fib_decomposition}
\lean{FUST.phi_pow_fib}
\leanok
For any integer $k$,
\[
\varphi^k = F_k\varphi + F_{k-1}, \quad \psi^k = F_k\psi + F_{k-1}
\]
where $F_k$ is the Fibonacci sequence.
\end{theorem}

\begin{proof}
By induction.
\begin{itemize}
\item $k=0$: $\varphi^0 = 1 = 0 \cdot \varphi + 1$.
\item $k=1$: $\varphi^1 = 1 \cdot \varphi + 0$.
\item $k \to k+1$: $\varphi^{k+1} = \varphi(F_k\varphi + F_{k-1}) = F_k(\varphi+1) + F_{k-1}\varphi = F_{k+1}\varphi + F_k$.
\end{itemize}
\end{proof}

\section{Binet Formula}
\label{sec:binet}

\begin{theorem}[Binet Formula]
\label{thm:binet}
\lean{FUST.binet_formula}
\leanok
For any integer $k$,
\[
\varphi^k - \psi^k = F_k \cdot \sqrt{5}
\]
\end{theorem}

\begin{proof}
From $\varphi^k = F_k\varphi + F_{k-1}$ and $\psi^k = F_k\psi + F_{k-1}$, we have
$\varphi^k - \psi^k = F_k(\varphi - \psi) = F_k \sqrt{5}$.
\end{proof}

\section{Frourio Exponential Function}
\label{sec:frourio_exp}

\begin{definition}[Fibonacci Factorial]
\label{def:fib_factorial}
Define the Fibonacci factorial as $n!_{\mathsf{F}} := \prod_{k=1}^{n} F_k$.
\end{definition}

\begin{definition}[Frourio Exponential]
\label{def:frourio_exp}
\[
\exp_{\mathsf{F}}(x) = \sum_{n=0}^{\infty} \frac{x^n}{n!_{\mathsf{F}}}
\]
\end{definition}

\section{Frourio Constant}
\label{sec:frourio_constant}

\begin{definition}[Frourio Constant]
\label{def:frourio_constant}
\[
\mathfrak{f} := \exp_{\mathsf{F}}(1) = \sum_{n=0}^{\infty} \frac{1}{n!_{\mathsf{F}}} \approx 3.7045
\]
\end{definition}

\textbf{Alternative representation} (Binet type):
\[
\mathfrak{f} = \frac{\exp_{\mathsf{F}}(\varphi) - \exp_{\mathsf{F}}(\psi)}{\sqrt{5}} = \frac{N_2[\exp_{\mathsf{F}}](1)}{\sqrt{5}}
\]

This is a constant that gives a natural scale to the golden ratio.
