% FUST Navier-Stokes - Introduction
\chapter{Global Existence for Navier-Stokes via D6 Planck-Scale Cutoff}
\label{chap:ns_introduction}

\section{Clay Millennium Problem: Fefferman's Statement (A)}
\label{sec:ns_clay_statement}

The Clay Mathematics Institute formulates the Navier-Stokes problem as follows (Fefferman 2000):

\begin{quote}
\textbf{Statement (A).}
Given $\nu > 0$ on $\mathbb{R}^3$, smooth divergence-free initial data $u_0$ with
\[
|\partial^\alpha u_0(x)| \leq C_{\alpha K}(1+|x|)^{-K} \quad \text{for all } \alpha, K,
\]
and external force $f \equiv 0$, prove the existence of smooth $u(x,t)$, $p(x,t)$ on $[0,\infty) \times \mathbb{R}^3$ satisfying the Navier-Stokes equations with
\[
\int_{\mathbb{R}^3} |u(x,t)|^2\, dx < C \quad \text{for all } t \geq 0.
\]
\end{quote}

\section{FUST Reformulation}
\label{sec:ns_fust_reformulation}

FUST replaces the continuous PDE with an algebraic mode evolution system.
The velocity field decomposes into D6 eigenmodes on the $\varphi$-scaled lattice:
\begin{equation}\label{eq:ns_mode_ode}
\frac{d\hat{u}_n}{dt} = -C_n^2 \hat{u}_n + \sum_{m+k=n} N_{m,k}\, \hat{u}_m \hat{u}_k
\end{equation}
where $C_n = \texttt{dissipationCoeff}(n)$ from $D_6$, and $N_{m,k} = \texttt{nonlinearCoeff}(m,k)$ from the $D_6$ Leibniz deviation.

\subsection{Key Structural Properties}
\begin{enumerate}
\item \textbf{Universal probe}: $D_6$ is defined for all $x \neq 0$ (covers the entire $\varepsilon$-$\delta$ domain)
\item \textbf{Dissipation}: $C_n^2 > 0$ for $n \geq 3$, with lower bound $C_n \geq \tfrac{1}{3}\varphi^{3n}$
\item \textbf{Energy monotonicity}: The nonlinear term redistributes but does not create energy
\item \textbf{Exponential mode decay}: $C_n \sim \varphi^{3n}$ grows exponentially $\Rightarrow$ analytic regularity
\item \textbf{Kernel stability}: $\ker(D_6) = \{1, x, x^2\}$ is time-invariant
\item \textbf{Planck-scale cutoff}: Below the structural minimum length $\ell_P = 25/12$, D6 sampling is unresolvable and energy thermally dissipates
\end{enumerate}

\section{Planck-Scale Thermal Cutoff}
\label{sec:ns_planck_cutoff}

The central innovation is the Planck-scale thermal cutoff argument:

\begin{enumerate}
\item \textbf{D6 sampling resolution}: $D_6$ evaluates at $\{\varphi^3 x, \varphi^2 x, \varphi x, \psi x, \psi^2 x, \psi^3 x\}$. When $x$ is too small, sampling points fall below the structural minimum length $\ell_P = 25/12$ (derived from D6 structure in \texttt{TimeTheorem.lean})

\item \textbf{Planck cutoff scale}: $x_P = \ell_P / \varphi^3$. For $x < x_P$, the outermost D6 sampling point $\varphi^3 x < \ell_P$ is physically unresolvable

\item \textbf{Finite mode cutoff}: For a system of size $L$, since $\varphi > 1$ implies $\varphi^n \to \infty$, there exists $N_{\max}$ such that for all $n \geq N_{\max}$: $L / \varphi^n < \ell_P$. Modes above $N_{\max}$ have sub-Planck scale

\item \textbf{Thermal dissipation}: The third law of thermodynamics (massive states have positive entropy) combined with the energy density hierarchy ensures that sub-Planck modes have already thermally dissipated

\item \textbf{Finite-dimensional reduction}: Only modes $0, 1, \ldots, N_{\max}$ are physical. $\ker(D_6)$ modes ($n \leq 2$) are stationary; dissipative modes ($3 \leq n \leq N_{\max}$) have $C_n^2 > 0$

\item \textbf{Global existence}: Finite-dimensional system + energy non-increasing $\Rightarrow$ no finite-time blowup $\Rightarrow$ global solution
\end{enumerate}

\section{Continuous Time Evolution}
\label{sec:ns_continuous_time}

The solution is constructed for continuous time $t \in \mathbb{R}$ ($t \geq 0$), matching Fefferman's requirement of solutions on $[0,\infty) \times \mathbb{R}^3$.
The decay factor $r_n^t$ uses real exponentiation (rpow), so the evolution is defined for all $t \geq 0$:
\begin{itemize}
\item $D_6$ is a universal 6-point sampling operator on the $\varphi^n$-lattice
\item Time $t \in [0,\infty)$ is continuous, consistent with the Clay formulation
\item The universal variable $x \in \mathbb{C}$ (with $D_n$ operators defined on $\mathbb{C}$) and time $t \in \mathbb{R}$ ($t \geq 0$)
\end{itemize}

\section{Solution Strategy}
\label{sec:ns_solution_strategy}

We construct a truncated continuous-time evolution. For a system of size $L$, compute $N_{\max}$ from the Planck mode cutoff, then define:
\[
\hat{u}_n(t) = \begin{cases}
\hat{u}_n(0) \cdot r_n^t & \text{if } n \leq N_{\max} \\
0 & \text{if } n > N_{\max}
\end{cases}
\]
where $r_n = 1/(1 + C_n^2)$ and $r_n^t$ denotes real exponentiation ($\texttt{rpow}$).
\begin{itemize}
\item $\ker(D_6)$ modes ($n \leq 2$): $C_n = 0 \Rightarrow r_n = 1$ (stationary for all $t$)
\item High modes ($3 \leq n \leq N_{\max}$): $C_n^2 > 0 \Rightarrow r_n < 1$ (exponential decay as $t \to \infty$)
\item Sub-Planck modes ($n > N_{\max}$): set to $0$ (thermally dissipated)
\end{itemize}
