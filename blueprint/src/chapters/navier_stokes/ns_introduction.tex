% FUST Navier-Stokes - Introduction
\chapter{Global Existence for Navier-Stokes in FUST Framework: Introduction}
\label{chap:ns_introduction}

\section{Overview}
\label{sec:ns_overview}

This paper proves that smooth solutions to the Navier-Stokes type equation exist for all time within the FUST (Frourio Universal Structure Theorem) framework. FUST is a ``discrete scale version'' of the standard Navier-Stokes equation, where finite-time blowup is structurally forbidden due to the golden ratio scaling structure.

\textbf{Main Result}: For the FUST-NS equation, smooth solutions exist for all time $t \in [0, \infty)$ for smooth initial conditions.

\section{Relation to Clay Millennium Problem}
\label{sec:ns_clay}

\subsection{Clay Problem Requirements}

The Clay Mathematics Institute's Navier-Stokes problem asks:

\begin{quote}
On $\mathbb{R}^3$ or $\mathbb{T}^3$, for smooth initial conditions, do smooth solutions to the NS equations exist for all time? Or does there exist an example of finite-time blowup?
\end{quote}

\subsection{FUST's Position}

FUST is not ``equivalent'' to standard NS, but has the following additional structure:

\begin{enumerate}
\item \textbf{Discrete scale structure}: Wavenumbers $k = \varphi^n$ (discrete, not continuous)
\item \textbf{Algebraically closed dissipation and nonlinear terms}: Derived from algebraic properties of the golden ratio
\item \textbf{Structural regularization}: Infinitesimal scales do not exist
\end{enumerate}

This additional structure makes it possible to prove global existence within FUST, which remains unsolved for standard NS.

\subsection{Physical Justification}

The discrete scale structure is justified by:
\begin{itemize}
\item Continuous space breaks down below the Planck scale
\item The continuum limit problem can be regarded as a ``mathematical artifact''
\item Physically meaningful scales are discrete
\end{itemize}

