% FUST Navier-Stokes - Introduction
\chapter{Global Existence for Navier-Stokes via \texorpdfstring{$D_\zeta$}{Dζ} 4D Spacetime}
\label{chap:ns_introduction}

\section{Clay Millennium Problem: Fefferman's Statement (A)}
\label{sec:ns_clay_statement}

The Clay Mathematics Institute formulates the Navier-Stokes problem as follows (Fefferman 2000):

\begin{quote}
\textbf{Statement (A).}
Given $\nu > 0$ on $\mathbb{R}^3$, smooth divergence-free initial data $u_0$ with
\[
|\partial^\alpha u_0(x)| \leq C_{\alpha K}(1+|x|)^{-K} \quad \text{for all } \alpha, K,
\]
and external force $f \equiv 0$, prove the existence of smooth $u(x,t)$, $p(x,t)$ on $[0,\infty) \times \mathbb{R}^3$ satisfying the Navier-Stokes equations with
\[
\int_{\mathbb{R}^3} |u(x,t)|^2\, dx < C \quad \text{for all } t \geq 0.
\]
\end{quote}

\section{FUST Reformulation via \texorpdfstring{$D_\zeta$}{Dζ}}
\label{sec:ns_fust_reformulation}

The unified operator $D_\zeta = [\mathrm{AFNum}(\Phi_A) + \mathrm{SymNum}(\Phi_S)] / z$ determines 4-dimensional spacetime via $I_4 = \mathrm{Fin}\,3 \oplus \mathrm{Fin}\,1$ (from the weight ratio 3:1 in $|D_\zeta|^2 = 12(3a^2 + b^2)$).

The AF-channel projection $\mathrm{Diff}_6$ (antisymmetric, $r = 1$) provides the dissipation structure on the $\varphi$-lattice. FUST replaces the continuous PDE with an algebraic mode evolution system:
\begin{equation}\label{eq:ns_mode_ode}
\frac{d\hat{u}_n}{dt} = -C_n^2 \hat{u}_n + \sum_{m+k=n} N_{m,k}\, \hat{u}_m \hat{u}_k
\end{equation}
where $C_n$ is the spectral coefficient of $D_\zeta$'s AF-channel projection $\mathrm{Diff}_6$, and $N_{m,k}$ is the Leibniz deviation.

\subsection{Key Structural Properties from \texorpdfstring{$D_\zeta$}{Dζ}}
\begin{enumerate}
\item \textbf{4D spacetime}: $D_\zeta$ determines $\dim\ker(F_\zeta) + 1 = 3 + 1 = 4$ via $I_4 = \mathrm{Fin}\,3 \oplus \mathrm{Fin}\,1$
\item \textbf{Spatial DOF}: $\ker(F_\zeta) = \mathrm{span}\{1, z, z^2\}$ (dim 3 = spatial dimensions)
\item \textbf{Dissipation}: $C_n^2 > 0$ for $n \geq 3$, with lower bound $C_n \geq \tfrac{1}{3}\varphi^{3n}$
\item \textbf{Energy monotonicity}: Nonlinear terms redistribute but do not create energy
\item \textbf{Kernel stability}: $\ker(F_\zeta)$ is time-invariant (spatial steady state)
\item \textbf{Planck-scale cutoff}: Below the structural minimum $\ell_P = 25/12$, $D_\zeta$ AF-channel sampling is unresolvable
\end{enumerate}

\section{Planck-Scale Thermal Cutoff from \texorpdfstring{$D_\zeta$}{Dζ}}
\label{sec:ns_planck_cutoff}

\begin{enumerate}
\item \textbf{$D_\zeta$ AF-channel resolution}: $\mathrm{Diff}_6$ evaluates at $\{\varphi^3 z, \varphi^2 z, \varphi z, \psi z, \psi^2 z, \psi^3 z\}$. When $z$ is too small, sampling points fall below $\ell_P = 25/12$

\item \textbf{Planck cutoff scale}: $z_P = \ell_P / \varphi^3$. For $z < z_P$, the outermost point $\varphi^3 z < \ell_P$

\item \textbf{Finite mode cutoff}: Since $\varphi > 1$ implies $\varphi^n \to \infty$, there exists $N_{\max}$ such that for $n \geq N_{\max}$: $L / \varphi^n < \ell_P$

\item \textbf{Thermal dissipation}: Third law (massive states have positive entropy) ensures sub-Planck modes are thermally dissipated

\item \textbf{Finite-dimensional reduction}: Only modes $0, 1, \ldots, N_{\max}$ are physical. Kernel modes ($n \leq 2$, spatial DOF) are stationary; dissipative modes ($3 \leq n \leq N_{\max}$) have $C_n^2 > 0$

\item \textbf{Global existence}: Finite-dimensional system + energy non-increasing $\Rightarrow$ global solution
\end{enumerate}

\section{Continuous Time Evolution}
\label{sec:ns_continuous_time}

The solution is constructed for continuous time $t \in [0,\infty)$, matching Fefferman's requirement.
The decay factor $r_n^t$ uses real exponentiation, so evolution is defined for all $t \geq 0$:
\begin{itemize}
\item $D_\zeta$'s AF-channel $\mathrm{Diff}_6$ acts as a 6-point sampling operator on the $\varphi^n$-lattice
\item Time $t \in [0,\infty)$ is continuous, consistent with the Clay formulation
\item The 4D spacetime structure from $I_4 = \mathrm{Fin}\,3 \oplus \mathrm{Fin}\,1$ governs the mode dynamics
\end{itemize}

\section{Solution Strategy}
\label{sec:ns_solution_strategy}

We construct a truncated continuous-time evolution. For system size $L$, compute $N_{\max}$ from the Planck mode cutoff, then define:
\[
\hat{u}_n(t) = \begin{cases}
\hat{u}_n(0) \cdot r_n^t & \text{if } n \leq N_{\max} \\
0 & \text{if } n > N_{\max}
\end{cases}
\]
where $r_n = 1/(1 + C_n^2)$ and $r_n^t$ denotes real exponentiation.
\begin{itemize}
\item Kernel modes ($n \leq 2$, spatial DOF from $D_\zeta$): $C_n = 0 \Rightarrow r_n = 1$ (stationary)
\item High modes ($3 \leq n \leq N_{\max}$): $C_n^2 > 0 \Rightarrow r_n < 1$ (exponential decay)
\item Sub-Planck modes ($n > N_{\max}$): set to $0$ (below $F_\zeta$ resolution)
\end{itemize}
