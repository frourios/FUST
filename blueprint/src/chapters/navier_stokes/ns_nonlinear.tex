% FUST Navier-Stokes - Nonlinear Terms
\chapter{Nonlinear Terms from \texorpdfstring{$D_\zeta$}{Dζ} Leibniz Deviation}
\label{chap:ns_nonlinear}

\section{Nonlinear Term as Leibniz Deviation}
\label{sec:ns_leibniz}

The deviation from the Leibniz rule of $D_\zeta$'s AF-channel projection $N_6$ gives the nonlinear term:

\begin{definition}[Nonlinear Coefficient]
\label{def:ns_nonlinear}
\lean{FUST.NavierStokes.nonlinearCoeff}
\leanok
\[
N_{m,n} = C_{m+n} - C_m - C_n
\]
\end{definition}

\begin{theorem}[Existence of Nonlinear Term]
\label{thm:ns_nonlinear_exists}
\lean{FUST.NavierStokes.nonlinearCoeff_1_2_ne_zero}
\leanok
\[
N_{1,2} = C_3 \neq 0
\]
\end{theorem}

This is physically significant: the product of two kernel elements $z \cdot z^2 = z^3$ exits the kernel ($C_3 \neq 0$), triggering nonlinear energy transfer from spatial modes to dissipative modes.

\section{Growth Estimate of Nonlinear Term}
\label{sec:ns_nonlinear_growth}

\begin{theorem}[Nonlinear Coefficient Growth]
\label{thm:ns_nonlinear_growth}
\lean{FUST.NavierStokes.nonlinear_coeff_growth}
\leanok
\[
|N_{m,n}| \leq 30 \cdot \varphi^{3(m+n)}
\]
\end{theorem}

\section{Dissipation Dominance in \texorpdfstring{$D_\zeta$}{Dζ} AF-Channel}
\label{sec:ns_dominance}

\begin{theorem}[Dissipation Dominates Nonlinearity]
\label{thm:ns_dominance}
At high modes in $D_\zeta$'s AF-channel:
\begin{itemize}
\item Dissipation: $C_n^2 \sim \varphi^{6n}$
\item Nonlinear: $N_{m,n}^2 \lesssim \varphi^{6(m+n)}$
\end{itemize}
\end{theorem}

Dissipation is certainly positive (with lower bound $C_n \geq \frac{1}{3}\varphi^{3n}$ for $n \geq 4$), while the nonlinear term only redistributes energy. This is the mechanism by which $D_\zeta$'s structure prevents finite-time blowup.
