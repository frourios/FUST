% FUST Navier-Stokes - Time Structure from Dζ
\chapter{Time Structure from \texorpdfstring{$D_\zeta$}{Dζ} and Kernel Invariance}
\label{chap:ns_time_structure}

\section{Arrow of Time from \texorpdfstring{$\varphi$}{φ}-Asymmetry}
\label{sec:ns_arrow_time}

\begin{theorem}[Golden Ratio Asymmetry]
\label{thm:ns_phi_psi}
\lean{FUST.phi_gt_one, FUST.abs_psi_lt_one}
\leanok
\[
\varphi > 1, \quad |\psi| < 1
\]
\end{theorem}

In $D_\zeta$'s hyperbolic component: $\varphi > 1$ causes scale expansion (future direction), while $|\psi| < 1$ causes exponential decay (past direction). This asymmetry is the origin of the arrow of time in $D_\zeta$'s 4D spacetime.

\section{Kernel Invariance under Time Evolution}
\label{sec:ns_kernel_invariant}

\begin{theorem}[Kernel Invariance]
\label{thm:ns_kernel_invariant}
\lean{FUST.ker_D6_invariant_timeEvolution}
\leanok
\[
f \in \ker(F_\zeta) \Rightarrow \mathrm{timeEvolution}(f) \in \ker(F_\zeta)
\]
\end{theorem}

The 3 spatial degrees of freedom (ker$(F_\zeta) = \mathrm{span}\{1, z, z^2\}$) are preserved under time evolution. This is the $D_\zeta$-structural reason why spatial geometry is stable.
