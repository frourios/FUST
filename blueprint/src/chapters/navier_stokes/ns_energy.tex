% FUST Navier-Stokes - Energy Decay
\chapter{Energy Decay}
\label{chap:ns_energy}

\section{Definition of Mode Energy}
\label{sec:ns_mode_energy}

\begin{definition}[Mode Energy]
\[
E_n = |\hat{u}_n|^2
\]
\end{definition}

\section{High Mode Energy and Dissipation}
\label{sec:ns_high_mode}

\begin{definition}[High Mode Energy and Dissipation]
\label{def:ns_high_mode}
\lean{FUST.highModeEnergy, FUST.highModeDissipation}
\leanok
\[
E_{\mathrm{high}} = \sum_{n \geq 3} E_n, \quad D_{\mathrm{high}} = \sum_{n \geq 3} C_n^2 E_n
\]
\end{definition}

\section{Energy Decay Theorem}
\label{sec:ns_energy_decay}

\begin{theorem}[Strictly Positive Dissipation]
\label{thm:ns_dissipation_positive}
\lean{FUST.dissipation_strictly_positive}
\leanok
\[
E_{\mathrm{high}} > 0 \Rightarrow D_{\mathrm{high}} > 0
\]
\end{theorem}

If energy exists in high modes, positive dissipation necessarily occurs.

\section{Consequence for Global Existence}
\label{sec:ns_global_consequence}

\begin{enumerate}
\item Energy in $\ker(D_6)$ does not dissipate ($C_n = 0$)
\item Energy outside $\ker(D_6)$ necessarily dissipates ($C_n^2 > 0$)
\item Due to the dissipation lower bound, high mode energy decays exponentially
\item Therefore, the solution remains smooth for all time
\end{enumerate}

