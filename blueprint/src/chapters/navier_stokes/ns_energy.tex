% FUST Navier-Stokes - Energy Decay in Dζ 4D Spacetime
\chapter{Energy Decay in \texorpdfstring{$D_\zeta$}{Dζ} 4D Spacetime}
\label{chap:ns_energy}

\section{Planck Resolution Limit from \texorpdfstring{$D_\zeta$}{Dζ}}
\label{sec:ns_planck_resolution}

\begin{definition}[Planck Cutoff Scale]
\label{def:ns_planck_cutoff}
\lean{FUST.NavierStokes.PlanckCutoff.planckCutoffScale}
\leanok
\[
z_P = \frac{t_P}{\varphi^3}, \qquad t_P = \texttt{planckSecond} = \frac{1}{20\sqrt{15}}
\]
\end{definition}

\begin{theorem}[Below Planck: $D_\zeta$ AF-Channel Unresolvable]
\label{thm:ns_below_planck}
\lean{FUST.NavierStokes.PlanckCutoff.D6_below_planck_unresolvable}
\leanok
\[
0 < z < z_P \Rightarrow \varphi^3 z < t_P
\]
\end{theorem}

\begin{theorem}[Above Planck: $D_\zeta$ Resolvable]
\label{thm:ns_above_planck}
\lean{FUST.NavierStokes.PlanckCutoff.D6_above_planck_resolvable}
\leanok
\[
z \geq z_P \Rightarrow \varphi^3 z \geq t_P
\]
\end{theorem}

\section{Finite Mode Cutoff}
\label{sec:ns_finite_mode}

\begin{theorem}[$\varphi^n$ Unbounded]
\label{thm:ns_phi_unbounded}
\lean{FUST.NavierStokes.PlanckCutoff.phi_pow_unbounded}
\leanok
\[
\forall M \in \mathbb{R},\; \exists N \in \mathbb{N},\; \varphi^N > M
\]
\end{theorem}

\begin{theorem}[Planck Mode Cutoff]
\label{thm:ns_planck_mode_cutoff}
\lean{FUST.NavierStokes.PlanckCutoff.planck_mode_cutoff}
\leanok
\[
\forall L > 0,\; \exists N \in \mathbb{N},\; \forall n \geq N: L / \varphi^n < t_P
\]
\end{theorem}

Since $\varphi > 1$, $\varphi^n \to \infty$ and the scale $L/\varphi^n$ eventually falls below the $D_\zeta$ structural minimum. Modes above this threshold are beyond $D_\zeta$'s resolution.

\section{Thermal Dissipation from \texorpdfstring{$D_\zeta$}{Dζ} Third Law}
\label{sec:ns_thermal_dissipation}

\begin{theorem}[Sub-Planck Thermal Dissipation]
\label{thm:ns_thermal_dissipation}
\lean{FUST.NavierStokes.PlanckCutoff.sub_planck_thermal_dissipation}
\leanok
Three facts justify setting sub-Planck modes to zero:
\begin{enumerate}
\item $t_P > 0$ (structural minimum length from $D_\zeta$ is positive)
\item Third law: $f \notin \ker(F_\zeta) \Rightarrow \exists t,\; S(f,t) > 0$ (massive states have positive entropy)
\item $\forall n \geq 3: C_n^2 > 0$ (dissipation is active outside $\ker(F_\zeta)$, the spatial DOF)
\end{enumerate}
\end{theorem}

\section{Decay Factor from \texorpdfstring{$D_\zeta$}{Dζ} AF-Channel}
\label{sec:ns_decay_factor}

\begin{definition}[Decay Factor]
\label{def:ns_decay_factor}
\lean{FUST.NavierStokes.PlanckCutoff.decayFactor}
\leanok
\[
r_n = \frac{1}{1 + C_n^2}
\]
\end{definition}

\begin{theorem}[Decay Factor Properties]
\label{thm:ns_decay_properties}
\lean{FUST.NavierStokes.PlanckCutoff.decayFactor_pos, FUST.NavierStokes.PlanckCutoff.decayFactor_le_one, FUST.NavierStokes.PlanckCutoff.decayFactor_lt_one, FUST.NavierStokes.PlanckCutoff.decayFactor_kernel}
\leanok
\begin{enumerate}
\item $r_n > 0$ for all $n$
\item $r_n \leq 1$ for all $n$
\item $r_n < 1$ for $n \geq 3$ (strict decay outside spatial kernel)
\item $r_n = 1$ for $n \leq 2$ (spatial DOF are stationary)
\end{enumerate}
\end{theorem}

$C_0 = C_1 = C_2 = 0$ implies $r_0 = r_1 = r_2 = 1$: the 3 spatial kernel modes (from $D_\zeta$'s $I_4 = \mathrm{Fin}\,3 \oplus \mathrm{Fin}\,1$) do not evolve. For $n \geq 3$, $C_n^2 > 0$ implies $r_n < 1$: modes beyond spatial DOF decay exponentially.

\section{Truncated Mode Evolution}
\label{sec:ns_truncated_evolution}

\begin{definition}[Truncated Evolution]
\label{def:ns_truncated_evolution}
\lean{FUST.NavierStokes.PlanckCutoff.truncatedEvolution}
\leanok
For mode sequence $\hat{u}_0$, cutoff $N = N_{\max}$, and time $t \geq 0$:
\[
\hat{u}_n(t) = \begin{cases}
\hat{u}_{0,n} \cdot r_n^t & \text{if } n \leq N \\
0 & \text{if } n > N
\end{cases}
\]
where $r_n^t$ denotes real exponentiation, defined for all $t \in \mathbb{R}$.
\end{definition}

\begin{theorem}[Initial Condition]
\label{thm:ns_initial}
\lean{FUST.NavierStokes.PlanckCutoff.truncatedEvolution_initial}
\leanok
\[
\hat{u}_n(0) = \begin{cases} \hat{u}_{0,n} & \text{if } n \leq N \\ 0 & \text{if } n > N \end{cases}
\]
\end{theorem}

\begin{theorem}[Finite-Dimensional]
\label{thm:ns_finite_dim}
\lean{FUST.NavierStokes.PlanckCutoff.truncatedEvolution_finite}
\leanok
\[
n > N \Rightarrow \hat{u}_n(t) = 0 \quad \forall t
\]
\end{theorem}

\begin{theorem}[Kernel Mode Invariance]
\label{thm:ns_kernel_mode_invariant}
\lean{FUST.NavierStokes.PlanckCutoff.truncatedEvolution_kernel}
\leanok
\[
n \leq 2,\; n \leq N \Rightarrow \hat{u}_n(t) = \hat{u}_{0,n} \quad \forall t \geq 0
\]
\end{theorem}

The 3 spatial kernel modes ($n \leq 2$, from $\dim\ker(F_\zeta) = 3$) remain stationary: $C_n = 0$ means no dissipation acts on them.

\section{Mode Energy and Its Decay}
\label{sec:ns_mode_energy}

\begin{definition}[Mode Energy]
\label{def:ns_mode_energy}
\lean{FUST.NavierStokes.modeEnergy}
\leanok
\[
E_n = |\hat{u}_n|^2
\]
\end{definition}

\begin{theorem}[Mode Energy Non-Increasing]
\label{thm:ns_mode_energy_decay}
\lean{FUST.NavierStokes.PlanckCutoff.truncatedEvolution_energy_noninc}
\leanok
\[
E_n(t) \leq E_n(0) \quad \forall t \geq 0,\; \forall n
\]
\end{theorem}

\section{Total Energy Bound}
\label{sec:ns_total_energy}

\begin{definition}[Total Energy]
\label{def:ns_total_energy}
\lean{FUST.NavierStokes.totalEnergy}
\leanok
\[
E(t) = \sum_{n=0}^{N} E_n(t) = \sum_{n=0}^{N} |\hat{u}_n(t)|^2
\]
\end{definition}

\begin{theorem}[Total Energy Non-Increasing]
\label{thm:ns_total_energy_decay}
\lean{FUST.NavierStokes.PlanckCutoff.truncatedEvolution_totalEnergy_noninc}
\leanok
\[
E(t) \leq E(0) \quad \forall t \geq 0
\]
\end{theorem}

This satisfies Fefferman's requirement $\int |u(x,t)|^2\, dx < C$ for all $t \geq 0$, with $C = E(0)$.

\section{High Mode Energy and Dissipation}
\label{sec:ns_high_mode}

\begin{definition}[High Mode Energy and Dissipation]
\label{def:ns_high_mode}
\lean{FUST.NavierStokes.highModeEnergy, FUST.NavierStokes.highModeDissipation}
\leanok
\[
E_{\mathrm{high}} = \sum_{n=3}^{N} E_n, \quad D_{\mathrm{high}} = \sum_{n=3}^{N} C_n^2 E_n
\]
\end{definition}

\begin{theorem}[Strictly Positive Dissipation]
\label{thm:ns_dissipation_positive}
\lean{FUST.NavierStokes.dissipation_strictly_positive}
\leanok
\[
E_{\mathrm{high}} > 0 \Rightarrow D_{\mathrm{high}} > 0
\]
\end{theorem}

Any energy in modes beyond the 3 spatial DOF ($n \geq 3$) is actively dissipated by $D_\zeta$'s AF-channel structure.
