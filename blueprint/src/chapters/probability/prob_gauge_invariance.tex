% Same-Degree Ratio Gauge Invariance

\chapter{Gauge Invariance of Same-Degree Ratios}
\label{chap:gauge_invariance}

\section{\texorpdfstring{Linearity of $D_6$}{Linearity of D6}}

\begin{theorem}[$D_6$ Linear Scalar]
\label{thm:D6_linear}
\lean{FUST.Probability.D6_linear_scalar}
\leanok
For any scalar $a \in \mathbb{C}$ and function $f$:
\[
D_6[a \cdot f](x) = a \cdot D_6[f](x)
\]
\end{theorem}

\section{Theorem 3.1: Gauge Invariance}

For same-degree monomials $f_A = a \cdot x^n$ and $f_B = b \cdot x^n$:
\[
\frac{|D_6[f_A](x)|}{|D_6[f_B](x)|} = \frac{|a|}{|b|}
\]
This is independent of $x$ (gauge-invariant).

\begin{theorem}[Same-Degree Ratio Gauge Invariance]
\label{thm:gauge_invariant}
\lean{FUST.Probability.same_degree_ratio_gauge_invariant}
\leanok
For $b \neq 0$ and $D_6[f](x) \neq 0$:
\[
\frac{|D_6[a \cdot f](x)|}{|D_6[b \cdot f](x)|} = \frac{|a|}{|b|}
\]
\end{theorem}

\begin{proof}
By linearity:
\begin{align*}
\frac{|D_6[a \cdot f](x)|}{|D_6[b \cdot f](x)|}
&= \frac{|a \cdot D_6[f](x)|}{|b \cdot D_6[f](x)|} \\
&= \frac{|a| \cdot |D_6[f](x)|}{|b| \cdot |D_6[f](x)|} \\
&= \frac{|a|}{|b|}
\end{align*}
\end{proof}

\begin{corollary}[Position Independence]
\label{cor:position_independent}
\lean{FUST.Probability.monomial_ratio_invariant}
\leanok
For any $x, y$ with $D_6[t^n](x) \neq 0$ and $D_6[t^n](y) \neq 0$:
\[
\frac{|D_6[a \cdot t^n](x)|}{|D_6[b \cdot t^n](x)|} =
\frac{|D_6[a \cdot t^n](y)|}{|D_6[b \cdot t^n](y)|}
\]
\end{corollary}

\section{Physical Interpretation}

The gauge invariance of same-degree ratios means that:
\begin{itemize}
\item Probability ratios are well-defined
\item They don't depend on the choice of ``gauge'' (position $x$)
\item This provides the foundation for FUST probability theory
\end{itemize}
