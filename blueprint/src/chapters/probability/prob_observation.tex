% φ-Scale Iteration and Observation Sequences

\chapter{Observation Sequences}
\label{chap:observation}

\section{\texorpdfstring{$\varphi$-Scale Transformation}{phi-Scale Transformation}}

The $\varphi$-scale transformation:
\[
x \mapsto \varphi^k \cdot x \quad (k \in \mathbb{Z})
\]
generates observation sequences that correspond to ``trials'' in probability theory.

\begin{definition}[Observation at Step $k$]
\label{def:observation}
\lean{FUST.Probability.observationAt}
\leanok
For state $f$ and base point $x_0$:
\[
A_f(k) := |D_6[f](\varphi^k \cdot x_0)|
\]
\end{definition}

\begin{theorem}[Observations Non-negative]
\label{thm:obs_nonneg}
\lean{FUST.Probability.observationAt_nonneg}
\leanok
\[
A_f(k) \geq 0 \quad \forall k \in \mathbb{Z}
\]
\end{theorem}

\section{\texorpdfstring{$\varphi$-Invariant Measure}{phi-Invariant Measure}}

The $\varphi$-scale invariant measure is uniquely determined:
\[
d\mu = \frac{dx}{x}
\]
This is the Haar measure on the multiplicative group, derived from
$\varphi$-scale structure (not assumed externally).

\begin{theorem}[Shift Invariance]
\label{thm:shift_invariance}
\lean{FUST.Probability.observation_shift}
\leanok
\[
A_f(k; \varphi \cdot x_0) = A_f(k+1; x_0)
\]
\end{theorem}

\section{Discrete Action}

\begin{definition}[Discrete Action]
\label{def:discrete_action}
\lean{FUST.Probability.discreteAction}
\leanok
\[
\mathcal{A}_N[f] := \sum_{k=-N}^{N} A_f(k)^2 \cdot \log\varphi
\]
\end{definition}

\begin{definition}[Haar Weight]
\label{def:haar_weight}
\lean{FUST.Probability.haarWeight}
\leanok
\[
w := \log\varphi
\]
\end{definition}

\begin{theorem}[Action Non-negative]
\label{thm:action_nonneg}
\lean{FUST.Probability.discreteAction_nonneg}
\leanok
\[
\mathcal{A}_N[f] \geq 0
\]
\end{theorem}

\begin{theorem}[Kernel States Have Zero Action]
\label{thm:action_zero_ker}
\lean{FUST.Probability.action_zero_for_ker}
\leanok
If $f \in \ker(D_6)$, then $\mathcal{A}_N[f] = 0$ for all $N$.
\end{theorem}
