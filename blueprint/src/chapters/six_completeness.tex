% Chapter 5: Six-Element Completeness Theorem
\chapter{Six-Element Completeness Theorem}
\label{chap:six_completeness}

\section{D6 Completeness}
\label{sec:fust_completeness}

The $D_n$ operators are defined for \textbf{arbitrary functions} $f : \mathbb{R} \to \mathbb{R}$.
The completeness theorem states that $\ker(D_7) = \ker(D_6)$ holds universally.

\subsection{Kernel Equality (Lean4 Proven)}

\begin{theorem}[Kernel Equality]
\label{thm:kernel_equality}
\lean{FUST.D7_kernel_equals_D6_kernel}
\leanok
\[
\ker(D_n)|_{\mathbb{R}[x]} = \ker(D_6)|_{\mathbb{R}[x]} = \mathrm{span}\{1, x, x^2\} \quad \text{for } n \ge 7
\]
\end{theorem}

\begin{enumerate}
\item $\ker(D_6) = \mathrm{span}\{1, x, x^2\}$ (3-dimensional)
\item $D_6[x^3] \ne 0$: $D_6$ detects cubic and higher degree polynomials
\item $\ker(D_7) = \ker(D_6)$: no new independent structure for $n \ge 7$
\end{enumerate}

\section{Elimination Conditions}
\label{sec:elimination}

Let $K_n$ denote the set of $k$ such that $D_n[x^k] = 0$:

\begin{center}
\begin{tabular}{|c|c|c|c|}
\hline
Operator & Coefficient sum & Eliminated monomials $K_n$ & $\dim \ker(D_n)$ \\
\hline
$D_2$ & $0$ & $\{0\}$ (constant) & 1 \\
$D_3$ & $0$ & $\{0\}$ (constant) & 1 \\
$D_4$ & $-\sqrt{5}$ & $\{2\}$ ($x^2$ only) & 1 \\
$D_5$ & $0$ & $\{0, 1\}$ (constant and $x$) & 2 \\
$D_6$ & $0$ & $\{0, 1, 2\}$ (constant, $x$, $x^2$) & 3 \\
\hline
\end{tabular}
\end{center}

\textbf{Gauge invariance}: When coefficient sum $\sigma(D) = 0$, we have $D[1] = 0$. $D_2, D_3, D_5, D_6$ are gauge invariant; $D_4$ breaks gauge invariance.

\textbf{Important}: $D_5, D_6$ have kernel dimensions greater than 1, which forms the basis for deriving gauge groups (Chapter~\ref{chap:gauge_groups}).

\section{Reduction of Integer Orders}
\label{sec:integer_reduction}

\begin{proposition}[Integer orders $n\ge7$ have no new structure]
\label{prop:n_ge_7}
\lean{FUST.D7_kernel_equals_D6_kernel}
\leanok
For any integer $n\ge7$, $\ker(D_n) = \ker(D_6)$.
\end{proposition}

\textbf{Important note}: This proposition asserts ``\textbf{identity of kernels},'' not ``linear combinations of function values.''

\subsection{Avoiding Misunderstanding}

$\varphi^k = F_k \varphi + F_{k-1}$ is a \textbf{relation between points (arguments)}, and for a general function $f$, $f(\varphi^k x) = F_k f(\varphi x) + F_{k-1} f(x)$ \textbf{does not hold} (unless $f$ is linear).

However, what this proposition asserts is:
\begin{itemize}
\item When $D_n$ coefficients are determined so that $\ker(D_n) \supseteq \ker(D_6)$
\item The result is $\ker(D_n) = \ker(D_6)$, and no new kernel elements appear
\end{itemize}

\section{Collapse of Half-Order Hierarchies}
\label{sec:half_collapse}

\begin{proposition}[Half-orders for $n\ge6$ collapse]
\label{prop:half_collapse}
For $n \ge 6$, half-order structures are absorbed into integer orders: $D_{n+0.5} \sim D_n$
\end{proposition}

\begin{proof}
By Binet's formula, antisymmetric components $f(\varphi^k x) - f(\psi^k x)$ are isomorphic to the structure of $D_2$ (including scaling). Since $D_6$ saturates elimination conditions, additional $D_2$ components are redundant.
\end{proof}

\section{Independence and Uniqueness of D5.5 (Lean4 Formal Proof)}
\label{sec:d55_uniqueness}

\begin{proposition}[Algebraic Independence of $D_{5.5}$]
\label{prop:d55_independence}
\lean{FUST.D5half_independence}
\leanok
$D_{5.5}$ is algebraically non-equivalent to $D_5$ and $D_6$:
\[
D_5[x] = D_6[x] = 0, \quad \text{but} \quad D_{5.5}[x] \neq 0
\]
\end{proposition}

\textbf{Important structural interpretation}:

\begin{center}
\begin{tabular}{|c|c|c|}
\hline
Operator & Polynomials in $\ker$ & Symmetry \\
\hline
$D_5$ & $\{1, x\}$ & Symmetric \\
$D_6$ & $\{1, x, x^2\}$ & Symmetric \\
$D_{5.5}$ & $\{1\}$ only & \textbf{Contains antisymmetric term} \\
\hline
\end{tabular}
\end{center}

$D_{5.5}$ does not ``break'' the symmetric structure of $D_5$ but \textbf{adds an orthogonal antisymmetric direction}. This shrinks the kernel of $D_5$ ($\{1,x\} \to \{1\}$), but $D_5$ itself is preserved.

\begin{proposition}[Uniqueness of $D_{5.5}$]
\label{prop:d55_unique}
The only algebraically non-equivalent half-order difference operator is $D_{5.5}$.
\end{proposition}

\begin{proof}
\begin{itemize}
\item $n < 5$: Half-order terms are absorbed into $D_{n+1}$
\item $n = 5$: There is a gap between $D_5$ and $D_6$, and the antisymmetric direction is independent
\item $n \ge 6$: Collapses by the above proposition
\end{itemize}
Therefore $D_{5.5}$ is the unique exception.
\end{proof}

\section{Six-Element Completeness Theorem}
\label{sec:main_theorem}

\begin{theorem}[Six-Element Completeness Theorem]
\label{thm:six_completeness}
\lean{FUST.six_element_completeness}
\leanok
In the Frourio difference hierarchy, the algebraically non-equivalent difference operators are limited to the \textbf{6 types}:
\[
\boxed{\{D_2, D_3, D_4, D_5, D_{5.5}, D_6\}}
\]
\end{theorem}

\begin{proof}[Proof outline]
\begin{enumerate}
\item \textbf{Integer orders $n \le 6$}: $D_2, D_3, D_4, D_5, D_6$ have different elimination patterns $K_n$ and are non-equivalent
\item \textbf{Integer orders $n \ge 7$}: Reduced to $D_6$ or lower by Fibonacci closure
\item \textbf{Half-integer orders $n \ge 6$}: Collapse to $D_n$ by Binet's formula
\item \textbf{Half-integer order $n = 5$}: Only $D_{5.5}$ is exceptional
\end{enumerate}
\end{proof}

\section{D6 Spectral Coefficients}
\label{sec:spectral_coefficients}

When $D_6$ acts on the monomial $x^n$, the result factors as
\[
D_6[x^n](x) = \frac{C_n}{(\varphi - \psi)^5} \cdot x^{n-1}
\]
where the \emph{spectral coefficient} $C_n$ is defined as follows.

\begin{definition}[D6 Spectral Coefficient]
\label{def:D6Coeff}
\lean{FUST.SpectralCoefficients.D6Coeff}
\leanok
\[
C_n := \varphi^{3n} - 3\varphi^{2n} + \varphi^n - \psi^n + 3\psi^{2n} - \psi^{3n}
\quad (n \in \mathbb{N})
\]
\end{definition}

\begin{theorem}[Fibonacci Representation]
\label{thm:D6Coeff_fibonacci}
\lean{FUST.SpectralCoefficients.D6Coeff_fibonacci}
\leanok
\[
C_n = \sqrt{5} \cdot \bigl(F_{3n} - 3F_{2n} + F_n\bigr)
\]
where $F_k$ denotes the $k$-th Fibonacci number.
\end{theorem}

\begin{theorem}[Polynomial Kernel Characterization]
\label{thm:D6Coeff_eq_zero_iff}
\lean{FUST.SpectralCoefficients.D6Coeff_eq_zero_iff}
\leanok
\[
C_n = 0 \iff n \le 2
\]
In particular, $C_0 = C_1 = C_2 = 0$ and $C_3 = 12\sqrt{5} \ne 0$.
\end{theorem}

\begin{theorem}[Factorization]
\label{thm:D6Coeff_factored}
\lean{FUST.SpectralCoefficients.D6Coeff_factored}
\leanok
\[
C_n = (\varphi^n - \psi^n)\bigl(\varphi^{2n} + \psi^{2n} + (-1)^n + 1 - 3(\varphi^n + \psi^n)\bigr)
\]
\end{theorem}

\begin{theorem}[Asymptotic Growth]
\label{thm:D6Coeff_asymptotic}
\lean{FUST.SpectralCoefficients.D6Coeff_asymptotic}
\leanok
For $n \ge 3$, there exists $C > 0$ such that
\[
|C_n - \varphi^{3n}| \le C \cdot \varphi^{2n}
\]
\end{theorem}

\subsection{Extended Kernel on \texorpdfstring{$\mathbb{Z}$}{Z}}
\label{subsec:extended_kernel}

Extending $C_n$ from $\mathbb{N}$ to $\mathbb{Z}$ using zpow yields the Laurent spectral coefficient.

\begin{definition}[Laurent D6 Coefficient]
\label{def:D6CoeffZ}
\lean{FUST.SpectralCoefficients.D6CoeffZ}
\leanok
\[
C_n^{\mathbb{Z}} := \varphi^{3n} - 3\varphi^{2n} + \varphi^n - \psi^n + 3\psi^{2n} - \psi^{3n}
\quad (n \in \mathbb{Z})
\]
\end{definition}

\begin{theorem}[Antisymmetry for Even Arguments]
\label{thm:D6CoeffZ_neg_even}
\lean{FUST.SpectralCoefficients.D6CoeffZ_neg_even}
\leanok
For even $n \in \mathbb{Z}$:
\[
C_{-n}^{\mathbb{Z}} = -C_n^{\mathbb{Z}}
\]
\end{theorem}

\begin{theorem}[Extended Kernel Dimension]
\label{thm:D6_extended_kernel_dim}
\lean{FUST.SpectralCoefficients.D6_extended_kernel_dim}
\leanok
The Laurent kernel of $D_6$ is 4-dimensional:
\[
\{n \in \mathbb{Z} \mid C_n^{\mathbb{Z}} = 0\} \supseteq \{-2, 0, 1, 2\}
\]
\end{theorem}

\begin{theorem}[Kernel Gap Structure]
\label{thm:D6_kernel_gap_structure}
\lean{FUST.SpectralCoefficients.D6_kernel_gap_structure}
\leanok
The extended kernel has a gap at $n = -1$:
\[
C_{-3}^{\mathbb{Z}} \ne 0, \quad C_{-2}^{\mathbb{Z}} = 0, \quad C_{-1}^{\mathbb{Z}} \ne 0, \quad
C_0^{\mathbb{Z}} = C_1^{\mathbb{Z}} = C_2^{\mathbb{Z}} = 0, \quad C_3^{\mathbb{Z}} \ne 0
\]
The polynomial kernel $\{0, 1, 2\}$ extends to the Laurent kernel $\{-2, 0, 1, 2\}$,
with $n = -2$ connected via the antisymmetry $C_{-2}^{\mathbb{Z}} = -C_2^{\mathbb{Z}} = 0$.
The isolated zero at $n = -2$ (separated by the nonzero gap $C_{-1}^{\mathbb{Z}} \ne 0$)
distinguishes the Laurent extension from the polynomial kernel.
\end{theorem}

\textbf{Remark}: The polynomial kernel $\ker(D_6) \cap \mathbb{R}[x] = \operatorname{span}\{1, x, x^2\}$ (dimension 3) is the physically relevant kernel used for spatial dimension derivation.
The Laurent kernel (dimension 4) arises naturally in the Mellin transform setting on $L^2(\mathbb{R}_+, dx/x)$.
