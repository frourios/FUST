% Chapter 5: Six-Element Completeness and Dζ Unification
\chapter{Six-Element Completeness and \texorpdfstring{$D_\zeta$}{Dζ} Unification}
\label{chap:six_completeness}

\section{\texorpdfstring{$D_6$}{D6} Completeness}
\label{sec:fust_completeness}

\begin{theorem}[Kernel Equality]
\label{thm:kernel_equality}
\lean{FUST.D7_kernel_equals_D6_kernel}
\leanok
\[
\ker(D_n)|_{\mathbb{C}[z]} = \ker(D_6)|_{\mathbb{C}[z]} = \mathrm{span}\{1, z, z^2\} \quad \text{for } n \ge 7
\]
\end{theorem}

\begin{enumerate}
\item $\ker(D_6) = \mathrm{span}\{1, z, z^2\}$ (3-dimensional)
\item $D_6[z^3] \ne 0$: $D_6$ detects cubic and higher degree polynomials
\item $\ker(D_7) = \ker(D_6)$: no new independent structure for $n \ge 7$
\end{enumerate}

\section{Elimination Conditions}
\label{sec:elimination}

\begin{center}
\begin{tabular}{|c|c|c|c|}
\hline
Operator & Coeff.\ sum & Eliminated monomials & $\dim \ker$ \\
\hline
$D_2$ & $0$ & $\{z^0\}$ (constant) & 1 \\
$D_3$ & $0$ & $\{z^0\}$ (constant) & 1 \\
$D_4$ & $-\sqrt{5}$ & $\{z^2\}$ ($z^2$ only) & 1 \\
$D_5$ & $0$ & $\{z^0, z^1\}$ (constant and $z$) & 2 \\
$D_6$ & $0$ & $\{z^0, z^1, z^2\}$ (constant, $z$, $z^2$) & 3 \\
\hline
\end{tabular}
\end{center}

\textbf{$D_\zeta$ perspective}: Each operator corresponds to a specific projection of $D_\zeta$ through the $\mathbb{Z}/6\mathbb{Z}$ Fourier decomposition. $D_2, D_4, D_6$ lie in the antisymmetric (AF) channel; $D_3, D_5$ lie in the symmetric (SY) channel. The kernel hierarchy $1 \to 2 \to 3$ is a structural consequence of the hyperbolic-elliptic composition.

\section{Integer Order Reduction}
\label{sec:integer_reduction}

\begin{proposition}[Integer orders $n\ge7$ have no new structure]
\label{prop:n_ge_7}
\lean{FUST.D7_kernel_equals_D6_kernel}
\leanok
For any integer $n\ge7$, $\ker(D_n) = \ker(D_6)$.
\end{proposition}

This means $D_6$ is the \textbf{completeness boundary}: the hyperbolic lattice cannot produce kernel dimensions greater than 3. In $D_\zeta$, this is reflected by the fact that $\zeta_6^6 = 1$ closes the elliptic cycle.

\section{Half-Order Collapse}
\label{sec:half_collapse}

\begin{proposition}[Half-orders for $n\ge6$ collapse]
\label{prop:half_collapse}
For $n \ge 6$, half-order structures are absorbed into integer orders: $D_{n+0.5} \sim D_n$.
\end{proposition}

\section{Independence and Uniqueness of \texorpdfstring{$D_{5\frac{1}{2}}$}{D5half}}
\label{sec:d55_uniqueness}

\begin{proposition}[Algebraic Independence of $D_{5\frac{1}{2}}$]
\label{prop:d55_independence}
\lean{FUST.D5half_independence}
\leanok
$D_{5\frac{1}{2}}$ is algebraically non-equivalent to $D_5$ and $D_6$:
\[
D_5[z] = D_6[z] = 0, \quad \text{but} \quad D_{5\frac{1}{2}}[z] \neq 0
\]
\end{proposition}

In the $D_\zeta$ framework, $D_{5\frac{1}{2}}$ bridges the AF and SY channels: it is $D_5$ (SY channel) plus $\mu \cdot \Delta_2$ (AF channel component). This direct sum structure
\[
D_{5\frac{1}{2}} \in \mathrm{ScaleQ}(-4,0) \oplus \mathrm{ScaleQ}(-1,1)
\]
is the unique half-order that crosses the channel boundary.

\section{Six-Element Completeness Theorem}
\label{sec:main_theorem}

\begin{theorem}[Six-Element Completeness Theorem]
\label{thm:six_completeness}
\lean{FUST.six_element_completeness}
\leanok
In the Frourio difference hierarchy, the algebraically non-equivalent difference operators are limited to the \textbf{6 types}:
\[
\boxed{\{D_2, D_3, D_4, D_5, D_{5\frac{1}{2}}, D_6\}}
\]
\end{theorem}

These 6 operators are the component projections of the unified $D_\zeta$:
\begin{itemize}
\item AF channel: $D_2$ ($r=5$), $D_4$ ($r=3$), $D_6$ ($r=1$)
\item SY channel: $D_3$ ($r=4$), $D_5$ ($r=2$)
\item Channel bridge: $D_{5\frac{1}{2}}$ (AF $\oplus$ SY)
\end{itemize}

\section{\texorpdfstring{$D_6$}{D6} Spectral Coefficients}
\label{sec:spectral_coefficients}

\begin{definition}[D6 Spectral Coefficient]
\label{def:D6Coeff}
\lean{FUST.SpectralCoefficients.D6Coeff}
\leanok
\[
C_n := \varphi^{3n} - 3\varphi^{2n} + \varphi^n - \psi^n + 3\psi^{2n} - \psi^{3n}
\quad (n \in \mathbb{N})
\]
\end{definition}

\begin{theorem}[Fibonacci Representation]
\label{thm:D6Coeff_fibonacci}
\lean{FUST.SpectralCoefficients.D6Coeff_fibonacci}
\leanok
$C_n = \sqrt{5} \cdot (F_{3n} - 3F_{2n} + F_n)$
\end{theorem}

\begin{theorem}[Polynomial Kernel Characterization]
\label{thm:D6Coeff_eq_zero_iff}
\lean{FUST.SpectralCoefficients.D6Coeff_eq_zero_iff}
\leanok
$C_n = 0 \iff n \le 2$. In particular, $C_0 = C_1 = C_2 = 0$ and $C_3 = 12\sqrt{5} \ne 0$.
\end{theorem}

\begin{theorem}[Factorization]
\label{thm:D6Coeff_factored}
\lean{FUST.SpectralCoefficients.D6Coeff_factored}
\leanok
\[
C_n = (\varphi^n - \psi^n)\bigl(\varphi^{2n} + \psi^{2n} + (-1)^n + 1 - 3(\varphi^n + \psi^n)\bigr)
\]
\end{theorem}

\begin{theorem}[Asymptotic Growth]
\label{thm:D6Coeff_asymptotic}
\lean{FUST.SpectralCoefficients.D6Coeff_asymptotic}
\leanok
For $n \ge 3$, there exists $C > 0$ such that $|C_n - \varphi^{3n}| \le C \cdot \varphi^{2n}$.
\end{theorem}

\subsection{Extended Kernel on \texorpdfstring{$\mathbb{Z}$}{Z}}

\begin{definition}[Laurent D6 Coefficient]
\label{def:D6CoeffZ}
\lean{FUST.SpectralCoefficients.D6CoeffZ}
\leanok
$C_n^{\mathbb{Z}} := \varphi^{3n} - 3\varphi^{2n} + \varphi^n - \psi^n + 3\psi^{2n} - \psi^{3n}$ for $n \in \mathbb{Z}$.
\end{definition}

\begin{theorem}[Extended Kernel Dimension]
\label{thm:D6_extended_kernel_dim}
\lean{FUST.SpectralCoefficients.D6_extended_kernel_dim}
\leanok
The Laurent kernel is 4-dimensional: $\{n \in \mathbb{Z} \mid C_n^{\mathbb{Z}} = 0\} \supseteq \{-2, 0, 1, 2\}$.
\end{theorem}

\begin{theorem}[Kernel Gap Structure]
\label{thm:D6_kernel_gap_structure}
\lean{FUST.SpectralCoefficients.D6_kernel_gap_structure}
\leanok
$C_{-3}^{\mathbb{Z}} \ne 0$, $C_{-2}^{\mathbb{Z}} = 0$, $C_{-1}^{\mathbb{Z}} \ne 0$, $C_0 = C_1 = C_2 = 0$, $C_3 \ne 0$.
The polynomial kernel $\{0, 1, 2\}$ extends to the Laurent kernel $\{-2, 0, 1, 2\}$.
\end{theorem}
