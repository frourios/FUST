% Chapter 5: Six-Element Completeness and Dζ Unification
\chapter{Six-Element Completeness and \texorpdfstring{$D_\zeta$}{Dζ} Unification}
\label{chap:six_completeness}

\section{\texorpdfstring{$\mathrm{Diff}_6$}{Diff6} Completeness}
\label{sec:fust_completeness}

\begin{theorem}[Kernel Equality]
\label{thm:kernel_equality}
\lean{FUST.Diff7_kernel_equals_Diff6_kernel}
\leanok
\[
\ker(\mathrm{Diff}_n)|_{\mathbb{C}[z]} = \ker(\mathrm{Diff}_6)|_{\mathbb{C}[z]} = \mathrm{span}\{1, z, z^2\} \quad \text{for } n \ge 7
\]
\end{theorem}

\begin{enumerate}
\item $\ker(\mathrm{Diff}_6) = \mathrm{span}\{1, z, z^2\}$ (3-dimensional)
\item $\mathrm{Diff}_6[z^3] \ne 0$: $\mathrm{Diff}_6$ detects cubic and higher degree polynomials
\item $\ker(\mathrm{Diff}_7) = \ker(\mathrm{Diff}_6)$: no new independent structure for $n \ge 7$
\end{enumerate}

\section{Elimination Conditions}
\label{sec:elimination}

\begin{center}
\begin{tabular}{|c|c|c|c|}
\hline
Operator & Coeff.\ sum & Eliminated monomials & $\dim \ker$ \\
\hline
$\mathrm{Diff}_2$ & $0$ & $\{z^0\}$ (constant) & 1 \\
$\mathrm{Diff}_3$ & $0$ & $\{z^0\}$ (constant) & 1 \\
$\mathrm{Diff}_4$ & $-\sqrt{5}$ & $\{z^2\}$ ($z^2$ only) & 1 \\
$\mathrm{Diff}_5$ & $0$ & $\{z^0, z^1\}$ (constant and $z$) & 2 \\
$\mathrm{Diff}_6$ & $0$ & $\{z^0, z^1, z^2\}$ (constant, $z$, $z^2$) & 3 \\
\hline
\end{tabular}
\end{center}

\textbf{$D_\zeta$ perspective}: Each operator corresponds to a specific projection of $D_\zeta$ through the $\mathbb{Z}/6\mathbb{Z}$ Fourier decomposition. $\mathrm{Diff}_2, \mathrm{Diff}_4, \mathrm{Diff}_6$ lie in the antisymmetric (AF) channel; $\mathrm{Diff}_3, \mathrm{Diff}_5$ lie in the symmetric (SY) channel. The kernel hierarchy $1 \to 2 \to 3$ is a structural consequence of the hyperbolic-elliptic composition.

\section{Integer Order Reduction}
\label{sec:integer_reduction}

\begin{proposition}[Integer orders $n\ge7$ have no new structure]
\label{prop:n_ge_7}
\lean{FUST.Diff7_kernel_equals_Diff6_kernel}
\leanok
For any integer $n\ge7$, $\ker(N_n) = \ker(\mathrm{Diff}_6)$.
\end{proposition}

This means $\mathrm{Diff}_6$ is the \textbf{completeness boundary}: the hyperbolic lattice cannot produce kernel dimensions greater than 3. In $D_\zeta$, this is reflected by the fact that $\zeta_6^6 = 1$ closes the elliptic cycle.

\section{Half-Order Collapse}
\label{sec:half_collapse}

\begin{proposition}[Half-orders for $n\ge6$ collapse]
\label{prop:half_collapse}
For $n \ge 6$, half-order structures are absorbed into integer orders.
\end{proposition}

\section{Six-Element Completeness Theorem}
\label{sec:main_theorem}

\begin{theorem}[Six-Element Completeness Theorem]
\label{thm:six_completeness}
\lean{FUST.six_element_completeness}
\leanok
In the Frourio difference hierarchy, the algebraically non-equivalent difference operators are limited to the \textbf{6 types}:
\[
\boxed{\{\mathrm{Diff}_2, \mathrm{Diff}_3, \mathrm{Diff}_4, \mathrm{Diff}_5, \Phi_S, \mathrm{Diff}_6\}}
\]
\end{theorem}

These 6 operators are the component projections of the unified $D_\zeta$:
\begin{itemize}
\item AF channel: $\mathrm{Diff}_2$ ($r=5$), $\mathrm{Diff}_4$ ($r=3$), $\mathrm{Diff}_6$ ($r=1$)
\item SY channel: $\mathrm{Diff}_3$ ($r=4$), $\mathrm{Diff}_5$ ($r=2$)
\item Channel bridge: $\Phi_S$ (AF $\oplus$ SY)
\end{itemize}
