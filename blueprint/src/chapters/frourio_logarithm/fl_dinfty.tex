\section{\texorpdfstring{D$_\infty$}{D∞} Generator Linearization}
\label{sec:dinfty-linearization}

\subsection{The Infinite Dihedral Group}

The infinite dihedral group $D_\infty$ has two generators:
\begin{itemize}
    \item $U$ (scale): $x \mapsto \varphi x$
    \item $R$ (reflection): $x \mapsto -1/x$ (using $\varphi \cdot \psi = -1$)
\end{itemize}

These satisfy the relations:
\[
R^2 = 1, \quad RUR = U^{-1}
\]

\subsection{Generators in Real Space}

\begin{definition}[Scale Generator]
\label{def:scale-u}
\lean{FUST.FrourioLogarithm.scaleU}
\leanok
\[
U(x) := \varphi \cdot x
\]
\end{definition}

\begin{definition}[Inverse Scale Generator]
\label{def:scale-u-inv}
\lean{FUST.FrourioLogarithm.scaleUInvPos}
\leanok
\[
U^{-1}(x) := \varphi^{-1} \cdot x = (\varphi - 1) \cdot x
\]
\end{definition}

\begin{definition}[Reflection Generator]
\label{def:reflect-r}
\lean{FUST.FrourioLogarithm.reflectR}
\leanok
\[
R(x) := -\frac{1}{x}
\]
\end{definition}

\begin{theorem}[Reflection Involution]
\label{thm:reflect-squared}
\lean{FUST.FrourioLogarithm.reflectR_squared}
\leanok
For $x \neq 0$:
\[
R(R(x)) = x
\]
\end{theorem}

\subsection{Linearization in Frourio Coordinates}

\begin{theorem}[Scale Generator in Frourio Coordinates]
\label{thm:scale-u-frourio}
\lean{FUST.FrourioLogarithm.scaleU_frourio}
\leanok
For $x > 0$:
\[
\log_{\mathfrak{f}}(U(x)) = \log_{\mathfrak{f}}(x) + \mathrm{phiStep}
\]
The scale generator becomes a translation by $\mathrm{phiStep}$.
\end{theorem}

\begin{theorem}[Inverse Scale in Frourio Coordinates]
\label{thm:scale-u-inv-frourio}
\lean{FUST.FrourioLogarithm.scaleUInvPos_frourio}
\leanok
For $x > 0$:
\[
\log_{\mathfrak{f}}(U^{-1}(x)) = \log_{\mathfrak{f}}(x) - \mathrm{phiStep}
\]
The inverse scale generator becomes a translation by $-\mathrm{phiStep}$.
\end{theorem}

\subsection{The \texorpdfstring{$\varphi\psi = -1$}{φψ = -1} Identity}

\begin{theorem}[Scale Composition Sign Flip]
\label{thm:scale-sign-flip}
\lean{FUST.FrourioLogarithm.scaleU_scaleUInv_eq}
\leanok
\[
U(U_\psi(x)) = \varphi \cdot (\psi \cdot x) = (\varphi \cdot \psi) \cdot x = -x
\]
where $U_\psi(x) = \psi \cdot x$.
\end{theorem}

This reflects the fundamental identity $\varphi \cdot \psi = -1$ in the group structure.
