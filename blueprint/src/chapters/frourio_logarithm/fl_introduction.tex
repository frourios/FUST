\chapter{Frourio Logarithm: The Natural Coordinate System}
\label{chap:frourio-logarithm}

\section{Introduction}

The Frourio logarithm provides the natural coordinate system for FUST, transforming the multiplicative structure of $\varphi$-scaling into an additive structure suitable for physical interpretation.

\subsection{Motivation}

In FUST, the fundamental symmetry is $\varphi$-scaling: $x \mapsto \varphi x$. In real space, this is a multiplicative operation. However, physics prefers additive structures (e.g., energy addition, time translation). The Frourio logarithm resolves this by defining:
\[
t := \log_{\mathfrak{f}}(x) = \frac{\log x}{\log \mathfrak{f}}
\]
where $\mathfrak{f}$ is the Frourio constant.

\subsection{Key Properties}

\begin{enumerate}
    \item \textbf{Linearization}: $\varphi$-scaling becomes translation: $x \mapsto \varphi x$ becomes $t \mapsto t + \Delta$
    \item \textbf{Time Interpretation}: The coordinate $t$ naturally interprets as proper time
    \item \textbf{D$_\infty$ Linearization}: The infinite dihedral group becomes linear operations
    \item \textbf{Entropy Connection}: Frourio entropy provides a natural information measure
\end{enumerate}

\subsection{Lean Formalization}

The Frourio logarithm is formalized in \texttt{FUST/FrourioLogarithm.lean}:

\begin{definition}[Frourio Logarithm]
\label{def:frourio-log}
\lean{FUST.FrourioLogarithm.frourioLog}
\leanok
For $x > 0$:
\[
\mathrm{frourioLog}(x) := \frac{\log x}{\log \mathfrak{f}}
\]
\end{definition}

\begin{definition}[Frourio Exponential]
\label{def:frourio-exp}
\lean{FUST.FrourioLogarithm.frourioExp}
\leanok
\[
\mathrm{frourioExp}(t) := \mathfrak{f}^t
\]
\end{definition}

\begin{theorem}[Inverse Properties]
\label{thm:frourio-inverse}
\lean{FUST.FrourioLogarithm.frourioExp_frourioLog, FUST.FrourioLogarithm.frourioLog_frourioExp}
\leanok
For $x > 0$ and all $t \in \mathbb{R}$:
\begin{enumerate}
    \item $\mathrm{frourioExp}(\mathrm{frourioLog}(x)) = x$
    \item $\mathrm{frourioLog}(\mathrm{frourioExp}(t)) = t$
\end{enumerate}
\end{theorem}
