\section{State Function Constraint: from \texorpdfstring{$\mathbb{Q}(\varphi)$}{Q(φ)} to \texorpdfstring{$\mathbb{Z}[\varphi, \zeta_6]$}{Z[φ,ζ₆]}}
\label{sec:state-function-constraint}

FUST theorems hold for universal variable $x$, but physically manifest particle states
occupy specific initial values $x_0$.  This section first proves that $x_0$ is
algebraically constrained to the golden quadratic field
$\mathbb{Q}(\varphi) = \{a + b\varphi \mid a, b \in \mathbb{Q}\}$,
then upgrades the constraint to the golden--Eisenstein integer ring
$\mathbb{Z}[\varphi, \zeta_6]$ via the integral operator
$\tilde{D}_\zeta = 5z \cdot D_\zeta$.

\subsection{Golden Quadratic Field \texorpdfstring{$\mathbb{Q}(\varphi)$}{Q(φ)}}

\begin{definition}[Membership in $\mathbb{Q}(\varphi)$]
\label{def:in-golden-field}
$x \in \mathbb{Q}(\varphi)$ iff $\exists\, a, b \in \mathbb{Q}$ such that $x = a + b\varphi$.
\end{definition}

\begin{theorem}[Field operations on $\mathbb{Q}(\varphi)$]
\label{thm:golden-field-ops}
\uses{def:in-golden-field}
$\mathbb{Q}(\varphi)$ is closed under addition, negation, and multiplication.
Multiplication closure uses $\varphi^2 = \varphi + 1$.
For all $n \in \mathbb{Z}$, $\varphi^n \in \mathbb{Q}(\varphi)$.
\end{theorem}

\subsection{Frourio-Admissible Subfield}

\begin{definition}[Frourio-admissible set]
\label{def:frourio-admissible}
A subset $K \subseteq \mathbb{R}$ is \emph{Frourio-admissible} if:
\begin{enumerate}
    \item (PBW) $\varphi \in K$ and $\varphi^{-1} \in K$
    \item (Field closure) $\mathbb{Q} \subseteq K$, and $K$ is closed under $+$, $\times$, negation
    \item (Quadratic closure from $\chi(U) = -1$) $\forall x \in K,\, \exists\, a, b \in \mathbb{Q}: x = a + b\varphi$
\end{enumerate}
These three conditions are independently derived from the Frourio algebra:
\begin{itemize}
    \item Condition 1 follows from the Frourio operator $\Delta = M_{1/x}(\alpha U - \beta R^{\varepsilon} U^{-1})$ requiring $\varphi, \varphi^{-1}$ as scale parameters.
    \item Condition 2 follows from the crossed product algebra requiring a coefficient field.
    \item Condition 3 follows from the group character $\chi(U) = -1$ forcing a degree-2 Galois extension.
\end{itemize}
\end{definition}

\begin{theorem}[Uniqueness of $\mathbb{Q}(\varphi)$]
\label{thm:golden-field-unique}
\uses{def:frourio-admissible, def:in-golden-field}
$\mathbb{Q}(\varphi)$ is the unique Frourio-admissible subfield of $\mathbb{R}$:
any $K$ satisfying all three conditions equals $\mathbb{Q}(\varphi)$.
\end{theorem}

\begin{proof}
\textbf{Minimality} ($\mathbb{Q}(\varphi) \subseteq K$):
For any $a + b\varphi \in \mathbb{Q}(\varphi)$, $a \in K$ and $b \in K$ by $\mathbb{Q} \subseteq K$,
$\varphi \in K$ by PBW, so $b\varphi \in K$ and $a + b\varphi \in K$ by closure.
\textbf{Maximality} ($K \subseteq \mathbb{Q}(\varphi)$):
By the quadratic closure condition, every $x \in K$ has the form $a + b\varphi$,
hence $x \in \mathbb{Q}(\varphi)$.
\end{proof}

\subsection{\texorpdfstring{$\varphi$}{φ}-Orbit and Galois Conjugation}

\begin{theorem}[Orbit closure in $\mathbb{Q}(\varphi)$]
\label{thm:orbit-in-golden-field}
\uses{def:in-golden-field}
If $x_0 \in \mathbb{Q}(\varphi)$, then the entire $\varphi$-orbit
$\mathcal{O}(x_0) = \{\varphi^n x_0 \mid n \in \mathbb{Z}\}$
lies in $\mathbb{Q}(\varphi)$.
\end{theorem}

The Galois conjugation $\sigma : a + b\varphi \mapsto (a+b) + (-b)\varphi$
is an involution preserving $\mathbb{Q}(\varphi)$.  The group character
$\chi(U) = -1$ has exact order 2, matching $[\mathbb{Q}(\varphi):\mathbb{Q}] = 2$.

\begin{theorem}[$\mathbb{Q}(\varphi) = \mathbb{Q}(\sqrt{5})$]
\label{thm:golden-field-eq-sqrt5}
\uses{def:in-golden-field}
$\mathbb{Q}(\varphi)$ and $\mathbb{Q}(\sqrt{5})$ coincide:
$a + b\varphi = (a + b/2) + (b/2)\sqrt{5}$ and
$p + q\sqrt{5} = (p - q) + 2q\varphi$.
\end{theorem}

\subsection{Physical Constraint: \texorpdfstring{$\mathbb{Z}[\varphi]$}{Z[φ]}}

The coefficient \emph{field} $\mathbb{Q}(\varphi)$ is required for PBW faithfulness,
but physical generation processes produce only $\mathbb{Z}[\varphi]$ coefficients.
The two-point difference operator preserves coefficient valuation over $\mathbb{Z}[\varphi]$:
$v_{\mathfrak{p}}(\Delta f) \ge v_{\mathfrak{p}}(f)$.
No denominators arise from finite compositions of Frourio operations,
so physically manifest initial values lie in $\mathbb{Z}[\varphi]$.

This $\mathbb{Z}[\varphi]$-constraint captures the symmetric (SY) channel
of $D_\zeta$.  The full operator also has an antisymmetric--Fibonacci (AF)
channel with coefficient $4\zeta_6 - 2 \in \mathbb{Z}[\zeta_6]$,
requiring the upgrade to $\mathbb{Z}[\varphi, \zeta_6]$ below.

\subsection{The Golden--Eisenstein Integer Ring}

\begin{definition}[{$\mathbb{Z}[\varphi, \zeta_6]$}]
\label{def:golden-eisenstein-int}
\lean{FUST.FrourioAlgebra.GoldenEisensteinInt}
\leanok
An element $x \in \mathbb{Z}[\varphi, \zeta_6]$ is represented as
$a + b\varphi + c\zeta_6 + d\varphi\zeta_6$ with $a,b,c,d \in \mathbb{Z}$,
subject to $\varphi^2 = \varphi + 1$ and $\zeta_6^2 = \zeta_6 - 1$.
\end{definition}

\begin{theorem}[Commutative ring]
\label{thm:gei-comm-ring}
\uses{def:golden-eisenstein-int}
$\mathbb{Z}[\varphi, \zeta_6]$ forms a commutative ring with the explicit
multiplication formula verified by \texttt{ext <;> ring}.
\end{theorem}

\begin{definition}[{Embedding $\mathbb{Z}[\varphi, \zeta_6] \hookrightarrow \mathbb{C}$}]
\label{def:gei-to-complex}
\lean{FUST.FrourioAlgebra.GoldenEisensteinInt.toComplex}
\leanok
\uses{def:golden-eisenstein-int}
$\mathrm{toComplex}(a + b\varphi + c\zeta_6 + d\varphi\zeta_6)
  = a + b\varphi + c\zeta_6 + d\varphi\zeta_6 \in \mathbb{C}$.
This map preserves addition and multiplication.
\end{definition}

\begin{theorem}[$\mathrm{toComplex}$ preserves multiplication]
\label{thm:gei-tocomplex-mul}
\lean{FUST.FrourioAlgebra.GoldenEisensteinInt.toComplex_mul}
\leanok
\uses{def:gei-to-complex}
$\mathrm{toComplex}(x \cdot y) = \mathrm{toComplex}(x) \cdot \mathrm{toComplex}(y)$.
\end{theorem}

\subsection{Galois Group \texorpdfstring{$\mathrm{Gal}(K/\mathbb{Q}) \cong \mathbb{Z}/2 \times \mathbb{Z}/2$}{Gal(K/Q) ≅ Z/2 x Z/2}}

\begin{definition}[Galois automorphisms $\sigma, \tau$]
\label{def:gei-galois}
\lean{FUST.FrourioAlgebra.GoldenEisensteinInt.sigma}
\leanok
\uses{def:golden-eisenstein-int}
\begin{align}
\sigma &: \varphi \mapsto \psi,\ \zeta_6 \mapsto \zeta_6
  & \sigma(a,b,c,d) &= (a+b,\, -b,\, c+d,\, -d) \\
\tau &: \varphi \mapsto \varphi,\ \zeta_6 \mapsto \bar{\zeta}_6
  & \tau(a,b,c,d) &= (a+c,\, b+d,\, -c,\, -d)
\end{align}
Both are ring homomorphisms and involutions; $\sigma\tau = \tau\sigma$.
\end{definition}

\begin{theorem}[Galois properties]
\label{thm:gei-galois-props}
\lean{FUST.FrourioAlgebra.GoldenEisensteinInt.sigma_involution}
\leanok
\uses{def:gei-galois}
\begin{itemize}
  \item $\sigma^2 = \mathrm{id}$, $\tau^2 = \mathrm{id}$, $\sigma\tau = \tau\sigma$
  \item $\sigma(xy) = \sigma(x)\sigma(y)$, $\tau(xy) = \tau(x)\tau(y)$
\end{itemize}
\end{theorem}

\subsection{Norm and Unique Factorization}

\begin{definition}[{Full norm $N : \mathbb{Z}[\varphi,\zeta_6] \to \mathbb{Z}$}]
\label{def:gei-norm}
\lean{FUST.FrourioAlgebra.GoldenEisensteinInt.norm}
\leanok
\uses{def:golden-eisenstein-int, def:gei-galois}
$N(x) = \mathrm{Nm}_{\mathbb{Z}[\varphi]}(x \cdot \tau(x))$,
where $\mathrm{Nm}_{\mathbb{Z}[\varphi]}$ is the norm of the golden integer
sub-ring. Since $h_K = 1$, irreducibility is detected by prime norm.
\end{definition}

\subsection{Integral Operator \texorpdfstring{$\tilde{D}_\zeta$}{D̃ζ}}

\begin{definition}[$\tilde{D}_\zeta = 5z \cdot D_\zeta$]
\label{def:dzeta-int}
\lean{FUST.IntegralDzeta.Dζ_int}
\leanok
The rescaled operator $\tilde{D}_\zeta$ replaces $\Phi_S$ by
$\Phi_{S,\mathrm{int}} = 5\Phi_S$ whose coefficients
$[10,\; 21{-}2\varphi,\; {-}50,\; 9{+}2\varphi,\; 10]$
all lie in $\mathbb{Z}[\varphi]$.
\end{definition}

\begin{theorem}[$\tilde{D}_\zeta = 5z \cdot D_\zeta$]
\label{thm:dzeta-int-eq}
\lean{FUST.IntegralDzeta.Dζ_int_eq}
\leanok
\uses{def:dzeta-int}
$\tilde{D}_\zeta f(z) = 5z \cdot D_\zeta f(z)$ for $z \ne 0$.
\end{theorem}

\begin{theorem}[Kernel: $\tilde{D}_\zeta(z^n) = 0$ for $n \equiv 0,2,3,4 \pmod{6}$]
\label{thm:dzeta-int-kernel}
\lean{FUST.IntegralDzeta.Dζ_int_vanish_mod6_0}
\leanok
\uses{def:dzeta-int}
The kernel of $\tilde{D}_\zeta$ on monomials is spanned by
$\{z^n : 6 \nmid (n \bmod 6) \in \{0,2,3,4\}\}$.
Only $n \equiv 1, 5 \pmod{6}$ are \emph{active}.
\end{theorem}

\begin{theorem}[Eigenvalue formula for active modes]
\label{thm:dzeta-int-eigenvalue}
\lean{FUST.IntegralDzeta.Dζ_int_eigenvalue_mod6_1}
\leanok
\uses{def:dzeta-int}
For $n = 6k + 1$:
\[
\tilde{D}_\zeta(z^n) = \bigl(5 c_A(n)\, (4\zeta_6 - 2) + 6\, c_S(n)\bigr) \cdot z^n
\]
where $c_A(n), c_S(n) \in \mathbb{Z}[\varphi]$ are explicit Fibonacci polynomials.
The eigenvalue lies in $\mathbb{Z}[\varphi, \zeta_6]$.
\end{theorem}

\subsection{Spectral Obstruction}

Every $\tilde{D}_\zeta$ eigenvalue has the form
$c_A \cdot (4\zeta_6 - 2) + c_S \cdot 6$ with $c_A, c_S \in \mathbb{Z}[\varphi]$.
The spectral obstruction theorems show this expression is always divisible by
$10 \cdot (1 + \zeta_6)$ in $\mathbb{Z}[\varphi, \zeta_6]$.

\begin{theorem}[Universal 2-divisibility]
\label{thm:eigenvalue-even}
\lean{FUST.SpectralObstruction.eigenvalue_even}
\leanok
\uses{def:golden-eisenstein-int}
For all $c_A, c_S \in \mathbb{Z}$:
\[
c_A \cdot (4\zeta_6 - 2) + c_S \cdot 6 = 2\mu
\quad \text{for some } \mu \in \mathbb{Z}[\varphi, \zeta_6].
\]
\end{theorem}

\begin{theorem}[Mod 3 structure]
\label{thm:half-eigenvalue-mod3}
\lean{FUST.SpectralObstruction.half_eigenvalue_mod3}
\leanok
\uses{thm:eigenvalue-even}
The half-eigenvalue $\mu$ satisfies
$\mu \equiv 2c_A \cdot (1+\zeta_6) \pmod{3}$.
\end{theorem}

\begin{definition}[$1+\zeta_6$: the prime above 3]
\label{def:one-plus-zeta6}
\lean{FUST.SpectralObstruction.one_plus_zeta6}
\leanok
\uses{def:golden-eisenstein-int}
$(1+\zeta_6)(2-\zeta_6) = 3$, so $1+\zeta_6$ is a prime element of norm~9.
\end{definition}

\begin{theorem}[Norm of obstruction ideal]
\label{thm:obstruction-norm}
\lean{FUST.SpectralObstruction.obstruction_ideal_norm}
\leanok
\uses{def:one-plus-zeta6, def:gei-norm}
$N(10 \cdot (1+\zeta_6)) = 90000 = 2^4 \cdot 3^2 \cdot 5^4$.
The three Euler factors $\{2, 3, 5\}$ encode the spectral gap.
\end{theorem}

\begin{theorem}[Shadow lattice]
\label{thm:shadow-lattice}
\lean{FUST.SpectralObstruction.phi_not_reachable}
\leanok
\uses{def:golden-eisenstein-int, thm:eigenvalue-even}
The elements $\varphi$, $\zeta_6$, $\varphi\zeta_6$ are
unreachable from $\mathbb{Z} + 2\mathbb{Z}[\varphi,\zeta_6]$:
eigenvalues cannot generate the full ring.
\end{theorem}

\subsection{Pairwise Dark Coupling}

\begin{theorem}[Active modes always couple to kernel]
\label{thm:pairwise-dark}
\lean{FUST.SpectralObstruction.pairwise_sum_in_kernel}
\leanok
If $a, b \equiv 1$ or $5 \pmod{6}$ (active residues), then
$a + b \not\equiv 1$ or $5 \pmod{6}$.
Pairwise products of active modes fall into the $\tilde{D}_\zeta$ kernel.
\end{theorem}

\begin{theorem}[Triple coupling can be active]
\label{thm:triple-active}
\lean{FUST.SpectralObstruction.triple_sum_can_be_active}
\leanok
$\exists\, a, b, c$ active such that $a + b + c$ is active
(e.g.\ $1 + 1 + 5 \equiv 1$).
\end{theorem}

\subsection{Summary}

The constraint on state functions has three layers:
\begin{enumerate}
  \item \textbf{Algebraic}: $\mathbb{Q}(\varphi)$ is the unique Frourio-admissible coefficient field (Theorem~\ref{thm:golden-field-unique}).
  \item \textbf{Physical}: Finite Frourio operations produce only $\mathbb{Z}[\varphi]$ coefficients (SY channel).
  \item \textbf{Spectral}: The AF channel introduces $\zeta_6$, and $\tilde{D}_\zeta$ closes on $\mathbb{Z}[\varphi, \zeta_6][z]$. The spectral obstruction $10 \cdot (1+\zeta_6)$ shows eigenvalues occupy a proper sub-ring, with the gap controlled by the primes $\{2, 3, 5\}$.
\end{enumerate}
