\section{Derivation of Integers and Primes}
\label{sec:integers-primes}

\subsection{Integers as Orbit Labels}

In FUST, integers are NOT primitive objects. They emerge as discrete labels of $\varphi$-scale orbits in Frourio logarithm space.

\begin{definition}[$\varphi$-Orbit]
\label{def:phi-orbit}
\lean{FUST.PhiOrbit.phiOrbit}
\leanok
For $x_0 \in \mathbb{Q}(\varphi)$, the $\varphi$-orbit is:
\[
\mathrm{Orbit}_\varphi(x_0) := \{ y \mid \exists n \in \mathbb{Z},\, y = \varphi^n \cdot x_0 \}
\]
The initial value $x_0$ is constrained to the golden quadratic field $\mathbb{Q}(\varphi) = \{a + b\varphi \mid a, b \in \mathbb{Q}\}$ by the Frourio admissibility conditions (see \S\ref{sec:phi-orbit-initial-value}).
\end{definition}

The key insight: integers $n$ index positions along the $\varphi$-orbit, forming a $\mathbb{Z}$-torsor in Frourio space. The Frourio coordinate is additive in the index:
\[
t(x_0, n+k) = t(x_0, n) + k \cdot \mathrm{phiStep}
\]

\subsection{Primes in FUST}

Primes are defined NOT by divisibility, but by structural irreducibility under three criteria:

\begin{definition}[Fibonacci Convolution Irreducibility]
\label{def:fib-conv-irred}
\lean{FUST.FrourioLogarithm.FibConvIrreducible}
\leanok
$n \ge 2$ is Fibonacci-convolution irreducible if:
\[
\forall a, b \ge 1: F(a) \cdot F(b) = F(n) \Rightarrow a = 1 \lor b = 1
\]
where $F$ is the FUST structure constant.
\end{definition}

\begin{definition}[D$_6$ Irreducibility]
\label{def:d6-irred}
\lean{FUST.FrourioLogarithm.D6Irreducible}
\leanok
$n \ge 2$ is D$_6$-irreducible if it cannot be decomposed in the D$_6$ kernel:
\[
\nexists (c_0, c_1, c_2) \in \mathbb{Z}^3 \setminus \{(0,0,0)\}: n = c_0 + c_1 \varphi + c_2 \varphi^2
\]
with all $c_i \neq 0$.
\end{definition}

\begin{definition}[$\varphi/\psi$ Resonance]
\label{def:phi-psi-resonance}
\lean{FUST.FrourioLogarithm.PhiPsiResonant}
\leanok
$n$ resonates if it is fixed under $\varphi \leftrightarrow \psi$ exchange:
\[
\frac{\varphi^n - \psi^n}{\sqrt{5}} = \frac{\psi^n - \varphi^n}{\sqrt{5}}
\]
\end{definition}

\begin{theorem}[No $\varphi/\psi$ Resonance]
\label{thm:no-resonance}
\lean{FUST.FrourioLogarithm.no_phi_psi_resonance}
\leanok
For $n \ge 1$: $n$ does NOT resonate.
\end{theorem}

\begin{proof}
Since $\varphi > \psi$ (in fact, $\varphi^n > \psi^n$ for all $n \ge 1$), we have $\varphi^n - \psi^n \neq 0$, so the resonance condition cannot hold.
\end{proof}

\begin{definition}[FUST Prime]
\label{def:fust-prime}
\lean{FUST.FrourioLogarithm.FUSTPrime}
\leanok
$n$ is a FUST prime if it satisfies all three irreducibility criteria:
\begin{enumerate}
    \item Fibonacci-convolution irreducible
    \item D$_6$-irreducible
    \item Not $\varphi/\psi$-resonant
\end{enumerate}
\end{definition}

\begin{theorem}[Standard Primes are Non-Resonant]
\label{thm:nat-prime-non-resonant}
\lean{FUST.FrourioLogarithm.nat_prime_not_resonant}
\leanok
If $p$ is a standard prime, then $p$ is not $\varphi/\psi$-resonant.
\end{theorem}

\subsection{Summary}

\begin{theorem}[Frourio Logarithm Properties]
\label{thm:frourio-log-properties}
\lean{FUST.FrourioLogarithm.frourio_log_properties}
\leanok
\begin{enumerate}
    \item $\forall x > 0$: $\mathrm{frourioExp}(\mathrm{frourioLog}(x)) = x$
    \item $\forall t$: $\mathrm{frourioLog}(\mathrm{frourioExp}(t)) = t$
    \item $\forall x > 0$: $\mathrm{frourioLog}(\varphi \cdot x) = \mathrm{frourioLog}(x) + \mathrm{phiStep}$
\end{enumerate}
\end{theorem}

The Frourio logarithm provides:
\begin{enumerate}
    \item \textbf{Linearization} of $\varphi$-scale generator
    \item \textbf{D$_6$ completeness} bounding observable physics
    \item \textbf{Integers} as orbit labels, not primitives
    \item \textbf{Primes} as structurally irreducible, not divisibility-defined
\end{enumerate}
