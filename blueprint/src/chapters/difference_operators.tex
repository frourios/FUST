% Chapter 3: Difference Operators
\chapter{Difference Operators}
\label{chap:difference_operators}

\section{Definition of Difference Operators (ZF)}
\label{sec:diff_op_def}

\begin{definition}[$D_2$: Frourio Golden 2-Point Difference]
\label{def:D2}
\lean{FUST.D2}
\leanok
\[
D_2 f(x) = \frac{f(\varphi x)-f(\psi x)}{(\varphi-\psi)x}
\]
\end{definition}

\begin{definition}[$D_3$: Frourio Golden 3-Point Difference]
\label{def:D3}
\[
D_3 f(x) = \frac{f(\varphi^2 x) - f(\varphi x) + f(\psi x)}{(\varphi-\psi)^2 x}
\]
\end{definition}

\begin{definition}[$D_4$: Frourio Golden 4-Point Difference]
\label{def:D4}
\[
D_4 f(x) = \frac{f(\varphi^2 x) - \varphi^2 f(\varphi x) + \psi^2 f(\psi x) - f(\psi^2 x)}{(\varphi-\psi)^3 x}
\]
\end{definition}

\begin{definition}[$D_5$: Frourio Golden 5-Point Difference]
\label{def:D5}
\lean{FUST.D5}
\leanok
\[
D_5 f(x) = \frac{f(\varphi^2 x) - a f(\varphi x) + b f(x) - a f(\psi x) + f(\psi^2 x)}{(\varphi-\psi)^4 x}
\]
\end{definition}

\begin{definition}[$D_6$: Frourio Golden 6-Point Difference]
\label{def:D6}
\lean{FUST.D6}
\leanok
\[
D_6 f(x) = \frac{f(\varphi^3 x) - A f(\varphi^2 x) + B f(\varphi x) - B f(\psi x) + A f(\psi^2 x) - f(\psi^3 x)}{(\varphi-\psi)^5 x}
\]
\end{definition}

\begin{definition}[$D_{5.5}$: Half-Order Difference Operator]
\label{def:D5half}
\lean{FUST.D5half}
\leanok
Define the half-order operator $D_{5.5}$ as:
\[
D_{5.5} f(x) := D_5 f(x) + \mu \cdot \Delta_2 f(x)
\]
where $\Delta_2 f(x) := f(\varphi x) - f(\psi x)$ is the \textbf{antisymmetric difference} (the numerator part of $D_2$).
\end{definition}

\textbf{Note}: $\Delta_2$ is different from $D_2$. We have $D_2 f(x) = \Delta_2 f(x) / ((\varphi - \psi) x)$, and $\Delta_2$ is the unnormalized raw difference.

\textbf{Parameter determination}: Solving the eigenvalue problem $D_{5.5}[F_{5.5}] = c \cdot F_{5.5}$, for eigenvalue $c = 0$ we obtain
\[
\mu = \frac{2}{\varphi + 2} = \frac{2}{\varphi^2 + 1} \approx 0.5528
\]
(value at $x = 1$). $D_{5.5}$ is a linear combination of $D_5$ (symmetric) and $\Delta_2$ (antisymmetric).

\section{Uniqueness Theorem for D5 and D6 Coefficients}
\label{sec:coeff_uniqueness}

\subsection{Determination of D5 Coefficients}
\label{subsec:D5_coeff}

$D_5$ is the critical hierarchy that ``permits structural fluctuations but not translations.'' Since there are 2 unknowns $(a,b)$, we impose the following \textbf{minimal 2 conditions}:

\begin{itemize}
\item \textbf{C0} Constant drift elimination: $D_5[1]=0$
\item \textbf{C1} Linear drift elimination: $D_5[x]=0$
\end{itemize}

\textbf{Calculation}:

\textbf{C0} $(f(x)=1)$:
$2-2a+b=0 \Rightarrow b=2a-2$

\textbf{C1} $(f(x)=x)$:
$(\varphi^2+\psi^2)-a(\varphi+\psi)+b=0$, so $3-a+b=0 \Rightarrow b=a-3$

Solving simultaneously:
\[
\boxed{a=-1,\quad b=-4}
\]

\textbf{Final form of $D_5$}:
\[
D_5 f(x)=\frac{f(\varphi^2x)+f(\varphi x)-4f(x)+f(\psi x)+f(\psi^2x)}{(\varphi-\psi)^4 x}
\]

\subsection{Determination of D6 Coefficients}
\label{subsec:D6_coeff}

$D_6$ needs to \textbf{completely eliminate} both translation (linear) and curvature drift (quadratic).

\begin{itemize}
\item \textbf{D1} $D_6[x]=0$
\item \textbf{D2} $D_6[x^2]=0$
\end{itemize}

\textbf{Justification of coefficient determination}:

These conditions are not ``physical selection principles'' but \textbf{mathematical requirements for difference operators}:

\begin{enumerate}
\item \textbf{$D_6[1] = 0$}: Automatically satisfied from the antisymmetric form $[1, -A, B, -B, A, -1]$
\item \textbf{$D_6[x] = 0$}: Eliminates linear functions (translations) --- analogous to differential operators
\item \textbf{$D_6[x^2] = 0$}: Eliminates quadratic functions (curvature) --- realizes the maximum kernel dimension for 6-point operators
\end{enumerate}

\textbf{Calculation} (using Fibonacci identities):

In general: $D_6[x^n] \propto F_{3n}-A F_{2n}+B F_n$

\textbf{n=1}: $2-A+B=0 \Rightarrow B=A-2$

\textbf{n=2}: $8-3A+B=0 \Rightarrow B=3A-8$

Solving simultaneously:
\[
\boxed{A=3,\quad B=1}
\]

\textbf{Final form of $D_6$}:
\[
D_6 f(x)=\frac{f(\varphi^3x)-3f(\varphi^2x)+f(\varphi x)-f(\psi x)+3f(\psi^2x)-f(\psi^3x)}{(\varphi-\psi)^5 x}
\]

\begin{theorem}[$D_5$ and $D_6$ Coefficient Uniqueness Theorem]
\label{thm:coeff_uniqueness}
\lean{FUST.D5_coefficients_unique, FUST.D6_coefficients_unique}
\leanok
In FUST, no other difference coefficients for $D_5$ and $D_6$ exist except
\[
a=-1,\quad b=-4,\quad A=3,\quad B=1
\]
\end{theorem}

This is not an empirical claim but a \textbf{necessary consequence derived only from hierarchical structure, zeros, and criticality}.
