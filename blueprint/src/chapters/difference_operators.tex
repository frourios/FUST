% Chapter 3: Difference Operators and the Unified Dζ
\chapter{Difference Operators and the Unified \texorpdfstring{$D_\zeta$}{Dζ}}
\label{chap:difference_operators}

\section{Component Operators \texorpdfstring{$D_n$}{Dn} (ZF + DC)}
\label{sec:diff_op_def}

All operators act on functions $f : \mathbb{C} \to \mathbb{C}$ with $z \neq 0$. These are the \textbf{hyperbolic components} of $D_\zeta$, acting on the $\varphi$-dilation lattice $\{\varphi^k z\}$.

\begin{definition}[$D_2$: 2-Point Difference]
\label{def:D2}
\lean{FUST.D2}
\leanok
\[
D_2 f(z) = \frac{f(\varphi z)-f(\psi z)}{(\varphi-\psi)z}
\]
\end{definition}

\begin{definition}[$D_3$: 3-Point Difference]
\label{def:D3}
\[
D_3 f(z) = \frac{f(\varphi^2 z) - f(\varphi z) + f(\psi z)}{(\varphi-\psi)^2 z}
\]
\end{definition}

\begin{definition}[$D_4$: 4-Point Difference]
\label{def:D4}
\[
D_4 f(z) = \frac{f(\varphi^2 z) - \varphi^2 f(\varphi z) + \psi^2 f(\psi z) - f(\psi^2 z)}{(\varphi-\psi)^3 z}
\]
\end{definition}

\begin{definition}[$D_5$: 5-Point Difference]
\label{def:D5}
\lean{FUST.D5}
\leanok
\[
D_5 f(z) = \frac{f(\varphi^2 z) + f(\varphi z) - 4f(z) + f(\psi z) + f(\psi^2 z)}{(\varphi-\psi)^4 z}
\]
\end{definition}

\begin{definition}[$D_6$: 6-Point Difference]
\label{def:D6}
\lean{FUST.D6}
\leanok
\[
D_6 f(z) = \frac{f(\varphi^3 z) - 3f(\varphi^2 z) + f(\varphi z) - f(\psi z) + 3f(\psi^2 z) - f(\psi^3 z)}{(\varphi-\psi)^5 z}
\]
\end{definition}

\begin{definition}[$D_{5\frac{1}{2}}$: Half-Order Operator]
\label{def:D5half}
\lean{FUST.D5half}
\leanok
\[
D_{5\frac{1}{2}} f(z) := D_5 f(z) + \mu \cdot \Delta_2 f(z), \quad \mu = \frac{2}{\varphi + 2}
\]
where $\Delta_2 f(z) := f(\varphi z) - f(\psi z)$ is the antisymmetric difference.
\end{definition}

\section{Elliptic Component: \texorpdfstring{$\zeta_6$}{ζ6} Operators}
\label{sec:zeta6_operators}

The $\zeta_6$-rotation operators act on the compact elliptic lattice $\{\zeta_6^k z\}_{k=0}^5$.

\begin{definition}[$\zeta_6$-DFT: Mode-3 Projection]
\label{def:zeta6_N6}
\lean{FUST.Zeta6.zeta6_N6}
\leanok
\[
\zeta_6\text{-}N_6(f)(z) = \sum_{k=0}^{5} (-1)^k f(\zeta_6^k z)
\]
\end{definition}

\begin{theorem}[$\zeta_6$-DFT Kernel]
\label{thm:zeta6_kernel}
\lean{FUST.Zeta6.zeta6_N6_const, FUST.Zeta6.zeta6_N6_linear, FUST.Zeta6.zeta6_N6_quadratic}
\leanok
$\ker(\zeta_6\text{-}N_6) \supseteq \mathrm{span}\{1, z, z^2\}$ and $\zeta_6\text{-}N_6[z^3] = 2z^3 \neq 0$.
The elliptic kernel has the same dimension 3 as the hyperbolic $\ker(D_6)$.
\end{theorem}

\section{Channel Decomposition: AFNum and SymNum}
\label{sec:channel_decomposition}

The $\mathbb{Z}/6\mathbb{Z}$ Fourier transform of $D_\zeta$ decomposes into two channels based on parity:

\begin{definition}[AFNum: Antisymmetric Channel]
\label{def:AFNum}
\lean{FUST.Zeta6.AFNum}
\leanok
\[
\mathrm{AFNum}(g)(z) = g(\zeta_6 z) + g(\zeta_6^2 z) - g(\zeta_6^4 z) - g(\zeta_6^5 z)
\]
Fourier coefficients $[0, 1, 1, 0, -1, -1]$: the \textbf{anti-Fibonacci sequence} on $\mathbb{Z}/6\mathbb{Z}$.
\end{definition}

\begin{definition}[SymNum: Symmetric Channel]
\label{def:SymNum}
\lean{FUST.Zeta6.SymNum}
\leanok
\[
\mathrm{SymNum}(g)(z) = 2g(z) + g(\zeta_6 z) - g(\zeta_6^2 z) - 2g(\zeta_6^3 z) - g(\zeta_6^4 z) + g(\zeta_6^5 z)
\]
Fourier coefficients $[2, 1, -1, -2, -1, 1]$.
\end{definition}

\begin{theorem}[Channel Fourier Coefficients]
\label{thm:channel_coefficients}
\lean{FUST.Zeta6.AF_coeff_eq, FUST.Zeta6.SY_coeff_val}
\leanok
On monomials $g(w) = w^s$:
\[
\mathrm{AF\_coeff} = \zeta_6 + \zeta_6^2 - \zeta_6^4 - \zeta_6^5 = 2i\sqrt{3}, \quad
\mathrm{SY\_coeff} = 6
\]
The antisymmetric coefficient is \textbf{purely imaginary}; the symmetric coefficient is \textbf{real}.
\end{theorem}

\section{Hyperbolic Channels: \texorpdfstring{$\Phi_A$}{φ_A} and \texorpdfstring{$\Phi_S$}{φ_S}}
\label{sec:phi_channels}

\begin{definition}[$\Phi_A$: Antisymmetric Hyperbolic Channel]
\label{def:PhiA}
\lean{FUST.Zeta6.Phi_A}
\leanok
\[
\Phi_A(f)(z) = f(\varphi^3 z) - 4f(\varphi^2 z) + (3+\varphi)f(\varphi z) - (3+\psi)f(\psi z) + 4f(\psi^2 z) - f(\psi^3 z)
\]
6-point antisymmetric stencil on the $\varphi$-lattice. The parabolic point $z = 1$ has coefficient 0 by antisymmetry ($\varphi \leftrightarrow \psi$ parity forces $c = -c \Rightarrow c = 0$).
\end{definition}

\begin{definition}[$\Phi_S$: Symmetric Hyperbolic Channel]
\label{def:PhiS}
\lean{FUST.Zeta6.Phi_S}
\leanok
\[
\Phi_S(f)(z) = 2f(\varphi^2 z) + (3+\mu)f(\varphi z) - 10f(z) + (3-\mu)f(\psi z) + 2f(\psi^2 z)
\]
5-point symmetric stencil \textbf{including} the parabolic point $f(z)$ with coefficient $-10$.
\end{definition}

\begin{theorem}[$\Phi_S$ Rank-Three Decomposition]
\label{thm:PhiS_rank_three}
\lean{FUST.Zeta6.Phi_S_rank_three}
\leanok
$\Phi_S = 2 \cdot N_5 + N_3 + \mu \cdot N_2$, and the $3 \times 3$ matrix of coefficients
$(\sigma_{N_5}, \sigma_{N_3}, \sigma_{N_2})$ evaluated at $s = 1, 5, 7$ has determinant $-6952(\varphi - \psi) \neq 0$.
Thus $\Phi_S$ carries $\mathrm{Fin}\,3$ worth of independent information.
\end{theorem}

\section{The Unified Operator \texorpdfstring{$D_\zeta$}{Dζ}}
\label{sec:unified_Dzeta}

\begin{definition}[$D_\zeta$: Unified Difference Operator]
\label{def:Dzeta}
\lean{FUST.Zeta6.Dzeta}
\leanok
\[
\boxed{D_\zeta(f)(z) = \frac{\mathrm{AFNum}(\Phi_A(f))(z) + \mathrm{SymNum}(\Phi_S(f))(z)}{z}}
\]
\end{definition}

$D_\zeta$ operates on the rank-2 lattice $\langle \varphi, \zeta_6 \rangle \cong \mathbb{Z} \times \mathbb{Z}/6\mathbb{Z}$ and composes all three geometries:

\begin{enumerate}
\item \textbf{Hyperbolic layer} ($\Phi_A, \Phi_S$): $\varphi$-dilation lattice evaluations ($c = +1$)
\item \textbf{Elliptic layer} ($\mathrm{AFNum}, \mathrm{SymNum}$): $\zeta_6$-rotation DFT ($c = -1$)
\item \textbf{Parabolic layer} ($\div z$): Division by $z$ extracts the degenerate root $x = 0$ of $x^2 = x$ ($c = 0$)
\end{enumerate}

\textbf{Composition order is forced}: hyperbolic $\to$ elliptic $\to$ parabolic. Moving $\div z$ before the elliptic layer would shift the Fourier mode index ($s \to s-1$), making $\mathrm{AF\_coeff}(0) = 0$ and destroying the antisymmetric channel.

\begin{theorem}[$D_\zeta$ Norm-Squared Decomposition]
\label{thm:Dzeta_normSq}
\lean{FUST.Zeta6.Dzeta_normSq_decomposition}
\leanok
\[
|6a + \mathrm{AF\_coeff} \cdot b|^2 = 12(3a^2 + b^2)
\]
The weight ratio 3:1 between symmetric and antisymmetric channels encodes $I_4 = \mathrm{Fin}\,3 \oplus \mathrm{Fin}\,1$ (spacetime decomposition).
\end{theorem}

\section{\texorpdfstring{$D_\zeta$}{Dζ} Encodes All Six Component Operators}
\label{sec:dzeta_encodes}

The component operators $D_2, \ldots, D_6$ are recovered as specific projections of $D_\zeta$:

\begin{center}
\begin{tabular}{|c|c|c|c|}
\hline
$D_\zeta$ projection & Operator & Channel & Normalization power $r$ \\
\hline
$\mathrm{AFNum} / (\zeta_6 - \bar{\zeta}_6)$ & $D_{\zeta,6}$ & AF (odd) & $r = 1$ \\
$\mathrm{SymNum} / (\zeta_6 - \bar{\zeta}_6)^2$ & $D_{\zeta,5}$ & SY (even) & $r = 2$ \\
$\mathrm{AFNum} / (\zeta_6 - \bar{\zeta}_6)^3$ & $D_{\zeta,4}$ & AF (odd) & $r = 3$ \\
$\mathrm{SymNum} / (\zeta_6 - \bar{\zeta}_6)^4$ & $D_{\zeta,3}$ & SY (even) & $r = 4$ \\
$\mathrm{AFNum} / (\zeta_6 - \bar{\zeta}_6)^5$ & $D_{\zeta,2}$ & AF (odd) & $r = 5$ \\
\hline
\end{tabular}
\end{center}

\section{Uniqueness of Coefficients}
\label{sec:coeff_uniqueness}

\begin{theorem}[$D_5$ and $D_6$ Coefficient Uniqueness]
\label{thm:coeff_uniqueness}
\lean{FUST.D5_coefficients_unique, FUST.D6_coefficients_unique}
\leanok
The only coefficients satisfying the kernel conditions are $a=-1, b=-4$ ($D_5$) and $A=3, B=1$ ($D_6$).
\end{theorem}

Since all five operators have distinct dimensions, $D_i f(z) + D_j f(z)$ ($i \neq j$) is a type error.
