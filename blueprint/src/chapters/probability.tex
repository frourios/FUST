% FUST Probability Theory
\chapter{FUST Probability Theory}
\label{chap:probability}

This chapter presents probability theory derived structurally from the
$F_\zeta = 5z \cdot D_\zeta$ operator, without external probability axioms or measure theory
assumptions. Probability emerges from:
\begin{enumerate}
\item $\varphi$-scale iteration generating observation sequences
\item Empirical distribution from frequency counts
\item Born rule as a structural consequence of $F_\zeta$ action
\end{enumerate}

\section{Gauge Invariance of Same-Degree Ratios}
\label{sec:gauge_invariance}

\subsection{\texorpdfstring{Linearity of $F_\zeta$}{Linearity of Fζ}}

\begin{theorem}[$F_\zeta$ Linear Scalar]
\label{thm:Fzeta_linear}
\lean{FUST.Probability.Fζ_linear_scalar}
\leanok
For any scalar $a \in \mathbb{C}$ and function $f$:
\[
F_\zeta[a \cdot f](x) = a \cdot F_\zeta[f](x)
\]
\end{theorem}

\subsection{Gauge Invariance}

For same-degree monomials $f_A = a \cdot x^n$ and $f_B = b \cdot x^n$:
\[
\frac{|F_\zeta[f_A](x)|}{|F_\zeta[f_B](x)|} = \frac{|a|}{|b|}
\]
This is independent of $x$ (gauge-invariant).

\begin{theorem}[Same-Degree Ratio Gauge Invariance]
\label{thm:gauge_invariant}
\lean{FUST.Probability.same_degree_ratio_gauge_invariant}
\leanok
For $b \neq 0$ and $F_\zeta[f](x) \neq 0$:
\[
\frac{|F_\zeta[a \cdot f](x)|}{|F_\zeta[b \cdot f](x)|} = \frac{|a|}{|b|}
\]
\end{theorem}

\begin{proof}
By linearity:
\begin{align*}
\frac{|F_\zeta[a \cdot f](x)|}{|F_\zeta[b \cdot f](x)|}
&= \frac{|a \cdot F_\zeta[f](x)|}{|b \cdot F_\zeta[f](x)|} \\
&= \frac{|a| \cdot |F_\zeta[f](x)|}{|b| \cdot |F_\zeta[f](x)|} \\
&= \frac{|a|}{|b|}
\end{align*}
\end{proof}

\begin{corollary}[Position Independence]
\label{cor:position_independent}
\lean{FUST.Probability.monomial_ratio_invariant}
\leanok
For any $x, y$ with $F_\zeta[t^n](x) \neq 0$ and $F_\zeta[t^n](y) \neq 0$:
\[
\frac{|F_\zeta[a \cdot t^n](x)|}{|F_\zeta[b \cdot t^n](x)|} =
\frac{|F_\zeta[a \cdot t^n](y)|}{|F_\zeta[b \cdot t^n](y)|}
\]
\end{corollary}

\subsection{Physical Interpretation}

The gauge invariance of same-degree ratios means that:
\begin{itemize}
\item Probability ratios are well-defined
\item They don't depend on the choice of ``gauge'' (position $x$)
\item This provides the foundation for FUST probability theory
\end{itemize}

\section{Observation Sequences}
\label{sec:observation}

\subsection{\texorpdfstring{$\varphi$-Scale Transformation}{phi-Scale Transformation}}

The $\varphi$-scale transformation:
\[
x \mapsto \varphi^k \cdot x \quad (k \in \mathbb{Z})
\]
generates observation sequences that correspond to ``trials'' in probability theory.

\begin{definition}[Observation at Step $k$]
\label{def:observation}
\lean{FUST.Probability.observationAt}
\leanok
For state $f$ and base point $x_0$:
\[
A_f(k) := |F_\zeta[f](\varphi^k \cdot x_0)|
\]
\end{definition}

\begin{theorem}[Observations Non-negative]
\label{thm:obs_nonneg}
\lean{FUST.Probability.observationAt_nonneg}
\leanok
\[
A_f(k) \geq 0 \quad \forall k \in \mathbb{Z}
\]
\end{theorem}

\subsection{\texorpdfstring{$\varphi$-Invariant Measure}{phi-Invariant Measure}}

The $\varphi$-scale invariant measure is uniquely determined:
\[
d\mu = \frac{dx}{x}
\]
This is the Haar measure on the multiplicative group, derived from
$\varphi$-scale structure (not assumed externally).

\begin{theorem}[Shift Invariance]
\label{thm:shift_invariance}
\lean{FUST.Probability.observation_shift}
\leanok
\[
A_f(k; \varphi \cdot x_0) = A_f(k+1; x_0)
\]
\end{theorem}

\subsection{Discrete Action}

\begin{definition}[Discrete Action]
\label{def:discrete_action}
\lean{FUST.Probability.discreteAction}
\leanok
\[
\mathcal{A}_N[f] := \sum_{k=-N}^{N} A_f(k)^2 \cdot \log\varphi
\]
\end{definition}

\begin{definition}[Haar Weight]
\label{def:haar_weight}
\lean{FUST.Probability.haarWeight}
\leanok
\[
w := \log\varphi
\]
\end{definition}

\begin{theorem}[Action Non-negative]
\label{thm:action_nonneg}
\lean{FUST.Probability.discreteAction_nonneg}
\leanok
\[
\mathcal{A}_N[f] \geq 0
\]
\end{theorem}

\begin{theorem}[Kernel States Have Zero Action]
\label{thm:action_zero_ker}
\lean{FUST.Probability.action_zero_for_ker}
\leanok
If $f \in \ker(F_\zeta)$, then $\mathcal{A}_N[f] = 0$ for all $N$.
\end{theorem}

\section{Empirical Distribution}
\label{sec:distribution}

\subsection{Empirical Distribution Function}

\begin{definition}[Count Below Threshold]
\label{def:count_below}
\lean{FUST.Probability.countBelow}
\leanok
\[
\#\{k \in [-N, N] \mid A_f(k) \leq t\}
\]
\end{definition}

\begin{definition}[Empirical Distribution at Finite $N$]
\label{def:empirical_dist}
\lean{FUST.Probability.empiricalDistN}
\leanok
\[
\rho_f^{(N)}(t) := \frac{\#\{k \in [-N, N] \mid A_f(k) \leq t\}}{2N + 1}
\]
\end{definition}

\begin{theorem}[Distribution Bounds]
\label{thm:dist_bounds}
\lean{FUST.Probability.empiricalDistN_bounds}
\leanok
\[
0 \leq \rho_f^{(N)}(t) \leq 1
\]
\end{theorem}

\begin{theorem}[Distribution at Infinity]
\label{thm:dist_infinity}
\lean{FUST.Probability.empiricalDistN_at_infinity}
\leanok
If all observations are bounded by $M < t$, then:
\[
\rho_f^{(N)}(t) = 1
\]
\end{theorem}

\subsection{FUST Probability}

For set $E \subseteq \mathbb{R}_{\geq 0}$:

\begin{definition}[FUST Probability at Finite $N$]
\label{def:fust_prob}
\lean{FUST.Probability.fustProbN}
\leanok
\[
P_f^{(N)}(E) := \frac{\#\{k \in [-N, N] \mid A_f(k) \in E\}}{2N + 1}
\]
\end{definition}

\begin{theorem}[Normalization]
\label{thm:normalization}
\lean{FUST.Probability.fustProb_normalization}
\leanok
\[
P_f^{(N)}(\mathbb{R}_{\geq 0}) = 1
\]
\end{theorem}

\begin{proof}
Since all observations are non-negative ($A_f(k) \geq 0$), every observation
is in $\mathbb{R}_{\geq 0}$.
\end{proof}

\begin{theorem}[Probability Non-negative]
\label{thm:prob_nonneg}
\lean{FUST.Probability.fustProbN_nonneg}
\leanok
\[
P_f^{(N)}(E) \geq 0
\]
\end{theorem}

\begin{theorem}[Probability Bounded]
\label{thm:prob_bounded}
\lean{FUST.Probability.fustProbN_le_one}
\leanok
\[
P_f^{(N)}(E) \leq 1
\]
\end{theorem}

\section{Born Rule}
\label{sec:born_rule}

\subsection{Action and Expectation}

The FUST action:
\[
\mathcal{A}[f] = \int |F_\zeta[f](x)|^2 \frac{dx}{x}
\]
equals the probability expectation:
\[
\mathcal{A}[f] = \mathbb{E}_{P_f}[|F_\zeta[f]|^2]
\]

This shows Born rule is NOT a new axiom but a structural consequence
of the $F_\zeta$ action.

\subsection{Discrete Version}

\begin{definition}[Discrete Expectation]
\label{def:discrete_expectation}
\lean{FUST.Probability.discreteExpectation}
\leanok
\[
\mathbb{E}_N[|F_\zeta[f]|^2] := \frac{1}{2N+1} \sum_{k=-N}^{N} A_f(k)^2
\]
\end{definition}

\begin{theorem}[Born Rule (Discrete)]
\label{thm:born_rule}
\lean{FUST.Probability.born_rule_discrete}
\leanok
\[
\mathcal{A}_N[f] = (2N + 1) \cdot \log\varphi \cdot \mathbb{E}_N[|F_\zeta[f]|^2]
\]
\end{theorem}

\begin{proof}
By definition:
\begin{align*}
\mathcal{A}_N[f] &= \sum_{k=-N}^{N} A_f(k)^2 \cdot \log\varphi \\
&= (2N+1) \cdot \log\varphi \cdot \frac{1}{2N+1} \sum_{k=-N}^{N} A_f(k)^2 \\
&= (2N+1) \cdot \log\varphi \cdot \mathbb{E}_N[|F_\zeta[f]|^2]
\end{align*}
\end{proof}

\subsection{Physical Interpretation}

The Born rule $P \propto |\psi|^2$ in quantum mechanics is usually
postulated as a fundamental axiom. In FUST:

\begin{enumerate}
\item $|F_\zeta[f]|^2$ naturally appears in the action functional
\item The $\varphi$-scale iteration generates a ``sample space''
\item Frequency counts give empirical probabilities
\item The action equals the expectation of $|F_\zeta[f]|^2$
\end{enumerate}

\textbf{Conclusion}: Born rule emerges from the $F_\zeta$ structure
without being postulated.

\subsection{Complete Summary}

\begin{theorem}[FUST Probability Theory]
\label{thm:fust_probability}
\lean{FUST.Probability.fust_probability_theory}
\leanok
\begin{enumerate}
\item[(A)] Same-degree ratio is gauge-invariant
\item[(B)] Observation values are non-negative
\item[(C)] Probability normalization: $P_f(\mathbb{R}_{\geq 0}) = 1$
\item[(D)] Born rule: Action $= (2N+1) \cdot \log\varphi \cdot$ Expectation
\item[(E)] $\ker(F_\zeta)$ states have zero action
\end{enumerate}
\end{theorem}
