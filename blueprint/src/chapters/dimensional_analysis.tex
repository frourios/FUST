% Chapter: Dimensional Analysis --- Structural Typing of FUST Quantities
\chapter{Dimensional Analysis}
\label{chap:dimensional_analysis}

\section{Motivation}
\label{sec:dim_motivation}

In FUST, quantities such as $t_{\mathrm{FUST}} = 25/12$ and $\Delta = 12/25$ are all defined as $\mathbb{R}$ in the Lean implementation.
This allows type-checked operations like $t_{\mathrm{FUST}} + \Delta$ that are physically meaningless.
Moreover, $t_{\mathrm{FUST}}$ (time) and $l_{\mathrm{FUST}}$ (length) have the same numerical value $25/12$
(\texttt{planck\_time\_space\_equivalence: rfl}), yet represent fundamentally different physical quantities.

This chapter introduces a \textbf{structural type system} $(\sqrt{5}, \delta, \tau)$ derived from FUST's internal algebraic structure,
classifying all quantities into three disjoint types: \textbf{ScaleQuantity}, \textbf{CountQuantity}, and \textbf{RatioQuantity}.

\section{Three Independent Dimensions}
\label{sec:dim_three_dimensions}

\begin{definition}[FUST Dimension]
\label{def:dim_fdim}
\lean{FUST.Dim.FDim}
\leanok
The dimension of a FUST ScaleQuantity is a triple $(\sqrt{5}, \delta, \tau) \in \mathbb{Z}^3$:
\begin{itemize}
\item $\sqrt{5}$: \textbf{$\sqrt{5}$-power dimension} --- exponent of $(\varphi - \psi) = \sqrt{5}$
\item $\delta$: \textbf{dissipation dimension} --- count of $C_d$ normalization factors
\item $\tau$: \textbf{time dimension} --- $\ker(D_6)^\perp$ direction count
\end{itemize}
\end{definition}

\subsection{Dimension I: \texorpdfstring{$\sqrt{5}$}{√5}-Power Dimension}

The discrete scale lattice $\varphi^k$ ($k \in \mathbb{Z}$) generates FUST's hierarchical structure.
Time evolution $f \mapsto f(\varphi \cdot)$ shifts $\sqrt{5}$ by $+1$.
The normalization denominator $(\varphi - \psi)^{m-1}$ of each $D_m$ contributes $-(m-1)$ to $\sqrt{5}$.

\subsection{Dimension II: Dissipation Dimension \texorpdfstring{$\delta$}{δ}}

The dissipation coefficient $C_d$ is the value of $D_6$ applied to $x^d$.
Even-$m$ operators ($D_2, D_4, D_6$) have antisymmetric coefficient patterns yielding $\sqrt{5} \cdot F_n$ (Fibonacci) factors,
giving $\delta = 1$.
Odd-$m$ operators ($D_3, D_5$) have symmetric patterns yielding Lucas integers, giving $\delta = 0$.

\subsection{Dimension III: Time Dimension \texorpdfstring{$\tau$}{τ}}

$\ker(D_6) = \mathrm{span}\{1, x, x^2\}$ defines the spatial sector.
Only $D_6$ detects $x^3 \notin \ker(D_6)$ (temporal sector), hence only $D_6$ carries $\tau = -1$.
All other operators detect modes within $\ker(D_6)$, hence $\tau = 0$.

\section{Type Hierarchy}
\label{sec:dim_type_hierarchy}

\begin{definition}[ScaleQuantity]
\label{def:dim_scaleq}
\lean{FUST.Dim.ScaleQ}
\leanok
$\mathrm{ScaleQ}(\sqrt{5}, \delta, \tau)$: a real-valued quantity carrying dimension $(\sqrt{5}, \delta, \tau)$.
Addition is restricted to quantities of identical dimension.
\end{definition}

\begin{definition}[CountQuantity]
\label{def:dim_countq}
\lean{FUST.Dim.CountQ}
\leanok
$\mathrm{CountQ}$: a natural number arising from kernel dimensions, pair counts, or tensor products.
Carries no $(\sqrt{5}, \delta, \tau)$ structure.
\end{definition}

\begin{definition}[RatioQuantity]
\label{def:dim_ratioq}
\lean{FUST.Dim.RatioQ}
\leanok
$\mathrm{RatioQ}$: a rational number formed as ratios of pair counts.
Dimensionless ($\sqrt{5} = \delta = \tau = 0$).
\end{definition}

\begin{theorem}[Type Separation]
\label{thm:dim_type_separation}
Addition across types is prohibited:
\begin{itemize}
\item $\mathrm{ScaleQ}(d_1) + \mathrm{ScaleQ}(d_2)$ requires $d_1 = d_2$
\item $\mathrm{ScaleQ} + \mathrm{CountQ}$: type error
\item $\mathrm{ScaleQ} + \mathrm{RatioQ}$: type error
\item $\mathrm{CountQ} + \mathrm{RatioQ}$: type error
\end{itemize}
\end{theorem}

\section{Algebra of Dimensions}
\label{sec:dim_algebra}

\begin{theorem}[Multiplicative Group]
\label{thm:dim_comm_group}
\lean{FUST.Dim.instCommGroupFDim}
\leanok
$(\sqrt{5}, \delta, \tau)$ forms a commutative group under componentwise addition:
\begin{align}
\dim(Q_1 \cdot Q_2) &= (\sqrt{5}_1 + \sqrt{5}_2,\; \delta_1 + \delta_2,\; \tau_1 + \tau_2) \\
\dim(Q_1 / Q_2) &= (\sqrt{5}_1 - \sqrt{5}_2,\; \delta_1 - \delta_2,\; \tau_1 - \tau_2)
\end{align}
\end{theorem}

\section{Difference Operator Output Dimensions}
\label{sec:dim_operator_output}

\begin{definition}[Operator Output Dimension]
\label{def:dim_derive_fdim}
\lean{FUST.Dim.deriveFDim}
\leanok
The output dimension of $D_m$ is:
\[
\dim(D_m f(x)) = (-(m-1),\; \delta_m,\; \tau_m)
\]
where $\delta_m = [m \bmod 2 = 0]$ and $\tau_m = -[m = 6]$.
\end{definition}

\begin{center}
\begin{tabular}{|c|c|c|c|c|}
\hline
Operator & Symmetry & $\delta$ & $\tau$ & $(\sqrt{5}, \delta, \tau)$ \\
\hline
$D_2$ & antisymmetric (Fibonacci) & 1 & 0 & $(-1, 1, 0)$ \\
$D_3$ & symmetric (Lucas) & 0 & 0 & $(-2, 0, 0)$ \\
$D_4$ & antisymmetric (Fibonacci) & 1 & 0 & $(-3, 1, 0)$ \\
$D_5$ & symmetric (Lucas) & 0 & 0 & $(-4, 0, 0)$ \\
$D_6$ & antisymmetric (Fibonacci) & 1 & $-1$ & $(-5, 1, -1)$ \\
\hline
\end{tabular}
\end{center}

\begin{theorem}[Operators Have Distinct Dimensions]
\label{thm:dim_operators_distinct}
For $i \neq j$, $\dim(D_i f) \neq \dim(D_j f)$, so $D_i f(x) + D_j f(x)$ is a type error.
\end{theorem}

\begin{theorem}[$D_6$ Output Is Inverse Time]
\label{thm:dim_D6_inverse_time}
\lean{FUST.Dim.observation_is_D6_output}
\leanok
\[
\dim(D_6 f) = (-5, 1, -1) = \dim(t_{\mathrm{FUST}})^{-1}
\]
\end{theorem}

\section{Structural Constants}
\label{sec:dim_structural_constants}

\begin{definition}[Time Dimension]
\label{def:dim_time}
\lean{FUST.Dim.dimTime}
\leanok
\[
\dim(t_{\mathrm{FUST}}) := (\dim(D_6 f))^{-1} = (5, -1, 1)
\]
\end{definition}

\begin{definition}[Length Dimension]
\label{def:dim_length}
\lean{FUST.Dim.dimLength}
\leanok
\[
\dim(l_{\mathrm{FUST}}) := (5, -1, 0)
\]
Length shares $(\sqrt{5}, \delta) = (5, -1)$ with time but has $\tau = 0$ (spatial).
\end{definition}

\begin{definition}[Light Speed Dimension]
\label{def:dim_lightspeed}
\lean{FUST.Dim.dimLightSpeed}
\leanok
\[
\dim(c) := \dim(l) - \dim(t) = (0, 0, -1)
\]
\end{definition}

\begin{theorem}[$l = c \cdot t$ Dimensional Consistency]
\label{thm:dim_l_eq_ct}
\lean{FUST.Dim.length_eq_speed_times_time}
\leanok
\[
(5, -1, 0) = (0, 0, -1) + (5, -1, 1) \quad \checkmark
\]
\end{theorem}

\begin{definition}[Lagrangian Dimension]
\label{def:dim_lagrangian}
\lean{FUST.Dim.dimLagrangian}
\leanok
\[
\dim((D_6 f)^2) = 2 \cdot (-5, 1, -1) = (-10, 2, -2)
\]
\end{definition}

\section{$D_{5\frac{1}{2}}$ Direct Sum Structure}
\label{sec:dim_d5half}

\begin{definition}[$D_{5\frac{1}{2}}$ Dimension]
\label{def:dim_d5half}
\lean{FUST.Dim.D5half_dim}
\leanok
$D_{5\frac{1}{2}} f(x) = D_5 f(x) + \frac{2}{\varphi} \cdot x \cdot D_2 f(x)$ has a direct sum type:
\[
D_{5\frac{1}{2}} f(x) \in \mathrm{ScaleQ}(-4, 0, 0) \oplus \mathrm{ScaleQ}(-2, 1, 0)
\]
The $D_5$ component is symmetric (Lucas, $\delta = 0$) and the $D_2$ component is antisymmetric (Fibonacci, $\delta = 1$).
This is a \textbf{dimension bridge operator} connecting the two sectors.
\end{definition}

\section{Physical Quantity Classification}
\label{sec:dim_classification}

\subsection{ScaleQuantity Instances}

\begin{center}
\begin{tabular}{|c|c|c|}
\hline
Quantity & $(\sqrt{5}, \delta, \tau)$ & Lean reference \\
\hline
$t_{\mathrm{FUST}} = 25/12$ & $(5, -1, 1)$ & \texttt{structuralMinTime\_dim} \\
$l_{\mathrm{FUST}} = 25/12$ & $(5, -1, 0)$ & \texttt{structuralMinLength\_dim} \\
$c = 1$ & $(0, 0, -1)$ & \texttt{dimLightSpeed} \\
$D_6 f(x)$ & $(-5, 1, -1)$ & \texttt{D6\_dim} \\
$(D_6 f)^2$ & $(-10, 2, -2)$ & \texttt{dimLagrangian} \\
$\Delta = 12/25 = 1/t_{\mathrm{FUST}}$ & $(-5, 1, -1)$ & \texttt{massGap}$\Delta$ \\
$\Delta^2 = 144/625$ & $(-10, 2, -2)$ & \texttt{massGap}$\Delta$\texttt{\_sq} \\
$m_s/m_d = \varphi^6$ & $(0, 0, 0)$ & \texttt{msMdRatio} (dimensionless) \\
$\alpha = \varphi^{-5}/4\pi$ & $(0, 0, 0)$ & \texttt{fineStructure} (dimensionless) \\
\hline
\end{tabular}
\end{center}

\subsection{CountQuantity Instances}

\begin{center}
\begin{tabular}{|c|c|c|}
\hline
Quantity & Value & Lean reference \\
\hline
$\dim\ker(D_5)$ & 2 & \texttt{kerDimD5} \\
$\dim\ker(D_6)$ & 3 & \texttt{kerDimD6} \\
Spacetime dimension & 4 & \texttt{spacetimeDim} \\
Active D-levels & 5 & \texttt{activeDLevels} \\
SM particle count & 37 & \texttt{smParticleCount} \\
\hline
\end{tabular}
\end{center}

\subsection{RatioQuantity Instances}

\begin{center}
\begin{tabular}{|c|c|c|}
\hline
Quantity & Value & Lean reference \\
\hline
$\sin^2\theta_W$ & $C(3,2)/(C(3,2)+C(5,2)) = 3/13$ & \texttt{weinbergAngle} \\
$\alpha_s$ & $C(3,2)/(C(5,2)+C(6,2)) = 3/25$ & \texttt{strongCoupling} \\
$m_u/m_d$ & $C(2,2)/2 = 1/2$ & \texttt{muMdRatio} \\
$m_W^2/m_Z^2$ & $C(5,2)/(C(3,2)+C(5,2)) = 10/13$ & \texttt{wzRatioSq} \\
$\sin^2\theta_{12}$ & $1/C(3,2) = 1/3$ & \texttt{solarMixing} \\
\hline
\end{tabular}
\end{center}

\section{Consistency Verification}
\label{sec:dim_consistency}

\begin{theorem}[Blueprint Consistency]
\label{thm:dim_blueprint_consistency}
The dimensional type system is consistent with all blueprint theorems:
\begin{center}
\begin{tabular}{|c|c|}
\hline
Blueprint theorem & Dimensional verification \\
\hline
$t_{\mathrm{FUST}} = 25/12$ & Scale$(5,-1,1)$ $\checkmark$ \\
$l_{\mathrm{FUST}} = 25/12$ & Scale$(5,-1,0)$ $\checkmark$ \\
$l = c \cdot t$ & $(5,-1,0) = (0,0,-1) + (5,-1,1)$ $\checkmark$ \\
$\Delta = 12/25 = 1/t_{\mathrm{FUST}}$ & Scale$(-5,1,-1)$ $\checkmark$ \\
$\sin^2\theta_W = 3/13$ & Ratio (Count/Count) $\checkmark$ \\
$(D_6 f)^2$ Hamiltonian & Scale$(-10,2,-2)$ $\checkmark$ \\
$D_6[\mathrm{timeEvolution}(f)] = \varphi \cdot D_6[f](\varphi \cdot)$ & Both sides: $(-5,1,-1)$ $\checkmark$ \\
\hline
\end{tabular}
\end{center}
\end{theorem}

\section{Energy Conservation in Proper Time}
\label{sec:dim_energy_conservation}

Time evolution amplifies $D_6$ by $\varphi$:
\[
D_6[\mathrm{timeEvolution}(f)](x) = \varphi \cdot D_6[f](\varphi x)
\]
Squaring: $H' = \varphi^2 \cdot H$ in coordinate time. The $\tau = -2$ component of $\dim((D_6 f)^2) = (-10, 2, -2)$ indicates inverse-square-time scaling. The conserved quantity in proper time is:
\[
E_{\mathrm{conserved}} = \frac{H}{\varphi^{2 \cdot \mathrm{step}}}
\]

\begin{center}
\begin{tabular}{|c|c|}
\hline
FUST & General Relativity \\
\hline
Coordinate step & Coordinate time $t$ \\
$(D_6 f)^2 : (-10, 2, -2)$ & Metric $g_{00}$ \\
$\varphi^2$ amplification & Gravitational redshift \\
$E_{\mathrm{conserved}}$ & $\nabla_\mu T^{\mu\nu} = 0$ \\
\hline
\end{tabular}
\end{center}

\section{\texorpdfstring{$\Delta^2 \leq E$}{Δ^2 ≤ E} Interpretation}
\label{sec:dim_spectral_gap}

\begin{theorem}[Spectral Gap Dimensional Interpretation]
\label{thm:dim_spectral_gap}
\lean{FUST.Hamiltonian.spectral_gap_squared}
\leanok
$\Delta = C_3/(\sqrt{5})^5 = 12/25$ is ScaleQuantity$(-5, 1, -1)$, and $\Delta^2 = 144/625$ is ScaleQuantity$(-10, 2, -2)$.
The comparison $\Delta^2 \leq E$ in \texttt{EnergyInSpectrum} compares quantities of the same dimension $(-10, 2, -2)$,
which is dimensionally consistent.
\end{theorem}

\section{Degree Constraint from Mass Condition}
\label{sec:dim_degree_constraint}

The BH formation condition $|D_6(x^d)(x_0)| < 1$ constrains admissible polynomial degrees.
For monomial state functions $g(x) = x^d$:
\[
D_6(x^d)(x_0) = C_d \cdot x_0^{d-1}, \quad C_d = \frac{F_{3d} - 3F_{2d} + F_d}{25}
\]
where $F_n$ is the $n$-th Fibonacci number. Since $C_d \sim \varphi^{3d}/(25\sqrt{5})$,
higher degrees produce larger mass, bounding degree from above.

\begin{theorem}[\texorpdfstring{$D_6(x^4)$}{D6(x\^{}4)} Quartic Formula]
\label{thm:dim_d6_quartic}
\lean{FUST.D6_quartic_eq}
\leanok
\[
D_6(x^4)(x_0) = \frac{84}{25} \cdot x_0^3
\]
where $84 = F_{12} - 3F_8 + F_4 = 144 - 63 + 3$.
\end{theorem}

\begin{theorem}[Cubic Admissible in Domain]
\label{thm:dim_cubic_admissible}
\lean{FUST.cubic_admissible_in_domain}
\leanok
For $x_0 \in (0, 1]$ (FUST coordinate domain):
\[
|D_6(x^3)(x_0)| = \frac{12}{25} \cdot x_0^2 \leq \frac{12}{25} < 1
\]
The cubic mode is admissible throughout the FUST domain.
\end{theorem}

\begin{theorem}[Quartic Inadmissible at \texorpdfstring{$x_0 = 1$}{x0 = 1}]
\label{thm:dim_quartic_inadmissible}
\lean{FUST.quartic_inadmissible_at_one}
\leanok
\[
D_6(x^4)(1) = \frac{84}{25} = 3.36 > 1
\]
The quartic mode exceeds the mass threshold at $x_0 = 1$.
\end{theorem}

\begin{definition}[Admissible Mode]
\label{def:dim_admissible_mode}
\lean{FUST.IsAdmissibleMode}
\leanok
A mode $(d, x_0)$ is \emph{admissible} if $x_0 \neq 0 \Rightarrow |D_6(x^d)(x_0)| < 1$.
\end{definition}

\begin{theorem}[\texorpdfstring{$d_{\max} = 3$}{dmax = 3} at \texorpdfstring{$x_0 = 1$}{x0 = 1}]
\label{thm:dim_dmax_at_one}
\lean{FUST.d_max_at_one}
\leanok
At $x_0 = 1$, the cubic mode is the unique admissible massive mode:
\[
\mathrm{IsAdmissibleMode}(3, 1) \quad \land \quad \lnot\, \mathrm{IsAdmissibleMode}(4, 1)
\]
\end{theorem}

\begin{center}
\begin{tabular}{|c|c|c|c|}
\hline
$k$ & $x_0 = \varphi^{-k}$ & $d_{\max}$ & Physical meaning \\
\hline
0 & 1 & 3 & Cubic only \\
1 & $\varphi^{-1} \approx 0.618$ & 4 & Cubic + quartic \\
2 & $\varphi^{-2} \approx 0.382$ & 6 & Low-degree modes \\
$\geq 3$ & $\leq \varphi^{-3} \approx 0.236$ & $\infty$ & All degrees allowed \\
\hline
\end{tabular}
\end{center}

The critical threshold $k = 3$ corresponds to the highest node degree $\varphi^3$ in $D_6$.
The net $\varphi$-exponent $d(3-k) + k$ becomes $d$-independent at $k = 3$,
with limit $\varphi^3/(25\sqrt{5}) \approx 0.076 < 1$.
